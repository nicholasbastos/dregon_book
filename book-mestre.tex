\documentclass[masters]{BYUPhys}
\usepackage[portuguese,noprefix]{nomencl}
\usepackage[portuges, brazil]{babel}

\usepackage{eso-pic}
%\usepackage{wallpaper}

\newcommand\BackgroundPic{
\put(0,0){
\parbox[b][\textheight]{\textwidth}{%\parbox[b][\paperheight]{\paperwidth}{
\vfill
\centering
%\includegraphics[height=1.5\textheight, width=1.05\textheight]{"page"}%
\vfill
}}}
% ------- Fill in these fields for the preliminary pages --------
%
% For Senior and honors this is the year and month that you submit the thesis
% For Masters and PhD, this is your graduation date
  \Year{2016}
  \Author{N.b.m. - V.1.2}

% If you have a long title, split it between two lines using the \\ command.
% A multiple line title should be an "inverted pyramid" with the top line(s)
% longer than the bottom.
  \Title{Livro do Mestre - Dragon}

\begin{document}

 % Start page counting in roman numerals
 \frontmatter

 % This command makes the formal preliminary pages.
 % You can comment it out during the drafting process if you want to save paper.

 \makepreliminarypages

 \singlespace

 % Make the table of contents.
 \tableofcontents

% \clearemptydoublepage

 %\listoffigures

 %\listoftables

 \doublespace

 % Start regular page counting at page 1
 \mainmatter
\AddToShipoutPicture{\BackgroundPic}


%%
%% Cap�tulo 1: mestre
%%


\chapter{Introdu��o}

\label{Cap:intro}

O mestre, tamb�m chamado de narrador, � responsavel por transportar os jogadores para o cen�rio de jogo, ou seja, ele � a ponte entre os jogadores e o mundo aonde os personagens existem. Al�m de descrever o cen�rio e seus personagens (por isso o nome de narrador) o mestre tamb�m deve dar din�mica ao jogo, criando situa��es, fazendo julgamentos, impondo barreiras, recompensando objetivos alcan�ados, entre outros v�rios atos importante dentro de jogo. 

Para ajudar o mestre em tal tarefa, muitas vezes complicada, aqui se segue v�rios pontos de ref�ncia para guiar o mestre. Quest�es como "esse equipamento � bom para essa situa��o" ou "que dificuldades devem passar os jogadores para percorrer essa caverna" ter�o respostas nesse livro, que deve ser usado como ponto de refer�ncia, e nunca como restri��o da narrativa do jogo. Nesse cap�tulo iremos mostrar como o mestre pode usar a liberadade a ele conferida para guiar o grupo e criar aventuras equilibradas.

\section{Liberdade}
A primeira e �nica regra que o mestre deve seguir fielmente � a chamada "Regra de Ouro". Ela diz que o mestre tem total controle sobre qualquer regra de um sistema. Ou seja, se em determinada situa��o o mestre julga que regra "X" est� atrapalhando o andamento do jogo, ent�o ele pode mudar essa regra. O sistema dragon foi criado dando liberdade para o mestre, ou seja, ele tem liberdade para alterar boa parte das regras sem prejudicar o sistema como um todo. � importante o mestre usar a "Regra de Ouro" com cautela. Seu uso excessivo pode trazer desequilibrio ao jogo.

Por exemplo, o mestre nunca deve usar seu poder para prejudicar os jogadores intensionalmente. Se algum personagem do grupo est� muito forte, provavelmente ele o fez por merecer, j� que o sistema foi criado para n�o gerar personagens desequilibradamente fortes dentro de um grupo, e mesmo que o sistema falhe o mestre pode corrigir esse erro. O mestre n�o deve colocar aquele personagem em especial em situa��es que o prejudiquem somente por que ele � forte, enquanto os outros ficam numa situa��o mais favor�vel. O que pode acontecer � uma distribui��o de tarefas, onde por exemplo, um personagem mais forte fisicamente enfrenta o inimigo mais forte do grupo enquanto os outros personagens o auxiliam. O mestre tamb�m deve seguir a "Regra de Platina", que diz que o jogo deve ser jogado para trazer divers�o justa aos jogadores. Derrotar um inimigo devido a real for�a e habilidades dos jogadores � melhor do que o mestre explodir um inimigo dos jogadores com seu poder de mestre s� por que o mestre n�o quer que os jogadores percam. Quando mais o mestre se vir no papel de narrador e deixar o jogo fluir, melhor para o grupo.

Em termos de sistema, listaremos alguma situa��es que geralmente o mestre pode (e deve) usar sua liberdade e altera-las.

\section{Custo de Habilidades}
 
Segundo o livro do sistema, sempre que um personagen desejar comprar uma habilidade fora de sua classe ele deve pagar um custo extra por estar comprando uma habilidade extra-classe. De acordo com o andamento da aventura, o mestre pode diminuir esse custo extra, e at� em alguns casos anula-lo. Por exemplo, um personagem da classe ladr�o deseja aprender espada de duas m�os, uma habilidade t�picamente guerreira. Durante sua aventura ele encontra um guerreiro experiente, que deseja ensin�-lo. Digamos que o custo extra seja de 4 (fora o custo normal da habilidade). O mestre decide reduzir esse custo extra para 2 j� que o guerreiro ensinando o ladr�o � excepcional em seu ensinamento, tendo este desenvolvido um m�todo eficaz de treinamento. Quanto maior for o atributo acuidade de quem aprender e de quem ensina, menor deve ser o custo extra da habiliade ap�s determinado tempo de treinamento. Al�m disso, o personagem pode reduzir o custo de habilidades extra classe com ajuda de companheiros do pr�prio grupo, e n�o apenas NPC que ele encontre ao longo da sua hist�ria.

Vale lembrar tamb�m que o mestre pode limitar o aprendizado de certas habilidades apenas quando determinado acontecimento ocorrer dentro de jogo. Por exemplo, elepode limitar o aprendizado da habilidade metamorfose, apenas quando o personagem gastar um determinado tempo treinando com um grupo de druidas ou em uma escola com magos azuis poderosos.
	
%N�o � aconselhado o mestre reduzir uma habilidade abaixo de seu custo m�nimo de aprendizado em xp, ou seja, n�o aconselhamos o mestre fazer com que um personagem aprenda uma habilidade que custe 10 de xp gastando apenas 6. Apenas em raros casos isso pode ocorrer, por exemplo, o personagem tem a habilidade materializar firani que custa 20 PS e 8 xpm. A habilidade materializar magia � uma versao melhorada dessa magia, onde o usu�rio pode materializar n�o somente a magia firani como qualquer magia que ele possua. O custo dessa habilidade � de 20PS e 12 xpm para magos negros e 30PS e 18 xpm para outras classes que n�o sejam desfavorecidas com mago negro. Como as duas habilidades s�o bastante parecidas, o mestre pode fazer com que o usu�rio da habilidade m�gica materializar firani, quando for aprender a habilidade materializar magia, desconte uma quantidade de PS e xpm igual �quela quantidade gasta para aprender materializar firani, ou seja, no lugar dele ter que gastar 30PS e 18 xpm, ele pode aprender essa mesma habilidade gastanto apenas 10PS e 10 xpm.	
		
%O mesmo pode ser usado para p�ricias, onde por exemplo, para um personagem que saiba usar espada curta e espada m�dia, o mestre pode conceder um b�nus para esse personagem aprender a usar a p�ricia b�lica espada longa.	
	
\section{Criando Novas Habilidades}
As habilidades das classes e magias listadas no livro do sistema comp�em apenas uma base do que aquela classe ou aquela magia pode proporcionar. Baseado nisso, o mestre e os jogadores tem total liberdade para criar habilidades durante a aventura. Existem duas formas de criar uma nova habilidade.

A primeira � criando uma habilidade do zero. De acordo com o conceito do seu personagen, muitas vezes voc� sente nescessidade que o seu personagem fa�a algo a mais, por�m esse "algo a mais" n�o encontra-se manifestado na forma de nenhuma habilidade no sistema. Para criar uma habilidade do zero o mestre ou jogador deve criar um rascunho dela, e ent�o procurar alguma habilidade semelhante. Por exemplo, se voc� est� criando uma habilidade que te conceda b�nus, seria interessante comparar com outras habilidades que concedam b�nus. Muitas habilidades criadas pelos jogadores podem ser aproximadas para habilidaes j� existentes. 

O mestre deve ter cuidado para n�o fazer com que tal habilidade fuja da �tica do personagem, desequilibrando o mesmo quando comparando com outros. Ent�o, se voc� � especializado em suporte, o normal � que voce crie habilidade voltadas para o suporte. Se o mestre concede a voc� uma habilidade de ataque de forma semelhante a se te concedesse uma de suporte, nesse momento o jogo tem um desequilibrio. Apesar disso, um jogador pode criar habilidades que n�o envolvam seu conceito, por�m ele estar� criando essas habilidades para outros personagens. Ent�o, voltando ao exemplo do personagem de suporte, ele pode auxiliar um personagem de ataque ao criar uma nova habilidade de ataque basedo em sua observa��o pessoal, mesmo ele n�o sendo um personagem de ofensivo. 

Geralmente quando um jogador cria uma nova habilidade do zero, a mesma habilidade, ou uma vers�o muito pr�xima aquela habilidade existe na lista de habilidades de outra classe. Quando isso acontecer o mestre pode fazer com que o jogador aprenda a habilidade sem necessidade de treino, apenas gastando a experi�ncia necess�rio.

A segunda forma de criar uma habilidade nova � melhorando alguma habilidade antiga. Geralmente isso � feito aumentando o b�nus que determinada habilidade concede, por�m aumentando tamb�m o seu custo (em PF, PM e em raros casos PV). Para reduzir o custo de uma habilidade geralmente � mais vantajoso aumentar os valores de PF ou PM diretamente com experi�ncia. Ou seja, se determinada habilidade consome 3 PF para ser usada, no lugar do personagem diminuir o custo dessa habilidade � melhor ele aumentar permanentemente os seus PF. Isso � uma dica para a maioria das habilidades, exceto por algumas magias. Se determinada magia custa muito PM para ser usada, geralmente o personagem pode baixar seu custo com gasto de PS na ordem de 10PS gastos para reduzir 1 PM no custo da magia. Geralmente uma habilidade n�o pode ser reduzida para um valor menor que a metade do custo original. Mais uma vez, o mestre deve ter cuidado para que uma habilidade/magia n�o seja melhorada demais, desequilibrando assim o jogo. 

O mestre tambem pode fazer com que uma vers�o melhorada da habilidade seja o jogador poder comprar a mesma habilidade novamente, com um custo maior, podendo asssim acumular os b�nus da habilidade. Por exemplo, a habilidade punho da stiga concede um b�nus do dano, proporcional ao n�mero de vezes que a mesma foi comprada, at� um valor m�ximo de 5 vezes. O mestre pode criar uma vers�o melhorada, aonde o b�nus � o mesmo, por�m o personagem pode comprar a habilidade mais 2 ou 3 vezes, podendo acumular o b�nus e aumentar ainda mais o dano. Como dito anteriormente, o mestre pode aumentar o custo da nova habilidade ou restringir seu aprendizado para situa��es espec�ficas dentro da hist�ria.

Em muitos casos o mestre pode criar novas habilidades que n�o se encaixem dentro de jogo. Por exemplo, em um cen�rio futurista o mestre pode incluir v�rias habilidades voltadas para tecnologia avan�ada. Em outras palavras, a lista de habilidades gerais mostradas no livro do sistema n�o � (nem de longe) definitiva.

Compara��o, equilibrio e imagina��o s�o os aspectos mais importantes no momento de se criar uma habilidade nova.


\section{Cuidados}

� extremamente importante que o mestre tenha cuidado com habilidades "Quebra Regra". Por exemplo, � normal magias serem de acerto autom�tico, ou seja, o usu�rio de uma magia n�o precisa realizar a jogada de acerto. Por�m, se o alvo da magia tiver a habilidade auridade ele pode tentar se esquivar de uma magia. Essa habilidade � considerada uma habilidade "Quebra Regra", pois ela vai de encontro com as regras fundamentais do sistema. Isso em determinadas situa��es pode trazer elementos interessantes para o cen�rio, por�m, seu uso excessivo pode causar problemas.

\section{Recompensas}

O mestre deve recompensar seus jogadores de v�rias formas. O mestre pode conceder aos jogadores ponto de b�nus por interpreta��o. Esse pontos de b�nus podem ser dados aos jogadores quando eles poderem gastar sua experi�ncia acumulada ou quando o mestre achar adecuado.

Sempre que um jogador alcan�ar um objetivo, o jogador deve ser recompensado com ponto de experi�ncia extra. O b�nus varia de 2 a 15 pontos, de acordo com a magnitude do objetivo alcan�ado. Por exemplo, os jogadores est�o a 20 sess�es tentando resgatar um membro da realeza que foi sequestrado. Para isso o grupo teve que encontrar boatos de quem o sequestrou, e ir atr�s do grupo. Al�m disso eles tiveram que combater e libertar a pessoa sequestrada. Ao final dessa quest, o mestre pode conceder 10 pontos de b�nus extras para os jogadores de acordo de como foi realizado o resgate (que tipo de repercus�es positivas essas a��es trouxeram para o grupo).

Outra forma de recompensar o jogador � atravez de itens e equipamentos. Por exemplo, em alguns locais do mundo existem po��es que aumentam permanentemente atributos ou pontos como PV, PF e PM. Por�m essa forma de recompensa deve ser a mais rara de todas, pois pode causar desequilibrio nos personagens, uma vez que somente 1 personagem do grupo pode usar uma dessas po��es por vez.

Existem outras recompensas menos diretamente relacionadas com a ficha do personagem. Por exemplo, encontrar determinado NPC (personagem do mestre), pode fazer com que aquela pessoa possa aprender certas habilidades, ou ganhar b�nus quando aprende-las, como dito anteriormente. Cabe ao mestre de acordo com o andamento do grupo decidir que tipo de recompensar seus jogadores merecem.

\section{Downtime}

Downtime � um termo usado para periodos longos onde o jogo fica de certa forma "parado". Dentro de um downtime o personagem pode treinar, estudar, viajar, entre outras a��es que exigam uma grande quantidade de tempo. Por exemplo, o objetivo de um certo personagem � viajar at� o reino de jotan e participar do torneio de cavaleiros, que acontece de 3 em 3 anos. Ap�s chegar ao reino, esse personagem percebe que chegou muito cedo. Ele n�o quer realizar nenhuma side-quest (aventura paralela). Ent�o o mestre anuncia um downtime de 6 meses, tempo necess�rio para que o torneio tenha inicio.


A��es realizadas no downtime s�o a��es prolongadas. Ou seja, o personagem n�o define o que vai fazer em um dia especif�co e sim durante 1 semana ou durante 1 m�s. Downtimes s�o importante entre aventuras grandes ou at� mesmo para alguns casos trazer coer�ncia � hist�ria do grupo.






\chapter{Explorando o Mundo}
\label{cap:explorando}

O personagem pode explorar determinadas áreas em 2 amplitudes diferentes. Explorações em locais grandes e em locais pequenos. Explorando um local grande é por exemplo, explorar um deserto, uma floresta, etc. Explorar um local pequeno seria explorar um castelo, um bairro abandonado, entre outros. Quando explorando um local grande, os personagens percorrem distância maiores e gastam mais tempo durante a exploração. O mestre deve levar em consideração o local a ser explorando para descrever o que está acontecendo, e também manipular a perda/ganho de PF.

Em um dia de exploração ou locomoção, o personagem passa por 3 estágios. Descansos curtos, longos e caminhadas. Um descanso longo dura cerca de 8 horas, enquanto que um curto cerca de 30 minutos. Dentro do tempo gasto para descanso um personagem pode dormir, fazer sua higiene pessoal e se alimentar. Ele consome cerca de 10 horas de caminhada em um dia de exploração. Tomando em consideração que a velocidade média de caminhada de um ser humano é cerca de 8 km por hora, então ele caminha cerca de 80 km por dia. Porém, esse valor é usado para caminhos retos, como estradas bem construídas ou planícies. O mestre deve reduzir esses 80 km diários de acordo com a situação do terreno. Existem montanhas que esse valor é reduzido para 10 km diários por exemplo. 

Durante a exploração, o mestre pode fazer com que os personagens encontrem criaturas, armadilhas, tesouros, ou outros elementos interessantes. O mestre deve usar sua criatividade para tornar a exploração interessante e desafiadora.

\section{Exploração de Locais Grandes}

Ao explorar locais grandes, os personagens podem encontrar:

\begin{itemize}
    \item Criaturas selvagens
    \item Ruínas antigas
    \item Recursos naturais
    \item Perigos ambientais
    \item Povos nativos
\end{itemize}

\section{Exploração de Locais Pequenos}

Ao explorar locais pequenos, os personagens podem encontrar:

\begin{itemize}
    \item Armadilhas
    \item Tesouros escondidos
    \item Documentos importantes
    \item Criaturas hostis
    \item Segredos antigos
\end{itemize}

\section{Regras de Exploração}

Durante a exploração, os personagens devem:

\begin{itemize}
    \item Fazer testes de percepção para detectar perigos
    \item Fazer testes de sobrevivência para navegar
    \item Fazer testes de conhecimento para identificar elementos
    \item Gerenciar recursos como comida e água
    \item Tomar decisões estratégicas sobre rotas
\end{itemize}

\section{Consequências da Exploração}

A exploração pode resultar em:

\begin{itemize}
    \item Ganho de XP por descobertas
    \item Encontros com criaturas perigosas
    \item Descoberta de tesouros valiosos
    \item Perda de PF por fadiga
    \item Descoberta de informações importantes
\end{itemize} 
%\include{inimigos} 	
%%
%% Capítulo : pericia
%%

\chapter{Equipamento}
\label{Cap:pericia}

As perícias bélicas nada mais são do que os equipamentos que seu personagem tem proficiência em usar, e em alguns casos a forma como o seu personagem utiliza seu corpo na batalha. Em seguida explicaremos detalhes sobre obtenção e uso das péricias bélicas no sistema dragon. Antes de entrarmos em tais detalhes, apenas algumas observações:

\begin{itemize}
	\item Normalmente, a curta distância, pode-se aparar ataques com as mãos, porém isso não quer dizer que um artista marcial pode usar o braço para defender como se fosse uma espada ou um escudo. Ele de alguma forma impede o golpe de ser desferido para ele (segurando a base de uma espada antes que o golpe seja desferido por exemplo). O mesmo vale para armas brancas leves quando forem defender uma arma grande a curta distância, porém o teste de força continua sendo necessário em certos casos.

	\item Em alguns casos o mestre pode optar por substituir perícias de determinada classe de acordo com a história do personagem. Um mago utilizando um arco seria um exemplo dessa troca de perícia de acordo com a história. 

	\item Algumas armas exóticas, não estão incluídas na listagem de perícias bélicas devido a sua exclusividade de cenário. Cabe ao mestre, de acordo com a campanha, criar péricias novas. 
	
	\item Mesmo com a perícia, certas armas exigem valores mínimos de atributos para serem utilizadas. Por exemplo, um machado grande pode exigir força 6 para ser usado.

	\item A perícia bélica arremesso serve para atacar a distância arremessando armas ou dardos. Todos os personagens sabem arremessar uma arma a qual ele saiba usar como péricia, porém não recebe bônus de especialização. 

	\item Todos os personagens têm acesso a péricia bélica armadura leve.

\end{itemize}

\section{Lista De Acesso}

Durante a criação de personagem toda classe tem acesso a todas as pericias listadas para aquela classe. Vale notar que quando não for especificado o tamanho da arma (apenas seu tipo) todos os tamanhos da arma serão englobados, por exemplo, dizer que samurai tem acesso a espada, é o mesmo que dizer que samurai tem acesso a espada curta, média e longa. Caso o personagem não tenha acesso a uma certa péricia, ele pode comprar a mesma gastando xpf normalmente de acordo com a lista de péricias.


\begin{itemize}
	\item Guerreiro: Todas as perícias, fora as armas de fogo.
	
	\item Samurai: Espadas, lanças, bastões, arco e flecha, armadura média e mais duas perícias, com exceção de armas de fogo leves.

	\item Ranger: Armas de porte pequeno/curto e médio, arco e flecha, besta, uma perícia defensiva e mais duas perícias extras.

	\item Ladrão: Arma de fogo leve e média, espada média e curta, arco, besta, bastão, bastão curto composto, manopla, armadura média e arremesso leve.

	\item Druida: Espadas, lanças, bastões, arco, esmagador, esmagador grande, arremesso, lutar desarmado, escudo médio, armadura média, chicote.

	\item Monge: Lutar desarmado, arremesso, bastão, esmagador, bastão curto composto, kusarigama, espada curta.

	\item Soldier: Arma de fogo, besta, bastão, manopla, armadura média, escudo, lutar desarmado.

	\item Guerreiro Mago: Três tipos de perícias não defensivas (com exceção de arma de fogo), duas perícias defensiva.

	\item Ninja: Espada, lança e bastão, bastao curto composto, esmagador, manopla, arremesso leve, arco médio.

	\item Mago azul: Todas as perícias.

	\item Magos: Perícias pequenas ou leves.
	
	\item Eremita: Bastão, bastão curto composto, esmagador, 1 perícia leve opcional, Lutar desarmado.
	
\end{itemize}



\section{Usando uma arma sem a perícia}

È normal que uma pessoa use armas que não saiba usar em horas de extremo perigo. Como ela não tem a habilidade suficiente para usar tal arma, terá redutores toda vez que for utilizá-la. Ao invés de se escolher o maior resultado do dado, o menor resultado será somado com a destreza. Logicamente que à medida que o personagem for utilizando uma arma desconhecida, ele vai se acostumando com ela e com o tempo poderá aprendê-la com XP.

\section{Usando uma arma com duas mãos}

Algumas armas devem ser usadas com duas mãos, devido ao seu peso e tamanho. Todas as armas grande exigem um valor mínimo de força para que ela possa ser usada com duas mãos. Personagem que desejem usar uma arma com as duas mãos sem ter o pré-requisito, podem ter redutores em destreza. Para usar uma arma de duas mãos com apenas 1 mão, o personagem deve ter um valor de força superior ao mínimo exigido pela arma mais 6.

Usando qualquer arma normal com 2 mãos, o consome de PF daquela arma é reduzido para o mesmo consumo de uma arma de categoria mais leve.



\subsection{Ambidestria}

Em dragon o uso da ambidestria (uso normal das duas mãos quando tratando de habilidades manuais) é negligenciado para facilitar o andamento das lutas. Quando uma pessoa tem certa perícia, ela pode usar essa pericia normalmente independente da mão que esta segurando a arma.

\section{Especialização}

Toda vez que você aprende a utilização de um equipamento, a perícia começa no nível básico. Porém uma pessoa pode ultrapassar os limites básicos de utilização de uma arma tornando-se especialista nesta. Isto pode ser feito com o uso de xp físico (para qualquer arma), e é necessário que a pessoa tenha o nível básico da perícia bélica em questão. Cada nível de especialização concede ao personagem um bônus em acerto-aparar ou dano automático. O personagem deve escolher aonde será o bônus no momento da compra. Por exemplo, um personagem com Lança 1 Ac, 3 Dano, sempre que usar uma lança recebe +1 para acertar-aparar e +3 no dano.  

Cada nível de especialização custa 3 xpf, até o máximo de 5 ou metade do nível físico do personagem (o menor valor entre os dois). Além disso, o valor de especialização para acerto-aparar e dano são tratados separadamento, como se fossem especializações diferentes. Outra restrição é de que o bônus concedido ao personagem não pode ultrapassar o atributo básico, ou seja, o bônus em acerto-aparar concedido por uma especialização não pode ultrapassar o atributo destreza assim como o bônus em dano não pode ultrapassar força. 

%Esse nível máximo pode ser aumentando até  10 com a ajuda de mentores, porém o custo também é aumentado em 1.

%O custo aumenta 1 a cada dois níveis individuais começando por 3 de xp físico para o primeiro nível de especialização. Ou seja, para comprar o nível 1 de dano você irá gastar somente 3 de xp físico, porém para aumentar do nível 5 para o nível 6 você gastará 5 de xp físico, independente do valor do bônus em acerto-aparar. Em outras palavras, o valor de especialização para acerto-aparar e dano são tratados separadamento, como se fossem especializações diferentes.

%Existem algumas restrinções em relação a evolução de uma especialização. O valor de uma especialização não pode ser maior que o nível físico. Outra restrição é de que o bônus concedido ao personagem não pode ultrapassar o atributo básico, ou seja, o bônus em acerto-aparar concedido por uma especialização não pode ultrapassar o atributo destreza assim como o bônus em dano não pode ultrapassar força. Esse bônus também não pode ser maior que 10. 

%O mestre também pode limitar o nível de especialização total, de acordo com a campanha. Esse nível máximo pode ser aumentando com a ajuda de mentores. O mestre também pode limitar o valor recebido no dano devido a especialização de acordo com a arma usada. 

%O mestre também pode limitar o máximo de bônus de especialização. O aconselhado é 20, sendo 10 para dano e 10 para acerto. Mas existe uma regra interessante para esse caso de limitação do nível da especiualização. O nível máximo é 10, podendo ser distribuitos da forma que o personagem desejar. Para liberar o uso de níveis superiores, o personagem deve encontrar dentro de jogo um mestre ou pergaminho ensinado o treinamento necessário.

Aumentar ponto em especialização requer treinamento. Pegue o numero atual da especialização e multiplique por 4. Esse é a quantidade de dias necessários. Outro detalhe é de que o bônus de especializações em armas de projeteis (arco e flecha, armas de fogo etc) não podem ultrapassar o atributo destreza. 

%Toda vez que você aprende a utilização de um equipamento, a perícia começa no nível básico. Porém uma pessoa pode ultrapassar os limites básicos de utilização de uma arma tornando-se especialista nesta. Isto pode ser feito com o uso de xp físico (para qualquer arma), e é necessário que a pessoa tenha o nível básico da perícia bélica em questão. Cada nível de especialização concede ao personagem um bônus em acerto-aparar ou dano automático. O personagem deve escolher aonde será o bônus no momento da compra. O custo é 3 de xp físico para os 5 primeiros níveis individuais (dano e acerto/aparar) e 6 de xp físico até o nível 10. O nível total de especialização de uma péricia é a soma dos dois níveis individuais (o nível de dano e o nível de acerto-aparar). Por exemplo, um personagem com Lança 4 (1 Ac, 3 Dano) sempre que usar uma lança ele recebe +1 para acertar-aparar e +3 no dano.

%Existem algumas restrições em relação a compra da especialização de uma péricia. 

%O nível de especialização total de uma arma não pode ultrapassar o nível físico do personagem. 
%O nível individual não pode ultrapassar 10.
%O bônus concedido ao personagem não pode ultrapassar o atributo básico, ou seja, o bônus em acerto-aparar concedido por uma especialização não pode ultrapassar o atributo destreza assim como o bônus em dano não pode ultrapassar força.  

\section{Acrobacias em Combate}

A péricia bélica Acrobacias em Combate tem finalidade única de conceder bônus em esquiva na forma de especialização. Sua compra não faz com que o personagem possa se esquivar em batalha usando seu atributo esquiva, qualquer personagem pode fazer isso. A compra dessa péricia bélica faz com que o personagem ganhe o bônus de especialização quando for desviar do golpe usando a esquiva, ou seja, a compra dessa péricia já concede o primeiro nível de especialização. Existem alguns detalhes sobre a compra e uso dessa péricia bélica especial.

\begin{itemize}
	\item Para cada 8 pontos em força o personagem pode usar 1 equipamento de porte grande/pesado quando usando a péricia acrobacias em combate.
	\item Para cada 5 pontos em força o personagem pode usar 1 equipamento de porte médio quando usando a péricia acrobacias em combate.
	\item O mestre pode aumentar esse valor mínimo de força de acordo com o peso do equipamento usado.
	\item De acordo com o peso do equipamento usado o mestre pode retirar mais PF que o normal quando usando essa péricia em combate.					
\end{itemize}
	


\section{Pontos de Resistência}

Um equipamento está sujeito a desgaste, seja esse devido ao uso ou ao passar do tempo. O que diz se um objeto está desgastado ou não são os seus PR (pontos de resistência). Os PR dizem o quão a arma ou armadura encontra-se desgastada, e o mestre deve atribuir penalidades ao equipamento quando a mesma encontra-se com poucos PR. As penalidades começam quando a quantidade de Pr é menor do que 20. Quanto menor o Pr do equipamento, maior sua penalidade. Para valores inferiores a 10, sempre que usada o equipamento tem uma chance de 50\% ser quebrado completamente.


Os PR podem ser recuperados ou terem sua quantidade total aumentada segundo alguns métodos alternativos, como magias, trabalhos de ferreiros, banhos em soluções químicas, entre outros. O PR total e atual de um equipamento deve ser anotado ao lado do mesmo na ficha, na parte de ''Info''. Cada vez que o equipamento é utilizado, o mestre deve analisar de que forma aquele uso contribuiu para o seu desgaste e assim retirar seus PR. Por exemplo, se um inimigo tem uma defesa do tipo automático alto, o mestre deve retirar mais PR da arma do que o normal (esse valor pode ser igual ao bônus de força do atacante por exemplo). Alguns inimigos explicitamente indicam que caso recebam dano ou acertem o alvo, uma quantidade de PR adicional é retirada do equipamento. A média de perda de uma arma ou armadura por batalha é de cerca de 5 PR. Armas grandes podem ser usadas para retirarem PR de outros equipamentos menores de forma semelhante a habilidade de guerreiro "Quebra de Equipamento". Por exemplo, uma armadura leve caso receba um golpe de um machado grande pode perder uma quantidade de PR igual a força do oponente, ou no máximo ate o dobro da mesma (a quantidade fica à criterio do mestre). Caso o personagem use a habilidade citada, a perda de PR é ainda maior, ou seja, 3 vezes o valor da força do usuário, normal da habilidade,  mais um bônus devido ao tamanho da arma. Esse bônus pode variar de acordo com o julgamento do mestre analisando a situação.


Algumas armas (incluindo armas extras) podem ser usadas em conjunto com a habilidade punho da stiga. Quando assim feitas, eles perdem uma quantidade de PR igual a defesa do alvo, com um limite de perda de PR maxima igual ao força mais bônus do punho da stiga do usuário. Por exemplo, um lutador tem força 6 e punho da stiga +4, e está usando um soco inglês +4. Ao atinjir um oponente com defesa 8, esse mesmo soco inglês perde 8 de PR. Porém se o inimigo tiver 20 de defesa, a arma perde apenas 10 PR (força mais bônus da habilidade punho da stiga).

Armas têm seu Pr variando entre 30 e 80, enquanto armaduras e escudos entre 50 e 100.

\section{Manopla}

A perícia manopla é vista como uma perícia defensiva de único nível, assim como armadura. Essa perícia cede a capacidade de, no momento do ataque, o golpe ter seu dano diminuído pela proteção do equipamento e a oportunidade maior de aparar ataques de armas brancas cortantes usando as mãos. Pode ser interpretada também como escudo pequeno, onde seu bônus de defesa e força não são tão grandes comparados a um escudo médio ou grande, porém seu consumo de PF é praticamente nulo. 

Em determinadas situações a manopla também pode ser usada para aparar projéteis. Nesses casos o personagem pode aparar projéteis atirados de uma distância próxima à si mesmo, pois é mais fácil prever a trajetória de uma bala/flecha a curta distância. Além disso o mestre pode conceder um redutor no teste de aparar de acordo com a situação (usar uma arma grande que prejudique o movimento da manopla por exemplo).

A absorção da manopla não acumulada com outras armaduras que o personagem esteja usando, podendo em alguns casos aumentar um pouco a absorção da armadura como um todo. Porém, o uso de uma manopla junto de uma armadura aumenta o fator cobertura da mesma.

Aqueles que tiverem a péricia bélica lutar desarmado podem usar livremente uma manopla para aparar golpes de armas cortantes, mas não recebem os bônus do equipamento para tal, recebendo bônus de especialização em artes marciais se aplicável.



\section{Armaduras}

Armaduras protegem várias partes do corpo (pontos específicos como partes vitais, articulações etc), e devido a seu grau de proteção, podem privar o personagem de alguns movimentos. Um personagem para movimentar o seu corpo enquanto usa uma armadura deve ter a péricia apropiada, ou seja, o uso de armadura (leve ou pesada) requer o conhecimento da perícia bélica respectiva. Se você deseja usar uma armadura sem a perícia, maiores penalidades serão impostas. Sua destreza e esquiva são reduzidas pela metade e o dobro da penalidade de locomoção da armadura será aplicado.
 
As perícias defensivas armadura média, pesada e manopla não possuem especialização. A única perícia bélica unicamente defensiva que possui especialização é a perícia escudo.


\section{Escudos}

Para um personagem poder usar um escudo ele deve ter a perícia bélica apropiada (escudo médio ou escudo grande). A perícia bélica escudo pode ser aumentada com níveis de especialização, sendo seu bônus alocado unicamente em aparar. Além disso, usar um escudo sem a devida perícia acarreta nas mesmas penalidades normalmente aplicadas nesse caso.

O escudo tem duas características especiais. Tamanho e resistência. O tamanho do escudo diz a área de proteção do mesmo, ou seja, o bônus de aparar. Em outras palavras, sempre que um personagem tenta se defender usando um escudo, ele joga destreza + bônus do tamanho do escudo. A resistência diz o quanto um escudo pode suportar impactos, livrando o defensor de sofrê-los. O valor de resistência de um escudo é manifestado como um bônus de força quando usando a manobra aparar (veja o capítulo batalha para maiores informações). Resumindo, além do escudo conceder bônus em aparar, ele pode ajudar o defensor a suportar a força do atacante pois mesmo conseguindo interceptar o golpe, o defensor pode não suportar a força imposta pelo mesmo. Para projéteis, o teste de força na maioria das vezes é ignorado. 

De acordo com certas situações de combate (defensor sendo flanqueado por 2 personagens agéis por exemplo), o personagem que deseje aparar com o escudo pode ser impedido de realizar tal manobra ou faze-la com redutores.


\section{Arco e Flecha}

O dano de um arco é calculado de acordo com a força do usuário. Ou seja, se um usuário de arco tem força 4 então o dano desse arco vai ser 4. Um dos diferenciais de cada arco é sua capacidade de suportar essa força. Cada arco tem um limite máximo de força suportada. Por exemplo, um arco que suporte até força 6 terá dano igual a 6 mesmo se a força do usuário for 10 e dano igual a 4 se a força do usuário for 4. 

O alcance de um arco depende muito de sua construção, resumidamente de sua força de arremesso. Quanto maior essa força, mais longe o arco pode arremessar o projétil. Dentro desse alcance existem duas distâncias importantes. A distância média e a distância longa. Na distância média a energia usada para arremessar a flecha encontra-se concentrada, então o dano nessa distância é o dano normal causado pela flecha. Na distância longa, a maior parte da energia já foi usada para levar a flecha para tão longe, portanto o dano aqui é reduzido. O mestre pode reduzir, e em alguns casos até anular o dano do tipo normal de acordo com a distância. 

Para saber a distância média e longa de um arco, o mestre deve se basear na força máxima que pode ser aplicada no arco. Esse valor será os valores de alcance máximo do arco. Por exemplo, um arco cuja força máxima é 4. Ele tem distância média de 0 até 35 metros e distância longa de 35 até 135 metros. Então qualquer pessoa com força 4 ou maior consegue atirar nessas duas distâncias. Na descrição de cada arco tem os valores de sua distância média e longa.

Esses valores ditos acima são geralmente utilizados para alvos estáticos, então existem alguns detalhes que devemos falar para englobar uma situação de combate real. O alcance médio de um arqueiro é duas vezes sua percepção em metros, até o limite do arco. Ou seja, se um arqueiro tem percepção 10, sua distância média é 20 metros, mesmo que seu arco permita que sua distância média seja de 50 metros. O usuário pode aumentar essa distância se concentrando em batalha. Cada turno mirando no alvo aumenta essa distância em concentração metros (até o máximo da distância média que o arco pode suportar). O mestre também pode fazer com que a distância média seja sempre a distância média do arco, quando usado com a força máxima que o arco pode suportar. Isso quem vai definir é o mestre. Quando o usuário não tem a força máxima do arco, o mestre pode diminuir o alcance da distância média. O mestre pode alterar esses valores como bem desejar de acordo com sua necessidade de realismo/fluidez no jogo. 

Vale lembrar também que mesmo que a distância longa do arco seja muito grande (200 metros por exemplo), fica dificil atirar com precisão nessa distância simplesmente devido ao fato do arqueiro não conseguir enxergar tão longe. Geralmente arcos assim são usados em guerra, dando tiros aleatórios em tropas a distâncias grandes.

Alguns arcos também tem uma redução no acerto devido ao seu dano elevado. Quanto maior a força aplicada, maior a distância e dano da flecha, porém fica mais dificil controlar o tiro nessas situações. Cada arco tem seu detalhe especial e é voltado para uma situação. Arcos menores são bons para batalhas a média e curta distância, pois são fáceis de usar. Arcos longos são mais usados em posições fixa para defesa de território.

Os arcos perdem muito pouco PR quando utilizados, sendo o consumo de PR evidenciado devido a fatores externos (água, tempo, etc). Arcos devem ser utilizados com ambas as mãos, porém consomem pouco PF. Cada ponto em especialização em arco reduz em 1 o requerimento de força mínima para se usar um arco.  
 
O dano total de um tiro usando arco e flecha é a soma do dano do arco e da flecha. A maioria das flechas pode ser usada independente do arco. Flechas geralmente tem dano automático grande, e pode se quebrar de acordo com a jogada de defesa do alvo. Por exemplo, uma flecha pode quebrar-se quando atirada em alguém com uma defesa automática maior do que o dano da flecha. O mestre também pode estabelecer uma porcentagem de chance da flecha quebrar. Algumas flechas são usadas para ataques contra armadura (tem dano normal maior que o normal), outras são usadas para combate próximo (têm a ponta mais pesada). O mestre tem liberdade para trabalhar esses fatores na hora de criar uma flecha. O dano de uma flecha geralmente não é alto.

\section{Armas Perfurantes}

Armas que tem o atributo perfurante ignoram um certo valor de armadura do alvo. Por exemplo, uma flecha 2+3 perfurante 2, ignora 2 pontos de armadura do alvo. 
Geralmente armas com atributo perfurante são armas cuja superficie de contato no mometo do dano é pequena em relação ao tamanho da arma em si, por exemplo, flechas, balas, etc.


\section{Armas de Fogo}

O uso de armas de fogo é bastante comum em alguns cenários. O uso dessa perícia muitas vezes mortal tem alguns detalhes, que iremos expor a seguir.

\subsection{Cadência de tiro}
 
Armas mais antigas disparam apenas uma bala por ataque (tem apenas um ataque por rodada de ataque), porém armas de fogo automáticas ou semi-automáticas podem disparar diversos tiros em um único turno de ataque. A quantidade de tiros que uma arma pode realizar por turno é chamada de cadência de tiro. Existem duas formas de usar a cadência de tiro de uma arma. 

A primeira é a forma manual. O atirador divide sua destreza normalmente, da mesma forma quando está usando a manobra multíplos ataques. O número máximo de ataques realizados não pode ultrapassar a cadência de tiro da arma. 

A segunda forma é a automática. Nessa forma a arma realiza automáticamente varios tiros por turno, sem a necessidade de dividir a destreza. Porém, a medida que a arma realiza os disparos ela vai se tornando mais difícil de se controlada. Esse efeito é chamado de repuxo da arma. O repuxo é manifestado em um redutor cumulativo no acerto, onde cada arma tem seu próprio redutor de empuxo. A maioria das armas automáticas podem ser colocadas em modo normal de operação. O atacante não pode usar a manobra ataques multíplos nessa situação. A pessoa que tenta se desviar dos ataques poderá jogar apenas uma vez sua jogada de esquiva normal, porem vários tiros poderão acerta-la. 

Vamos a um exemplo geral: Biggs tem uma arma com cadência de tiro 4 e um repuxo de -1. Ele irá atacar com os 3 tiros graças a cadência de tiro da arma. O primeiro ataque será com sua destreza normal, o segundo com -1 e o terceiro com -2. Wedge tenta pular para se esquivar, porém apenas consegue um resultado maior do que o terceiro tiro, ou seja, dois tiros o acertam com dano normal. Vale lembrar que ele poderá jogar sua defesa para os dois danos dos tiros recebidos; uma jogada de defesa para cada tiro.

\subsection{Pontos de resitência de armas de fogo}

Observações sobre pontos de resistência em relação as armas de fogo.
\begin{itemize}
	\item Armas de fogo se desgastam a medida que são usadas, principalmente quando usadas em modo automático. Porém a perda de PF de uma arma de fogo é bem menor em comparação com outras armas e armaduras. 

	\item Ao contrário das armas normais, uma arma de fogo com baixo PR não tem redução em seus atributos, e sim uma chance de falha de funcionamento. Essa chance é de 10\% para valores de PR abaixo de 30\% do total, e de 50\% para valores de PR abaixo de 10\% do total. 

	\item Para recuperar os PR de uma arma de fogo, é necessário a habilidade manutenção de armas de fogo, e não forjar como na maioria das armas e armaduras normais.

	\item Geralmente armas de fogo básicas têm valor inicial de 30 PR. 
\end{itemize}

\subsection{Balas}

O dano total de uma arma de fogo é dividido em duas partes. O dano do impacto da arma e o dano da bala. Ou seja, se uma arma de fogo tem dano 6+4 e uma bala tem dano +6, o dano total será 6+10 para cada tiro. Uma arma de fogo geralmente é feita para ser compatível com apenas um tipo de munição, porém existem casos de armas de fogos que são feitas para dispararem diversas balas diferentes de acordo com seu modo de operação. Vale notar que, ao contrário de uma flecha que talvez possa ser reaproveitada após seu uso, sempre que uma bala é usada ela é descartada.


\subsection{Efeito mangueira}

Algumas armas têm a capacidade disparar todo o pente a elas equipado, dando um poder de ataque em área maior. Essa capacidade é chamada de efeito mangueira. A arma atinge todos os alvos dentro do campo do efeito mangueira, que é por volta de 45 a 60 graus partindo do atacante. Esse ataque é igual para todos os alvos, e tem um bônus de acordo com o bônus de efeito mangueira da arma. Os alvos atingidos por um ataque usando o efeito mangueira recebem uma quantidade de tiros igual a cadência da arma - 1. Além de perder todas as balas de um pente, a arma perde 10 PR sempre que o efeito mangueira for utilizado.

Para realizar o efeito mangueira existem algumas limitações, sitadas a seguir:

\begin{itemize}
	\item O alvo deve gastar todo o turno de ataque para realizar o efeito  mangueira;

	\item Algumas armas exigem uma força mínima para realizarem o efeito mangueira. Essa força mínima pode ser reduzida com a ajuda de suportes;

	\item Uma arma só pode realizar efeito mangueira se tiver munição suficiente dentro do seu pente;

	\item Mesmo que vários pentes possam ser acoplados a uma arma, ela deve esperar aproximadamente 2 turnos para resfriar e poder ser usada novamente.

\end{itemize}


\section{Lista Das Perícias}

Na próxima página segue a lista do XP necessário para aprender uma perícia caso ela não faça parte da sua classe. O custo é igual para todos os personagens. O mestre pode reduzir esses custos de acordo com as péricias que o personagem já possua. Por exemplo, se o personagem possui a péricia bélica de espada curta e média, o mestre pode reduzir o custo para que ele aprenda espada longa. 


\begin{table}[htbp]
\begin{center}
\begin{tabular}{|c|c|c|} \hline 
Espada curta: 3 &	 Espada média: 4&	 Espada longa: 4 \\\cline{1-3} 
Lança: 4&	 Bastão: 5 &	Esmagador: 3 \\\cline{1-3} 
Esmagador Grande: 4 & Arco: 5 & Acrobacia em Combate: 4 \\\cline{1-3} 
Arma de fogo leve: 5 & Arma de fogo média: 7&	 Arma de fogo pesada: 6	  \\\cline{1-3} 
Manopla: 3&	 Armadura média: 4 & Armadura pesada: 5 \\\cline{1-3} 
Foice Pequena: 4 &	 Foice Longa: 4 &	 Escudo: 4 \\\cline{1-3} 
Zarabatana: 4&	Arremesso: 4&	 KusariGama: 5 \\\cline{1-3} 
Bastão Curto Composto: 5&	 Lutar desarmado: 4 &	 Chicote: 4 \\\cline{1-3} 
Besta: 4&	Balista: 8  & --	
\\ \hline
\end{tabular}
\end{center}
\caption{Lista de Perícias}
\label{}
\end{table}


\section{Limitação de Itens Mágicos}

Alguns equipamentos normais são melhorados graças a poderes sobrenaturais. Esses itens únicos e poderosos são conhecidos como itens mágicos ou artefatos. Uma espada artefato tem poderes especiais quando comparado a uma espada normal. Mas o uso desse tipo de item tem suas limitações. A maioria dos artefatos permite que um certo limite de outros artefatos sejam utilizados simultaneamente. Em outras palavras, se o limite de um item mágico é 2, então o usuário desse item mágico pode equipar no máximo outros 2 itens mágicos. Se um item for mágico, ele deve ser identificado como tal, e também ter sua limitação exposta.

Com conhecimento de magia um personagem pode "desligar/ligar" um artefato mágico gastando cerca de 5 minutos. Dessa forma ele pode carregar vários artefatos, porém usar somente os que desejar. 

 
%%%
%% Capítulo 4: Batalha
%%

\chapter{Batalha}
\label{Cap:batalha}

O objetivo principal do sistema de batalha de Dregon é fazer com que a luta se torne mais dinâmica possível. É muito importante em um sistema de RPG que o seu sistema de combate tenha três principais características: liberdade, velocidade e coerência. 
	 
Vários detalhes como habilidades, perícias, magias e etc, serão explicados em capítulos posteriores. Vale notar que o jogador não é obrigado a conhecer as regras de combate, quem deve conhecê-las é o mestre. O desconhecimento do sistema de batalha não impede que um bom jogador realize lutas ruins. O conhecimento do mesmo torna mais rápido o fluxo da batalha.

O sistema de batalha de Dregon será descrito através de diversas situações de combate. Cabe ao mestre encaixar as ações desejadas pelos jogadores, nessas ações descritas neste capítulo. Esse conjunto de situações básicas podem dar origem a outras situações mais específicas. Por exemplo, atacar é uma ação básica. Desarmar o oponente, é uma variação da ação básica atacar, porém com outro objetivo. Se um personagem desejar desarmar o oponente, ele tem duas opções. Seguir a orientação de regra dado nesse capítulo, ou alterar ao bel prazer a regra de ataque. Isso cabe ao mestre e jogador fazer durante o jogo, deixando o jogo mais fluido e divertido. Ou seja, nesse capítulo são dadas ações de combate básicas, e sugestões de como o mestre deve proceder caso o jogador queira outras ações mais complexas.

\section{Momento de Ação}

Um momento de ação corresponde a um intervalo de tempo que é igual à soma de dois turnos, um turno de ataque e um turno de esquiva. Durante o turno de ataque o personagem pode realizar ações mais complexas, como atacar, usar uma habilidade ou uma magia por exemplo. Durante o turno de esquiva, ações mais reflexivas devem ser feitas tais como pular rapidamente para poder se desviar de um ataque, dar um grito sinalizando uma ordem, etc. Portanto podemos dizer que durante o turno de ataque o tempo para se realizar ações é maior do que o tempo para se realizar ações no tempo de esquiva. Se alguém tenta realizar uma ação mais longa no turno de esquiva, terá penalidades para realizar tal ação, por outro lado quem desejar realizar ações simples no turno de ataque terá bônus.

	No momento que acontece o seu turno de ataque, acontece o turno de esquiva do inimigo, e assim por diante. Mas isso não quer dizer que você não possa tentar realizar um ataque em seu turno de esquiva (manobra contra-ataque). Mais detalhes explicados posteriormente.
	
	Em suma, para termos de sistema um momento de ação é dividido em dois turnos. Um turno de ataque e um turno de esquiva. Um momento de ação tem mais ou menos duração de 3 a 4 segundos.  O mestre deve usar esse valor apenas como referência para basear-se na duração de determinadas batalhas. Mas o contexto da situação de combate descreve melhor a duração de todo o combate. 

\section{Iniciativa}

Para calcular a ordem das ações, todos jogam ágilidades para ações físicas e percepção para ações mentais ou ativação de habilidades. A ordem dos sucessos determina a ordem das ações. 
O personagem pode também, caso ganhe a iniciativa, optar por não atacar. Nesse caso ele terá um bônus em qualquer ação que ele quiser realizar no turno de esquiva (apenas em ações que não sejam de ataque, ou seja, ações de esquiva) e esse bônus depende de sua ágilidade/percepção. 

\section{Ataque}

\subsection{Um ataque normal}

O atacante realiza um teste de destreza para atacar. Guarde esse valor para usar posteriormente se o defensor for esquivar. Em outras palavras, ele faz um teste combatido contra o defensor. Se o atacante conseguir ganhar no teste ele consegue acertar o golpe. O número de sucessos extras no ataque concede um aumento no dano final (detalhes explicados posteriormente).
Ataques usando magias não obedecem essa regra, tendo regra própria explicando em capítulo posterior.

\subsection{Ataque Inconsequente}

Nesta manobra, o personagem ataca de forma louca e sem precaução, visando apenas a destruição do oponente. Ele vai perder o turno de esquiva e não pode realizar as manobras de combate ataque em alvo específico ou múltiplos ataques. O mestre deve considerar também o bônus de força para determinar a diferença da força atacante-defensor para fins da manobra aparar.

O personagem que realizar a manobra ataque inconsequente ganha bônus de atributo em destreza e em força para 1 ataque nesse turno. Essa manobra só é válida para ataques físicos feitos corpo a corpo, ou seja, não é usada para armas projetáveis, como arcos, armas de fogo, etc. \textbf{Um personagem só pode realizar um ataque inconsequente por batalha.} 

No seu próximo turno de esquiva, como dito anteriormente, o personagem não pode esquivar. Ou seja, o valor do teste de esquiva do usuário é igual a sua esquiva, sem a jogada de dados (ele não pode substituir esse valor pelo usado na manobra aparar). Use esse valor como base para calcular bônus no dano proveniente dos sucessos obtidos no ataque.

\section{Esquiva}

O personagem pode usar 3 manobras de esquiva.

\subsection{Esquiva Normal}

Ele pode esquivar-se normalmente jogando sua esquiva e testando contra a destreza de quem o atacou. Em outras palavras, o que define se o defensor se esquivou ou não é o resultado do teste combatido da destreza (do atacante) contra a esquiva (do defensor). O número de sucessos extras na esquiva serve apenas para descrever o grau de êxito da mesma.

\subsection{Aparar}

Ele pode aparar o golpe usando um objeto. Uma espada, lança, escudo, ou algum equipamento que ele saiba utilizar pode ser usado para desviar o golpe do atacante. Nesse caso ele joga destreza no lugar de esquiva. É um movimento bastante útil para quem tem baixa esquiva. %porém esse movimento cansa mais do que uma simples esquiva (consome mais pf no momento da batalha). Cada 2 ou 3 usos da manobra aparar (de acordo com a tamanho da arma, diferença entre forças defensor/atacante, etc), o usuário perde 1 PF. Além disso o mestre pode retirar uma quantidade extra de PR da arma por ser usada como objeto parar aparar o golpe do adversário.

No caso do defensor ter uma força menor em relação ao atacante em 4 ou mais pontos, ou da arma do atacante ser pelo menos 1 porte maior que a arma do defensor, o atacante recebe um bônus em sua jogada de acerto igual ao seu bônus de força. Além disso, o mestre tem total liberdade para alterar os valores nesse teste, por exemplo, no caso do atacante usar armas leves a uma curta distância de um defensor que tente aparar com uma arma grande, o defensor recebe uma penalidade. O mestre pode usar qualquer interpretação da situação de combate para torna-lo mais dinâmico.
	
%Por exemplo, quando a diferença entre forças é considerável, o atacante pode jogar destreza com um bônus de força contra a destreza do defensor. O mestre também pode dar bônus ao atacante de acordo com a distância de ataque, natureza favorável contra a arma do atacante/defensor, etc. Além disso o mestre pode reduzir o dano de acordo com a situação, por exemplo, em caso de empate no teste de aparar. 
	
Se o defensor pretende usar essa manobra para se defender de um projétil (um ataque a distância), dobre o valor do teste final da destreza do atacante, já que a área de um projétil é bem menor que de uma espada, por exemplo. Além disso, o mestre pode decidir aumentar ainda mais a penalidade para se defender projéteis (alguns não permitem a defesa de projéteis usando armas). A maioria dos projéteis pode ser defendida independente da força do defensor. Alguns projéteis não podem ser aparados usando essa manobra, o caso de esferas de energia ou projéteis pequenos.

 
\section{Tipos de Dano}

\subsection{Dano Normal}

Representa o dano que pode ser reduzido sem auxilio de armadura especial. Dano por contusão (socos, chutes) e afins são exemplos de danos normais. O dano normal é representado por uma " / " 
	Todo dano usando apenas o atributo força (sem auxilio de habilidade especial) é considerado como dano normal. Você pode somar todos os danos normais para fins de cálculo de dano normal total. Por exemplo, se personagem com força 6 usa um porrete /3, pode-se considerar a como dano normal total 9 (/6 + /3 = /9).


\subsection{Dano Automático}

O dano letal é um tipo de dano mais difícil de absorver sem armadura apropriada. Normalmente é causado por ataques cortantes, quentes, ácidos e afins. é representado por um \" + \". Ou seja, quando a força de um personagem mostrar 5 + 1 não quer dizer que a força dele é 6, e sim que a força dele é 5 porém está associada com +1 de dano automático. Para minimizar-se +1 de dano é necessário +1 de defesa automática. Apesar de ser mais difícil de defender, 1 de dano automático retira a mesma quantidade de dano que 1 de dano normal. 
Vamos supor que uma pessoa seja atacada por uma espada +2 e tenha conseguido no seu resultado de defesa um valor igual a /5. Ela conseguirá reduzir +1 de dano, reduzindo o dano total para +1. Então ela vai perder apenas 1 de dano e não 3. Se por exemplo a espada fosse um porrete, seria /2 (dano normal), e a pessoa com /5 de defesa poderia reduzir o dano total a 0, não perdendo nenhum PV com o ataque.

\subsection{Dano Crítico}

Quando um ataque resulta em um acerto crítico (explicado no capítulo de jogadas), o dano causado por esse ataque é considerado dano crítico. O dano crítico possui duas características importantes:

\begin{itemize}
	\item \textbf{Não pode ser curado durante o combate}: Magias de cura, poções e outras formas de recuperação de PV não afetam o dano crítico.
	\item \textbf{Reduz temporariamente o PV máximo}: O dano crítico reduz o PV máximo do personagem temporariamente até que ele realize um descanso adequado (geralmente uma noite de sono).
\end{itemize}

Por exemplo, se um personagem com 20 PV máximo recebe 5 de dano crítico, ele fica com 15 PV máximo até descansar. Durante o combate, ele não pode recuperar esses 5 PV perdidos por dano crítico.

\subsection{Dano de Toque}

O dano de toque é um tipo especial de dano que representa ataques que ignoram parcialmente a proteção física. Este tipo de dano possui duas características importantes:

\begin{itemize}
	\item \textbf{Reduz a defesa pela metade}: A defesa do alvo é reduzida pela metade (arredondada para baixo) para fins de redução de dano.
	\item \textbf{Ignora armadura}: Equipamentos de proteção (armaduras, coletes, etc.) não fornecem proteção contra dano de toque.
\end{itemize}

Por exemplo, se um personagem com defesa natural /6 e armadura /2+1 recebe um ataque de dano de toque /4, apenas sua defesa natural (/6) é considerada, mas reduzida pela metade (/3), resultando em /1 de dano efetivo.

\subsection{Lembrete Sobre Dano}

Tanto o dano normal quanto o dano automático são semelhantes em termos de dano (quantidade de pv tirada), o que os diferencia é a capacidade de um ser absorvido facilmente enquanto o outro não . +1 de dano automático pode ser absorvido com +1 de defesa automática, porém /1 de dano normal pode ser absorvido com /1 de defesa normal.

	Da mesma forma existe a defesa automática e normal. A defesa automática serve para defender igualmente +1 dano, porém na mesma intensidade de defender /1 de dano. Em outras palavras, uma armadura +4 defende da mesma forma +4 ou /4.
	
\section{Dano}

Para calcular-se o dano de um ataque (dano final), deve-se calcular o dano inicial e a redução de dano. O dano que o personagem vai receber é o dano inicial menos a redução de dano. 

Para calcular-se o dano inicial, execute o seguinte procedimento. Primeiramente some no dano do atacante o bônus do sucesso no teste de acerto como sendo dano normal. Esse bônus não pode exceder o valor do bônus de força permanente do mesmo para ataques corpo a corpo, e no caso de ataques a distância o bônus proveniente do sucessos extras não pode ultrapassar o dano base da arma (não do projétil). Se por exemplo o atacante consegue 17 no teste de acerto e o defensor consegue 10 no teste de esquiva, a diferença é de 7. Então o atacante terá um bônus de /3 no dano (3 é o valor do bônus de 7). Se o atacante tiver força 5, esse bônus será /2, pois o bônus de sua força é 2. Isso pode ser explicado analisando que quanto maior o sucesso do atacante em cima do defensor, mais eficiente será o golpe. Existem algumas armas e habilidades que aumentam esse limite. Após calcular o dano proveniente de sucessos extras, o atacante soma separadamente os tipos de dano (automático e normal). Esse valor será o dano inicial.

Para calcular-se a redução de dano, compare diretamente os valores de dano com os valores de defesa do defensor. A defesa é calculada diretamente sem jogadas de dados, usando apenas os valores base do personagem e equipamentos. Subtraia dano automático com defesa automática e dano normal com defesa normal. O excedente da defesa automática pode reduzir o excedente de dano normal, mas o excedente da defesa normal não pode reduzir dano automático. O dano após os cálculos é chamado de dano final.

A regra explicada acima vale para danos físicos, detalhes sobre o cálculo de dano mágico explicado no capítulo sobre magias.
	O mestre pode interpretar como o alvo sofreu o dano de acordo com a quantidade máxima de pv. O mestre pode interpretar um golpe que tire 15 de PV de um personagem que tem 20 de PV total como quase um braço decepado ou um olho perfurado por exemplo. A mesma quantidade de dano em um personagem com 100 de PV pode ser vista como um arranhão.
	
\textbf{Exemplo de cálculo de dano.}

O atacante tem força 5 e uma faca +4. Ele teve 16 no teste de destreza e o defensor 10 no de esquiva. Portanto, o atacante teve 6 sucessos extras, dando um bônus em seu dano de /2. O seu dano inicial vai ser /7+4. O defensor tem defesa /4+2 (valores base + equipamento). Primeiro diminuímos os tipos de danos semelhantes: /7 de ataque é reduzido pelos /4 de defesa, sobrando /3 de dano normal. Os +4 de ataque são reduzidos em +2 pela defesa +2 do defensor, sobrando +2 de dano automático. O excedente da defesa automática (+2) pode reduzir o excedente de dano normal (/3), reduzindo-o para /1. No final o defensor perde 3 PV (/1 + +2).


\section{Gastando PF}

Durante uma batalha é normal o desgaste físico dos combatentes. Isso é refletido na perda de PF. Normalmente em uma batalha um lutador, atacando e se esquivando 1 vez a cada turno usando armas leves, perde 1 PF a cada 4 turnos aproximadamente, 1 PF a cada 3 turnos quando usando equipamento de porte médio e 1 PF a cada 2 turnos quando usando equipamento de porte pesado. Mas existem alguns fatores que aceleram a perda de PF:

\begin{itemize}

	\item Uso de abusivo de equipamento pesado. Quanto maior for o número de equipamentos pesados usados durante o combate, mais PF é perdido. O uso de 3 equipamentos pesados ao  mesmo tempo acarreta a perda de 1 PF por turno. 
	\item Uso de habilidades específicas. A maioria das habilidades consomem PF.

	\item Locomoção. Seja para fugir ou para alcançar alvos, o mestre deve exigir o gasto de PF quanto maior for a locomoção do alvo na luta.

	\item Condições climáticas desfavoráveis também podem ser usados como fatores agravantes na perda de PF em relação ao desgaste físico natural sofrido nas batalhas.
\end{itemize}

No lugar de monitorar o que cada personagem faz em certos intervalos e ir tirando os PF de acordo com as regras acima, no final da batalha o mestre retira de 1 a 4 PF dos personagems por critério pessoal, levando em consideração condições que retirem mais PF que o normal (usar muitas manobras de combate, peso do equipamento, duração do combate etc).


\section{Usando Itens}

é comum durante as lutas os personagems usares determinados medicamentos de ação instantânea para recuperarem seus PV,PM ou PF, ou drogas para aumento temporário de atributos. Um item pode ser usado em você mesmo ou em um aliado dentro do seu campo de ação durante o seu turno de ataque. Para realizar tal ação durante a batalha o mestre deve ficar atento as situações de luta para atribuir penalidades ou até mesmo privar o jogador de realizar ataques nesse turno. 

Sempre que um personagem for usar um item em um aliado, ele perde todo o turno de ataque realizando essa ação. O mestre pode privar o uso para os aliados ele próximo, ou dentro do seu campo de ação. Caso um personagem deseje usar um item nele mesmo, e esse item encontra-se de fácil acesso, o personagem pode usa-lo sem penalidade nenhuma em seu ataque. Porém se o personagem usa alguma armadura ou encontra-se com as mãos ocupadas (segurando um arco e flecha por exemplo) o mestre pode fazer com que o personagem perca seu turno de ataque localizando o item.

 

\section{Regras Opcionais para Combate}

Abaixo segue a explicação de várias manobras de combate utilizadas no sistema Dregon. Essas manobras servem para tornar o combate mais dinâmico possível. Sempre que o mestre tiver dúvida como proceder perante uma manobra, ele pode usar as informações a seguir para ajudar em sua decisão final de como prosseguir. Todas as manobras de combate consumem mais PF que as ações padrões normais. 

As regras abaixo tem o único objetivo de mostrar como as regras básicas de combate são alteradas para que diferentes situações mais complexas de combate possam ser usados. Em outras palavras, o mestre pode alterá-las ou até mesmo não utilizar as regras abaixo que nada vai atrapalhar o funcionamento básico do sistema. Elas foram criadas para dar um maior grau de detalhes aos combates.
	
\subsection{Ataques Múltiplos}

Um personagem pode optar por no lugar de dar apenas um golpe bem visado, dar vários golpes porém com sua destreza menor. Isso pode ser feito se o personagem dividir a destreza para realizar um golpe, ou seja, um personagem que tem destreza 4 dar dois golpes com destreza 2, realizando 1 ataque extra. Cada mestre pode estabelecer um número limite de golpes extras que o personagem pode dar. Aqui iremos dar uma média que pode ser usada. De cada 4 em 4 pontos em destreza o personagem pode dar um golpe extra, ou seja, um personagem com destreza 7 poderia dar 2 golpes (1 normal e um extra) com destreza dividida (4 para um e 3 para outro ou 5 para um e 2 para outro) mas não poderia dar sete golpes com destreza 1. Vale observar que que qualquer bônus recebido em destreza entra no cálculo antes da divisão de ataques. Caso alguma habilidade conceda ao personagem bônus em um ataque, o personagem deve escolher em qual dos ataques (caso executando a manobra ataques múltiplos) esse bônus deve contar.

A distribuição de destreza entre os golpes deve ser escolhida pelo jogador, obedecendo a seguinte regra: O valor da destreza dividida para cada ataque não deve ser maior que o bônus de destreza do usuário. Por exemplo, um personagem com destreza 8 (bônus igual a 3), pode dividir 2 ataques em um ataque com a destreza 5 e o outro ataque com a destreza 3, mas não pode alocar 2 de destreza para 1 ataque e 6 para outro ataque, pois a diferença entre o primeiro ataque que tem destreza 2 e o segundo ataque que tem destreza 6 é 4, valor esse maior que o bônus de destreza que é 3.

Uma regra opcional para a forma como a destreza é dividida é que caso o personagem aloque 2 vezes consecutivas 1 ponto quando dividindo a destreza, o defensor não tem penalidade na terceira esquiva (penalidade essa descrita mais a frente). O uso de duas armas pode aumentar o número de golpes extras e reduzir um pouco o consumo de PF ao usar essa manobra. 

Quando um personagem utiliza a manobra ataques múltiplos no próximo turno de esquiva qualquer ação envolvendo destreza ou esquiva tem um redutor igual ao número de ataques extras.
	
No caso de você está dando vários golpes e o oponente não poder esquivar, o mestre pode atribuir um valor de dificuldade mínimo para você acertar o golpe. Isso fica a critério do narrador.
	
Tanto a quantidade de golpes, quanto as penalidades que cada golpe tiver (se tiver) podem ser estabelecidas pelo mestre. Existem mestres que não limitam a quantidade de golpes outros se guiam pelo bônus de destreza (cada bônus de destreza é um golpe a mais). Isso cabe ao mestre decidir. Aqui nós apenas mostramos uma média que pode ser usada, uma vez que o narrador pode utilizar do modo como bem desejar. Existem algumas habilidades que retiram esses limites, cedendo ataques extras ou bônus aos atacantes. 

\subsection{Esquivas Múltiplas}

Quando um personagem recebe múltiplos ataques, seja de um único oponente realizando vários ataques ou de múltiplos oponentes, ele recebe penalidades cumulativas em suas esquivas.

\textbf{Regra Simplificada}: Para cada ataque extra recebido, o defensor recebe um redutor de -2 na esquiva.

\textbf{Exemplos}:
\begin{itemize}
	\item Um oponente realiza 3 ataques: primeiro ataque sem penalidade, segundo ataque -2, terceiro ataque -4.
	\item Três oponentes atacam simultaneamente: primeiro ataque sem penalidade, segundo ataque -2, terceiro ataque -4.
\end{itemize}

O mestre pode permitir que o defensor use uma única esquiva para ataques quase simultâneos, mas isso consome mais PF e pode exigir um teste de percepção dependendo da situação.
 


%\subsection{Detalhes sobre Aparar}

%A regra a seguir pode ser usada por mestres mais exigentes em relação a detalhes da batalha.

%Caso o defensor passe no teste (consiga aparar o golpe) ele deve realizar um teste de força ou destreza (ele pode escolher) contra a força do atacante em duas situações distintas. A primeira é quando a arma do atacante é de grande porte, o defensor esté usando uma arma média para aparar e sua força é menor do que a força do atacante. A segunda é quando as duas armas são de mesmo porte, porém a força do atacante é maior que a força do defensor em 4 pontos ou mais. No caso do defensor escolher destreza para jogar contra a força do alvo, ele usa o mesmo resultado do teste inicial de aparar. 

%Esses dois testes são interpretados da seguinte forma. O primeiro serve para o posicionar o objeto que vai ser usado para aparar o golpe. Esse teste é o teste de aparar em si, aonde bônus de habilidades e equipamentos que concedam bônus em aparar podem ser usados. O segundo teste serve para anular ou direcionar a força proveniente do atacante. Para realizar tal feito, o defensor pode fazé-lo usando sua força diretamente contra a do oponente ou utilizando sua destreza para desviar essa força em outra direééo. Ou seja, o atributo aparar do equipamento deve ser usado como um bônus em qualquer teste de destreza envolvendo essa manobra. De forma semelhante, o atributo resiténcia do equipamento deve ser usado como um bônus em qualquer teste de força.


%Se o defensor não conseguir passar nesse teste de força combatido, isso pode ser interpretado que o defensor conseguiu posicionar algum objeto na trajetória do ataque, mas a força do ataque deslocou o objeto. Então o defensor recebe o golpe, porém reduzido. O valor da redução do golpe deve ser definida pelo mestre. Por exemplo, o mestre pode reduzir do dano recebido pelo defensor um valor igual a força do defensor. O defensor também pode ter feito um teste de força cujo resultado foi téo baixo, que o golpe não perdeu nada de sua força original. Nesse contexto, no lugar do defensor levar o golpe reduzido, o mestre também pode retirar uma quantidade maior de PR da arma ou até mesmo fazer com que a arma caia no chéo, próximo ao defensor. O mestre pode interpretar a manobra da forma como desejar.

%Em outras palavras, para realizar a manobra aparar o defensor deve realizar no méximo dois testes. O primeiro é de destreza contra a destreza do oponente. O segundo deve ser realizado caso se encaixe na situação descrita acima. Esse segundo teste é de força ou destreza (o defensor escolhe qual atributo usar) contra a força do oponente. Caso escolha a destreza, ele usa o mesmo valor obtido no primeiro teste. 

%Em algumas situações o mestre pode (e deve) proibir o uso dessa manobra. Por exemplo, alguém tentando aparar um martelo grande com uma espada curta. Apesar da perécia bélica "lutar desarmado" poder ser usada para aparar ataques, o mestre deve considera-lé como arma curta ou média de acordo com a situação. Por exemplo, ela pode ser considerada arma média caso o defensor esteja próximo do oponente, tornando assim o uso da péricia mais fácil. No caso inverso, ou seja, o defensor usa alguma arma para se defender de um atacante que usa lutar desarmado, o defensor não pode usar a manobra aparar (apesar com alguma péricia defensiva como escudo ou manopla) caso o atacante esteja muito próximo. Isso é determinado pelo mestre de acordo com a batalha, onde o mesmo pode exigir teste por parte do atacante para se aproximar a tal ponto do defensor (um teste de saltar por exemplo).	


\subsection{Atacando Alvos Específicos}

Para tornar mais ágil e simples o combate, em Dregon adotamos a seguinte interpretação para ataques em alvos específicos. Ao invés de acertar um local específico de um alvo para retirar uma quantidade maior de PV, sempre que um atacante retirar um sucesso muito grande (dano e acerto) esse ataque bem sucedido é interpretado como sendo nesse local. Por exemplo, uma grande quantidade de dano retirada subitamente pode ser interpretada como um membro decepado por exemplo.

Caso o personagem deseje atacar algum alvo específico do oponente (furar uma bolsa com itens, por exemplo), a jogada de ataque é feita normalmente, porém, o defensor tem um bônus em sua jogada de esquiva (esquivando ou aparando) de acordo com o tamanho do objeto alvo. Quanto menor o alvo, maior o bônus, variando de 2 até 8.

O mestre também pode usar a seguinte regra opcionais para redução de atributo de acordo com dano causando em certas partes do corpo. Caso o ataque acerte com êxito membros do corpo (usando os redutores citados acima), o mestre pode impor redutores em atributos (força, destreza ou esquiva) de acordo com o dano sofrido. A quantidade de redução de atributo será avaliada perante a quantidade de dano sofrido. O redutor é igual ao bônus do dano (um dano de 4 irá causar um redutor de -1 e assim por diante).
  

%\begin{itemize}
%	\item Alvos médios (bolsas, braços, etc) = bônus de destreza ou esquiva +2.
%	\item Alvos pequenos(pescoéo, cabeéa, mãos, etc) = bônus de ágilidade  + 4.
%	\item Alvos muito pequenos(olhos, brincos, etc) = bônus de ágilidade  + 6.
%\end{itemize}

O personagem também pode desejar atacar um alvo específico do oponente com o objetivo de reduzir a absorção da armadura do alvo. Para fazer isso é simples. Para cada ponto reduzido no teste de atacar, o dano (caso o ataque acerte) ignora 1 ponto da armadura do alvo (começando pela defesa normal). Outro detalhe é que para cada ponto de cobertura da armadura, a mesma consegue ignorar 1 ponto essa redução. Por exemplo, caso o atacante deseje retirar 2 pontos de absorção de uma armadura com cobertura zero ele deve jogar o teste de atacar com um redutor igual a 2. Se ele deseja reduzir a mesma quantidade de uma armadura com cobertura 3, então ele deve ter um redutor de 5 em seu ataque. A penalidade recebida não pode ser maior que o atributo destreza - 1. Vale observar que armaduras pesadas podem ter sua absorção reduzida até a metade com o uso dessa manobra.


%Caso o ataque atinja partes vitais e cause dano, o alvo sofre um dano extra igual a uma percentagem do PV do alvo (o mestre decide se 25\% ou 50\% do PV total).

Vale destacar que determinados inimigos não tem penalidades de receber danos em alvos específicos, seja devido a sua natureza ou cobertura excepcional de alguma armadura. Além disso, como dito anteriormente, o mestre pode interpretar os efeitos da perda de PV, ou seja, não será somente com o uso dessa manobra que um alvo pode ter seu braço ferido ou decepado. Cabe ao mestre escolher de acordo com a situação de combate quando essa manobra pode ser usada e como.

\subsection{Ataque em área}

Determinadas ocasiões permitem você acertar vários oponentes com apenas um golpe. Seja devido a arma ser grande (um machado ou espada montante), ou por que vários alvos estão muito próximos a você. No caso da arma poder fazer isso, vem especificado nela quantos metros ela pode alcançar com um golpe. No caso de personagems próximo demais, quem decide é o mestre. Nesse caso se faz um teste de destreza(no caso quem ataca). O alvo mais próximo recebe o golpe com destreza normal e os alvos consecutivos vão esquivar com um bonus cumulativo de +2 e uma penalidade cumulativa no dano para o atacante, em outras palavras, o ataque vai perdendo a força ao longo do trajeto. De acordo com a descrição da cena, o dano pode diminuir em 3 ate 10 para cada alvo atingido. Por exemplo, se alguém com uma espada longa ataca vários alvos e o segundo absorve tudo, o mestre pode decidir por reduzir o dano em 13 para o terceiro alvo, sendo 3 do primeiro e 10 do segundo. O mestre pode também interromper o ataque, dizendo que o ataque em área não efetuou sua trajetória completa. 

\subsection{Contra Ataque - Atacando no Turno de Esquiva}

Você pode no seu turno de esquiva optar por tentar atacar o oponente no lugar de esquivar-se de seu golpe (algo como um contra ataque). Primeiramente você tem que fazer um teste para ver se você é rápido o suficiente para poder atacar no turno de esquiva. Você vai fazer um teste de ágilidade contra o oponente que está te atacando nesse turno de ataque. Porém, como você esté atacando no turno de esquiva e o oponente não (pois o intervalo de tempo que ele dispõe para realizar uma ação e mais bem ajustado) você vai ter penalidades nesse teste de ágilidade. A penalidade será igual ao bônus de ágilidade do oponente (se for um oponente lento o turno de ataque dele poderá ser facilmente mais lento proporcionando oportunidade do seu ataque). Se você passar no teste poderá atacar normalmente o oponente. Algumas observações devem ser levantadas a respeito de tal manobra.
\begin{itemize}
	\item Se vários alvos estiverem te atacando no turno de esquiva, você poderá atacar eles. Faça um teste ágilidade contra cada um (com redutores baseados nas ágilidades de cada um separadamente). Os que você passar poderá atacar normalmente (dividindo a destreza com regras de múltiplos ataques), porém se alguém for mais rápido (ganhar no teste de ágilidade), você não poderá mais contra-atacar nesse turno.
	\item O alvo que recebe essa manobra não pode tentar se mais rápido que você(um contra-contra-ataque).
	\item Nem você nem o alvo terão direito de esquiva. O maximo que pode se feito é o que está no turno de ataque fazer um teste de percepção para poder usar a manobra defesa máxima. Esse teste é um teste contra a ágilidade de quem tenta dar o contra golpe.
	\item Essa manobra não pode ser feita no primeiro momento de ação da batalha.
	\item O mestre tem total direito para alterar tais regras segundo sua interpretação. Em casos como já foi citado, se os números do teste de ágilidade ou de destreza forem próximos ou iguais, o mestre pode interpretar de modo próprio. Os golpes podem se chocar, os dois atacam ao mesmo tempo e assim por diante.
	\item Como outras manobras de combate não usuais, ela consome mais PF do que o normal.
\end{itemize}


\subsection{Esquivando no Turno de Ataque }

Ao invés de atacar o oponente, você pode desviar de um provável golpe que pode vir a receber no próximo turno. Você ganha um bônus de ágilidade em destreza ou esquiva para qualquer ação que você fizer no turno seguinte a este em que você ficou parado (no caso seu turno de esquiva). Você amplia seu turno de ataque para se fundir com o turno de esquiva de modo a ganhar bônus para se esquivar ou defender usando a destreza (esse bônus e determinado pela ágilidade). O bônus não pode ser maior que o atributo usado. Por exemplo, certo personagem tem destreza 8 e esquiva 2, totalizando ágilidade 10. O bônus da sua ágilidade é 4, porém ele so pode usar 2 pontos desse bônus caso use a esquiva para se esquivar do ataque.


\subsection{Movimento Durante a Batalha}

Qualquer personagem pode se mover até um valor igual ao seu bônus de esquiva em metros durante um turno normal (podendo atacar e se esquivar normalmente). Se o personagem tirar o turno só para se mover, esse valor sobe para o seu valor completo de esquiva em metros.

O mestre pode reduzir esse valor de acordo com cargas levadas pelo personagem ou outras condições específicas da situação.
	
	
\subsection{Desarmar Oponente}

A manobra desarmar é usada para tirar a arma de um oponente de suas mãos. O uso dessa manobra é bastante simples, por parte do atacante um teste de destreza mais um bônus, definido de acordo com a situação do desarme. Esse bônus é igual ao bônus do atributo força caso o atacante use esse atributo para desarmar o oponente ou de acordo com a arma utilizada. De forma semelhante, o bônus é de destreza para desarmes usando armas que possibilitem essa opção (armas leves ou pequenas geralmente se encaixam nessa situação).

Por parte do defensor, um teste de destreza ou esquiva mais um bônus. Da mesma forma que o atacante escolhe atribuir o bônus de força ou destreza, o defensor também pode faze-lo. O mestre pode adicionar outros bônus extras de acordo com a situação, por exemplo, conceder um bônus maior de força se a arma utilizada for de duas mãos, se ela tiver algum acessério que ligue a arma ao defensor ou até mesmo de acordo com a descrição da manobra (distância dos alvos, usar uma arma média atrás de um escudo, etc). é muito importante que o mestre preste atenção nesses detalhes, por exemplo, na maioria dos casos é muito difícil desarmar um oponente usando determinadas armas (espadas, machados, etc). Boa parte das armas são feitas para causar dano, e seu formato não auxilia na manobra desarmar. Para facilitar, o mestre pode usar o seguinte referencial. Se uma arma não conceder bônus em aparar, o defensor recebe seu bônus (de destreza ou esquiva) duas vezes para se defender. Se a descrição da manobra por parte do atacante for coerente, o defensor joga o atributo mais bônus de atributo mais bônus de distância (que pode ser ignorado de acordo com a situação).

Essa manobra não retira dano do defensor caso seja bem sucedida. O número de sucessos determina qual longe a arma foi arremessada. Em alguns de sucesso muito baixo favorecendo o atacante, o defensor pode realizar um teste de concentração para tentar pegar a arma "no ar", ou atacar usando a arma no próximo turno com um redutor (como se uma parte do turno de ataque dele tivesse sido usada para recuperar a arma caída).
 
O mestre deve decidir quais atributos serão usados e os bônus que serão estábelecidos de acordo com a situação. Por exemplo, um samurai com uma espada média tentando desarmar um ranger com uma lança. O samurai pode escolher destreza (recebendo assim também o bônus de destreza) para tentar desarmar o ranger. O ranger escolhe pular para o lado enquanto puxa sua lança. Nesse caso ele joga esquiva mais bônus de destreza mais um bônus definido pelo mestre de acordo com a distância que o samurai encontra-se do ranger.

Em termos de sistema não é uma manobra complexa, porém seu uso da brecha para muitas possibilidades diferentes, possibilidades essas que podem conceder bônus e ônus tantos para o defensor quanto para o atacante. O mestre deve ficar atento para a descrição da manobra. Na dúvida, favorecer a defesa.

\subsection{Agarrar Oponente}

Para realizar a manobra agarrar é simples. Basta realizar sua jogada de ataque normal, ou seja, destreza contra esquiva ou destreza do oponente. Caso acerte o alvo, o atacante deve realizar outro teste, porém dessa vez usando força contra a força ou contra a destreza do oponente. Se obtiver sucesso mais uma vez, o defensor é considerado agarrado pelo atacante. O mestre deve incluir bônus ou ônus de acordo com a situação, por exemplo, um atacante pequeno tentando prender um alvo grande apenas com as mãos, ou o atacante usando cordas para amarrar o alvo. Quando agarrado, dependendo da situação do agarrão, o mestre deve definir as limitações de ações durante o agarrão. Por exemplo, uma pessoa com os braços imobilizados, não pode atacar usando os mesmos. O mestre também deve definir o dano que o alvo fica recebendo enquanto fica agarrando. Esse dano também pode ser representado por perda de PF (em situação de estrangulamento), ou perda de atributo (no caso de um membro quebrado).

Para se soltar do agarrão, o defensor deve realizar um teste novamente de destreza ou força com a força do oponente.

%\subsection{Regra Opcional sobre PF}

%Para mestres mais exigentes exige um método mais detalhista para monitoramento de perda de PF para cada personagem. Para cada personagem presente na batalha o mestre deve criar uma barra, com 10 campos, variando de 10\% até 100\%, como mostrado a seguir. Vamos chamar essa barra de "félego".

%\begin{table}[htbp]
%\begin{center}
%\begin{tabular}{|c|c|c|c|c|c|c|c|c|c|} \hline 
%10\%&	 20\%&	30\%&	 40\%&	 50\%&	 60\%&	 70\%&	 80\%&	 90\%&	 100\%\\\cline{1-10} 
% &  &   & &  &  &  &  &   &  
%\\ \hline
%\end{tabular}
%\end{center}
%\caption{Barra de acompanhamento de perda de PF}
%\label{}
%\end{table}

%Sempre que o personagem realizar uma ação, o mestre deve marcar com um "X" os campos equivalentes a percentagem que aquela ação consome. Por exemplo, se um personagem realizar um ataque que consuma 30\% dessa barra, a tabela deve ficar como a mostrada a seguir (partindo do princépio que o félego inicialmente estáva vazio). 
  

%\begin{table}[htbp]
%\begin{center}
%\begin{tabular}{|c|c|c|c|c|c|c|c|c|c|} \hline 
%10\%&	 20\%&	30\%&	 40\%&	 50\%&	 60\%&	 70\%&	 80\%&	 90\%&	 100\%\\\cline{1-10} 
%X &X  &X   & &  &  &  &  &   &  
%\\ \hline
%\end{tabular}
%\end{center}
%\caption{Exemplo usando ataque que consuma 30\% de seu félego}
%\label{}
%\end{table}

%A percentagem de quanto cada ação consome do félego é mostrada na tabela a seguir.

%\begin{table}[htbp]
%\begin{center}
%\begin{tabular}{|c|c|c|c|} \hline 
%Equip/Aééo &	 Esquiva &	Ataque&	 Aparar\\\cline{1-4} 
%Leve & 5\%  & 20\%   &  10\% \\\cline{1-4} 
%Médio & 10\%  & 30\%   &  20\% \\\cline{1-4} 
%Pesado & 30\%  & 50\%   &  30\%  
%\\ \hline
%\end{tabular}
%\end{center}
%\caption{Gasto de percentagem de PF}
%\label{}
%\end{table}


%Sempre que essa barra atingir 100\% o personagem perde 1 PF. Para cada ação extra realizada no mesmo turno, o personagem perde metade do que normalmente perde quando realizado aquela ação pela primeira vez. Em outras palavras, para cada 2 ações extras o personagem perde a percentagem associada aquela ação na tabela acima. Por exemplo, se um personagem realizar 3 ataques com uma arma média no mesmo turno ele perde 60\% de seu félego. O consumo do félego da esquiva esté associado na tabela acima está relacionado com o peso do equipamento usado. Em outras palavras, em um turno um personagem com uma arma grande e uma armadura média, perde 60\% de seu félego quando realizado um ataque e uma esquiva. 

%Vale lembrar que em certas batalhas um personagem não é atingido ou realiza nenhum ataque. Nessas situações um personagem não perde PF na batalha. Isso também cria situações de que um grupo salvar os PF de um personagem para serem gastos em situações mais urgentes. 

%Porém vale observar que em algumas batalhas o consumo de PF por ser crucial para definir o seu resultado. Por exemplo, um personagem usando uma arma de porte grande (um machado por exemplo), pode usar a manobra aparar várias vezes (uma vez que usando uma arma de grande porte, ele pode aparar armas de porte médio e pesado). Caso a destreza desse personagem seja muito grande, pode acontecer de que nenhum de seus oponentes consigam atingi-lo. Nesse caso, os oponentes podem bolar uma estratégia de consumo de PF, fazendo com que o alvo use a arma de porte grande várias vezes para aparar, consumindo seus PF lentamente com isso. Nesse caso o mestre deve retirar PF cautelosamente dos personagems e não somente uma vez ao termino da luta, pois o consumo de PF durante ela é decisivo nesse caso.

%Para ágilizar todo o processo de perda de PF durante a luta, o mestre pode ir anotando as ações realizadas por cada personagem (1 ataque, 3 aparar, 2 ataques, não foi atacado, etc), e de 4 em 4 turnos ir retirando de cada personagem o PF perdido até aquele momento. Vale lembrar que o mestre tem total liberdade para interpretar a perda de PF e usa-lé dentro de jogo como bem desejar. Alguns mestres retiram PF apenas quando habilidades ou manobras são usadas.



 


\ClearShipoutPicture

 %\include{AppendixA}
 %\include{AppendixB}

 %\phantomsection 
 %\addcontentsline{toc}{chapter}{Index}
 %\renewcommand{\baselinestretch}{1} \small \normalsize
 %\printindex


\end{document}
