\documentclass[masters]{BYUPhys}
\usepackage[portuguese,noprefix]{nomencl}
\usepackage[portuges, brazil]{babel}

\usepackage{eso-pic}
%\usepackage{wallpaper}

\newcommand\BackgroundPic{
\put(0,0){
\parbox[b][\textheight]{\textwidth}{%\parbox[b][\paperheight]{\paperwidth}{
\vfill
\centering
%\includegraphics[height=1.5\textheight, width=1.05\textheight]{"page"}%
\vfill
}}}
% ------- Fill in these fields for the preliminary pages --------
%
% For Senior and honors this is the year and month that you submit the thesis
% For Masters and PhD, this is your graduation date
  \Year{2016}
  \Author{N.b.m. - V.1.2}

% If you have a long title, split it between two lines using the \\ command.
% A multiple line title should be an "inverted pyramid" with the top line(s)
% longer than the bottom.
  \Title{Livro do Mestre - Dragon}

\begin{document}

 % Start page counting in roman numerals
 \frontmatter

 % This command makes the formal preliminary pages.
 % You can comment it out during the drafting process if you want to save paper.

 \makepreliminarypages

 \singlespace

 % Make the table of contents.
 \tableofcontents

% \clearemptydoublepage

 %\listoffigures

 %\listoftables

 \doublespace

 % Start regular page counting at page 1
 \mainmatter
\AddToShipoutPicture{\BackgroundPic}


%%
%% Cap�tulo 1: mestre
%%


\chapter{Introdu��o}

\label{Cap:intro}

O mestre, tamb�m chamado de narrador, � responsavel por transportar os jogadores para o cen�rio de jogo, ou seja, ele � a ponte entre os jogadores e o mundo aonde os personagens existem. Al�m de descrever o cen�rio e seus personagens (por isso o nome de narrador) o mestre tamb�m deve dar din�mica ao jogo, criando situa��es, fazendo julgamentos, impondo barreiras, recompensando objetivos alcan�ados, entre outros v�rios atos importante dentro de jogo. 

Para ajudar o mestre em tal tarefa, muitas vezes complicada, aqui se segue v�rios pontos de ref�ncia para guiar o mestre. Quest�es como "esse equipamento � bom para essa situa��o" ou "que dificuldades devem passar os jogadores para percorrer essa caverna" ter�o respostas nesse livro, que deve ser usado como ponto de refer�ncia, e nunca como restri��o da narrativa do jogo. Nesse cap�tulo iremos mostrar como o mestre pode usar a liberadade a ele conferida para guiar o grupo e criar aventuras equilibradas.

\section{Liberdade}
A primeira e �nica regra que o mestre deve seguir fielmente � a chamada "Regra de Ouro". Ela diz que o mestre tem total controle sobre qualquer regra de um sistema. Ou seja, se em determinada situa��o o mestre julga que regra "X" est� atrapalhando o andamento do jogo, ent�o ele pode mudar essa regra. O sistema dragon foi criado dando liberdade para o mestre, ou seja, ele tem liberdade para alterar boa parte das regras sem prejudicar o sistema como um todo. � importante o mestre usar a "Regra de Ouro" com cautela. Seu uso excessivo pode trazer desequilibrio ao jogo.

Por exemplo, o mestre nunca deve usar seu poder para prejudicar os jogadores intensionalmente. Se algum personagem do grupo est� muito forte, provavelmente ele o fez por merecer, j� que o sistema foi criado para n�o gerar personagens desequilibradamente fortes dentro de um grupo, e mesmo que o sistema falhe o mestre pode corrigir esse erro. O mestre n�o deve colocar aquele personagem em especial em situa��es que o prejudiquem somente por que ele � forte, enquanto os outros ficam numa situa��o mais favor�vel. O que pode acontecer � uma distribui��o de tarefas, onde por exemplo, um personagem mais forte fisicamente enfrenta o inimigo mais forte do grupo enquanto os outros personagens o auxiliam. O mestre tamb�m deve seguir a "Regra de Platina", que diz que o jogo deve ser jogado para trazer divers�o justa aos jogadores. Derrotar um inimigo devido a real for�a e habilidades dos jogadores � melhor do que o mestre explodir um inimigo dos jogadores com seu poder de mestre s� por que o mestre n�o quer que os jogadores percam. Quando mais o mestre se vir no papel de narrador e deixar o jogo fluir, melhor para o grupo.

Em termos de sistema, listaremos alguma situa��es que geralmente o mestre pode (e deve) usar sua liberdade e altera-las.

\section{Custo de Habilidades}
 
Segundo o livro do sistema, sempre que um personagen desejar comprar uma habilidade fora de sua classe ele deve pagar um custo extra por estar comprando uma habilidade extra-classe. De acordo com o andamento da aventura, o mestre pode diminuir esse custo extra, e at� em alguns casos anula-lo. Por exemplo, um personagem da classe ladr�o deseja aprender espada de duas m�os, uma habilidade t�picamente guerreira. Durante sua aventura ele encontra um guerreiro experiente, que deseja ensin�-lo. Digamos que o custo extra seja de 4 (fora o custo normal da habilidade). O mestre decide reduzir esse custo extra para 2 j� que o guerreiro ensinando o ladr�o � excepcional em seu ensinamento, tendo este desenvolvido um m�todo eficaz de treinamento. Quanto maior for o atributo acuidade de quem aprender e de quem ensina, menor deve ser o custo extra da habiliade ap�s determinado tempo de treinamento. Al�m disso, o personagem pode reduzir o custo de habilidades extra classe com ajuda de companheiros do pr�prio grupo, e n�o apenas NPC que ele encontre ao longo da sua hist�ria.

Vale lembrar tamb�m que o mestre pode limitar o aprendizado de certas habilidades apenas quando determinado acontecimento ocorrer dentro de jogo. Por exemplo, elepode limitar o aprendizado da habilidade metamorfose, apenas quando o personagem gastar um determinado tempo treinando com um grupo de druidas ou em uma escola com magos azuis poderosos.
	
%N�o � aconselhado o mestre reduzir uma habilidade abaixo de seu custo m�nimo de aprendizado em xp, ou seja, n�o aconselhamos o mestre fazer com que um personagem aprenda uma habilidade que custe 10 de xp gastando apenas 6. Apenas em raros casos isso pode ocorrer, por exemplo, o personagem tem a habilidade materializar firani que custa 20 PS e 8 xpm. A habilidade materializar magia � uma versao melhorada dessa magia, onde o usu�rio pode materializar n�o somente a magia firani como qualquer magia que ele possua. O custo dessa habilidade � de 20PS e 12 xpm para magos negros e 30PS e 18 xpm para outras classes que n�o sejam desfavorecidas com mago negro. Como as duas habilidades s�o bastante parecidas, o mestre pode fazer com que o usu�rio da habilidade m�gica materializar firani, quando for aprender a habilidade materializar magia, desconte uma quantidade de PS e xpm igual �quela quantidade gasta para aprender materializar firani, ou seja, no lugar dele ter que gastar 30PS e 18 xpm, ele pode aprender essa mesma habilidade gastanto apenas 10PS e 10 xpm.	
		
%O mesmo pode ser usado para p�ricias, onde por exemplo, para um personagem que saiba usar espada curta e espada m�dia, o mestre pode conceder um b�nus para esse personagem aprender a usar a p�ricia b�lica espada longa.	
	
\section{Criando Novas Habilidades}
As habilidades das classes e magias listadas no livro do sistema comp�em apenas uma base do que aquela classe ou aquela magia pode proporcionar. Baseado nisso, o mestre e os jogadores tem total liberdade para criar habilidades durante a aventura. Existem duas formas de criar uma nova habilidade.

A primeira � criando uma habilidade do zero. De acordo com o conceito do seu personagen, muitas vezes voc� sente nescessidade que o seu personagem fa�a algo a mais, por�m esse "algo a mais" n�o encontra-se manifestado na forma de nenhuma habilidade no sistema. Para criar uma habilidade do zero o mestre ou jogador deve criar um rascunho dela, e ent�o procurar alguma habilidade semelhante. Por exemplo, se voc� est� criando uma habilidade que te conceda b�nus, seria interessante comparar com outras habilidades que concedam b�nus. Muitas habilidades criadas pelos jogadores podem ser aproximadas para habilidaes j� existentes. 

O mestre deve ter cuidado para n�o fazer com que tal habilidade fuja da �tica do personagem, desequilibrando o mesmo quando comparando com outros. Ent�o, se voc� � especializado em suporte, o normal � que voce crie habilidade voltadas para o suporte. Se o mestre concede a voc� uma habilidade de ataque de forma semelhante a se te concedesse uma de suporte, nesse momento o jogo tem um desequilibrio. Apesar disso, um jogador pode criar habilidades que n�o envolvam seu conceito, por�m ele estar� criando essas habilidades para outros personagens. Ent�o, voltando ao exemplo do personagem de suporte, ele pode auxiliar um personagem de ataque ao criar uma nova habilidade de ataque basedo em sua observa��o pessoal, mesmo ele n�o sendo um personagem de ofensivo. 

Geralmente quando um jogador cria uma nova habilidade do zero, a mesma habilidade, ou uma vers�o muito pr�xima aquela habilidade existe na lista de habilidades de outra classe. Quando isso acontecer o mestre pode fazer com que o jogador aprenda a habilidade sem necessidade de treino, apenas gastando a experi�ncia necess�rio.

A segunda forma de criar uma habilidade nova � melhorando alguma habilidade antiga. Geralmente isso � feito aumentando o b�nus que determinada habilidade concede, por�m aumentando tamb�m o seu custo (em PF, PM e em raros casos PV). Para reduzir o custo de uma habilidade geralmente � mais vantajoso aumentar os valores de PF ou PM diretamente com experi�ncia. Ou seja, se determinada habilidade consome 3 PF para ser usada, no lugar do personagem diminuir o custo dessa habilidade � melhor ele aumentar permanentemente os seus PF. Isso � uma dica para a maioria das habilidades, exceto por algumas magias. Se determinada magia custa muito PM para ser usada, geralmente o personagem pode baixar seu custo com gasto de PS na ordem de 10PS gastos para reduzir 1 PM no custo da magia. Geralmente uma habilidade n�o pode ser reduzida para um valor menor que a metade do custo original. Mais uma vez, o mestre deve ter cuidado para que uma habilidade/magia n�o seja melhorada demais, desequilibrando assim o jogo. 

O mestre tambem pode fazer com que uma vers�o melhorada da habilidade seja o jogador poder comprar a mesma habilidade novamente, com um custo maior, podendo asssim acumular os b�nus da habilidade. Por exemplo, a habilidade punho da stiga concede um b�nus do dano, proporcional ao n�mero de vezes que a mesma foi comprada, at� um valor m�ximo de 5 vezes. O mestre pode criar uma vers�o melhorada, aonde o b�nus � o mesmo, por�m o personagem pode comprar a habilidade mais 2 ou 3 vezes, podendo acumular o b�nus e aumentar ainda mais o dano. Como dito anteriormente, o mestre pode aumentar o custo da nova habilidade ou restringir seu aprendizado para situa��es espec�ficas dentro da hist�ria.

Em muitos casos o mestre pode criar novas habilidades que n�o se encaixem dentro de jogo. Por exemplo, em um cen�rio futurista o mestre pode incluir v�rias habilidades voltadas para tecnologia avan�ada. Em outras palavras, a lista de habilidades gerais mostradas no livro do sistema n�o � (nem de longe) definitiva.

Compara��o, equilibrio e imagina��o s�o os aspectos mais importantes no momento de se criar uma habilidade nova.


\section{Cuidados}

� extremamente importante que o mestre tenha cuidado com habilidades "Quebra Regra". Por exemplo, � normal magias serem de acerto autom�tico, ou seja, o usu�rio de uma magia n�o precisa realizar a jogada de acerto. Por�m, se o alvo da magia tiver a habilidade auridade ele pode tentar se esquivar de uma magia. Essa habilidade � considerada uma habilidade "Quebra Regra", pois ela vai de encontro com as regras fundamentais do sistema. Isso em determinadas situa��es pode trazer elementos interessantes para o cen�rio, por�m, seu uso excessivo pode causar problemas.

\section{Recompensas}

O mestre deve recompensar seus jogadores de v�rias formas. O mestre pode conceder aos jogadores ponto de b�nus por interpreta��o. Esse pontos de b�nus podem ser dados aos jogadores quando eles poderem gastar sua experi�ncia acumulada ou quando o mestre achar adecuado.

Sempre que um jogador alcan�ar um objetivo, o jogador deve ser recompensado com ponto de experi�ncia extra. O b�nus varia de 2 a 15 pontos, de acordo com a magnitude do objetivo alcan�ado. Por exemplo, os jogadores est�o a 20 sess�es tentando resgatar um membro da realeza que foi sequestrado. Para isso o grupo teve que encontrar boatos de quem o sequestrou, e ir atr�s do grupo. Al�m disso eles tiveram que combater e libertar a pessoa sequestrada. Ao final dessa quest, o mestre pode conceder 10 pontos de b�nus extras para os jogadores de acordo de como foi realizado o resgate (que tipo de repercus�es positivas essas a��es trouxeram para o grupo).

Outra forma de recompensar o jogador � atravez de itens e equipamentos. Por exemplo, em alguns locais do mundo existem po��es que aumentam permanentemente atributos ou pontos como PV, PF e PM. Por�m essa forma de recompensa deve ser a mais rara de todas, pois pode causar desequilibrio nos personagens, uma vez que somente 1 personagem do grupo pode usar uma dessas po��es por vez.

Existem outras recompensas menos diretamente relacionadas com a ficha do personagem. Por exemplo, encontrar determinado NPC (personagem do mestre), pode fazer com que aquela pessoa possa aprender certas habilidades, ou ganhar b�nus quando aprende-las, como dito anteriormente. Cabe ao mestre de acordo com o andamento do grupo decidir que tipo de recompensar seus jogadores merecem.

\section{Downtime}

Downtime � um termo usado para periodos longos onde o jogo fica de certa forma "parado". Dentro de um downtime o personagem pode treinar, estudar, viajar, entre outras a��es que exigam uma grande quantidade de tempo. Por exemplo, o objetivo de um certo personagem � viajar at� o reino de jotan e participar do torneio de cavaleiros, que acontece de 3 em 3 anos. Ap�s chegar ao reino, esse personagem percebe que chegou muito cedo. Ele n�o quer realizar nenhuma side-quest (aventura paralela). Ent�o o mestre anuncia um downtime de 6 meses, tempo necess�rio para que o torneio tenha inicio.


A��es realizadas no downtime s�o a��es prolongadas. Ou seja, o personagem n�o define o que vai fazer em um dia especif�co e sim durante 1 semana ou durante 1 m�s. Downtimes s�o importante entre aventuras grandes ou at� mesmo para alguns casos trazer coer�ncia � hist�ria do grupo.






\chapter{Explorando o Mundo}
\label{cap:explorando}

O personagem pode explorar determinadas áreas em 2 amplitudes diferentes. Explorações em locais grandes e em locais pequenos. Explorando um local grande é por exemplo, explorar um deserto, uma floresta, etc. Explorar um local pequeno seria explorar um castelo, um bairro abandonado, entre outros. Quando explorando um local grande, os personagens percorrem distância maiores e gastam mais tempo durante a exploração. O mestre deve levar em consideração o local a ser explorando para descrever o que está acontecendo, e também manipular a perda/ganho de PF.

Em um dia de exploração ou locomoção, o personagem passa por 3 estágios. Descansos curtos, longos e caminhadas. Um descanso longo dura cerca de 8 horas, enquanto que um curto cerca de 30 minutos. Dentro do tempo gasto para descanso um personagem pode dormir, fazer sua higiene pessoal e se alimentar. Ele consome cerca de 10 horas de caminhada em um dia de exploração. Tomando em consideração que a velocidade média de caminhada de um ser humano é cerca de 8 km por hora, então ele caminha cerca de 80 km por dia. Porém, esse valor é usado para caminhos retos, como estradas bem construídas ou planícies. O mestre deve reduzir esses 80 km diários de acordo com a situação do terreno. Existem montanhas que esse valor é reduzido para 10 km diários por exemplo. 

Durante a exploração, o mestre pode fazer com que os personagens encontrem criaturas, armadilhas, tesouros, ou outros elementos interessantes. O mestre deve usar sua criatividade para tornar a exploração interessante e desafiadora.

\section{Exploração de Locais Grandes}

Ao explorar locais grandes, os personagens podem encontrar:

\begin{itemize}
    \item Criaturas selvagens
    \item Ruínas antigas
    \item Recursos naturais
    \item Perigos ambientais
    \item Povos nativos
\end{itemize}

\section{Exploração de Locais Pequenos}

Ao explorar locais pequenos, os personagens podem encontrar:

\begin{itemize}
    \item Armadilhas
    \item Tesouros escondidos
    \item Documentos importantes
    \item Criaturas hostis
    \item Segredos antigos
\end{itemize}

\section{Regras de Exploração}

Durante a exploração, os personagens devem:

\begin{itemize}
    \item Fazer testes de percepção para detectar perigos
    \item Fazer testes de sobrevivência para navegar
    \item Fazer testes de conhecimento para identificar elementos
    \item Gerenciar recursos como comida e água
    \item Tomar decisões estratégicas sobre rotas
\end{itemize}

\section{Consequências da Exploração}

A exploração pode resultar em:

\begin{itemize}
    \item Ganho de XP por descobertas
    \item Encontros com criaturas perigosas
    \item Descoberta de tesouros valiosos
    \item Perda de PF por fadiga
    \item Descoberta de informações importantes
\end{itemize} 
%\include{inimigos} 	
%%
%% Capítulo : Equipamento
%%

\chapter{Equipamento}
\label{Cap:equipamento}

Os equipamentos são as armas, armaduras e acessórios que seu personagem utiliza em combate e aventuras. Em seguida explicaremos detalhes sobre obtenção e uso dos equipamentos no sistema Dregon. Antes de entrarmos em tais detalhes, apenas algumas observações:

\begin{itemize}
	\item Normalmente, a curta distância, pode-se aparar ataques com as mãos, porém isso não quer dizer que um artista marcial pode usar o braço para defender como se fosse uma espada ou um escudo. Ele de alguma forma impede o golpe de ser desferido para ele (segurando a base de uma espada antes que o golpe seja desferido por exemplo). O mesmo vale para armas brancas leves quando forem defender uma arma grande a curta distância, porém o teste de força continua sendo necessário em certos casos.

	\item Em alguns casos o mestre pode optar por restringir certos equipamentos de acordo com a história do personagem ou contexto da campanha. 

	\item Algumas armas exóticas podem não estar incluídas nas listagens devido a sua exclusividade de cenário. Cabe ao mestre, de acordo com a campanha, criar equipamentos novos. 
	
	\item Certas armas exigem valores mínimos de atributos para serem utilizadas. Por exemplo, um machado grande pode exigir força 6 para ser usado.

	\item Todos os personagens podem arremessar armas ou dardos para atacar a distância, porém sem bônus especiais. 

	\item Todos os personagens podem usar armaduras leves sem restrições.

\end{itemize}



\section{Tipos de Armas}

Existem três tipos principais de armas no sistema Dregon, cada uma causando um tipo específico de dano:

\begin{itemize}
	\item \textbf{Contusão:} Armas que causam dano através de impacto e força bruta, como martelos, clavas e socos. Exemplos incluem maças, martelos de guerra e armas de combate desarmado.
	
	\item \textbf{Corte:} Armas que causam dano através de lâminas afiadas, como espadas, machados e facas. Exemplos incluem espadas, machados, foices e adagas.
	
	\item \textbf{Energia:} Armas que causam dano através de energia mágica ou tecnológica, como armas de fogo, varinhas mágicas e armas de energia. Exemplos incluem pistolas, rifles, varinhas e armas de plasma.
\end{itemize}

O dano causado por todos os tipos de armas é calculado da mesma forma. A diferença está na interação com armaduras e proteções específicas. Algumas armaduras podem ter bônus ou reduções específicas contra certos tipos de dano, oferecendo proteção especializada contra um tipo específico de arma.

\section{Pontos de Resistência}

Um equipamento está sujeito a desgaste, seja esse devido ao uso ou ao passar do tempo. O que diz se um objeto está desgastado ou não são os seus PR (pontos de resistência). Os PR dizem o quão a arma ou armadura encontra-se desgastada, e o mestre deve atribuir penalidades ao equipamento quando a mesma encontra-se com poucos PR. As penalidades começam quando a quantidade de Pr é menor do que 20. Quanto menor o Pr do equipamento, maior sua penalidade. Para valores inferiores a 10, sempre que usada o equipamento tem uma chance de 50\% ser quebrado completamente.


Os PR podem ser recuperados ou terem sua quantidade total aumentada segundo alguns métodos alternativos, como magias, trabalhos de ferreiros, banhos em soluções químicas, entre outros. O PR total e atual de um equipamento deve ser anotado ao lado do mesmo na ficha, na parte de ''Info''. Cada vez que o equipamento é utilizado, o mestre deve analisar de que forma aquele uso contribuiu para o seu desgaste e assim retirar seus PR. Por exemplo, se um inimigo tem uma defesa do tipo automático alto, o mestre deve retirar mais PR da arma do que o normal (esse valor pode ser igual ao bônus de força do atacante por exemplo). Alguns inimigos explicitamente indicam que caso recebam dano ou acertem o alvo, uma quantidade de PR adicional é retirada do equipamento. A média de perda de uma arma ou armadura por batalha é de cerca de 5 PR. Armas grandes podem ser usadas para retirarem PR de outros equipamentos menores de forma semelhante a habilidade de guerreiro "Quebra de Equipamento". Por exemplo, uma armadura leve caso receba um golpe de um machado grande pode perder uma quantidade de PR igual a força do oponente, ou no máximo ate o dobro da mesma (a quantidade fica à criterio do mestre). Caso o personagem use a habilidade citada, a perda de PR é ainda maior, ou seja, 3 vezes o valor da força do usuário, normal da habilidade,  mais um bônus devido ao tamanho da arma. Esse bônus pode variar de acordo com o julgamento do mestre analisando a situação.


Algumas armas (incluindo armas extras) podem ser usadas em conjunto com a habilidade punho da stiga. Quando assim feitas, eles perdem uma quantidade de PR igual a defesa do alvo, com um limite de perda de PR maxima igual ao força mais bônus do punho da stiga do usuário. Por exemplo, um lutador tem força 6 e punho da stiga +4, e está usando um soco inglês +4. Ao atinjir um oponente com defesa 8, esse mesmo soco inglês perde 8 de PR. Porém se o inimigo tiver 20 de defesa, a arma perde apenas 10 PR (força mais bônus da habilidade punho da stiga).

Armas têm seu Pr variando entre 30 e 80, enquanto armaduras e escudos entre 50 e 100.


\section{Armaduras}

Armaduras protegem várias partes do corpo (pontos específicos como partes vitais, articulações etc), e devido a seu grau de proteção, podem privar o personagem de alguns movimentos. O uso de armaduras pesadas pode impor penalidades de movimento, reduzindo destreza e esquiva conforme o peso e tipo da armadura.

Algumas armaduras podem oferecer proteção especializada contra certos tipos de dano. Por exemplo, uma armadura de couro reforçado pode ter defesa normal 5+5 (5 de defesa normal e 5 de defesa automática) mas conceder uma redução adicional de +5 contra danos de corte, tornando-a especialmente eficaz contra espadas e lâminas. Da mesma forma, uma armadura de metal pode ser mais resistente a danos de contusão, enquanto uma armadura tecnológica pode oferecer proteção extra contra danos de energia.

Essas proteções específicas são indicadas na descrição da armadura e podem variar conforme o material, construção e tecnologia utilizada.
 

\section{Escudos}

Qualquer personagem pode usar escudos. O uso de escudos oferece bônus em aparar ataques.

O escudo tem duas características especiais. Tamanho e resistência. O tamanho do escudo diz a área de proteção do mesmo, ou seja, o bônus de aparar. Em outras palavras, sempre que um personagem tenta se defender usando um escudo, ele joga destreza + bônus do tamanho do escudo. A resistência diz o quanto um escudo pode suportar impactos, livrando o defensor de sofrê-los. O valor de resistência de um escudo é manifestado como um bônus de força quando usando a manobra aparar (veja o capítulo batalha para maiores informações). Resumindo, além do escudo conceder bônus em aparar, ele pode ajudar o defensor a suportar a força do atacante pois mesmo conseguindo interceptar o golpe, o defensor pode não suportar a força imposta pelo mesmo. Para projéteis, o teste de força na maioria das vezes é ignorado. 

De acordo com certas situações de combate (defensor sendo flanqueado por 2 personagens agéis por exemplo), o personagem que deseje aparar com o escudo pode ser impedido de realizar tal manobra ou faze-la com redutores.


\section{Arco e Flecha}

\subsection{Dano e Força}

O dano de um arco é igual à força do usuário, limitado pela capacidade máxima do arco. Cada arco tem um limite de força que pode suportar. Por exemplo, um arco que suporte até força 6 terá dano 6 mesmo se o usuário tiver força 10, mas terá dano 4 se o usuário tiver força 4.

\subsection{Alcance e Distâncias}

O alcance de um arco depende de sua construção e força máxima. Existem duas distâncias importantes:

\begin{itemize}
	\item \textbf{Distância Média:} A energia da flecha está concentrada, causando dano normal completo.
	\item \textbf{Distância Longa:} A energia foi dissipada para alcançar a distância, causando dano reduzido.
\end{itemize}

\textbf{Exemplo:} Um arco com força máxima 4 tem distância média de 0-35 metros e distância longa de 35-135 metros.

\subsection{Alcance em Combate}

Em situações de combate real, o alcance efetivo de um arqueiro é limitado por sua percepção. O alcance médio em combate é igual ao dobro da percepção do arqueiro (em metros), limitado pela capacidade do arco.

\textbf{Exemplo:} Um arqueiro com percepção 10 tem alcance médio de 20 metros em combate, mesmo que seu arco permita 50 metros.

O arqueiro pode aumentar esse alcance concentrando-se: cada turno mirando aumenta o alcance em concentração metros (até o máximo do arco).

\subsection{Características dos Arcos}

\begin{itemize}
	\item \textbf{Arcos Pequenos:} Ideais para combate próximo e médio alcance, fáceis de usar.
	\item \textbf{Arcos Longos:} Melhores para posições fixas e defesa territorial.
	\item \textbf{Arcos Pesados:} Podem ter redução no acerto devido ao dano elevado.
\end{itemize}

\subsection{Flechas}

O dano total é a soma do dano do arco + dano da flecha. Flechas podem quebrar quando atingem alvos com defesa automática maior que seu dano. Existem diferentes tipos:

\begin{itemize}
	\item \textbf{Flechas Anti-armadura:} Dano normal maior para penetrar armaduras.
	\item \textbf{Flechas de Combate Próximo:} Ponta mais pesada para maior impacto.
\end{itemize}

\subsection{Características Gerais}

\begin{itemize}
	\item Arcos devem ser usados com ambas as mãos
	\item Consomem pouco PF
	\item Perdem poucos PR por uso (principalmente por fatores externos como água e tempo)
	\item Distâncias muito longas são difíceis de mirar com precisão
\end{itemize}

\section{Armas Perfurantes}

Armas que tem o atributo perfurante ignoram um certo valor de armadura do alvo. Por exemplo, uma flecha 2+3 perfurante 2, ignora 2 pontos de armadura do alvo. 
Geralmente armas com atributo perfurante são armas cuja superficie de contato no mometo do dano é pequena em relação ao tamanho da arma em si, por exemplo, flechas, balas, etc.

\section{Armas de Fogo}

O uso de armas de fogo é bastante comum em alguns cenários. O uso dessas armas muitas vezes mortal tem alguns detalhes, que iremos expor a seguir.

\subsection{Cadência de tiro}
 
Armas mais antigas disparam apenas uma bala por ataque (tem apenas um ataque por rodada de ataque), porém armas de fogo automáticas ou semi-automáticas podem disparar diversos tiros em um único turno de ataque. A quantidade de tiros que uma arma pode realizar por turno é chamada de cadência de tiro. Existem duas formas de usar a cadência de tiro de uma arma. 

A primeira é a forma manual. O atirador divide sua destreza normalmente, da mesma forma quando está usando a manobra multíplos ataques. O número máximo de ataques realizados não pode ultrapassar a cadência de tiro da arma. 

A segunda forma é a automática. Nessa forma a arma realiza automáticamente varios tiros por turno, sem a necessidade de dividir a destreza. Porém, a medida que a arma realiza os disparos ela vai se tornando mais difícil de se controlada. Esse efeito é chamado de repuxo da arma. O repuxo é manifestado em um redutor cumulativo no acerto, onde cada arma tem seu próprio redutor de empuxo. A maioria das armas automáticas podem ser colocadas em modo normal de operação. O atacante não pode usar a manobra ataques multíplos nessa situação. A pessoa que tenta se desviar dos ataques poderá jogar apenas uma vez sua jogada de esquiva normal, porem vários tiros poderão acerta-la. 

Vamos a um exemplo geral: Biggs tem uma arma com cadência de tiro 4 e um repuxo de -1. Ele irá atacar com os 3 tiros graças a cadência de tiro da arma. O primeiro ataque será com sua destreza normal, o segundo com -1 e o terceiro com -2. Wedge tenta pular para se esquivar, porém apenas consegue um resultado maior do que o terceiro tiro, ou seja, dois tiros o acertam com dano normal. Vale lembrar que ele poderá jogar sua defesa para os dois danos dos tiros recebidos; uma jogada de defesa para cada tiro.

\subsection{Pontos de resitência de armas de fogo}

Observações sobre pontos de resistência em relação as armas de fogo.
\begin{itemize}
	\item Armas de fogo se desgastam a medida que são usadas, principalmente quando usadas em modo automático. Porém a perda de PF de uma arma de fogo é bem menor em comparação com outras armas e armaduras. 

	\item Ao contrário das armas normais, uma arma de fogo com baixo PR não tem redução em seus atributos, e sim uma chance de falha de funcionamento. Essa chance é de 10\% para valores de PR abaixo de 30\% do total, e de 50\% para valores de PR abaixo de 10\% do total. 

	\item Para recuperar os PR de uma arma de fogo, é necessário a habilidade manutenção de armas de fogo, e não forjar como na maioria das armas e armaduras normais.

	\item Geralmente armas de fogo básicas têm valor inicial de 30 PR. 
\end{itemize}

\subsection{Balas}

O dano total de uma arma de fogo é dividido em duas partes. O dano do impacto da arma e o dano da bala. Ou seja, se uma arma de fogo tem dano 6+4 e uma bala tem dano +6, o dano total será 6+10 para cada tiro. Uma arma de fogo geralmente é feita para ser compatível com apenas um tipo de munição, porém existem casos de armas de fogos que são feitas para dispararem diversas balas diferentes de acordo com seu modo de operação. Vale notar que, ao contrário de uma flecha que talvez possa ser reaproveitada após seu uso, sempre que uma bala é usada ela é descartada.


\subsection{Efeito mangueira}

Algumas armas têm a capacidade disparar todo o pente a elas equipado, dando um poder de ataque em área maior. Essa capacidade é chamada de efeito mangueira. A arma atinge todos os alvos dentro do campo do efeito mangueira, que é por volta de 45 a 60 graus partindo do atacante. Esse ataque é igual para todos os alvos, e tem um bônus de acordo com o bônus de efeito mangueira da arma. Os alvos atingidos por um ataque usando o efeito mangueira recebem uma quantidade de tiros igual a cadência da arma - 1. Além de perder todas as balas de um pente, a arma perde 10 PR sempre que o efeito mangueira for utilizado.

Para realizar o efeito mangueira existem algumas limitações, sitadas a seguir:

\begin{itemize}
	\item O alvo deve gastar todo o turno de ataque para realizar o efeito  mangueira;

	\item Algumas armas exigem uma força mínima para realizarem o efeito mangueira. Essa força mínima pode ser reduzida com a ajuda de suportes;

	\item Uma arma só pode realizar efeito mangueira se tiver munição suficiente dentro do seu pente;

	\item Mesmo que vários pentes possam ser acoplados a uma arma, ela deve esperar aproximadamente 2 turnos para resfriar e poder ser usada novamente.

\end{itemize}


\section{Limitação de Itens Mágicos}

Alguns equipamentos normais são melhorados graças a poderes sobrenaturais. Esses itens únicos e poderosos são conhecidos como itens mágicos ou artefatos. Uma espada artefato tem poderes especiais quando comparado a uma espada normal. Mas o uso desse tipo de item tem suas limitações. A maioria dos artefatos permite que um certo limite de outros artefatos sejam utilizados simultaneamente. Em outras palavras, se o limite de um item mágico é 2, então o usuário desse item mágico pode equipar no máximo outros 2 itens mágicos. Se um item for mágico, ele deve ser identificado como tal, e também ter sua limitação exposta.

Com conhecimento de magia um personagem pode "desligar/ligar" um artefato mágico gastando cerca de 5 minutos. Dessa forma ele pode carregar vários artefatos, porém usar somente os que desejar. 


Aqui mostraremos o padrão usado para criação e uso dos equipamentos. Tudo aqui mostrado é uma referência, podendo ser alterado pelo mestre.

\section{Dinheiro}
Segue a base do sistema monetário usado no mundo de ederu.
\begin{itemize}
	\item 1 ponto de ouro equivale a 10 ponto de jade.
	\item 1 ponto de jade equivale a 10 pontos de prata.
	\item 1 ponto de jade equivale a aproximadamente 10 dolares.
	\item 1 ponto de jade é poder monetário suficiente para dormir 1 noite em uma estalagem, com comida.
	\item Com 1 ponto de jade pode-se comprar 1 provisão que alimente por 1 dia durante uma exploração longa.	
	\item Ouro e jade são usados em qualquer parte do mundo.
	%\item Ouro e jade são normais em montanhas, ou seja, tem custo padrão.
	%\item Jade tem um valor maior em áreas costeiras.
	%\item Ouro tem um valor maior em florestas e cidades grandes.
	%\item Jade não tem muito valor em desertos, mas ouro tem muito valor.
	%\item Ouro nao tem muito valor em areas geladas, mas jade sim.
	\item Todo o dinheiro aceito dentro de jogo tem uma marcação realizada únicamente por algum governo local, dificultando ou impedindo a falsificação do dinheiro.
		\item Apenas humanos e gorions tem moedas proprias.
	
\end{itemize}

\section{Criando um equipamento}


A estrutura básica de criação de um equipamento é semelhante a criação de um personagem. Da mesma forma que para criar um personagem, o jogador escolhe uma raça e classe, para criar um equipamento, o mestre escolhe o seu tipo (escudo, espada média, armadura leve, etc). O tipo da arma vai determinar algumas limitações. Armas tem seu PR inicial igual a 60, armaduras Pr inicial igual a 80 e escudos e escudos têm seu PR inicial igual a 30. O valor máximo de PR que uma arma pode alcançar varia de acordo com a natureza da mesma.

Após escolher o tipo da arma, o mestre deve distribuir os pontos da mesma, semelhante a como um jogador distribui os pontos nos atributos de seu personagem, seguindo as regras mostradas a seguir. Para as informações abaixo leve em consideração que a pontuação base de um equipamento mundano é igual a 10. Vale notar que nem todo equipamento deve começar com 10 pontos, essa pontuação é a uma base. Uma faca simples de um ladrão pode ser construida com 5 pontos, enquanto que uma espada longa feita de um material raro pode ser construida com 14 pontos, por exemplo. Uma arma que use a pontuação máxima permitida, já considerando todos as restrições que concedem bônus, é chamada de obra prima.

\begin{enumerate}
	\item Cada ponto extra concedido em aparar ou acerto deve ser descontado da pontuação total da arma.
	\item Da mesma forma, cada ponto extra retirado em destreza ou esquiva deve ser acrescentado na pontuação total da arma.
	\item Cada ponto de dano/defesa do tipo automático do equipamento, deve ser descontado diretamente da pontuação total da arma.
	\item Cada 3 pontos de dano/defesa do tipo normal do equipamento, devem retirar 2 pontos da pontuação total da arma.
	\item Equipamentos com consumo médio de PF devem acrescentar 2 pontos na pontuação total da arma.
	\item Equipamentos com consumo grande de PF devem acrescentar 3 pontos na pontuação total da arma.
	\item Cada 10 pontos de PR retirados/acrescentados do limite do equipamento, acrescentam/retiram 1 ponto de sua pontuação total.
	\item Para armaduras, cada 2 pontos de cobertura devem reduzir 1 ponto da pontuação total da arma.
	\item Algumas armas têm restrições de atributo para serem usadas. Armas com restrição de atributo igual a 8 ou 9, recebem 1 ponto em sua pontuação total. Restrições de atributo a 10, acrescentam 2 pontos na pontuação total da arma.
	\item Outras restrições (usar a arma com duas mãos, locais aberto, etc) podem concedem de 1 até 4 pontos. Essa quantidade será definida pelo mestre segundo a severidade da restrição. 
	\item Restrições de material (que material a arma é feita) e também de manufatura podem concender bônus definidos pelo mestre. Por exemplo, a manufatura de armas de fogo é bastante restrita, o que concede a armas de fogo um bônus considerável em sua pontuação inicial
	\item Poderes e vantagens extras da arma podem reduzir pontos da pontuação total da arma. Por exemplo, uma arma de fácil criação ou com baixo consumo de PF pode ter sua pontuação máxima reduzida severamente.
\end{enumerate}

%O que "pontuação total máxima" quer dizer? Isso representa a pontuação usada para criar uma determinada obra prima (equipamento no limite mundano). Ou seja, se de acordo com os pontos acima, uma determinada arma fica com 14 pontos para serem usados em sua criação, quer dizer que se todos os 14 pontos forem usados nos atributos da arma (dano, acerto, Pr, etc), então aquela arma criada é considerada uma obra prima. Em poucos momentos da campanha os personagens vão se deparar com armas desse tipo. Essa medida acima é usada para orientar o mestre no momento de criar qualquer equipamento, seja ele obra prima ou não.

%Armas de ataque a distância (arco e flecha, bestas, armas de fogo, etc) seguem as mesmas regras de criação de armas, porém de sua pontuação total é descontado/acrescentado alguns fatores relacionados a natureza da arma. Pelo fato delas podem ser usadas a distância do oponente, isso concede uma grande vantagem na maioria das situações. Por isso são descontados de 2 a 4 pontos da pontuação da arma de ataque a distância no momento de sua criação. Essa redução varia de acordo com o alcance efetivo da arma e velocidade de uso, ou seja, a maioria das armas de fogo são criadas com 6 pontos como base. Porém o fator restritivo "construção" pode ser usado para conceder mais pontos a uma arma de ataque a distância. Por exemplo, uma besta ou uma arma de fogo são de dificil acesso para a maioria das pessoas, pois sua construção é dificil, exigindo uma tecnologia maior. Devido a esse fator, o mestre pode acrescentar de 2 a 6 pontos na arma de acordo com a tecnologia usada para sua criação. Outro fator restritivo importante é que uma arma de ataque a distância requer munição para ser usada. Devido a essa restrição o mestre pode acrescentar de até 4 pontos de acordo com a raridade da munição da arma. Essa exceção é mais usada para armas de fogo. Como esse tipo de arma tem um funcionamento diferente da maioria das armas, mais a frente é dedicado uma sub-sessão voltado unicamente para sua explicação. 

Armas de ataque a distância (arco e flecha, bestas, armas de fogo, etc) seguem as mesmas regras de criação de armas, porém de sua pontuação total é descontado/acrescentado alguns fatores relacionados a natureza da arma. Existem pontos positivos (distância de ataque), e pontos negativos (manufatura complexa) que colocam as armas de fogo em uma situação diferente das demais armas. Arcos são criados usando valores máximos de pontuação variando entre 8 e 12, enquanto que armas de fogo usando valores máximos entre 12 e 16. A criação de projéteis deve ser realizada usando menos pontos (6 pontos). Outro detalhe é se o projétil é descartável ou não, ou seja, um projétil não tem PR e sim uma certa percentagem dele se quebrar ou não após seu uso. A maioria das flechas poder ser utilizada caso o defensor não tenha tido um teste muito grande de defesa, o que não acontece em balas, onde elas são descartadas depois de seu uso. Geralmente projéteis reaproveitáveis concedem menos pontos do que os descartáveis.

Armas de arremesso indireto, ou seja, que o dano não é baseado na força a qual os projeta (granadas por exemplo), devem ter seus valores baseados na tecnologia que a constroi.

Alguns equipamentos podem conceder bônus auxiliares, sem contar o dano ou defesa do equipamento. Por exemplo, certas armas concedem bônus no acerto ou para o teste de aparar. Em armaduras esse bônus auxiliar é manifestado na forma de um incremento no Pr normal. Geralmente armas com um valor grande de dano automático tem um valor baixo de PR ou não concedem bônus em aparar ao usuário. Um machado com dano /8 pode conceder +2no teste de aparar e ter 80 PR.

Na hora de criar um equipamento o mestre deve levar em consideração bônus, restrições e a natureza da arma (não seria lógico um porrete ter dano automático grande). Mais a frente mostraremos uma lista de alguns exemplos de equipamentos criados com esses parametros para ajudar o mestre a criar seus proprios equipamentos.


Os equipamentos de ataque/defesa mundanos normalmente têm seu dano/absorção variando de 2 até 10. Quanto maior for o bônus automático do equipamento, mais avançada ele é. Ou seja, uma espada +8 é considerada uma arma de boa qualidade, porém uma espada +12 pode ser considerada uma obra prima de um artesão. Uma espada média com dano 10 + 10 não é considerada mundana e sim uma arma mágica, pois ela ultrapassa os limites de uma arma mundana. É considerado equipamento mágico qualquer equipamento que se encaixe em um dos fatores descritos a seguir:

\begin{itemize}
	\item Qualquer arma ou armadura que conceda bônus relacionados a fatores não diretamente relacionados com o equipamento; 
	\item Qualquer arma ou armadura que ultrapasse excessivamente os valores analisados para equipamentos mundanos;
	\item Qualquer equipamento auxiliar que conceda bônus (um anel que conceda bônus em força por exemplo).
\end{itemize}

Um anel mágico que conceda bônus em sabedoria ou dano mágico, uma armadura que conceda bônus em PV ou PF, ou uma bota que conceda bônus em esquiva e destreza são exemplos de iténs mágicos. Qualquer equipamento que esteja no seu limite de melhoria é chamado de obra prima.
	
Para determinar se uma arma ou armadura ultrapassou os valores mundanos o mestre deve analisar todos os atributos do equipamento, não somente o dano ou sua defesa. Por exemplo, uma espada média com dano 6+6, +2 acerto, +2 aparar e 40 PR, não é considerada arma mágica. Somando os bônus de dado e acerto temos 14 pontos. Porém considerando a retirada de 20 PR, concede a arma + 2 pontos em sua pontuação. Finalmente, devido a ela ser uma arma de consumo mediano de PF, ela recebe + 2, totalizando 14 pontos a ser distribuido. Como o bônus automático não é tão grande, o mestre nessa situação considera essa espada média como no limite de um equipamento mundano. Qualquer alteração nessa espada deve ser feita por um ferreiro especial, para transforma-la em arma mágica. O mestre também pode considerar alguns fatores negativos para equilibrar a arma fazendo com que ela seja considerada mundana, como exigir um valor mínimo de atributo para usa-lá, ou restringir para que ela só possa ser usada com 2 mãos. Por exemplo, o bastão longo e pesado conhecido como dai bo. O dai bo tem dano /12, PR 80, +2 acerto. Totalizando seus bônus temos ao todo 12 pontos distribuidos na arma. Levando em consideração que todo o seu dano é do tipo normal (mais fácil de se reduzir com defesa) e que é uma arma que deve ser usada com duas mãos, consideramos essa arma como mundana. Na verdade podemos ainda melhora-la em alguns pontos para que ela fique em seu limite de melhoramento, tornando essa arma uma obra prima. 

O mestre deve levar em conta todos esses fatores (tipo do dano, bônus extras, penalidades, natureza da arma, restrições de uso, etc) para criar uma arma ou limitar seu melhoramento. Lembre-se que da mesma forma que pontos negativos no uso de uma arma concedem bônus para a melhoria da mesma, o contrario pode ocorrer, como por exemplo, uma arma com baixo consumo de PF que pode ser facilmente escondida não terá tanto dano quanto um machado de guerra que só pode ser usado com duas mãos.	


Na hora de criar um equipamento o mestre deve ter conhecimento sobre a arma criada, e distribuir seus pontos e determinar seus limites de acordo com a natureza da arma. Por exemplo, não tem muita lógica criar uma espada curta com alto PR ou alto dano, já que elas são armas de porte pequeno e com pouco consumo de PF. A mesma linha de pensamento deve ser usada para a melhoria do equipamento, aonde o mestre pode atribuir redução no custo básico para aumentar certos atributos, por exemplo, fazer com que uma faca possa aumentar seu dano automático de forma mais rápida do que uma espada grande, esta que teria seu dano normal aumentado mais rápido. O mestre tem liberdade para alterar os fatores gerais de melhoria e criação de uma arma de acordo com a natureza da mesma. Os dados mostrados aqui são apenas para orientar o mestre. O erro de calculo de poucos pontos não torna o jogo desbalanceado.


\section{Melhorando e Criando Equipamentos}
Pode-se melhorar o dano ou defesa de um equipamento obedecendo a seguinte regra.

\begin{itemize}
	\item O personagem deve gastar 6 de jade para aumentar o dano/defesa em 1 ponto do tipo normal.
	\item O personagem deve gastar 8 de jade para converter 1 ponto do tipo normal em 1 ponto do tipo automático.
	\item O personagem deve gastar 4 de jade para aumentar o PR máximo da arma em 5 pontos.
	\item O personagem deve gastar 1 PO para inserir um bônus auxilar na arma (acerto por exemplo).
	\item O personagem deve gastar 1 ponto de jade para recuperar 3 Pr perdidos. %1 pr 4 pj magico
	\item O personagem deve gastar 4 ponto de jade para recuperar 1 Pr perdido de uma arma mágica. %1 pr 4 pj magico
	\item Uma arma mágica pode ser melhorada gastando os mesmos pontos de uma arma normal, porém deve-se usar material mágico como materia prima conseguindo dentro de jogo.
	\item Para aumentar a cobertura de uma armadura o personagem deve gastar uma quantidade de pontos de jade igual a 4 vezes o  nível desejado.
	\item Cada 2 pontos de jade gastos no teste de forjar/gunsmith reduz em 1 ponto a dificuldade do mesmo. Essa redução não pode ser maior do que o valor original do habilidade geral forjar/gunsmith.	
\end{itemize}

Caso o personagem tenha a habilidade geral forjar e equipamentos de forja, ele pode criar/melhorar uma arma fazendo um teste de destreza ou concentração + forjar. A dificuldade desse teste é igual a 12 + dano/defesa do tipo normal desejado + dobro do valor de dano/defesa do tipo automático desejado + eventuais bônus auxiliares desejados (acerto por exemplo). Por exemplo, se um personagem deseja criar uma arma 2+2, ele deve fazer um teste de forjar com dificuldade 18. Para melhorar essa mesma arma para 2+4, ele deve realizar outro teste com dificuldade 22. Cada teste dura mais ou menos 3 horas. Alguns materiais especiais podem ser usados para diminuir a dificuldade desse teste. Por exemplo, um pouco de mythril pode diminuir em ate 4 a dificuldade de um teste.

Para criar/melhorar um equipamento o personagem deve ter ferramentas. Além disso ele deve ter uma quantidade inicial de material de forja para construir cada equipamento. Em termos de material usado, o personagem gasta 1 PJ para cada ponto desejado de dano/defesa do tipo normal, e o dobro para cada ponto do tipo automático. Esse valores são usados para equipamentos com dano/defesa abaixo de 6 pontos. Acima disso, o mestre deve usar os mesmos valores usados na melhora de equipamento. Por exemplo, o personagem deseja fazer uma espada  média 3 + 3. Ele deve gastar 7 de jade para fazer uma espada 3 + 2 (totalizando até o momento 5 pontos limite inicial). Após isso ele deve gastar  mais 6 PJ para adicionar /1 de dano, e em seguida converter esse dano normal em automático, gastando 8 PJ nesse último passo. Totalizando 21 PJ no processo todo, que é aproximadamente 2 PO. Analisando todo o processo de criação vemos que os testes de criação e melhora devem ser feitos separadamente. Em relação a melhora ou recuperação de PR nenhum teste é exigido. Além diso, não é necessário nenhum teste por parte do ferreiro para inserir cobertura da armadura.


Para balas/flechas o processo é o mesmo, porém, o bônus do sucesso obtido determina a quantidade de balas/flechas criadas, sendo o valor máximo de 4 por teste. O personagem gasta 1 PJ por teste para balas e a mesma quantia para 5 flechas. Para testes relacionados com balas o personagem deve usar gunsmith, enquanto que para flechas o personagem pode usar tanto forjar ou oficios. Além disso projéteis dificilmente podem ser melhorados após a sua criação. O que ocorre é do ferreiro desmontar o projétil e refazer ele do zero, de uma forma melhor. Quando isso ocorre o personagem pode receber um certo valor em material para ser usado na redução da dificuldade do teste.

%Essa regra citada acima é usada para criar armas normais de forma rápida, que dura no máximo algumas horas de trabalho pesado. Em alguns casos especiais, o processo de criação de uma arma pode levar meses, fazendo com que o processo de criação seja mais complexo, envolvendo várias fases (preparação do material, melhora da base do equipamento, etc). Para tais casos o mestre, para termos de sistema, pode considerar todo o processo de criação como um processo de criação simples seguido de vários processos de melhoria do equipamento.


O melhorando de um equipamento não pode ultrapassar o valor básico de um equipamento mundando de acordo com sua natureza, ou seja, se um equipamento já distribuiu todos os pontos disponíveis a ele, o máximo que o artesão pode fazer é criar penalidades para aumentar a pontuação da arma ou de alguma forma redistribuir os pontos do equipamento.

A maioria dos ferreiros encontrados dentro de jogo têm forjar/gunsmith e atributos usados nos testes (destreza e/ou concentração) variando entre 6 a 8, ou seja, os valores usados para testes varia entre de 12 até 16. Além disso, boa parte dos ferreiros usa equipamentos que concedem bônus nos testes. De modo geral, testes com dificuldade variando entre 20 até 25 podem ser realizados sem problemas por ferreiros comuns. Qualquer valor acima disso, deve ser cobrado um preço adicional, que é de mão de obra mais pontos de jade adicionais usados na redução do teste.

O mestre deve levar em consideração que esses preço são apenas valores aproximados. Alguns ferreiros podem cobrar mais ou menos de acordo com a situação. Por exemplo, um ferreiro pode alugar gratuitamente seus equipamentos e sua forja para um grupo, se esse grupo prender um grupo de ladrões que o atormenta há algumas semanas. O mestre tem total liberdade para manipular tais fatores. 


\section{Material de Forja/Alquimia/Herbalismo/Gunsmith}


Material de Forja/Alquimia/Herbalismo/Gunsmith é uma espécie de moeda de cambio, porém que só pode ser usada para Forja/Alquimia/Herbalismo/Gunsmith. A relação é que 1 ponto de jade equivale a 1 ponto de material. Por exemplo, no lugar do personagem ir para um ferreiro e gastar 3 pontos de jade para recuperar 1 pr de sua arma, ele pode recuperar a mesma quantia usando 3 pontos de Material para Forja. Esse material pode ser obtido ao longo da aventura ou até mesmo comprado. Em certos locais abundantes desse material (minas, montanhas, etc), ele pode ser comprado a um valor muito baixo (1 ponto de jade valendo até 10 pontos de material para forja por exemplo).

Ao longo da aventura o mestre também pode (e DEVE) criar materiais especiais para serem usados pelos personagens com o mesmo objetivo do material para forja. Por exemplo, ele pode criar material para gunsmith, material para herbalismo, madeira especial (abaixa o custo de fazer flechas por exemplo), entre outros. Para facilitar o uso quantitativo, todo e qualquer material do mestre deve ser visto como uma moeda usada especificamente para aquele fim. Alguns materias podem valer mais do que outros, por exemplo, cada ponto de uma madeira especial (material criado pelo mestre) pode valer 1 po quando se tratando de compra de materiais feitos com madeira. Para determinados equipamentos (uma espada próxima de se tornar uma obra prima), o mestre pode incluir na lista de materiais necessários alguns materiais raros (jade vermelho do deserto de lorén por exemplo). Isso é criado com o objetivo de valorizar as habilidades dos personagens e também de incentivar a exploração de materiais, seja eles raros ou não.

\section{Poções}
Um personagem pode usar poções mágicas para PV, PF, PM, ou para atacar inimigos. As poções de cura de efeito imediato emitem uma espécie de radiação mágica e são consideradas poções mágicas. Cada personagem tem um limite de quantas poções mágicas pode carregar, caso contrário essa radiação mágica pode causar efeitos não desejáveis (envenenamento do personagem ou até explosão das poções). Levando isso em consideração, a quantidade máxima de poções mágicas que um personagem pode carregar é de 4. A baixo segue uma referência para poções de cura de efeitos imediatas ou não, além de algumas poções de ataque.

As poções podem ser criadas com as mesmas parecidas da criação de equipamento, onde o usuário deve realizar um teste com dificuldade igual a 12 + cura/dano da poção (dobre esse último valor caso a poção de cura tenha efeito automático). Cada ponto de jade/material gasto ao criar a poção vale 2 pontos. Por exemplo, para criar uma poção de cura de 8 PV de efeito imediato, o personagem deve realizar um teste com dificuldade 26, gastando 4 pontos de jade/material. Ele também pode usar 2 pontos de jade/material para diminuir a dificuldade do test em 1 até um valor igual ao nível mental do personagem.

\begin{itemize}

	\item Potion 10 pv, Efeito após 5 minutos, Custo de 4 de jade.

	\item Potion 20 pv, Efeito após 5 minutos, Custo de 1 PO.

	\item Potion 10 pv, Efeito imediato, Custo de 1 PO.

	\item Potion 20 pv, Efeito imediato, Custo de 3 PO.

	\item Ginseng 10 pf, Efeito após 5 minutos, Custo de 1 PO.

	\item Ginseng 10 pf, Efeito imediato, Custo de 3 PO.

	\item Ether 10 pm, Efeito imediato, Custo de 1 PO.

	\item Ether 18 pm, Efeito imediato, Custo de 2 PO.

	\item Poção de fogo, +16 de dano físico de fogo em uma área de 3 metros, Custo de 6 a 12 de jade.

	\item Granada Incendiária, /8+18 de dano físico de fogo em uma área de 5 metros, Custo de 2 a 4 PO.

	\item Granada de Projéteis /20+10 de dano físico em uma área de 5 metros, Custo de 2 a 4 PO.

	\item Bomba Venenosa, Os alvos dentro de uma área de 4 metros devem fazer teste de resistência df de 20 a 25 para não ficar envenenados (veneno 2). Custo de 1 a 2 PO.

		\item Bomba Paralisadora, 1 alvo , resistencia 25 , se passar -2dex-2esquiva(resi20 ou mais nao pega). 1 ou 2 PO.

\end{itemize}

\section{Equipamento Mundano}
Aqui mostraremos os dados e informações básicas de alguns equipamentos que podem ser usadas no sistema. A listagem aqui feita é baseada na lista de péricias bélicas. Vale lembrar que as armas aqui mostradas não são armas mágicas.

\subsection{Espada Curta}
Armas leves de fácil uso. Consomem pouco PF quando utilizadas, por isso é a melhor arma para lutadores que usem vários ataques. Além disso podem ser usadas como armas de arremesso. Apesas do dano relativamente pequeno, algumas concedem bônus no acerto devido a sua leveza. Não podem ser usadas para aparar armas grandes. 

\begin{itemize}
	\item Dagger = Dano +2, PR 30, Custo 2 PJ (ponto de jade).
	\item Peixeira Ginzu = Dano 2+2, PR 40, Custo 1 PO.
	\item Silver Dagger = Dano +4, +2 acerto, PR 40, Custo 5 PO.
	\item Katara = 3+3, +3 acerto, PR 40. 6 PO.
	\item Faca Élfica = 4+4, PR 50. 7 PO.
	\item Kris Malai = Dano 2+3, +1 acerto, PR 45, Custo 3 PO.
	\item Adaga de Assassino = Dano +3, +2 acerto, PR 35, Custo 4 PO.
\end{itemize}

\subsection{Espada Média}
Armas extramamente versáteis, podendo ser usadas para defesa ou ataque. Consomem uma quantidade normal de PF quando usada em batalha (1 pf é consumido a cada 3 ataques aproximadamente).

\begin{itemize}
	\item Espada Média = Dano 3+3, PR 40, Custo 2 PO.
	\item Espada Romana = Dano 5+2, PR 60, Custo 3 PO.
	\item Espada de Prata = Dano 4+4, PR 50, Custo 5 PO.
	\item Katana = Dano 1+5, +2 Acerto, PR 40, Custo 4 PO.
	\item Sabre de Cavalaria = Dano 4+3, +1 acerto, PR 55, Custo 4 PO.
	\item Espada Bastarda = Dano 3+4, +1 aparar, PR 45, Custo 3 PO.
\end{itemize}

\subsection{Espada Longa}

Armas pesadas que podem ser usadas apenas com duas mãos. O seu peso dificulta que outros oponentes usem a manobra aparar quando atacados por espadas longas. Consomem mais PF que as espadas médias e podem ser usadas para atacar oponentes em área. Recebem uma penalidade ou não podem ser usadas para aparar armas de pequeno porte quando o inimigo encontra-se muito próximo.

\begin{itemize}
	\item Espada Grande = Dano 4+4, PR 80, Custo 4 PO, Min força 6 com 2 mãos.
	\item Katana Grande = Dano 6+4, PR 60, Custo 5 PO, Min força 6 com 2 mãos.
	\item Buster Sword = Dano 8+4, -1 Destreza e Esquiva, PR 90, Custo 8 PO. Requer força 8 ou mais para usar essa arma com 2 mãos.
	\item Claymore Escocesa = Dano 5+5, +1 aparar, PR 85, Custo 6 PO, Min força 7 com 2 mãos.
	\item Zweihänder = Dano 7+3, +2 aparar, PR 95, Custo 7 PO, Min força 8 com 2 mãos.
\end{itemize}

\subsection{Lança}
A péricia bélica Lança concede ao personagem usar qualquer tipo de lança, seja ela média ou longa.

Lanças médias são armas boas pois podem receber bônus no acerto devido ao seu tamanho. Geralmente não PR elevados e são usadas em conjunto com outra arma ou escudo. Consomem a mesma quantidade de Pf que outras armas de porte médio. Devido a sua natureza de perfuração tem um bom dano automático.

\begin{itemize}

	\item Javelin De Mogno = Dano +3, PR 30, +1 acerto, Custo 6 de Jade.
	\item Javelin = Dano 1+4, PR 30, +2 acerto, Custo 3 PO.
	\item Javelin de Prata = Dano 2+7, +4 acerto e 50 PR, Custo 12 PO.
	\item Tridente = Dano 2+6, +2 aparar, PR 40, Custo 7 PO.
	\item Lança de Caça = Dano 1+5, +1 acerto, PR 35, Custo 4 PO.
	\item Pilum Romano = Dano 2+3, +1 aparar, PR 45, Custo 5 PO.
	
\end{itemize}

Lanças longas são armas usadas com as duas mãos, podendo ser usadas para o ataque ou para defesa, devido ao seu tamanho prolongado que pode ser usado para impedir a aproximação de oponentes. Têm um bom dano, porém baixo PR. Consomem mais PF que o normal. Recebem uma penalidade ou não podem ser usadas para aparar armas de pequeno porte quando o inimigo encontra-se muito próximo.

\begin{itemize}	
	\item Bambu Yari = Dano +5, PR 30, +2 acerto, -2 esquiva, Custo 2 PO, Min força 5 com 2 mãos.
	\item Alabarda = Dano 4+4, PR 50, +4 aparar, +2 acerto, -3 esquiva, Custo 8 PO, Min força 7 com 2 mãos.
	\item Naginata = Dano 1+7, PR 40, +2 aparar, +4 acerto, -2 esquiva, Custo 8 PO , Min força 6 com 2 mãos.
	\item Lança de Cavalaria = Dano 3+6, +3 aparar, PR 45, Custo 6 PO, Min força 6 com 2 mãos.
	\item Guisarme = Dano 2+8, +3 aparar, +1 acerto, PR 55, Custo 7 PO, Min força 7 com 2 mãos.
\end{itemize}	


\subsection{Bastão}
O bastão médio assemelhasse muito a lança média. Tem um dano maior, porém do tipo normal. Consome a mesma quantidade de PF que uma arma de porte médio.

\begin{itemize}	
	\item Jo = Dano /6, +4 aparar (2 mãos), +2 aparar(1 mão), +1 acerto,  PR 70, Custo 3 PO.
	\item Jo de Bambu = Dano /4, +2 acerto, +2 aparar(2 mãos)  PR 60, Custo 2 de jade.
	\item Bastão de Carvalho = Dano /5, +3 aparar, PR 65, Custo 2 PO.
	\item Cajado de Peregrino = Dano /3, +1 acerto, +1 aparar, PR 55, Custo 1 PO.
\end{itemize}	

O bastão longo consome mais PF que o normal, assim como as espadas longas, e tem um dano do tipo normal maior. Também tem grande quantidade de PF, mas devido ao seu peso e tamanho podem causar redutores na agilidade do usuário. Devem ser usadas com ambas as mãos.

\begin{itemize}	
	\item Bo = Dano /8, PR 70, +1 acerto, Custo 2 PO, Min força 6 com 2 mãos.	
	\item Daibo = Dano /12, PR 80, +2 acerto, -2 esquiva, Custo 5 PO, Min força 7 com 2 mãos.	
	\item Bastão de Ferro = Dano /10, +2 aparar, PR 85, Custo 4 PO, Min força 7 com 2 mãos.
	\item Quarterstaff = Dano /9, +1 aparar, +1 acerto, PR 75, Custo 3 PO, Min força 6 com 2 mãos.
\end{itemize}	



\subsection{Esmagador}
Um esmagador simples usa como principio a capacidade da arma em esmagar objetos. Têm alto dano normal, são facilmente encontradas, e tem uma boa quantidade de PR. O consumo de PF varia de acordo com o tamanho do esmagador, pois uma maior força será usada para esmagadores simples maiores. Provavelmente é o tipo de arma mais usado desde a antiguidade, levando em consideração que qualquer pedaço de pau pode ser usasdo como um esmagador. Alguns tipos de esmagadores, conhecidos como esmagadores compostos, são versões melhoradas de um esmagador simples (um machado que é uma versao melhorada de um martelo por exemplo). Um mangual é considerado um esmagador composto.


\begin{itemize}	
	\item Tonfa = Dano /3, PR 80, +3 aparar. Custo 3 de jade.
	\item Porrete = Dano /5, PR 80, +2 aparar, +2 agarrar. Custo 6 de jade.		
	\item Takapi = Dano /4+1, PR 60, Custo 6 de jade.
	\item Maça = Dano /6, PR 90, Cada ataque bem	sucedido retira 5 PR mais bônus de força da armadura/escudo. Custo 3 PO.
	\item Kanabo = Dano /6+1, PR 80, Custo 2 PO.	
	\item Clava de Guerra = Dano /4, PR 75, +1 aparar, Custo 4 de jade.
	\item Mangual = Dano /5+1, PR 70, +1 agarrar, Custo 5 de jade.		
\end{itemize}	


\subsection{Esmagador Grande}

\begin{itemize}
	\item Martelo de Guerra = Dano /8, +2 Aparar, PR100, Min força 6 com 2 mãos, -1 esquiva.	Custo 4 PO.
	\item Kanabo Grande = Dano /10+2, PR 90, Min força 7 com 2 mãos, -2 esquiva, Custo 4 PO.	
	\item Goremaul de Jade Reforçado (obra-prima) = Dano /14+2, +2 aparar, -2 Agilidade, Min força 8 com 2 mãos, Cada ataque bem sucedido retira 6 PR mais bônus de força da armadura/escudo, PR 100, Custo 18 PO.
	\item Maça de Duas Mãos = Dano /9, +1 aparar, PR 95, Min força 6 com 2 mãos, Custo 5 PO.
	\item Martelo de Thor = Dano /12, +3 aparar, PR 110, Min força 8 com 2 mãos, -2 esquiva, Custo 8 PO.
\end{itemize}	

\subsection{Arma de Fogo Leve}
Armas de fogo leve estão entre as armas mais usadas no mundo. Relativamente baratas em comparação com as outras armas, podem causar um grande dano se usado por um expert. Geralmente são armas não-autômatica e de cadência de tiro baixa, mas devido a sua leveza podem ser usadas em ambas as mãos. Não podem ser usadas para aparar armas. Seu alcance atinge até 50 metros.

\begin{itemize}
	\item Bereta = Dano 8, Cadência de Tiro 1, PR 20, Custo 8 PO.
	\item M1911 = Dano 8, Acerto +2, Cadência de Tiro 2 manual, PR 20, Custo 10 PO.
	\item Deseart Eagle Dano = 11, Acerto -2, PR 25, Custo 11 PO.
	\item Glock 17 = Dano 7, +1 acerto, Cadência de Tiro 2 manual, PR 20, Custo 9 PO.
	\item Revolver .357 = Dano 9, +1 acerto, Cadência de Tiro 1, PR 25, Custo 10 PO.
\end{itemize}

A maioria das armas leves aceita balas de calibre 2+2 (custo de 8 PJ por caixa), +4(custo de 2 PO por caixa), 6+2 (custo de 3 PO por caixa). O calibre da bala especifica o dano causado pela mesma. Cada caixa vêm com geralmente 6 balas. 

\subsection{Arma de Fogo Média}
Mais caras do que as armas de fogo, têm um poder de destruição maior, seja pela poder do calibre ou pela cadência de tiro elevada. A maioria das armas de fogo média devem ser usadas com ambas as mãos. Algumas requerem um maior nível de gunsmith para que seus PR sejam reparados. O seu alcance pode variar até 500 metros, dependendo do tipo da arma. Vale apenas observar que o dano de muitas armas automáticas não é o dano de 1 so tiro, mas sim de uma sequencia de tiros. Mesmo assim, cada ataque deve ser considerado, para termo de sistema, como 1 tiro separado, assim como o gasto daquela "bala".

\begin{enumerate}
	\item Sniper Rifle = Dano +8, Acerto +4, Alcance entre 20m e 500m, Calibre + 7, Cadência de Tiro 1, Custo de 15 PO, 1 PO por bala, PR30.
	\item Shotgun = Dano 12, Acerto -2, Alcance até 10m e alvos próximos, Calibre +4, Custo de 10 PO, 4 PJ por bala, PR30. 
	\item AK-47 = Dano 8+2, Acerto -2, Cadência de Tiro 3 Auto, Calibre 2+4, Custo 12 PO, Custo de 2 PO por caixa da bala, PR30. 
	\item M16 = Dano 7+3, +1 acerto, Cadência de Tiro 3 Auto, Calibre 2+3, Custo 13 PO, Custo de 2 PO por caixa da bala, PR30.
	\item Winchester = Dano 9, +2 acerto, Cadência de Tiro 1, Alcance até 200m, Custo 11 PO, 3 PJ por bala, PR30.
\end{enumerate}

Vale lembrar que o mostrado aqui para as balas é apenas um padrao. Então não seria surpresa um jogador poder encontrar uma bala com dano 4+6 para a arma AK-47.

\subsection{Arma de Fogo Pesada}
Armas raramente encontradas, tem munição limitada e preço elevado. São extremamente perigosas e de dificil  mauseio.

\begin{enumerate}
	\item RPG-7V = Dano /10, Alcance entre 50m e 100m, Calibre /10+15, Cadência de Tiro 1, Custo de 10 PO, 3 PO por bala, PR30.
	\item Rail Gun = Dano +22, dano elétrico. Acerto +2, Alcance até 100m, Calibre feito de balas especiais, 1 turno para concentrar. Custo de 20 PO. PR50. A rail gun é uma arma que descarrega uma corrente elétrica no alvo. Cada tiro consome um pouco de sua bateria, de forma que a bateria deve ser trocada por uma nova sempre que for descarregada. Em termo de sistema, cada tiro custa 4 PJ.	
	\item Bazuca = Dano /12, Alcance entre 30m e 80m, Calibre /12+18, Cadência de Tiro 1, Custo de 12 PO, 4 PO por bala, PR30.
	\item Metralhadora Pesada = Dano 10+5, Cadência de Tiro 5 Auto, Alcance até 300m, Calibre 3+5, Custo de 18 PO, 3 PO por caixa da bala, PR40.
\end{enumerate}


\subsection{Manopla}
Manopla é um equipamento que pode ser usado para reduzir o dano ou para aparar ataques, servindo como um escudo pequeno. A absorção da manopla não acumulada com outras armaduras (média e pesada) que o personagem esteja usando, podendo em alguns casos aumentar um pouco a absorção da armadura como um todo. A grande vantagen da manopla em relação ao escudo e armadura, é que o seu consumo de PF é muito baixo. Os PR máximos e mínimos de uma manopla são a metade dos considerados normais para uma armadura.

\begin{itemize}
	\item Bracelete Samurai = +2 Aparar, /2 Absorção, PR 60, Custo 4 de jade. Se usado com armadura concede cobertura +1.
	\item Manopla de Espinhos  =  +3 Aparar, PR50.	Custo 5 de jade.
	\item Manopla de Couro = +1 Aparar, /1 Absorção, PR 40, Custo 2 de jade.
	\item Manopla de Ferro = +2 Aparar, /3 Absorção, PR 70, Custo 6 de jade.
\end{itemize}	


\subsection{Armadura Leve}
Qualquer personagem pode usar uma armadura leve sem nenhuma restrição. Geralmente sua área de proteção é restrita a algumas partes do corpo, fazendo assim com que o atributo cobertura desse tipo de armadura não ultrapasse 3. A faixa de PR de uma armadura leve é igual a usada em armas, ou seja, de 30 a 80.

\begin{itemize}
	\item Colete Simples = Absorção /3, PR 40, Custo 2 de jade.
	\item Cota de Couro  =  Absorção /2+2, PR 50, Custo 1 PO.
	\item Kimono Reforçado  =  Absorção /4, PR 70, Custo 1 PO.	
	\item Armadura Romana  =  Absorção /2+4, PR 50, cobertura 2, Custo 7 PO.	
	\item Cota élfica de mythril =  Absorção +8, PR 60, Custo 13 PO.	
	\item Armadura de Couro = Absorção /3+1, PR 45, Custo 3 de jade.
	\item Armadura de Tecido = Absorção /2, PR 35, Custo 1 de jade.		
\end{itemize}	


\subsection{Armadura Média}
Facilmente encontadas, protegem boa parte do corpo. Aqueles que não estão acostumados com esse tipo de equipamento tem dificuldade na locomoção e movimento de membros. A cobertura máxima desse tipo de armadura é 6. A redução máxima em um atributo é igual a 3. Geralmente não apresentam requisito de força para seu uso.

\begin{enumerate}
	\item Armadura de Ferro = Absorção /4+2, PR 80, Custo 3 PO.
	\item Armadura de Mitril = Absorção /2+8, PR 50, cobertura 1. Custo 11 PO.
	\item Armadura Samurai = Absorção /5+5, PR 60, cobertura 1. Custo 6 PO.
	\item Armadura Jade Escuro = Absorção /8+2, PR 100, cobertura 2. Custo 7 PO.
	\item Colete Tatsunamei = Absorção +8, PR 30, Custo 6 PO.		
	\item Armadura Média Obra-Prima Anã = Absorção +12, PR 100, cobertura 4, -2 esquiva. Custo 20 PO.	
	\item Armadura de Aço = Absorção /5+3, PR 85, Custo 4 PO.
	\item Armadura de Bronze = Absorção /3+4, PR 75, Custo 3 PO.		
\end{enumerate}


\subsection{Armadura Pesada}
Armaduras pesadas sempre concedem penalidades no movimento do usuário, porém combrem uma área de proteção geralmente maior que 90\% do corpo do usuário, fazendo assim com que a cobertura natural dessa armadura (sem contar os pontos de distribuição de equipamento) seja igual ao maior redutor em um atributo. Mover-se com esse tipo de equipamento pesado consome uma quantidade considerável de PF. 

\begin{enumerate}
	\item Armadura Completa = Absorção /8+4, PR 100, -3 esquiva, -3 Dex, cobertura extra 3 (6 no total), Min força 6. Custo 8 PO.
	\item Armadura Pesada de Veridiun Vermelho = Absorção /8+8, -4 esquiva, Min força 8, -4 Dex, PR 100. Custo 13 PO.
	\item Colete Tatsunamei Anti Explosão = Absorção +13, -4 esquiva, -2 Dex, PR 40. Custo 13 PO.
	\item Armadura De Couro de Dragão (obra-prima) = Absorção /15+8, Min força 10, -5 esquiva, -2 Dex, PR 120, cobertura extra 2 (7 no total). Custo 18 PO.
	\item Armadura de Placas = Absorção /10+6, PR 110, -4 esquiva, -3 Dex, cobertura extra 2 (5 no total), Min força 7. Custo 10 PO.
	\item Armadura Gótica = Absorção /9+5, PR 105, -3 esquiva, -2 Dex, cobertura extra 1 (4 no total), Min força 6. Custo 9 PO.
\end{enumerate}


\subsection{Foice Pequena}
Essas armas são adaptações de instrumentos usados na agricultura. De fácil criação, podem ser fatais se usadas contra oponentes desprotegidos. Devido ao seu formato curvo, podem ser usadas para aparar algumas armas.

\begin{enumerate}
	\item Foice Simples = Dano 5+1, PR 50, +1 aparar. Custo 8 de jade.
	\item Foice Gorion = Dano 5+3, PR 60, +2 aparar. Custo 5 PO .
	\item Makraka = Dano 8+1, PR 70, +1 aparar. Custo 4 PO.
	\item Foice de Caça = Dano 4+2, PR 45, +1 aparar, Custo 6 de jade.
	\item Foice de Batalha = Dano 6+2, PR 55, +2 aparar, Custo 1 PO.
\end{enumerate}


\subsection{Foice Longa}
As armas da categoria foice longa também são conhecidas como gadanhas. São fatais a distâncias médias, recebendo um bônus no acerto e em aparar devido ao seu formato.

\begin{itemize}	
	\item Gadanha Simples = Dano +5, PR 30, +2 acerto, -2 esquiva, Custo 1 PO, Min força 5 com 2 mãos.
	\item Foice Longa de Cobre Negro = Dano 4+4, PR 50, +4 aparar, +2 acerto, -3 esquiva, Custo 7 PO, Min força 7 com 2 mãos.
	\item Gadanha de Guerra = Dano 3+6, PR 40, +3 aparar, +1 acerto, -2 esquiva, Custo 3 PO, Min força 6 com 2 mãos.
	\item Foice de Duas Mãos = Dano 5+3, PR 45, +3 aparar, +2 acerto, -2 esquiva, Custo 4 PO, Min força 6 com 2 mãos.
\end{itemize}	

\subsection{Escudo}
Escudo são excelentes equipamentos de proteção, uma vez que eles podem aparar ataques com mais facilidade que uma arma comum. Podem ser usados para aparar projéteis. Geralmente concedem bônus no teste de aparar (bônus esse variante de acordo com o tamanho do escudo) e no teste para resistir a força do oponente (bônus esse variante a resistência do escudo).

\begin{enumerate}
	\item Escudo de Madeira = Tamanho (+1 aparar). PR30.  Custo 1 de jade.
	\item Escudo do Guerreiro = Tamanho (+3 aparar), Resitência (+3). PR70.  Custo 4 PO.
	\item Escudo Nobre = Tamanho (+4 aparar), Resitência (+2). PR80.  Custo 5 PO.	
	\item Escudo Pesado Anão = Tamanho (+5 aparar), Resitência (+4), - 2 Esquiva, -1 Destreza. PR100.  Custo 8 PO.
	\item Escudo de Couro = Tamanho (+2 aparar), Resistência (+1). PR40. Custo 2 de jade.
	\item Escudo de Ferro = Tamanho (+3 aparar), Resistência (+2). PR60. Custo 3 PO.
\end{enumerate}



\subsection{Zarabatana}
São armas de simples, leves e facilmente camufladas. Devido a esses fatores seu dano não é tão elevado. A periculosidade dessa arma concentra-se na natureza do projétil usado, esse sendo muitas vezes venenoso. Geralmente atiram apenas um projétil por vez. O que diferencia uma zarabatana da outra é sua quantidade de PR e alguns bônus que podem ser concedidos na destreza para acerto. Como é uma arma de sopro, geralmente a força usada serve apenas para mover o projétil até seu alvo. Abaixo seguem algumas sugestões de projéteis que podem ser usados com a zarabatana. Alguns desses projeteis somente tem efeito caso acertem a pele do alvo, ou seja, o mestre pode incluir uma penalidade no acerto devido a cobertura da armadura que o alvo esteja usando.


\begin{enumerate}

	\item Dardo Venenoso = Causa Veneno (2) se o alvo não passar em um teste de resistência DF 15. 3 por 2 pontos de jade.	 
	\item Dardo Tapu = Causa Veneno (3) se o alvo não passar em um teste de resistência DF 20. 4 Pontos de Jade. 
	\item Dardo Enfraquecedor = O alvo recebe um redutor na força de 4 se não passar em um teste de resistência DF 15. 3 Pontos de jade.
	\item Dardo Ácido = Dano +5. Ignora armadura. 2 pontos de jade por dardo.
	\item Dardo Sonífero = O alvo deve fazer teste de resistência DF 18 ou ficar inconsciente por 1d4 turnos. 5 pontos de jade por dardo.
	\item Dardo Paralisante = O alvo recebe -3 destreza e -3 esquiva se não passar em teste de resistência DF 16. 4 pontos de jade por dardo.
	
\end{enumerate}


\subsection{KusariGama}
Arma exótica criadas pelos omayushas. Consiste de uma foice pequena anexada à uma corrente, com um peso extra no final desta. Podem ser usada de dois modos, de perto (usando a foice) ou a distância (usando a corrente).

\begin{enumerate}
	\item KusariGama (Essa arma pode ser usada de dois modos, usando a pequena foice para atacar ou a bola de ferro ). Modo Foice = Dano +4, Aparar +2. Modo Bola de Ferro = Dano /6, Acerto +4, Agarrar +4, Alcance de 2 até 6 metros. Nessa distância a arma é considerada um projétil e recebe os bônus de acerto e agarrar. PR 50. 5 PO. 
	\item HatariGama (Essa arma pode ser usada de dois modos, usando a pequena foice para atacar ou a estrela de ferro ). Modo Foice = Dano /6, Aparar +3. Modo Estrela de Ferro = Dano +6, Acerto +2, Agarrar +2, Alcance de 2 até 6 metros. Nessa distância a arma é considerada um projétil e recebe os bônus de acerto e agarrar. PR 50. 5 PO. 
	\item KusariGama Simples (Modo Foice = Dano +3, Aparar +1. Modo Bola = Dano /4, Acerto +2, Agarrar +2, Alcance 2-4m). PR 40. 3 PO.
	\item KusariGama Avançado (Modo Foice = Dano +5, Aparar +3. Modo Bola = Dano /8, Acerto +5, Agarrar +5, Alcance 2-8m). PR 60. 7 PO.
\end{enumerate}


\subsection{Bastão Curto Composto}
As armas enquadradas na categoria de bastão curto composto são todas as armas de porte pequeno e médio ligadas por uma corrente, como por exemplo Nunchaku, martelo meteoro, etc. Apesar de não terem um dano elevado, são facilmente transportadas e consomem pouco PF durante a luta. Além disso, podem ser usada para defender armas de porte médio de modo mais seguro. Algumas podem ser usadas como bastões pequenos.

\begin{enumerate}
	\item Nunchaku = Dano /4, PR 60, +2 Destreza. Se usado com 1 mão a arma perde o bônus em destreza. Custo 8 de jade.	
	\item Nunchaku de Ferro = Dano /5, PR 70, +2 Destreza, +1 aparar. Custo 1 PO.
	\item Martelo Meteoro = Dano /6, PR 65, +1 Destreza, +1 agarrar. Custo 6 de jade.
\end{enumerate}


\subsection{Lutar Desarmado}
Poucas armas são usadas para atribuir bônus a péricia bélica lutar desarmado. Armas que se enquadram nessa categoria são soqueiras e luvas.


\begin{itemize}	
	\item Soco Inglês = Dano /2, PR 30. Custo 2 Pontos Jade.
	\item Bracelete de Espinhos = Dano +1, PR 50, +2 aparar. Custo 3 PO.	
	\item Luva de Couro = Dano /1, PR 25, +1 aparar. Custo 1 ponto de jade.
	\item Soqueira de Ferro = Dano /3, PR 40, +1 aparar. Custo 4 de jade.
\end{itemize}	


\subsection{Chicote}
O chite é uma arma que não é considerava uma arma de ataque a distância, mas pode realizar ataques a distâncias maiores que uma arma branca normal. Também tem bônus para agarrar. Algumas podem causar dano extra, como dano por eletricidade ou veneno por exemplo.

\begin{enumerate}

\item Chicote = Dano /4, PR 40, +2 Destreza. Custo 2 de jade.
\item Chicotede Prata = Dano /4+2, PR 50, +1 Destreza. Custo 4 PO.
\item Chicote Tatsunamei = Dano /8, PR 30, +2 Destreza. O dano dessa arma ignora armadura, mas o usuário não usa a força para calculo de dano. So o alvo aparar, também leva o dano. Custo 10 PO.
\item Chicote de Couro = Dano /3, PR 35, +1 Destreza, +1 agarrar. Custo 1 de jade.
\item Chicote de Guerra = Dano /6, PR 45, +2 Destreza, +2 agarrar. Custo 3 PO.

  
\end{enumerate}


\subsection{Bestas}
Bestas são armas feitas com a capacidade de arremessar projéteis sem a necessidade de força do usuário. Geralmente são mais caras e de manutenção mais complexa do que arcos. Bestas pequenas podem ser usadas com uma mão, enquanto bestas médias só podem ser usadas com uma mão caso a força do usuário seja alta. A dano da besta não é somado com a força do usuário. O dano final de uma besta é seu dano base mais o dano do projétil acoplado à besta. O alcance da besta geralmente é menor do que um arco, porém o mestre pode usar a mesma logica de alcance médio e longo usada em arco para bestas. Os dados de munição para bestas podem ser os mesmos das flechas usadas para arcos, porém uma flecha de besta não pode ser usada em um arco. 

\begin{itemize}	
	\item Besta Simples = Dano /5, Alcance Máximo(Médio = 15m, Longo = 35m), PR 30, Custo 2 PO.	
	\item Besta do Caçador = Dano /3+2, +1 Acerto, Alcance Máximo(Médio  = 10m, Longo = 25m), PR 30, Custo 5 PO.
	\item Besta Mata Urso = Dano /10 , Alcance Máximo(Médio  = 15m, Longo = 25m), -3 acerto, Força acima de 8 pode usar com 1 mão, PR 30, Custo 7 PO.
	\item Besta Leve = Dano /4, Alcance Máximo(Médio = 12m, Longo = 30m), +1 acerto, PR 25, Custo 1 PO.
	\item Besta Pesada = Dano /8, Alcance Máximo(Médio = 18m, Longo = 40m), -1 acerto, Min força 6, PR 35, Custo 4 PO.
\end{itemize}	



\section{Artefatos Mágicos}


\subsection{Natureza}
Alguns materiais são capazes de emanar energia mágica, alterando assim a propriedade da materia ao seu redor. Essa alteração pode ser algo útil ou perigoso para todos aqueles em volta do material. O nível de seguranca de um material mágico é determinado pela sua natureza e também pela forma como ele é tratado. Por exemplo, exposição continua a materiais mágicos pode causar doenças físicas e até mentais. Alguns animais expostos a determinados materiais podem se tornar bestas violentas e perigosas. Podemos fazer uma analógia de um material mágico com algum material radiativo. Da mesma forma, equipamentos construidos com esses materiais mágicos tem sua propriedade alterada. Por exemplo, fisicamente falando é impossível uma espada de madeira cortar uma parede de ferro. Porém, se essa mesma espada de madeira tiver sido construida com algum material mágico (uma madeira mágica por exemplo), essa espada de madeira mágica pode cortar a parede. Equipamentos construidos com material mágico são chamados de artefatos mágicos.

\subsection{Slots}

O slot é uma propriedade de uma arma mágica que diz a sua capacidade de equilibrio com outros itens mágicos. Quanto maior o slot de uma arma, maior é o número de artefatos mágicos que podem "conviver" com aquela arma. 

A energia mágica emanada por um artefato não é tao intensa quanto a materia prima encontada na natureza. Isso ocorre devido ao fato de, no momento da sua criação, o material é lapidado de forma a emenar a energia mágica de forma mais harmoniosa. Porém, mesmo mais seguro que um material magico bruto, um artefato mágico ainda pode apresentar problemas em relação a sua emanação de energia mágica. Se a energia mágica emanada por vários artefatos entrar em ressonância, uma reação mágica perigosa acontece, como por exemplo, a explosão dos artefatos.

Devido a esse fator, um personagem deve carregar um número máximo de itens mágicos. O mesmo acontece com as poções mágicas, como explicado anteriormente. O límite de armas que um personagem pode carregar é igual ao menor slot de uma arma que ele esteja carregando. Por exemplo, se um personagem tem 3 artefatos com slots 8, 6 e 2, respectivamente, ele deve escolher somente 2 artefatos para poder usar.

O usuário pode gastar 2 PM para "desligar" um artefato mágico. Dessa forma ele pode carregar consigo esse artefato porém sem usufruir de seus poderes. Esse dura alguns minutos, e não pode ser feito instantâneamente. Ele pode ligar o mesmo artefato sem custo de PM extra, gastando alguns minutos concentrado-se.

Artefatos mágicos com slots "nulos" são raros ou muito fracos, pois eles conseguem conviver com qualquer outro tipo de equipamento mágico.

\subsection{Pr dos Artefatos}
O consumo de PR de artefatos mágico ocorre de mesma forma que equipamentos mundanos. A diferença é que boa parte dos artefatos mágicos não é exposta a perda de PR. Por exemplo, se uma armadura recebe um bônus devido a uma pedra mágica acoplada à ela, quando a armadura recebe dano quem perde PR é a armadura e não a pedra mágica. Mas vale notar que um item mágico não é indestrutível. O mestre deve levar em consideração esses fatores durante a campanha. 


\subsection{Acumulo de Bônus}
Não pode existir acumulo de bônus em artefatos mágicos, ou seja, se determinado personagem usa 2 artefatos que concedam bônus para 1 atributo em comum, apenas o maior é considerado.

Além disso, bônus temporários de habilidades que duram múltiplos turnos não acumulam com bônus de equipamentos mágicos no mesmo atributo. Quando um personagem possui tanto um bônus de habilidade quanto um bônus de equipamento mágico no mesmo atributo, apenas o maior é considerado. Por exemplo, se uma habilidade concede +10 em força durante 5 turnos e um equipamento mágico concede +5 em força, o personagem recebe apenas o bônus de +10 da habilidade.

Bônus que duram apenas um instante (como bônus de teste ou bônus de ataque único) não são afetados por esta regra e podem acumular normalmente com outros bônus.

\subsection{Exemplos}

\begin{itemize}	

\item Acessório Mágico = Um acessório mágico consiste de qualquer acessório (anel, brinco, luva, etc) capaz de conceder bônus em um atributo. Esse bônus é igual a 3 pontos em um atributo. Ou seja, um anel mágico da força, faz com que o usuário aumente sua força em 3 enquanto estiver usando o artefato. Esse bônus aumenta também atributo geral, porém o bônus do atributo continua o mesmo. Esse bônus também pode ser convertido em PF (4 pontos), PM ou PV (6 pontos). O slot de um acessório mágico desse nível é igual a 3. O seu custo varia de 7 até 10 PO em uma loja de itens mágicos.

\item Sigil do Carvalho = O sigil do carvalho é um símbolo mágico escrito com tinta mágica. Qualquer equipamento que tenha o sigil do carvalho tem sua quantidade de PR aumentada em 100. Slot 4. Custo de 5 a 7 PO.

\item Espada Mágica = Espadas desse tipo não possuem a lâmina, sendo esta feita de energia mágica. O usuário gasta 4 PM para "ligar" a lâmina de energia, que tem duração em minutos igual a sabedoria do usuário. Quanto maior for a concentração do usuário, mais bela e harmiosa é a lâmina. Qualquer ataque realizado por essa arma não pode ser aparado ou impedido por qualquer meio físico. O dano final é igual ao bônus de inteligência ou sabedoria (a escolha do atributo é feita pelo forjador da arma) do usuário + 6, ou seja,  não tem influência da força do usuário ou outros bônus relacionado, como especialização ou habilidades. Além disso o oponente deve jogar espírito para reduzir o dano, uma vez que a arma causa dano de toque. Arma de slot 3. Custo de 10 a 13 PO.

\item Capa de Proteção Mágica = Essa capa concede 6 de bônus contra ataques de dano mágico. Slot 3. Custo de 7 a 10 PO.

\item Pedra de Proteção = Essa pedra concede +6 de bônus contra ataques de dano físico. Slot 3. Custo de 10 a 14 PO.

\item Firane gun, Terrato shot gun = sabedoria + 6 firane, dano magico toque, o cara joga dex para acertar contra esquiva. Projetil carregado com PM. 4 projeteis, 8 PM. 8 a 12 PO. Tem que ter a pericia respectiva para poder usar a arma.

\item Twilight Armour = Absorção /5+7. PR 50. Além disso concede 4 de defesa mágica para o usuário. O usuário pode gastar reflexivamente 1 PM para receber +1 de defesa mágica por e ataque, até o máximo de 4. Slot 3. Custo de 13 a 16 PO.

\item Earth Protection = Absorção /8, +4PF, Slot 4, PR100 preco de 8 ate 10 PO.

% sugestoes de itens
% colar da sorte, o cara gasta tantos pm para poder jogar um dado a mais no teste
% magic essence gun, o cara pode atirar magias com um cajado;arma etc
% colar da virtude, o cara ganha bonus em todos os atributo sociais
%	Dark fog cloath +5 em testes de furtividade, slot 4
%	Armadura da dor. +15 de defesa fisica e +7 defesa magica. Custa 6PV acometido para ativar durante 1 hora. 40 pr slot2
% sussuros da noite.Por 4 pm o alvo pode usar seu focus como bonus automatico em defesa fisica/magica (max20). +6 fisica/ Magica Slot2 Requer consciencia maior que 15. Sempre que a habilidade for usda, a armadura perde 5 pr.

% para melhorar;criar item magico flexivel.

% pode usar as mesmas regras de forjar normalmente usadas, porem, poucas diferencas.
% cada ponto de atributo vale 2 PO.
% cada ponto aumenta 3 a DF basica.
% Df basica igual a 20.
% -1df basica igual a 3 pj.

% slot nullo, 
% slot 3, entre 10 e 20 XP (podendo ir ate 30 com uso de loot nivel 10 relacionado ao artefato ou acrescentar metade do xp maximo de acordo com uma restricao de igual custo em xp).
% slot 2, entre 30 e 45 XP (podendo ir ate 60 com uso de loot nivel 10 relacionado ao artefato ou acrescentar metade do xp maximo de acordo com uma restricao de igual custo em xp).
% slot 1, entre 80 e 100 XP (podendo ir ate 120 com uso de loot nivel 10 relacionado ao artefato ou acrescentar metade do xp maximo de acordo com uma restricao de igual custo em xp).



 
% os item embutidos podem dar poderes extras free de acordo com sua natureza.
% maior desvantagen de um item magico embutido eh q se a armadura quebrar o item tb quebra.
% alem disso, voce pode querer apenas uma caracteristica de um item embutido, e nao ele todo 
% por exemplo, um martelo que de sabedoria. um mago q nao use um martelo nunca vai querer.
% melhorar um item magico eh mais caro q o normal pq tem q usar item magico, entao um embutido pode ser um problema.


\end{itemize}	
 
%%%
%% Capítulo 4: Batalha
%%

\chapter{Batalha}
\label{Cap:batalha}

O objetivo principal do sistema de batalha de Dregon é fazer com que a luta se torne mais dinâmica possível. É muito importante em um sistema de RPG que o seu sistema de combate tenha três principais características: liberdade, velocidade e coerência. 
	 
Vários detalhes como habilidades, perícias, magias e etc, serão explicados em capítulos posteriores. Vale notar que o jogador não é obrigado a conhecer as regras de combate, quem deve conhecê-las é o mestre. O desconhecimento do sistema de batalha não impede que um bom jogador realize lutas ruins. O conhecimento do mesmo torna mais rápido o fluxo da batalha.

O sistema de batalha de Dregon será descrito através de diversas situações de combate. Cabe ao mestre encaixar as ações desejadas pelos jogadores, nessas ações descritas neste capítulo. Esse conjunto de situações básicas podem dar origem a outras situações mais específicas. Por exemplo, atacar é uma ação básica. Desarmar o oponente, é uma variação da ação básica atacar, porém com outro objetivo. Se um personagem desejar desarmar o oponente, ele tem duas opções. Seguir a orientação de regra dado nesse capítulo, ou alterar ao bel prazer a regra de ataque. Isso cabe ao mestre e jogador fazer durante o jogo, deixando o jogo mais fluido e divertido. Ou seja, nesse capítulo são dadas ações de combate básicas, e sugestões de como o mestre deve proceder caso o jogador queira outras ações mais complexas.

\section{Momento de Ação}

Um momento de ação corresponde a um intervalo de tempo que é igual à soma de dois turnos, um turno de ataque e um turno de esquiva. Durante o turno de ataque o personagem pode realizar ações mais complexas, como atacar, usar uma habilidade ou uma magia por exemplo. Durante o turno de esquiva, ações mais reflexivas devem ser feitas tais como pular rapidamente para poder se desviar de um ataque, dar um grito sinalizando uma ordem, etc. Portanto podemos dizer que durante o turno de ataque o tempo para se realizar ações é maior do que o tempo para se realizar ações no tempo de esquiva. Se alguém tenta realizar uma ação mais longa no turno de esquiva, terá penalidades para realizar tal ação, por outro lado quem desejar realizar ações simples no turno de ataque terá bônus.

	No momento que acontece o seu turno de ataque, acontece o turno de esquiva do inimigo, e assim por diante. Mas isso não quer dizer que você não possa tentar realizar um ataque em seu turno de esquiva (manobra contra-ataque). Mais detalhes explicados posteriormente.
	
	Em suma, para termos de sistema um momento de ação é dividido em dois turnos. Um turno de ataque e um turno de esquiva. Um momento de ação tem mais ou menos duração de 3 a 4 segundos.  O mestre deve usar esse valor apenas como referência para basear-se na duração de determinadas batalhas. Mas o contexto da situação de combate descreve melhor a duração de todo o combate. 

\section{Iniciativa}

Para calcular a ordem das ações, todos jogam ágilidades para ações físicas e percepção para ações mentais ou ativação de habilidades. A ordem dos sucessos determina a ordem das ações. 
O personagem pode também, caso ganhe a iniciativa, optar por não atacar. Nesse caso ele terá um bônus em qualquer ação que ele quiser realizar no turno de esquiva (apenas em ações que não sejam de ataque, ou seja, ações de esquiva) e esse bônus depende de sua ágilidade/percepção. 

\section{Ataque}

\subsection{Um ataque normal}

O atacante realiza um teste de destreza para atacar. Guarde esse valor para usar posteriormente se o defensor for esquivar. Em outras palavras, ele faz um teste combatido contra o defensor. Se o atacante conseguir ganhar no teste ele consegue acertar o golpe. O número de sucessos extras no ataque concede um aumento no dano final (detalhes explicados posteriormente).
Ataques usando magias não obedecem essa regra, tendo regra própria explicando em capítulo posterior.

\subsection{Ataque Inconsequente}

Nesta manobra, o personagem ataca de forma louca e sem precaução, visando apenas a destruição do oponente. Ele vai perder o turno de esquiva e não pode realizar as manobras de combate ataque em alvo específico ou múltiplos ataques. O mestre deve considerar também o bônus de força para determinar a diferença da força atacante-defensor para fins da manobra aparar.

O personagem que realizar a manobra ataque inconsequente ganha bônus de atributo em destreza e em força para 1 ataque nesse turno. Essa manobra só é válida para ataques físicos feitos corpo a corpo, ou seja, não é usada para armas projetáveis, como arcos, armas de fogo, etc. \textbf{Um personagem só pode realizar um ataque inconsequente por batalha.} 

No seu próximo turno de esquiva, como dito anteriormente, o personagem não pode esquivar. Ou seja, o valor do teste de esquiva do usuário é igual a sua esquiva, sem a jogada de dados (ele não pode substituir esse valor pelo usado na manobra aparar). Use esse valor como base para calcular bônus no dano proveniente dos sucessos obtidos no ataque.

\section{Esquiva}

O personagem pode usar 3 manobras de esquiva.

\subsection{Esquiva Normal}

Ele pode esquivar-se normalmente jogando sua esquiva e testando contra a destreza de quem o atacou. Em outras palavras, o que define se o defensor se esquivou ou não é o resultado do teste combatido da destreza (do atacante) contra a esquiva (do defensor). O número de sucessos extras na esquiva serve apenas para descrever o grau de êxito da mesma.

\subsection{Aparar}

Ele pode aparar o golpe usando um objeto. Uma espada, lança, escudo, ou algum equipamento que ele saiba utilizar pode ser usado para desviar o golpe do atacante. Nesse caso ele joga destreza no lugar de esquiva. É um movimento bastante útil para quem tem baixa esquiva. %porém esse movimento cansa mais do que uma simples esquiva (consome mais pf no momento da batalha). Cada 2 ou 3 usos da manobra aparar (de acordo com a tamanho da arma, diferença entre forças defensor/atacante, etc), o usuário perde 1 PF. Além disso o mestre pode retirar uma quantidade extra de PR da arma por ser usada como objeto parar aparar o golpe do adversário.

No caso do defensor ter uma força menor em relação ao atacante em 4 ou mais pontos, ou da arma do atacante ser pelo menos 1 porte maior que a arma do defensor, o atacante recebe um bônus em sua jogada de acerto igual ao seu bônus de força. Além disso, o mestre tem total liberdade para alterar os valores nesse teste, por exemplo, no caso do atacante usar armas leves a uma curta distância de um defensor que tente aparar com uma arma grande, o defensor recebe uma penalidade. O mestre pode usar qualquer interpretação da situação de combate para torna-lo mais dinâmico.
	
%Por exemplo, quando a diferença entre forças é considerável, o atacante pode jogar destreza com um bônus de força contra a destreza do defensor. O mestre também pode dar bônus ao atacante de acordo com a distância de ataque, natureza favorável contra a arma do atacante/defensor, etc. Além disso o mestre pode reduzir o dano de acordo com a situação, por exemplo, em caso de empate no teste de aparar. 
	
Se o defensor pretende usar essa manobra para se defender de um projétil (um ataque a distância), dobre o valor do teste final da destreza do atacante, já que a área de um projétil é bem menor que de uma espada, por exemplo. Além disso, o mestre pode decidir aumentar ainda mais a penalidade para se defender projéteis (alguns não permitem a defesa de projéteis usando armas). A maioria dos projéteis pode ser defendida independente da força do defensor. Alguns projéteis não podem ser aparados usando essa manobra, o caso de esferas de energia ou projéteis pequenos.

 
\section{Tipos de Dano}

\subsection{Dano Normal}

Representa o dano que pode ser reduzido sem auxilio de armadura especial. Dano por contusão (socos, chutes) e afins são exemplos de danos normais. O dano normal é representado por uma " / " 
	Todo dano usando apenas o atributo força (sem auxilio de habilidade especial) é considerado como dano normal. Você pode somar todos os danos normais para fins de cálculo de dano normal total. Por exemplo, se personagem com força 6 usa um porrete /3, pode-se considerar a como dano normal total 9 (/6 + /3 = /9).


\subsection{Dano Automático}

O dano letal é um tipo de dano mais difícil de absorver sem armadura apropriada. Normalmente é causado por ataques cortantes, quentes, ácidos e afins. é representado por um \" + \". Ou seja, quando a força de um personagem mostrar 5 + 1 não quer dizer que a força dele é 6, e sim que a força dele é 5 porém está associada com +1 de dano automático. Para minimizar-se +1 de dano é necessário +1 de defesa automática. Apesar de ser mais difícil de defender, 1 de dano automático retira a mesma quantidade de dano que 1 de dano normal. 
Vamos supor que uma pessoa seja atacada por uma espada +2 e tenha conseguido no seu resultado de defesa um valor igual a /5. Ela conseguirá reduzir +1 de dano, reduzindo o dano total para +1. Então ela vai perder apenas 1 de dano e não 3. Se por exemplo a espada fosse um porrete, seria /2 (dano normal), e a pessoa com /5 de defesa poderia reduzir o dano total a 0, não perdendo nenhum PV com o ataque.

\subsection{Dano Crítico}

Quando um ataque resulta em um acerto crítico (explicado no capítulo de jogadas), o dano causado por esse ataque é considerado dano crítico. O dano crítico possui duas características importantes:

\begin{itemize}
	\item \textbf{Não pode ser curado durante o combate}: Magias de cura, poções e outras formas de recuperação de PV não afetam o dano crítico.
	\item \textbf{Reduz temporariamente o PV máximo}: O dano crítico reduz o PV máximo do personagem temporariamente até que ele realize um descanso adequado (geralmente uma noite de sono).
\end{itemize}

Por exemplo, se um personagem com 20 PV máximo recebe 5 de dano crítico, ele fica com 15 PV máximo até descansar. Durante o combate, ele não pode recuperar esses 5 PV perdidos por dano crítico.

\subsection{Dano de Toque}

O dano de toque é um tipo especial de dano que representa ataques que ignoram parcialmente a proteção física. Este tipo de dano possui duas características importantes:

\begin{itemize}
	\item \textbf{Reduz a defesa pela metade}: A defesa do alvo é reduzida pela metade (arredondada para baixo) para fins de redução de dano.
	\item \textbf{Ignora armadura}: Equipamentos de proteção (armaduras, coletes, etc.) não fornecem proteção contra dano de toque.
\end{itemize}

Por exemplo, se um personagem com defesa natural /6 e armadura /2+1 recebe um ataque de dano de toque /4, apenas sua defesa natural (/6) é considerada, mas reduzida pela metade (/3), resultando em /1 de dano efetivo.

\subsection{Lembrete Sobre Dano}

Tanto o dano normal quanto o dano automático são semelhantes em termos de dano (quantidade de pv tirada), o que os diferencia é a capacidade de um ser absorvido facilmente enquanto o outro não . +1 de dano automático pode ser absorvido com +1 de defesa automática, porém /1 de dano normal pode ser absorvido com /1 de defesa normal.

	Da mesma forma existe a defesa automática e normal. A defesa automática serve para defender igualmente +1 dano, porém na mesma intensidade de defender /1 de dano. Em outras palavras, uma armadura +4 defende da mesma forma +4 ou /4.
	
\section{Dano}

Para calcular-se o dano de um ataque (dano final), deve-se calcular o dano inicial e a redução de dano. O dano que o personagem vai receber é o dano inicial menos a redução de dano. 

Para calcular-se o dano inicial, execute o seguinte procedimento. Primeiramente some no dano do atacante o bônus do sucesso no teste de acerto como sendo dano normal. Esse bônus não pode exceder o valor do bônus de força permanente do mesmo para ataques corpo a corpo, e no caso de ataques a distância o bônus proveniente do sucessos extras não pode ultrapassar o dano base da arma (não do projétil). Se por exemplo o atacante consegue 17 no teste de acerto e o defensor consegue 10 no teste de esquiva, a diferença é de 7. Então o atacante terá um bônus de /3 no dano (3 é o valor do bônus de 7). Se o atacante tiver força 5, esse bônus será /2, pois o bônus de sua força é 2. Isso pode ser explicado analisando que quanto maior o sucesso do atacante em cima do defensor, mais eficiente será o golpe. Existem algumas armas e habilidades que aumentam esse limite. Após calcular o dano proveniente de sucessos extras, o atacante soma separadamente os tipos de dano (automático e normal). Esse valor será o dano inicial.

Para calcular-se a redução de dano, compare diretamente os valores de dano com os valores de defesa do defensor. A defesa é calculada diretamente sem jogadas de dados, usando apenas os valores base do personagem e equipamentos. Subtraia dano automático com defesa automática e dano normal com defesa normal. O excedente da defesa automática pode reduzir o excedente de dano normal, mas o excedente da defesa normal não pode reduzir dano automático. O dano após os cálculos é chamado de dano final.

A regra explicada acima vale para danos físicos, detalhes sobre o cálculo de dano mágico explicado no capítulo sobre magias.
	O mestre pode interpretar como o alvo sofreu o dano de acordo com a quantidade máxima de pv. O mestre pode interpretar um golpe que tire 15 de PV de um personagem que tem 20 de PV total como quase um braço decepado ou um olho perfurado por exemplo. A mesma quantidade de dano em um personagem com 100 de PV pode ser vista como um arranhão.
	
\textbf{Exemplo de cálculo de dano.}

O atacante tem força 5 e uma faca +4. Ele teve 16 no teste de destreza e o defensor 10 no de esquiva. Portanto, o atacante teve 6 sucessos extras, dando um bônus em seu dano de /2. O seu dano inicial vai ser /7+4. O defensor tem defesa /4+2 (valores base + equipamento). Primeiro diminuímos os tipos de danos semelhantes: /7 de ataque é reduzido pelos /4 de defesa, sobrando /3 de dano normal. Os +4 de ataque são reduzidos em +2 pela defesa +2 do defensor, sobrando +2 de dano automático. O excedente da defesa automática (+2) pode reduzir o excedente de dano normal (/3), reduzindo-o para /1. No final o defensor perde 3 PV (/1 + +2).


\section{Gastando PF}

Durante uma batalha é normal o desgaste físico dos combatentes. Isso é refletido na perda de PF. Normalmente em uma batalha um lutador, atacando e se esquivando 1 vez a cada turno usando armas leves, perde 1 PF a cada 4 turnos aproximadamente, 1 PF a cada 3 turnos quando usando equipamento de porte médio e 1 PF a cada 2 turnos quando usando equipamento de porte pesado. Mas existem alguns fatores que aceleram a perda de PF:

\begin{itemize}

	\item Uso de abusivo de equipamento pesado. Quanto maior for o número de equipamentos pesados usados durante o combate, mais PF é perdido. O uso de 3 equipamentos pesados ao  mesmo tempo acarreta a perda de 1 PF por turno. 
	\item Uso de habilidades específicas. A maioria das habilidades consomem PF.

	\item Locomoção. Seja para fugir ou para alcançar alvos, o mestre deve exigir o gasto de PF quanto maior for a locomoção do alvo na luta.

	\item Condições climáticas desfavoráveis também podem ser usados como fatores agravantes na perda de PF em relação ao desgaste físico natural sofrido nas batalhas.
\end{itemize}

No lugar de monitorar o que cada personagem faz em certos intervalos e ir tirando os PF de acordo com as regras acima, no final da batalha o mestre retira de 1 a 4 PF dos personagems por critério pessoal, levando em consideração condições que retirem mais PF que o normal (usar muitas manobras de combate, peso do equipamento, duração do combate etc).


\section{Usando Itens}

é comum durante as lutas os personagems usares determinados medicamentos de ação instantânea para recuperarem seus PV,PM ou PF, ou drogas para aumento temporário de atributos. Um item pode ser usado em você mesmo ou em um aliado dentro do seu campo de ação durante o seu turno de ataque. Para realizar tal ação durante a batalha o mestre deve ficar atento as situações de luta para atribuir penalidades ou até mesmo privar o jogador de realizar ataques nesse turno. 

Sempre que um personagem for usar um item em um aliado, ele perde todo o turno de ataque realizando essa ação. O mestre pode privar o uso para os aliados ele próximo, ou dentro do seu campo de ação. Caso um personagem deseje usar um item nele mesmo, e esse item encontra-se de fácil acesso, o personagem pode usa-lo sem penalidade nenhuma em seu ataque. Porém se o personagem usa alguma armadura ou encontra-se com as mãos ocupadas (segurando um arco e flecha por exemplo) o mestre pode fazer com que o personagem perca seu turno de ataque localizando o item.

 

\section{Regras Opcionais para Combate}

Abaixo segue a explicação de várias manobras de combate utilizadas no sistema Dregon. Essas manobras servem para tornar o combate mais dinâmico possível. Sempre que o mestre tiver dúvida como proceder perante uma manobra, ele pode usar as informações a seguir para ajudar em sua decisão final de como prosseguir. Todas as manobras de combate consumem mais PF que as ações padrões normais. 

As regras abaixo tem o único objetivo de mostrar como as regras básicas de combate são alteradas para que diferentes situações mais complexas de combate possam ser usados. Em outras palavras, o mestre pode alterá-las ou até mesmo não utilizar as regras abaixo que nada vai atrapalhar o funcionamento básico do sistema. Elas foram criadas para dar um maior grau de detalhes aos combates.
	
\subsection{Ataques Múltiplos}

Um personagem pode optar por no lugar de dar apenas um golpe bem visado, dar vários golpes porém com sua destreza menor. Isso pode ser feito se o personagem dividir a destreza para realizar um golpe, ou seja, um personagem que tem destreza 4 dar dois golpes com destreza 2, realizando 1 ataque extra. Cada mestre pode estabelecer um número limite de golpes extras que o personagem pode dar. Aqui iremos dar uma média que pode ser usada. De cada 4 em 4 pontos em destreza o personagem pode dar um golpe extra, ou seja, um personagem com destreza 7 poderia dar 2 golpes (1 normal e um extra) com destreza dividida (4 para um e 3 para outro ou 5 para um e 2 para outro) mas não poderia dar sete golpes com destreza 1. Vale observar que que qualquer bônus recebido em destreza entra no cálculo antes da divisão de ataques. Caso alguma habilidade conceda ao personagem bônus em um ataque, o personagem deve escolher em qual dos ataques (caso executando a manobra ataques múltiplos) esse bônus deve contar.

A distribuição de destreza entre os golpes deve ser escolhida pelo jogador, obedecendo a seguinte regra: O valor da destreza dividida para cada ataque não deve ser maior que o bônus de destreza do usuário. Por exemplo, um personagem com destreza 8 (bônus igual a 3), pode dividir 2 ataques em um ataque com a destreza 5 e o outro ataque com a destreza 3, mas não pode alocar 2 de destreza para 1 ataque e 6 para outro ataque, pois a diferença entre o primeiro ataque que tem destreza 2 e o segundo ataque que tem destreza 6 é 4, valor esse maior que o bônus de destreza que é 3.

Uma regra opcional para a forma como a destreza é dividida é que caso o personagem aloque 2 vezes consecutivas 1 ponto quando dividindo a destreza, o defensor não tem penalidade na terceira esquiva (penalidade essa descrita mais a frente). O uso de duas armas pode aumentar o número de golpes extras e reduzir um pouco o consumo de PF ao usar essa manobra. 

Quando um personagem utiliza a manobra ataques múltiplos no próximo turno de esquiva qualquer ação envolvendo destreza ou esquiva tem um redutor igual ao número de ataques extras.
	
No caso de você está dando vários golpes e o oponente não poder esquivar, o mestre pode atribuir um valor de dificuldade mínimo para você acertar o golpe. Isso fica a critério do narrador.
	
Tanto a quantidade de golpes, quanto as penalidades que cada golpe tiver (se tiver) podem ser estabelecidas pelo mestre. Existem mestres que não limitam a quantidade de golpes outros se guiam pelo bônus de destreza (cada bônus de destreza é um golpe a mais). Isso cabe ao mestre decidir. Aqui nós apenas mostramos uma média que pode ser usada, uma vez que o narrador pode utilizar do modo como bem desejar. Existem algumas habilidades que retiram esses limites, cedendo ataques extras ou bônus aos atacantes. 

\subsection{Esquivas Múltiplas}

Quando um personagem recebe múltiplos ataques, seja de um único oponente realizando vários ataques ou de múltiplos oponentes, ele recebe penalidades cumulativas em suas esquivas.

\textbf{Regra Simplificada}: Para cada ataque extra recebido, o defensor recebe um redutor de -2 na esquiva.

\textbf{Exemplos}:
\begin{itemize}
	\item Um oponente realiza 3 ataques: primeiro ataque sem penalidade, segundo ataque -2, terceiro ataque -4.
	\item Três oponentes atacam simultaneamente: primeiro ataque sem penalidade, segundo ataque -2, terceiro ataque -4.
\end{itemize}

O mestre pode permitir que o defensor use uma única esquiva para ataques quase simultâneos, mas isso consome mais PF e pode exigir um teste de percepção dependendo da situação.
 


%\subsection{Detalhes sobre Aparar}

%A regra a seguir pode ser usada por mestres mais exigentes em relação a detalhes da batalha.

%Caso o defensor passe no teste (consiga aparar o golpe) ele deve realizar um teste de força ou destreza (ele pode escolher) contra a força do atacante em duas situações distintas. A primeira é quando a arma do atacante é de grande porte, o defensor esté usando uma arma média para aparar e sua força é menor do que a força do atacante. A segunda é quando as duas armas são de mesmo porte, porém a força do atacante é maior que a força do defensor em 4 pontos ou mais. No caso do defensor escolher destreza para jogar contra a força do alvo, ele usa o mesmo resultado do teste inicial de aparar. 

%Esses dois testes são interpretados da seguinte forma. O primeiro serve para o posicionar o objeto que vai ser usado para aparar o golpe. Esse teste é o teste de aparar em si, aonde bônus de habilidades e equipamentos que concedam bônus em aparar podem ser usados. O segundo teste serve para anular ou direcionar a força proveniente do atacante. Para realizar tal feito, o defensor pode fazé-lo usando sua força diretamente contra a do oponente ou utilizando sua destreza para desviar essa força em outra direééo. Ou seja, o atributo aparar do equipamento deve ser usado como um bônus em qualquer teste de destreza envolvendo essa manobra. De forma semelhante, o atributo resiténcia do equipamento deve ser usado como um bônus em qualquer teste de força.


%Se o defensor não conseguir passar nesse teste de força combatido, isso pode ser interpretado que o defensor conseguiu posicionar algum objeto na trajetória do ataque, mas a força do ataque deslocou o objeto. Então o defensor recebe o golpe, porém reduzido. O valor da redução do golpe deve ser definida pelo mestre. Por exemplo, o mestre pode reduzir do dano recebido pelo defensor um valor igual a força do defensor. O defensor também pode ter feito um teste de força cujo resultado foi téo baixo, que o golpe não perdeu nada de sua força original. Nesse contexto, no lugar do defensor levar o golpe reduzido, o mestre também pode retirar uma quantidade maior de PR da arma ou até mesmo fazer com que a arma caia no chéo, próximo ao defensor. O mestre pode interpretar a manobra da forma como desejar.

%Em outras palavras, para realizar a manobra aparar o defensor deve realizar no méximo dois testes. O primeiro é de destreza contra a destreza do oponente. O segundo deve ser realizado caso se encaixe na situação descrita acima. Esse segundo teste é de força ou destreza (o defensor escolhe qual atributo usar) contra a força do oponente. Caso escolha a destreza, ele usa o mesmo valor obtido no primeiro teste. 

%Em algumas situações o mestre pode (e deve) proibir o uso dessa manobra. Por exemplo, alguém tentando aparar um martelo grande com uma espada curta. Apesar da perécia bélica "lutar desarmado" poder ser usada para aparar ataques, o mestre deve considera-lé como arma curta ou média de acordo com a situação. Por exemplo, ela pode ser considerada arma média caso o defensor esteja próximo do oponente, tornando assim o uso da péricia mais fácil. No caso inverso, ou seja, o defensor usa alguma arma para se defender de um atacante que usa lutar desarmado, o defensor não pode usar a manobra aparar (apesar com alguma péricia defensiva como escudo ou manopla) caso o atacante esteja muito próximo. Isso é determinado pelo mestre de acordo com a batalha, onde o mesmo pode exigir teste por parte do atacante para se aproximar a tal ponto do defensor (um teste de saltar por exemplo).	


\subsection{Atacando Alvos Específicos}

Para tornar mais ágil e simples o combate, em Dregon adotamos a seguinte interpretação para ataques em alvos específicos. Ao invés de acertar um local específico de um alvo para retirar uma quantidade maior de PV, sempre que um atacante retirar um sucesso muito grande (dano e acerto) esse ataque bem sucedido é interpretado como sendo nesse local. Por exemplo, uma grande quantidade de dano retirada subitamente pode ser interpretada como um membro decepado por exemplo.

Caso o personagem deseje atacar algum alvo específico do oponente (furar uma bolsa com itens, por exemplo), a jogada de ataque é feita normalmente, porém, o defensor tem um bônus em sua jogada de esquiva (esquivando ou aparando) de acordo com o tamanho do objeto alvo. Quanto menor o alvo, maior o bônus, variando de 2 até 8.

O mestre também pode usar a seguinte regra opcionais para redução de atributo de acordo com dano causando em certas partes do corpo. Caso o ataque acerte com êxito membros do corpo (usando os redutores citados acima), o mestre pode impor redutores em atributos (força, destreza ou esquiva) de acordo com o dano sofrido. A quantidade de redução de atributo será avaliada perante a quantidade de dano sofrido. O redutor é igual ao bônus do dano (um dano de 4 irá causar um redutor de -1 e assim por diante).
  

%\begin{itemize}
%	\item Alvos médios (bolsas, braços, etc) = bônus de destreza ou esquiva +2.
%	\item Alvos pequenos(pescoéo, cabeéa, mãos, etc) = bônus de ágilidade  + 4.
%	\item Alvos muito pequenos(olhos, brincos, etc) = bônus de ágilidade  + 6.
%\end{itemize}

O personagem também pode desejar atacar um alvo específico do oponente com o objetivo de reduzir a absorção da armadura do alvo. Para fazer isso é simples. Para cada ponto reduzido no teste de atacar, o dano (caso o ataque acerte) ignora 1 ponto da armadura do alvo (começando pela defesa normal). Outro detalhe é que para cada ponto de cobertura da armadura, a mesma consegue ignorar 1 ponto essa redução. Por exemplo, caso o atacante deseje retirar 2 pontos de absorção de uma armadura com cobertura zero ele deve jogar o teste de atacar com um redutor igual a 2. Se ele deseja reduzir a mesma quantidade de uma armadura com cobertura 3, então ele deve ter um redutor de 5 em seu ataque. A penalidade recebida não pode ser maior que o atributo destreza - 1. Vale observar que armaduras pesadas podem ter sua absorção reduzida até a metade com o uso dessa manobra.


%Caso o ataque atinja partes vitais e cause dano, o alvo sofre um dano extra igual a uma percentagem do PV do alvo (o mestre decide se 25\% ou 50\% do PV total).

Vale destacar que determinados inimigos não tem penalidades de receber danos em alvos específicos, seja devido a sua natureza ou cobertura excepcional de alguma armadura. Além disso, como dito anteriormente, o mestre pode interpretar os efeitos da perda de PV, ou seja, não será somente com o uso dessa manobra que um alvo pode ter seu braço ferido ou decepado. Cabe ao mestre escolher de acordo com a situação de combate quando essa manobra pode ser usada e como.

\subsection{Ataque em área}

Determinadas ocasiões permitem você acertar vários oponentes com apenas um golpe. Seja devido a arma ser grande (um machado ou espada montante), ou por que vários alvos estão muito próximos a você. No caso da arma poder fazer isso, vem especificado nela quantos metros ela pode alcançar com um golpe. No caso de personagems próximo demais, quem decide é o mestre. Nesse caso se faz um teste de destreza(no caso quem ataca). O alvo mais próximo recebe o golpe com destreza normal e os alvos consecutivos vão esquivar com um bonus cumulativo de +2 e uma penalidade cumulativa no dano para o atacante, em outras palavras, o ataque vai perdendo a força ao longo do trajeto. De acordo com a descrição da cena, o dano pode diminuir em 3 ate 10 para cada alvo atingido. Por exemplo, se alguém com uma espada longa ataca vários alvos e o segundo absorve tudo, o mestre pode decidir por reduzir o dano em 13 para o terceiro alvo, sendo 3 do primeiro e 10 do segundo. O mestre pode também interromper o ataque, dizendo que o ataque em área não efetuou sua trajetória completa. 

\subsection{Contra Ataque - Atacando no Turno de Esquiva}

Você pode no seu turno de esquiva optar por tentar atacar o oponente no lugar de esquivar-se de seu golpe (algo como um contra ataque). Primeiramente você tem que fazer um teste para ver se você é rápido o suficiente para poder atacar no turno de esquiva. Você vai fazer um teste de ágilidade contra o oponente que está te atacando nesse turno de ataque. Porém, como você esté atacando no turno de esquiva e o oponente não (pois o intervalo de tempo que ele dispõe para realizar uma ação e mais bem ajustado) você vai ter penalidades nesse teste de ágilidade. A penalidade será igual ao bônus de ágilidade do oponente (se for um oponente lento o turno de ataque dele poderá ser facilmente mais lento proporcionando oportunidade do seu ataque). Se você passar no teste poderá atacar normalmente o oponente. Algumas observações devem ser levantadas a respeito de tal manobra.
\begin{itemize}
	\item Se vários alvos estiverem te atacando no turno de esquiva, você poderá atacar eles. Faça um teste ágilidade contra cada um (com redutores baseados nas ágilidades de cada um separadamente). Os que você passar poderá atacar normalmente (dividindo a destreza com regras de múltiplos ataques), porém se alguém for mais rápido (ganhar no teste de ágilidade), você não poderá mais contra-atacar nesse turno.
	\item O alvo que recebe essa manobra não pode tentar se mais rápido que você(um contra-contra-ataque).
	\item Nem você nem o alvo terão direito de esquiva. O maximo que pode se feito é o que está no turno de ataque fazer um teste de percepção para poder usar a manobra defesa máxima. Esse teste é um teste contra a ágilidade de quem tenta dar o contra golpe.
	\item Essa manobra não pode ser feita no primeiro momento de ação da batalha.
	\item O mestre tem total direito para alterar tais regras segundo sua interpretação. Em casos como já foi citado, se os números do teste de ágilidade ou de destreza forem próximos ou iguais, o mestre pode interpretar de modo próprio. Os golpes podem se chocar, os dois atacam ao mesmo tempo e assim por diante.
	\item Como outras manobras de combate não usuais, ela consome mais PF do que o normal.
\end{itemize}


\subsection{Esquivando no Turno de Ataque }

Ao invés de atacar o oponente, você pode desviar de um provável golpe que pode vir a receber no próximo turno. Você ganha um bônus de ágilidade em destreza ou esquiva para qualquer ação que você fizer no turno seguinte a este em que você ficou parado (no caso seu turno de esquiva). Você amplia seu turno de ataque para se fundir com o turno de esquiva de modo a ganhar bônus para se esquivar ou defender usando a destreza (esse bônus e determinado pela ágilidade). O bônus não pode ser maior que o atributo usado. Por exemplo, certo personagem tem destreza 8 e esquiva 2, totalizando ágilidade 10. O bônus da sua ágilidade é 4, porém ele so pode usar 2 pontos desse bônus caso use a esquiva para se esquivar do ataque.


\subsection{Movimento Durante a Batalha}

Qualquer personagem pode se mover até um valor igual ao seu bônus de esquiva em metros durante um turno normal (podendo atacar e se esquivar normalmente). Se o personagem tirar o turno só para se mover, esse valor sobe para o seu valor completo de esquiva em metros.

O mestre pode reduzir esse valor de acordo com cargas levadas pelo personagem ou outras condições específicas da situação.
	
	
\subsection{Desarmar Oponente}

A manobra desarmar é usada para tirar a arma de um oponente de suas mãos. O uso dessa manobra é bastante simples, por parte do atacante um teste de destreza mais um bônus, definido de acordo com a situação do desarme. Esse bônus é igual ao bônus do atributo força caso o atacante use esse atributo para desarmar o oponente ou de acordo com a arma utilizada. De forma semelhante, o bônus é de destreza para desarmes usando armas que possibilitem essa opção (armas leves ou pequenas geralmente se encaixam nessa situação).

Por parte do defensor, um teste de destreza ou esquiva mais um bônus. Da mesma forma que o atacante escolhe atribuir o bônus de força ou destreza, o defensor também pode faze-lo. O mestre pode adicionar outros bônus extras de acordo com a situação, por exemplo, conceder um bônus maior de força se a arma utilizada for de duas mãos, se ela tiver algum acessério que ligue a arma ao defensor ou até mesmo de acordo com a descrição da manobra (distância dos alvos, usar uma arma média atrás de um escudo, etc). é muito importante que o mestre preste atenção nesses detalhes, por exemplo, na maioria dos casos é muito difícil desarmar um oponente usando determinadas armas (espadas, machados, etc). Boa parte das armas são feitas para causar dano, e seu formato não auxilia na manobra desarmar. Para facilitar, o mestre pode usar o seguinte referencial. Se uma arma não conceder bônus em aparar, o defensor recebe seu bônus (de destreza ou esquiva) duas vezes para se defender. Se a descrição da manobra por parte do atacante for coerente, o defensor joga o atributo mais bônus de atributo mais bônus de distância (que pode ser ignorado de acordo com a situação).

Essa manobra não retira dano do defensor caso seja bem sucedida. O número de sucessos determina qual longe a arma foi arremessada. Em alguns de sucesso muito baixo favorecendo o atacante, o defensor pode realizar um teste de concentração para tentar pegar a arma "no ar", ou atacar usando a arma no próximo turno com um redutor (como se uma parte do turno de ataque dele tivesse sido usada para recuperar a arma caída).
 
O mestre deve decidir quais atributos serão usados e os bônus que serão estábelecidos de acordo com a situação. Por exemplo, um samurai com uma espada média tentando desarmar um ranger com uma lança. O samurai pode escolher destreza (recebendo assim também o bônus de destreza) para tentar desarmar o ranger. O ranger escolhe pular para o lado enquanto puxa sua lança. Nesse caso ele joga esquiva mais bônus de destreza mais um bônus definido pelo mestre de acordo com a distância que o samurai encontra-se do ranger.

Em termos de sistema não é uma manobra complexa, porém seu uso da brecha para muitas possibilidades diferentes, possibilidades essas que podem conceder bônus e ônus tantos para o defensor quanto para o atacante. O mestre deve ficar atento para a descrição da manobra. Na dúvida, favorecer a defesa.

\subsection{Agarrar Oponente}

Para realizar a manobra agarrar é simples. Basta realizar sua jogada de ataque normal, ou seja, destreza contra esquiva ou destreza do oponente. Caso acerte o alvo, o atacante deve realizar outro teste, porém dessa vez usando força contra a força ou contra a destreza do oponente. Se obtiver sucesso mais uma vez, o defensor é considerado agarrado pelo atacante. O mestre deve incluir bônus ou ônus de acordo com a situação, por exemplo, um atacante pequeno tentando prender um alvo grande apenas com as mãos, ou o atacante usando cordas para amarrar o alvo. Quando agarrado, dependendo da situação do agarrão, o mestre deve definir as limitações de ações durante o agarrão. Por exemplo, uma pessoa com os braços imobilizados, não pode atacar usando os mesmos. O mestre também deve definir o dano que o alvo fica recebendo enquanto fica agarrando. Esse dano também pode ser representado por perda de PF (em situação de estrangulamento), ou perda de atributo (no caso de um membro quebrado).

Para se soltar do agarrão, o defensor deve realizar um teste novamente de destreza ou força com a força do oponente.

%\subsection{Regra Opcional sobre PF}

%Para mestres mais exigentes exige um método mais detalhista para monitoramento de perda de PF para cada personagem. Para cada personagem presente na batalha o mestre deve criar uma barra, com 10 campos, variando de 10\% até 100\%, como mostrado a seguir. Vamos chamar essa barra de "félego".

%\begin{table}[htbp]
%\begin{center}
%\begin{tabular}{|c|c|c|c|c|c|c|c|c|c|} \hline 
%10\%&	 20\%&	30\%&	 40\%&	 50\%&	 60\%&	 70\%&	 80\%&	 90\%&	 100\%\\\cline{1-10} 
% &  &   & &  &  &  &  &   &  
%\\ \hline
%\end{tabular}
%\end{center}
%\caption{Barra de acompanhamento de perda de PF}
%\label{}
%\end{table}

%Sempre que o personagem realizar uma ação, o mestre deve marcar com um "X" os campos equivalentes a percentagem que aquela ação consome. Por exemplo, se um personagem realizar um ataque que consuma 30\% dessa barra, a tabela deve ficar como a mostrada a seguir (partindo do princépio que o félego inicialmente estáva vazio). 
  

%\begin{table}[htbp]
%\begin{center}
%\begin{tabular}{|c|c|c|c|c|c|c|c|c|c|} \hline 
%10\%&	 20\%&	30\%&	 40\%&	 50\%&	 60\%&	 70\%&	 80\%&	 90\%&	 100\%\\\cline{1-10} 
%X &X  &X   & &  &  &  &  &   &  
%\\ \hline
%\end{tabular}
%\end{center}
%\caption{Exemplo usando ataque que consuma 30\% de seu félego}
%\label{}
%\end{table}

%A percentagem de quanto cada ação consome do félego é mostrada na tabela a seguir.

%\begin{table}[htbp]
%\begin{center}
%\begin{tabular}{|c|c|c|c|} \hline 
%Equip/Aééo &	 Esquiva &	Ataque&	 Aparar\\\cline{1-4} 
%Leve & 5\%  & 20\%   &  10\% \\\cline{1-4} 
%Médio & 10\%  & 30\%   &  20\% \\\cline{1-4} 
%Pesado & 30\%  & 50\%   &  30\%  
%\\ \hline
%\end{tabular}
%\end{center}
%\caption{Gasto de percentagem de PF}
%\label{}
%\end{table}


%Sempre que essa barra atingir 100\% o personagem perde 1 PF. Para cada ação extra realizada no mesmo turno, o personagem perde metade do que normalmente perde quando realizado aquela ação pela primeira vez. Em outras palavras, para cada 2 ações extras o personagem perde a percentagem associada aquela ação na tabela acima. Por exemplo, se um personagem realizar 3 ataques com uma arma média no mesmo turno ele perde 60\% de seu félego. O consumo do félego da esquiva esté associado na tabela acima está relacionado com o peso do equipamento usado. Em outras palavras, em um turno um personagem com uma arma grande e uma armadura média, perde 60\% de seu félego quando realizado um ataque e uma esquiva. 

%Vale lembrar que em certas batalhas um personagem não é atingido ou realiza nenhum ataque. Nessas situações um personagem não perde PF na batalha. Isso também cria situações de que um grupo salvar os PF de um personagem para serem gastos em situações mais urgentes. 

%Porém vale observar que em algumas batalhas o consumo de PF por ser crucial para definir o seu resultado. Por exemplo, um personagem usando uma arma de porte grande (um machado por exemplo), pode usar a manobra aparar várias vezes (uma vez que usando uma arma de grande porte, ele pode aparar armas de porte médio e pesado). Caso a destreza desse personagem seja muito grande, pode acontecer de que nenhum de seus oponentes consigam atingi-lo. Nesse caso, os oponentes podem bolar uma estratégia de consumo de PF, fazendo com que o alvo use a arma de porte grande várias vezes para aparar, consumindo seus PF lentamente com isso. Nesse caso o mestre deve retirar PF cautelosamente dos personagems e não somente uma vez ao termino da luta, pois o consumo de PF durante ela é decisivo nesse caso.

%Para ágilizar todo o processo de perda de PF durante a luta, o mestre pode ir anotando as ações realizadas por cada personagem (1 ataque, 3 aparar, 2 ataques, não foi atacado, etc), e de 4 em 4 turnos ir retirando de cada personagem o PF perdido até aquele momento. Vale lembrar que o mestre tem total liberdade para interpretar a perda de PF e usa-lé dentro de jogo como bem desejar. Alguns mestres retiram PF apenas quando habilidades ou manobras são usadas.



 


\ClearShipoutPicture

 %\include{AppendixA}
 %\include{AppendixB}

 %\phantomsection 
 %\addcontentsline{toc}{chapter}{Index}
 %\renewcommand{\baselinestretch}{1} \small \normalsize
 %\printindex


\end{document}
