%%
%% Capítulo : magia
%%

\chapter{Magias}
\label{Cap:Magias}


\section{Natureza da Magia}

Existem vários detalhes sobre a natureza da magia em Dregon. Portanto, para seu uso do ponto de vista do sistema deve-se compreender seu funcionamento básico e sua natureza. Aqui iremos mostrar como a magia é estudada, interpretada e usada no mundo de Ederu, o mundo básico usado no sistema Dregon. Os mestres podem fazer quaisquer adaptações desejáveis nas explicações mostradas a seguir. Aqui tentaremos não nos prender a detalhes que prejudiquem a compreensão do leitor, portanto focar na explicação da magia para ser usada no sistema de forma simples.

Em poucas palavras, a magia é a manifestação da energia mágica quando trazida para o meio físico através de um corpo espiritual. O que isso quer dizer? Para compreender esta afirmação é necessário explicar um pouco sobre a teoria dos planos de existência. Dentro do universo conhecido, existe vários planos interagindo entre si. Imagine os planos como líquidos em um copo d'água. Se misturarmos água, areia e gás dentro de um recipiente, poderemos ver uma relação entre os planos de existência. Imagine que o "universo" é o copo, e cada plano de existência (plano físico, mágico e espiritual) sejam esses elementos que colocamos dentro do copo, ou seja, água, areia e gás. Apesar desses elementos serem de naturezas diferentes, eles podem interagir entre si até um certo ponto.

Existem partes em cada plano onde sua proximidade com outros planos é mais acentuada. Por exemplo, em certos locais mal assombrados, podemos dizer que é mais fácil ver espíritos e coisas relacionadas ao mundo espíritual. Ou seja, nesses locais o mundo físico encontra-se  mais próximo do plano espiritual.

Cada plano tem uma infinidade de peculiaridades em relação a seu comportamento existencial. Por exemplo, o comportamento de objetos no plano físico é compreendido através das leis da física. Porém, existe algo que é universal a qualquer plano de existência conhecido. Esse algo é a energia. A energia é manifestada de diversas formas diferentes. Ou seja, se você está segurando uma pedra, essa pedra armazena uma certa quantidade de energia. Chamamos essa energia "parada" de energia potencial. Mas a medida que movemos a pedra, essa energia potencial vai se tornando em "energia de movimento", conhecida como energia cinetica. Da mesma forma que essa energia se manifesta no plano físico como energia potencial ou cinetica, ela é manifestada em outros planos de existencia de outras formas. Aqui nos deparamos com o conceito de vários tipos de energia diferentes, cada um ligado a um plano de existencia.

De uma forma mais simples, a energia do plano físico é a energia física (chamada de ki), a energia do plano mágico é energia mágica (chamada de mana) e a do plano espiritual energia espiritual (conhecida também como aura). Da mesma forma como o exemplo da pedra, uma pessoa tem certo controle sobre a transformação dessa magia. Então voltemos à nossa afirmação sobre o que é magia: "magia é a manifestação da energia mágica quando trazida para o meio físico atravéz de um corpo espiritual". Ou seja, se trouxermos a energia mágica para o plano físico, poderemos ver uma reação mágica, ou seja, a magia em execução. Porém, como o plano físico e mágico são muito diferentes em sua natureza (como a água e o gás no exemplo anterior do copo) para trazer a energia do mundo mágico para o físico é necessário uma espécie de ponte. Essa ponte é o mundo espiritual. 

Em poucas palavras, quando se realiza uma magia o usuário, primeiramente, concentra a energia mágica no espírito do alvo, ou seja, primeiramente ele concentra a magia no espírito do alvo e não no seu corpo. Após o espírito ser atingido, ocorre imediatamente uma reação física. Todo esse procedimento é chamado de efeito mágico.

Aqui podemos ver dois detalhes iniciais importantes sobre a magia. O primeiro é que a natureza da magia não é fisica, ou seja, o alvo da magia não pode usar nada físico para se defender ou melhorar uma magia. Em outras palavras, o guerreiro com a maior defesa física do mundo pode levar dano mágico facilmente caso sua defesa mágica seja baixa. A defesa mágica é representada pelo atributo espírito. A quantidade de energia mágica que um mago pode manipular para realizar um efeito mágico é representada pelo atributo sabedoria. Outro detalhe importante é que, como a magia é acionada no espirito do alvo, apenas criaturas com espirito podem ser afetadas. Portanto se alguém solta uma bola de fogo em uma parede, a magia atravessa a parede, se desfazendo aos poucos à medida que ela prossegue no mundo espiritual. Mesmo com essa limitação alguns magos conseguem materializar a magia, usando seu espirito como catalizador de planos, trazendo todos os efeitos mágicos para o plano físico sem receber qualquer dano no processo. Porém, ao fazê-lo os alvos da magia podem defender com atributos físicos normalmente. 

Seguindo essa lógica, deduzimos que o alvo não pode usar seu atributo esquiva para esquivar de magia. Na verdade, se o alvo não tiver uma habilidade especial chamada de auridade, ele não pode esquivar de uma magia uma vez que ele não tem a capacidade de locomover seu espírito no plano espiritual de modo a esquivar da magia. Semelhantemente, um usuário de magia não precisa mover seu corpo para realizar um efeito mágico, podendo fazê-lo apenas com o poder da mente.

Mesmo a magia sendo uma força que tem início no plano espiritual, sua manifestação é visível por qualquer sensor visual (olhos humanos, câmeras, etc) presente no plano físico.

A quantidade de energia mágica que um usuário de magia pode manipular é determinada pelos seus PM, apenas sua intensidade é determinada pelo seu atributo sabedoria. Em palavras mais simples, o quanto de energia mágica que você pode manipular é determinado pelos seus PM. Quanto maior for sua sabedoria, mais energia você pode manipular eficientemente de uma só vez.

Existem outros vários detalhes da magia que não podemos explicar aqui, devido ao espaço e foco do texto. Por exemplo, as magias são divididas em cores. Magia branca, negra, azul, etc. Cada magia tem uma cor devido a um fato curioso. Imagine se você colocar seu braço em um lago, de superficie calma. Você vai gerar uma onda, que irá percorrer todo o lago. De forma semelhante, quando o usuário da magia "puxa" a energia do plano mágico para o plano espiritual ele cria uma onda. Se as informações dessa onda (comprimento de onda, frequência, etc) forem medidas, ela pode ser associada a uma cor no espectro visível de luz. Ou seja, quando alguém usa uma magia vermelha, ao trazer a energia mágica para o plano físico ele cria uma onda com comprimento de onda e frequência semelhantes à cor vermelha.  


As magias são divididas de acordo com uma cor relacionada a elas. Os tipos são: Magia negra, branca, verde, roxa, vermelha e azul. A magia negra é usada ofensivamente, onde a maioria dos seus usos tem objetivo destrutivo. Também são chamadas como magia de guerra. As magias brancas são usadas mais para defesa ou cura. A magia verde é representada pela magia Ciarú, representante das plantas, vegetais e florestas. A magia roxa é associada à deterioração da matéria. A magia vermelha é conhecida como magia alternativa, sendo dividida em 3 sub-caminhos (forma, mente e espírito). E finalmente a magia azul é manifestada na compreensão da realidade. 

\section{Magia no Sistema}

As explicações feitas na seção anterior têm o propósito de construir uma base para o uso e compreensão da magia, dando uma maior profundidade às campanhas. Nessa seção iremos resumir os pontos que serão usados diretamente no sistema.

\begin{itemize}
	\item Magias, a priori, não podem ser esquivadas com esquiva ou defendidas por qualquer tipo de proteção física (escudos, muros, etc).
	\item Magias tem seu efeito reduzido de acordo com o atributo espirito. Espirito é a defesa magica, assim como defesa é a redução de dano fisico.
	\item Magias, a priori, não podem atingir objetos sem espiritos, como portas, robos, carros, entre outros.
	\item Quando atigindo por uma magia, o alvo não joga espirito. O valor de sua absorção magica é o valor total de seu atributo.
	\item O alvo de uma mágia apenas joga seus espírito para reduzir dano mágico caso a magia realizada seja de toque.
	\item Uma magia é considerada de toque, a priori, se o corpo físico do usuário da magia tocar o corpo físico do alvo da mesma.
	\item O alvo da magia pode anular seu espírito para receber todo a cura de uma magia.	
\end{itemize}


\section{Magia Pura}

A magia pura é a manifestação mais pura, mais natural de um caminho mágico. Ela representa a verdadeira natureza daquela magia, e é manifestada geralmente como uma exploção do elemento relacionado a magia. Por exemplo, dentro das magias negras existe o caminho da magia da terra. Esse caminho é chamado de Terrato. A magia pura de terrato pode ser manifestada como o arremesso de uma grande rocha ou um pilar de pedra que surge no meio da terra. O usuário da magia que deve definir como é a manifestação visual da magia pura. 

Uma magia pura sempre tem 3 dados principais. O primeiro desses dados é chamado de bônus da magia pura. Esse bônus representa o poder da magia pura. Quando usada em ataque, o usuário deve somar sua sabedoria ao bônus da magia pura para saber a quantidade de dano causado. Voltemos ao exemplo de terrato. Quando comprada inicialmente, o bônus da magia pura de terrato é +6. Um mago com sabedoria 8 ao utiliza-la pode causar 14 de dano mágico. Após aprender a magia pura, o usuário da mesma pode aumentar seu bônus em +1 gastando 10 PS. Esse bônus so pode ser aumentando até um limite estabelecido de acordo com o circulo da mágia e o tipo da mesma. O outro dado importanto é a quantidade de PM que uma magia pura consome ao ser utilizada. O usuário pode gastar 10 PS para diminuir o custo da magia pura em 1 PM até o limite mínimo de metade do seu custo inicial. O terceiro e último dado de uma magia pura é o TC necessário para realiza-la. As regras para TC (tempo de concentração) usadas nesse caso são as mesmas descritas no capítulo sobre habilidades.

Mesmo que a magia pura represente uma "explosão energetica", o usuário da magia pura pode explorar sua natureza para obter efeitos especiais. Ou seja, devido a natureza diferente de cada caminho, cada magia pura tem um efeito especial associado a ela. Por exemplo, vejamos a magia branca acua (elemental água). A magia pura de acua pode ser representada como um jato de agua. Porém, o usuário pode explorar de forma melhor a natureza da magia, podendo usa-la para curar pequenos ferimentos.

Alguns caminhos não tem uma magia pura propriamente dita a ela associados, como é o caso da magia vermelha e magia azul. Explicações mais detalhadas sobre essas magias nas suas respectivas sessões.

Para um personagem poder comprar uma magia pura, ele deve determinar o custo total da magia de acordo com a lista favorecimento de habilidades. Por exemplo, firani é uma magia negra. A sua magia pura é considerada uma habilidade de mago negro para determinar o seu custo total. O custo normal de firano é 30 PS, ou seja, um mago negro pode aprende-la normalmente gastando-se 30 PS. Um ladrão pode aprende-la, pagando um custo extra de 50\%, pagando 45 PS. Porém, um guerreiro (classe desfavorecida com a classe mago negro) deve pagar 60PS para aprender a magia firani. Abaixo segue a lista da relação do tipo da magia com a classe.

\begin{itemize}
	\item Magia Negra (firani, malon, terrato, zorin): Toda magia negra pura é considerada habilidade de mago negro para determinar o custo total.

	\item Magia Branca(heris, windan, acua): Toda magia branca pura é considerada habilidade de mago branco para determinar o custo total.

	\item Magia Roxa (Veru):A magia pura veru é considerada habilidade de necromante para determinar o custo total. Além disso necromantes têm livre acesso aos subcaminhos da magia.

	\item Magia vermelha (toronto): A magia vermelha não é considerada uma habilidade de mago vermelho, porém os magos vermelhos têm livre acesso aos seus subcaminhos. Aqueles que não forem magos vermelhos, devem gastar 20 PS e 8xpm (desvaforecidos para a classe mago vermelho devem pagar 24 PS e 14xpm) para abilitar 1 sub-caminho da magia vermelha. 

	\item Magia Verde (ciaru): A magia verde pura ciaru é considerada habilidade de druida para determinar seu custo total.
\end{itemize}

Algumas raças dão acesso a compra de certas magias pelo custo em PS da classe. Por exemplo, zenfru pode comprar zorin pelo custo de 30 PS. Da mesma forma magin da terra e magin do fogo podem comprar firani e terrato, por 30 PS, respectivamente.

\section{Controle Elemental e Habilidades Mágicas}
Após aprender a magia pura, o usuário tem acesso de aprendizado as habilidades daquela magia. O custo de qualquer habilidade mágica relacionada a uma magia é o MESMO independentemente da classe. Ou seja, um guerreiro, mesmo desfavorecido com mago negro, pode comprar as habilidades de firani sem pagar nenhum custo extra. Alguns mestres podem exigir alguns outros pre-requisitos, como por exemplo, controle elemental da magia usada ou um determinado valor mínimo de um atributo mental.


Após aprender a magia pura, o usuário pode gastar PS para melhorar seu bônus ou diminuir seu custo. Além disso, ele pode explorar a natureza básica da magia pura para poder usar um efeito especial, associado a natureza da magia. Porém isso não quer dizer que ele possa fazer somente isso para explorar aquele caminho mágico. Na verdade, isso é apenas o começo.

Um passo básico para a exploração de um caminho mágico é chamado de controle elemental. Gastando 6PS e 8 xpm (independente da classe) o usuário pode comprar a habilidade mágica controle elemental (uma compra para cada magia pura). Essa habilidade permite ao usuário controlar o elemento associado aquela magia pura de que ele comprou o controle elemental. O nível de controle é associado ao atributo geral consciência, ou seja, quanto maior a consciência de um usuário de controle elemental da magia firani, maior vai ser o controle sobre o fogo que o personagem terá. Com o controle elemental o usuário não cria o elemento, e sim o controla de acordo com sua existência no ambiente. Baseado nisso o mestre deve determinar de que forma o usuário vai usa-la dentro de jogo. Abaixo citamos alguns exemplos:

\begin{itemize}
	\item Você pode gastar menos PM quando usar a magia pura. É como se o elemento te desse alguns PM para que a magia possa ser realizada mais facilmente.
	\item Você pode atacar fisicamente com o elemental, causando apenas sua sabedoria de dano. Esse dano pode ser aumentado de acordo com a abundância do elemento no ambiente.
	\item Você pode usar o elemento a seu favor (criar um buraco para seus perseguidores cairem, um ventaval para que você possa pular mais alto etc).
\end{itemize}

As seguintes magias não tem controle elemental devido a sua natureza. Magias vermelhas, azuis e magias do caminho heris. Ou seja, para qualquer magia que não tenha sido citada anteriormente, o usuário da magia pura pode comprar a habilidade mágica controle elemental.

\subsection {Contra Magia(8xpm,10PS)}

Usuários da habilidade contra magia podem aparar ataques mágicos de magias puras elementais durante seu turno de esquiva. O usuário deve usar uma magia de mesmo elemento ou de um elemento favorecido para poder usar esta habilidade. Além disso, magias de um círculo superior não podem ser aparadas por magias de um círculo inferior. O usuário deve realizar um teste de concentração com dificuldade igual ao TC da magia usada para aparar. Cada tentativa de aparar outra magia, faz com que a dificuldade desse teste aumente em 2 pontos. Outros aspectos do ambiente podem aumentar a dificuldade desse teste (o usuário recebe vários ataques físicos ou se ele esta tentando aparar uma magia realizada em um aliado distante). O custo em PM é igual a metade (arredondado para cima) do utilizado normalmente.



\subsection{Materializar Magia(16ps,12xpm)}
 
Essa é uma habilidade geral especial que apenas usuários de magias elementais podem aprender. Por 2 PM o usuário pode materializar uma magia, transformando seus efeitos mágicos em físicos. O oponente absorve o dano da magia jogando defesa normalmente como faria se recebesse um ataque físico e pode se esquivar com esquiva, caso a magia materializada seja de ataque. O usuário deve jogar destreza ou inteligência para acertar o alvo. A magia usada é considerada como um projétil pequeno (mesmo se for uma bolo de fogo enorme) para fins da manobra aparar. O dano causado pela magia é o mesmo dano causado quando lançada normalmente. A parte do dano proveniente da sabedoria é do tipo normal e o restante do tipo automático. Quando não usado em combate, os usuários dessa magia podem gerar uma certa quantidade do elemento gastando PM. Por exemplo, um usuário de firani pode acender uma fogueira gastando poucos PM, enquanto que um usuário de ácua pode enxer um galão de água por aproximadamente 6 PM. Quando usado em grande quantidade, mais dificil é converter energia mágica em física. Por exemplo, gastando-se PM para criar água, o usuário pode beber da mesma sem nenhum problema. Porém, se ele o fizer em grande quantidade (encher uma caixa dágua para várias pessoas beberem), existe uma chance dessa água ter efeitos mágicos nocivos a saúde, devido a presença de energia mágica em sua estrutura. Em poucas palavras, materializar o elemento em poucas quantidade é facil. O "residuo" de energia mágica no elemento materializado pode causar danos ás criaturas vivas. O mestre deve levar em consideração esses fatores. De acordo com o ambiente o dano da magia pode ser aumentando. Por exemplo, um usuário de terrato em uma caverna pode usar materializar magia mais controle elemental para aumentar consideravelmente o dano normal causado.

Essa é uma habilidade reflexiva. Requer nível mental 4 ou mais para aprender a habilidade.  

\subsection {Armas Elementais}
 
Usuários de materializar magia que possuam controle elemental, podem invocar armas de seus respectivos elementais. O custo para invocar essas armas é igual ao custo normal da magia mais 4. A arma tem uma quantidade de pontos para serem distribuidas igual ao bônus da magia, onde esse bônus não pode ultrapassar a inteligência do usuário. Essa quantidade de pontos diz respeito a regra de criação de equipamentos, descrita detalhadamente no capítulo sobre equipamentos. Basicamente 1 ponto concede +1 de dano do tipo automático, 2 pontos concedem /3 de dano do tipo normal e 1 ponto concede 10 PR. A quantidade de PR inicial da arma é igual ao focus do usuário vezes 2. A arma não tem redução de atributos devido a perda de PR, apenas quando o mesmo é reduzido a zero. A arma é considerada uma arma mágica de slot nulo (pode ser usada com qualquer arma mágica) que causa dano físico. A arma tem duração em minutos igual a concentração do usuário. Outras pessoas com conhecimento de magia ou forjar podem perceber que a arma é magica. Considere essa habilidade como uma magia de TC 10.

Essa habilidade mágica especial pode ser usada para criar projéteis, com bônus semelhantes as armas normais. Ao usar o projétil, o mesmo é destruido imediatamente após seu uso. O usuário pode criar uma quantidade de projéteis igual ao bônus de focus - 1, gastando a mesma quantidade de PM gasta normalmente ao criar uma arma.

Armas criadas com os elementais terra, planta e gelo, podem ser usadas por qualquer pessoa. Armas criadas com os demais elementais podem apenas ser usadas por usuários que tenham controle elemental dos respectivos elementos.

Caso o usuário deseje criar uma arma que ele não tenha a péricia bélica, o mestre pode exigir um teste de inteligência para tal.


\section{Alvos magicamente imunes}
A magia em sua forma mais simples, tem efeito em alvos cuja existência energetica possa ser manifestada no plano mágico ou planos próximos à ele, como o plano espiritual por exemplo. Pondo essa afirmação de um jeito mais simples, uma magia básica só pode atingir um alvo caso o mesmo tenha espírito, ou alguma relação com o plano mágico. Com isso em vista construtos, objetos da natureza, entre outros, são imunes a mágia. Mortos vivos são imunes á basicamente qualquer tipo de magia, com exceção da magia de cura. Essa causa danos em mortos vivos.

Alvos magicamente imunes podem ser acertados por uma magia em seu estado materializado, ou por exemplo, por um elemento da natureza que está sob efeito da habilidade mágica controle elemental. O fogo gerado por uma magia de fogo, por exemplo, pode ser um fogo físico ou mágico, de acordo com a situação. Como o efeito de uma magia é físico, caso usado em um alvo com espírito (que é o elemento de ligação entre o plano mágico/espiritual e o plano físico), aquele fogo gerado pela mágica também é fisico. Porém, ele tem duração somente enquanto a magica tem efeito. Por exemplo, em termos de sistema, se o dano de uma bola de fogo mágica for muito grande, isso pode ser interpretado como uma grande chama em volta do alvo. Se o usuário da mesma tiver controle elemental, ele pode depois se aproveitar disso para controlar e aumentar o fogo. Usando o mesmo para atingir alvos magicamente imunes.



\section{Conseguindo e Usando PS}

O PS (ponto de spiritum) represente a criatividade e desenvoltura mágica. Ou seja, sempre que um personagem usar sua magia com criatividade ou ver um efeito mágico desconhecido, o mestre pode dar uma certa quantidade de PS para o mesmo. Cada magia pura ou habilidade mágica (exceto controle elemental) usada pela primeira vez concede ao personagem de 4 a 6 PS. Um personagem pode obter até 10 PS por sessão de acordo com o mestre. Todo personagem quando usando sua magia tem um grau de liberdade para descreve-la. Por exemplo, firani é a magia pura de fogo. Com ela o personagem realiza um ataque usando essa magia. Esse ataque é descrito pelo personagem, podendo ser uma simples bola de fogo até um redemoinho de chamas azuis. Quanto mais criativo for o personagem em relação ao uso de suas magias, mais o mestre deve recompensa-lo com PS. O personagem também pode ganhar PS de acordo com feitos magicos realizados durante a aventura. Encontrar com uma criatura magica, ou adentrar um local fortalecido magicamente, podem render PS extras para os jogadores. Quem decide isso é o mestre. 


Os PS são usados para diversas finalidades. A principal delas é comprar magias ou habilidades ligadas a magia, seja de classe ou habilidades da magia. Por exemplo, para o personagem comprar windam, que é a magia pura de vento, ele de gastar PS. Para comprar mãos de vento, habilidade relacionada a magia windan, ele deve gastar PS e também xpm. O personagem também pode gastar PS para abaixar o custo de uma magia pura ou aumentar seu dano. O custo de uma magia pode ser reduzida em 1 gastando-se 10 PS até o mínimo da metade do custo original. Seu bônus de dano pode ser extendido de acordo com a círculo da magia.

Existem alguns lugares no mundo conhecidos como santuários. Nesses locais o jogador pode trocar 1 xpm por 2 PS atravéz de meditação. Existem outros efeitos especiais que variam de santuário para santuário, por exemplo, em raros santuários voce pode trocar 1 xpf por 1 PS, receber um bônus sempre que aumentar PM, trocar pontos de PV ou PF por PM, e vice versa. Geralmente nesses últimos casos o mestre deve limitar a quantidade de vezes que ele pode evoluir rapidamente. Essa limitação é anula ou mensal de acordo com um atributo mental. Alguns desses locais são protegidos e até mesmo procurados por ceder essa habilidade de evolução rápida.

\section{Círculo da Magia}

\begin{itemize}
 
	\item Primeiro círculo : Magias brancas e verdes podem ter seu bônus aumentado até +10, geralmente com valor inicial de +4. Magias negras e roxas podem ter seu bônus aumentado até +15, geralmente com valor inicial de +6. O límite de sabedoria usado para calculo de dano/cura é igual a 20. Geralmente têm TC 14, custo inicial de 6 PM (podendo ser reduzido até 3 com o uso de PS) e custo de 30 PS.

	\item Segundo círculo : Magias brancas e verdes podem ter seu bônus aumentado até + 30, geralmente com valor inicial de +20. Magias negras e roxas podem ter seu bônus aumentado até + 40, geralmente com valor inicial de +25. O límite de sabedoria usado para calculo de dano/cura é igual a 30. Geralmente têm TC 18, custo inicial de 12 PM (podendo ser reduzido até 6 PM com o uso de PS) e custo de 60 PS.

%	\item Terceiro círculo : Magias brancas e verdes podem ter seu bônus aumentado até + 50, geralmente com valor inicial de +40. Magias negras e roxas podem ter seu bônus aumentado até + 60, geralmente com valor inicial de +45. Sem o límite de sabedoria usado para calculo de dano/cura. Geralmente têm TC 20, custo inicial de 20 PM (podendo ser reduzido até 10 PM com o uso de PS) e custo de 100 PS.


\end{itemize}


%primeiro circulo: 6pm/3pm -> +4 ate +10 ou +6..+15. TC10. (sab stack on 20)
%segundo circulo: 12pm/6pm -> +15 ate +25 ou +20..+35. TC10. hab extra. (sab stack on 30)




\section{Carregando uma Magia}

Como dito anteriormente, o tempo de concentração necessário para usar uma magia varia de acordo com o TC da mesma. Porém o usuário da magia pode concentrar uma quantidade de turnos extras para que a magia receba um bônus. Para cada turno extra gasto carregando a magia, ela recebe um bônus no dano igual ao bônus de sua sabedoria, porém ao custo de 2 pm adicionais por turno extra concentrado. Essa regra somente é válida para magias puras. Um personagem só pode concentrar uma magia até um número de turnos extras igual ao seu valor de concentração. Além disso o bônus em dano de uma magia não pode ultrapassar o valor máximo de permitido pelo seu círculo. Um personagem so recebe esse bônus se assim desejar, ou seja, ele pode concentrar uma magia por turnos extras esperando o melhor momento para utiliza-la sem receber o bônus e a perda de PM adicionais. Em condições normais, independente da sua concentração, ele pode carregar somente uma magia por vez. 


\section{Magias de Primeiro Nível}

Além da limitação de aumento do seu bônus, as magias de primeiro círculo tem uma limitação em relação a sabedoria do usuário que pode ser usada em conjunto com elas. Para magias de primeiro círculo esse limite é de 20, ou seja, mesmo que um personagem com sabedoria 30 use uma magia de primeiro círculo, para fins de valor total usado, a magia consegue suportar até sabedoria 20. Em um exemplo mais simples, se um usuário de sabedoria 30 soltar uma magia de primeiro círculo que tem bônus +3, o dano total não vai ser 33 e sim 23.

A seguir listaremos todas as magias de primeiro círculo.

\subsection{Firani}

\begin{itemize}
	\item Magia Pura: 6pm, +6, 30PS. Magia Negra.TC14. Elemento fogo.\newline
--->Especial Primeiro círculo (chamas da concentração): Para cada turno extra concentrado, o dano da magia aumenta em +3.
\newline
--->Especial Segundo círculo (chamas da perseverança): O alvo recebe dano mágico de toque, que ignora a armadura, igual ao bônus de sabedoria do usuário da magia. A duração do efeito é igual ao bônus de inteligência do usuário da magia. O efeito é considerado como um status negativo, e pode ser cancelado com magia de fogo ou de àgua. Esse efeito tem ativação automática quando firani 2 é usado.

	\item Barreira de Fogo(20ps,10xpm): Custo normal de firani + 4 pm. TC 14.\newline
O usuário cria uma barreira mágica que o envolve, reduzindo todo o dano mágico que entrar na barreira em Sabedoria + 2 durante concentração turnos. O usuário também pode gastar 1 pm para aumentar a redução da barreira em 1 até o valor de sua inteligência. Nos últimos 2 turnos a proteção da barreira cai pela metade. Para qualquer magia de vento e de fogo a barreira tem o dobro da defesa. Porém para qualquer magia de agua ela tem apenas metade. Caso o usuário tenha a habilidade materializar magia, pelo mesmo custo ela pode criar uma barreira física, com dados e comportamentos semelhantes se ela fosse invocada como barreira mágica, mas como a natureza da magia é diferente são necessárias algumas observações. A barreira criada tem um valor de defesa automática igual a Sabedoria + 2 (a defesa da barreira pode ser aumentada da mesma forma se ela fosse mágica), e agora ela tem uma certa quantidade de PV igual ao dobro da consciencia do alvo + o bônus da magia pura. O oponente tem duas opções quando ataca um usuário envolto pela barreira manifestada fisicamente. A primeira é atacar diretamente a barreira, afim de destrui-la. O atacante sempre acerta a barreira, porém não recebe bônus de acerto ao faze-lo. A barreira defende com sua defesa normal e todo o dano é reduzido dos PV totais da barreira. O atacante também pode optar por tentar ultrapassar a barreira. Dessa forma ele tem um bônus de destreza durante o ataque (como se estivesse realizando um ataque inesperado), porém recebe dano por tal feito. O dano infligido pela barreira é de natureza físico, e é igual ao dobro do dano causado pela magia pura (o dano de sabedoria é considerado normal e o bônus da magia é considerado automático). Se o dano causado for maior que o bônus de resistência do alvo, o mesmo é arremessado para fora da barreira. Sua fraqueza e resistência a elementais também continuam as mesmas. 

	\item Chuva de Fogo(10 ps, 16 xpm): Custo normal de firani + 6 pm. TC normal de firani + 6.\newline
Ataque de firani em area com raio de sabedoria metros em todos os alvos próximos a uma área escolhida pelo usuário. O dano de firani é incrementado em +10. O usuário pode gastar 2 PM extras para livrar 1 aliado dentro da área da magia. O número máximo de aliados livrados dessa forma não pode ultrapassar o bônus de inteligência do usuário. Caso a magia já seja considerada de área, a mesma recebe um aumento em sua area de ataque de 50\%.

	\item Fúria do Fogo(20 ps, 10 xmp): Custo normal de firani + 2pm. TC14.\newline
Teste do dano de firani contra força de vontade do alvo. Durante 2 + sucessos turnos, o alvo fica com as condição negativa beserk.

	\item Fire Steel(10 ps, 12xpm): Custo de 6 PM. TC 14.\newline
O usuario concentra a magia olhando para algum equipamento do alvo. Se o alvo falhar em um teste de força de vontade com DF 22, ele deve soltar imediatamente a arma. Se passar no teste e desejar resistir ao calor emanado pela arma, o usuario recebe 4 PV de dano sem poder absorve-lo. Para cada PM extra gasto ao usar essa magia, a dificuldade do teste aumenta em 1 ponto.  

	\item Explosão de Fogo(10 xpm): Dobro do custo normal de firani. TC normal de firani + 4. \newline
O usuário ataca com firani normalmente causando apenas o bônus da magia de dano (sem contar sabedoria, porém outros bônus podem ser inseridos, como os das habilidades focus mágico ou focus elemental). O usuário recebe dano físico (do tipo automático) e dano mágico ao mesmo tempo (cada tipo de dano absorvido com sua respectiva defesa). Para fins de esquiva esse poder é considerado uma magia. Pré-requisito: Materializar Magia.

\end{itemize}


\subsection{Malon}

\begin{itemize}
	\item Magia Pura: 6pm, +6, 30PS. Magia Negra. TC14. Elemento Raio.\newline
--->Especial Primeiro círculo (condutividade mágica): Se o alvo tiver muitos equipamentos metalicos, o mestre pode adicionar um bônus no dano da magia. Além disso se outro alvo estiver próximo ao alvo inicial, o usuario pode gastar reflexivamente a mesma quantidade de PM para acertar essa outra pessoa como se tivesse usando a mesma magia, ou o mestre pode ajustar o custo e o dano como bem entender.\newline
--->Especial Segundo círculo (Extashock): O alvo realiza um teste de resistência contra sabedoria mais bônus da magia do usuário da mesma. Caso não passe no teste, o alvo recebe um redutor igual ao bônus de sabedoria do usuário da magia em qualquer teste durante o próximo turno.


	\item Barreira de Raio(20ps,10xpm): Custo normal de malon + 4 pm. TC 14.\newline
O usuário cria uma barreira mágica que o envolve, reduzindo todo o dano mágico que entrar na barreira em Sabedoria + 2 durante concentração turnos. O usuário também pode gastar 1 pm para aumentar a redução da barreira em 1 até o valor de sua inteligência. Nos últimos 2 turnos a proteção da barreira cai pela metade. Para qualquer magia de raio e de fogo a barreira tem o dobro da defesa. Porém para qualquer magia de terra ela tem apenas metade. Caso o usuário tenha a habilidade materializar magia, pelo mesmo custo ela pode criar uma barreira física, com dados e comportamentos semelhantes se ela fosse invocada como barreira mágica, mas como a natureza da magia é diferente são necessárias algumas observações. A barreira criada tem um valor de defesa automática igual a Sabedoria + 2 (a defesa da barreira pode ser aumentada da mesma forma se ela fosse mágica), e agora ela tem uma certa quantidade de PV igual ao dobro da consciencia do alvo + o bônus da magia pura. O oponente tem duas opções quando ataca um usuário envolto pela barreira manifestada fisicamente. A primeira é atacar diretamente a barreira, afim de destrui-la. O atacante sempre acerta a barreira, porém não recebe bônus de acerto ao faze-lo. A barreira defende com sua defesa normal e todo o dano é reduzido dos PV totais da barreira. O atacante também pode optar por tentar ultrapassar a barreira. Dessa forma ele tem um bônus de destreza durante o ataque (como se estivesse realizando um ataque inesperado), porém recebe dano por tal feito. O dano infligido pela barreira é de natureza físico, e é igual ao dobro do dano causado pela magia pura (o dano de sabedoria é considerado normal e o bônus da magia é considerado automático). Se o dano causado for maior que o bônus de resistência do alvo, o mesmo é arremessado para fora da barreira. Sua fraqueza e resistência a elementais também continuam as mesmas. 

	\item Relâmpago(16 ps, 12 xpm): Custo do malon normal + 6 pm. TC normal de malon + 6.\newline
Ataque de malon em area com raio de focus metros em todos os alvos próximos a uma área escolhida pelo usuário. O dano do malon é incrementado em +5. O usuário pode gastar 2 PM extras para livrar 1 aliado dentro da área da magia. O número máximo de aliados livrados dessa forma não pode ultrapassar a concentração do usuário.

	\item Paralisia(20PS,12xpm): Custo de 8 PM. TC14.\newline
Teste entre resistência do alvo e focus do usuário da magia. Se o usuário ganhar, o alvo não pode realizar ataques físicos durante uma quantidade de turnos igual ao sucesso obtido no teste pelo usuário  (max bônus de concentração do usuário). O usuário pode gastar 5 PS para reduzir o custo da magia em 1 PM até o mínimo de 4 PM. Vale lembrar que qualquer bônus cedido ao atributo geral resistência pode ser usado no teste contra essa magia. Pré-requisito: Materializar Malon ou Materializar Magia. Essa magia não tem efeito em alvos cujo mais de 80\% do corpo seja composto de materiais não-magnéticos. 

	\item Campo Magnetico(12PS,10xpm): Custo normal de malon + 4 pm. TC 14.\newline
O usuário cria um campo magnetico em sua volta durante concentração turnos. Todos os projéteis pequenos ou médios, que não são feitos de materiais não-magnéticos, são desviados autómaticamente em angulos definidos pelo mestre de acordo com as propriedades mágneticas do projétil. 

\end{itemize}

\subsection{Terrato}

\begin{itemize}
	\item Magia Pura: 6pm, +6, 30PS. Magia Negra. TC14. Elemento Terra.\newline 
--->Especial Primeiro círculo (Magic Sand) : O usuário pode recuperar PR ou atributos perdido dos equipamentos com o uso de PM. Cada PM recupera aproximadamente 5PR/1 atributo. O usuário não pode recuperar mais do que inteligência mais bônus da magia pura de um único equipamento por dia.\newline
--->Especial Segundo círculo (Earth breath) : Caso o alvo da mágia esteja em solo, a magia é considerada de toque. Caso algum efeito de toque já esteja sendo realizado, a magia recebe um bônus no seu dano igual ao bônus de sabedoria do usuário.


	\item Barreira de Terra(20ps,10xpm): Custo normal de terrato + 4 pm. TC 14.\newline
O usuário cria uma barreira mágica que o envolve, reduzindo todo o dano mágico que entrar na barreira em Sabedoria + 2 durante concentração turnos. O usuário também pode gastar 1 pm para aumentar a redução da barreira em 1 até o valor de sua inteligência. Nos últimos 2 turnos a proteção da barreira cai pela metade. Para qualquer magia de terra e de raio a barreira tem o dobro da defesa. Porém para qualquer magia de gelo ela tem apenas metade. Caso o usuário tenha a habilidade materializar magia, pelo mesmo custo ela pode criar uma barreira física, com dados e comportamentos semelhantes se ela fosse invocada como barreira mágica, mas como a natureza da magia é diferente são necessárias algumas observações. A barreira criada tem um valor de defesa automática igual a Sabedoria + 2 (a defesa da barreira pode ser aumentada da mesma forma se ela fosse mágica), e agora ela tem uma certa quantidade de PV igual ao dobro da consciencia do alvo + o bônus da magia pura. O oponente tem duas opções quando ataca um usuário envolto pela barreira manifestada fisicamente. A primeira é atacar diretamente a barreira, afim de destrui-la. O atacante sempre acerta a barreira, porém não recebe bônus de acerto ao faze-lo. A barreira defende com sua defesa normal e todo o dano é reduzido dos PV totais da barreira. O atacante também pode optar por tentar ultrapassar a barreira. Dessa forma ele tem um bônus de destreza durante o ataque (como se estivesse realizando um ataque inesperado), porém recebe dano por tal feito. O dano infligido pela barreira é de natureza físico, e é igual ao dobro do dano causado pela magia pura (o dano de sabedoria é considerado normal e o bônus da magia é considerado automático). Se o dano causado for maior que o bônus de resistência do alvo, o mesmo é arremessado para fora da barreira. Sua fraqueza e resistência a elementais também continuam as mesmas. 

\item Muro De Pedra (14xpm): O usuário gasta reflexivamente o custo normal de terrato quando concentrando a magia barreira de terra. A defesa da barreira recebe um bônus igual ao bônus da magia pura terrato.

\item Cristal Mágico (10PS,10xpm): O usuário gasta reflexivamente x PM no momento do uso da magia terrato. Caso o usuário esteja usando alguma proteção mágica proveniente de equipamento, esse equipamento recebe uma redução de 2x em sua defesa mágica, até o maximo do bonus de focus do usuário. 

\item Pele de Pedra(12xpm): Custo normal de terrato + 2 PM. TC14.\newline
Cria uma crosta de pedra na pele do alvo que concede defesa autómatica + 4. Pode ser gastos PM extras para aumentar a proteção da barreira em 2 PM gastos pra +1 de defesa aumentado, até o máximo da inteligência do usuário. O efeito dura concentração turnos. A crosta criada tem uma quantidade de PR igual ao dobro do focus do usuário mais o bônus da magia pura. A crosta protetora não tem redutor em sua defesa variando de acordo com sua perda de PR, apenas se esse chegar a zero. 

\item Tempestade de Areia(10ps, 12xpm): Custo de 10 PM. TC16.\newline
Cria uma tempestate de areia em um raio de focus metros em sua volta. Para cada PM extra gasto ao concentrar a magia, o raio é aumentado em 1 metro. Pode-se gastar até o valor de concentração em PM dessa forma. O usuário pode sentir os oponentes dentro da tempestade, mas dificilmente consegue exergar quem encontra-se fora dela. Todos aqueles dentro da tempestade recebem uma penalidade na destreza e esquiva igual ao bônus de sabedoria do usuário da magia. Aqueles que desejem atacar a distância e estiverem fora da tempestade, recebem o dobro dessa penalidade em suas jogadas de ataque. Se algum dos alvos dentro da tempestade tiver algum controle sobre o elemento terra (seja com a magia terrato ou com a habilidade ton no jutsu) ele pode tentar cancelar o efeito apenas em volta dele (com gasto de PM/PF determinado pelo mestre), se passar em um teste de sabedoria ou espírito contra a sabedoria do criador da tempestade. O usuário pode gastar 1 PM extra para cada aliado  dentro da tempestade para retirar 1 ponto na penalidade aplicada a sua destreza e sua esquiva. Com um maior gasto de PM extra ele pode reduzir a penalidade ainda mais. A magia tem duração em turnos igual a concentração do usuário.

	\item Mal da Areia(20ps, 8xpm): Custo normal de terrato + 4 PM. TC14.\newline
O usuário causa dano ao equipamento dos oponentes. Sempre que usa essa magia o equipamento do alvo perde uma quantidade de PR igual ao bônus da magia pura + o dobro de pm extras usados (max sabedoria). Gastando 6 PM extra, essa magia é considerada em área para todos os alvos dentro da área de efeito da habilidade mágica tempestade de areia.

	%\item Corpo Pesado(10PS, 12xpm): Custo normal de terrato + 2 PM. TC10.\newline
%O usuário realiza um teste de sabedoria contra o espírito do alvo (caso a mágia seja considerada de toque o sucesso é automático igual a sabedoria do usuário). Caso obtenha sucesso o alvo tem uma redução em sua esquiva e destreza iguais a metade do bônus da magia pura. A redução fica ativa durante 2 + sucessos turnos.


\end{itemize}


\subsection{Zorin}

\begin{itemize}
	\item Magia Pura: 6pm, +6, 30PS. Magia Negra. TC14. Elemento Gelo.\newline
--->Especial Primeiro círculo (Essência Elemental): Ao concentrar a magia pura o usuário pode trocar o tipo do elemento para fins de determinação de dano gastando-se 1 PM extra.\newline
--->Especial Segundo círculo (Ajuda Elemental): O dano da magia é aumentado de acordo com a temperatura ambiente. O bônus concedido pelo ambiente é no máximo igual a sabedoria do usuário.

\item Barreira de Gelo(20ps,10xpm): Custo normal de Zorin + 4 pm. TC 14.\newline
O usuário cria uma barreira mágica que o envolve, reduzindo todo o dano mágico que entrar na barreira em Sabedoria + 2 durante concentração turnos. O usuário também pode gastar 1 pm para aumentar a redução da barreira em 1 até o valor de sua inteligência. Nos últimos 2 turnos a proteção da barreira cai pela metade. Para qualquer magia de gelo e de veneno a barreira tem o dobro da defesa. Porém para qualquer magia de planta ela tem apenas metade. Caso o usuário tenha a habilidade materializar magia, pelo mesmo custo ela pode criar uma barreira física, com dados e comportamentos semelhantes se ela fosse invocada como barreira mágica, mas como a natureza da magia é diferente são necessárias algumas observações. A barreira criada tem um valor de defesa automática igual a Sabedoria + 2 (a defesa da barreira pode ser aumentada da mesma forma se ela fosse mágica), e agora ela tem uma certa quantidade de PV igual ao dobro da consciencia do alvo + o bônus da magia pura. O oponente tem duas opções quando ataca um usuário envolto pela barreira manifestada fisicamente. A primeira é atacar diretamente a barreira, afim de destrui-la. O atacante sempre acerta a barreira, porém não recebe bônus de acerto ao faze-lo. A barreira defende com sua defesa normal e todo o dano é reduzido dos PV totais da barreira. O atacante também pode optar por tentar ultrapassar a barreira. Dessa forma ele tem um bônus de destreza durante o ataque (como se estivesse realizando um ataque inesperado), porém recebe dano por tal feito. O dano infligido pela barreira é de natureza físico, e é igual ao dobro do dano causado pela magia pura (o dano de sabedoria é considerado normal e o bônus da magia é considerado automático). Se o dano causado for maior que o bônus de resistência do alvo, o mesmo é arremessado para fora da barreira. Sua fraqueza e resistência a elementais também continuam as mesmas.


\item Ice Steel(20ps, 10xpm): Custo normal de zorin + 4 PM. TC14.\newline
O usuário causa dano ao equipamento dos oponentes. Sempre que usa essa magia o equipamento do alvo perde uma quantidade de PR igual ao bônus da magia pura + o dobro de pm extras usados (max sabedoria). Gastando 6 PM extra, essa magia é considerada em área para todos os alvos dentro da área de efeito da habilidade mágica Nevasca

	\item Corpo Frio(8PS, 12xpm): Custo normal de zorin + 4 PM. TC14.\newline
O usuário realiza um teste de sabedoria contra o espírito do alvo (caso a mágia seja considerada de toque o sucesso é automático igual a sabedoria do usuário). Caso obtenha sucesso o alvo tem uma redução em sua força e defesa iguais a metade do bônus da magia pura. A redução fica ativa durante 2 + sucessos turnos.

	\item Nevasca(20ps, 12xpm): Custo normal de zorin + 6 PM. TC normal de zorin + 6.\newline
Cria uma nevasca em um raio de focus metros em sua volta. Para cada PM extra gasto ao concentrar a magia, o raio é aumentado em 1 metro. Pode-se gastar até o valor de concentração em PM dessa forma. É dificil de se enxergar dentro da nevasca, causando um redutor de 2 em destreza e esquiva para todos dentro da nevasca. O usuário ou usuários de controle elemental do gelo, podem sentir os oponentes dentro da tempestade, anulando assim esse redutor, mas dificilmente consegue exergar quem encontra-se fora dela. Todos aqueles dentro da tempestade recebem dano físico do tipo automático igual ao bônus da magia pura zorin. Esse dano é elemental do tipo gelo e ignora qualquer armadura no momento da jogada de absorção. Pré-requisito: Materializar Magia. A magia tem duração em turnos igual a concentração do usuário.

	\item Sangue Frio(12xpm): Custo normal de zorin + 2 PM. TC14.\newline 
O usuário joga focus contra a resistência do alvo. Caso falhe, o alvo perde uma quantidade de PF igual a metade do bônus (arredondado para cima) da magia pura zorin.

\end{itemize}



\subsection{Acua}

\begin{itemize}
	\item Magia Pura: 6pm, +4, 30PS. Magia Branca. TC14. Elemento Água.\newline
--->Especial Primeiro Círculo (Água da cura): O usuário pode curar pequenas quantidades de PV com o uso normal da magia. O PV recuperado é igual a sabedoria do usuário.\newline
--->Especial Segundo Círculo (Water Life): O especial de primeiro círculo também pode ser usado para recuperar PF.

\item Barreira de Água(20ps,10xpm): Custo normal de Acua + 4 pm. TC 10.\newline
O usuário cria uma barreira mágica que o envolve, reduzindo todo o dano mágico que entrar na barreira em Sabedoria + 2 durante concentração turnos. O usuário também pode gastar 1 pm para aumentar a redução da barreira em 1 até o valor de sua inteligência. Nos últimos 2 turnos a proteção da barreira cai pela metade. Para qualquer magia de agua e de planta a barreira tem o dobro da defesa. Porém para qualquer magia de veneno ela tem apenas metade. Caso o usuário tenha a habilidade materializar magia, pelo mesmo custo ela pode criar uma barreira física, com dados e comportamentos semelhantes se ela fosse invocada como barreira mágica, mas como a natureza da magia é diferente são necessárias algumas observações. A barreira criada tem um valor de defesa automática igual a Sabedoria + 2 (a defesa da barreira pode ser aumentada da mesma forma se ela fosse mágica), e agora ela tem uma certa quantidade de PV igual ao dobro da consciencia do alvo + o bônus da magia pura. O oponente tem duas opções quando ataca um usuário envolto pela barreira manifestada fisicamente. A primeira é atacar diretamente a barreira, afim de destrui-la. O atacante sempre acerta a barreira, porém não recebe bônus de acerto ao faze-lo. A barreira defende com sua defesa normal e todo o dano é reduzido dos PV totais da barreira. O atacante também pode optar por tentar ultrapassar a barreira. Dessa forma ele tem um bônus de destreza durante o ataque (como se estivesse realizando um ataque inesperado), porém recebe dano por tal feito. O dano infligido pela barreira é de natureza físico, e é igual ao dobro do dano causado pela magia pura (o dano de sabedoria é considerado normal e o bônus da magia é considerado automático). Se o dano causado for maior que o bônus de resistência do alvo, o mesmo é arremessado para fora da barreira. Sua fraqueza e resistência a elementais também continuam as mesmas. 

	\item Canhao D'Agua(10xpm): Custo normal de Acua. TC 14.\newline 
O alvo é empurrado sabedoria + falhas metros se não passar em um teste combatido de força contra o dano da magia. Esse ataque é considerado de natureza física podendo ser desviado com o atributo esquiva. O usuário pode usar concentração ou destreza para jogar seu acerto. Pré-requisito: Materializar Magica ou Materializar Acua.

	\item Pureza do Liquido(6xpm): O usuário pode verificar se existem impurezas na agua com o gasto reflexivo de 1 PM. Quanto maior for sua concentração maior ele pode saber dos detalhes sobre a impureza. Ele também pode purificar esse liquido, tornado-o puro com o uso de PM extras definido pelo mestre de acordo com o grau de impureza do liquido em relação ao seu estado original.

	\item Comunhão com a Água(10ps,6xpm): Gastando-se 2 pm por hora o mago pode respirar em baixo d´agua normalmente. Além disso o usuário pode gastar a mesma quantia para andar sobre qualquer tipo de líquido. Vale lembrar que andar sobre um líquido quente ainda pode retirar PV do usuário. Magia reflexiva.

	\item Água Sagrada(12xpm): Uma vez por dia o usuário pode gastar 1 uso de acua normal para criar um liquido especial que deve ser armazenado imediatamente. Aquele que beber desse líquido dentro de 1 hora recupera uma quantidade de PF igual ao bônus da magia acua.
	
\item Jaula De Água(20ps, 12xpm): Custo normal de acua + 8. TC normal de acua + 6.\newline
O usuario cria uma bolha de água preenchida por água mágica em torno de 1 alvo. O alvo perde 2 PF por cada turno (exceto aqueles que podem respirar em baixo dágua) e recebe um redutor em destreza, esquiva e força igual ao bônus da magia pura acua. Se o alvo tiver algum controle sobre o elemento água (seja com a magia acua ou com a habilidade ton no jutsu) ele pode tentar cancelar o efeito apenas em volta dele (com gasto de PM/PF determinado pelo mestre), se passar em um teste de sabedoria ou espírito contra a sabedoria do criador da magia. A magia tem duração em turnos igual a concentração do usuário.

	\item Manipular o Sangue(12xpm): Custo de 10 PM. TC14.\newline
Teste entre resistência do alvo e focus do usuário da magia. Se o usuário ganhar, o alvo não pode realizar ataques físicos durante uma quantidade de turnos igual ao sucesso obtido no teste pelo usuário. O usuário pode gastar 5 PS para reduzir o custo da magia em 1 PM até o mínimo de 4 PM. Vale lembrar que qualquer bônus cedido ao atributo geral resistência pode ser usado no teste contra essa magia. Pré-requisito: Materializar Magia. Essa magia não tem efeito sobre inimigos que não tem liquido em seu corpo.
	
	
\end{itemize}

\subsection{Heris}

\begin{enumerate}
	\item Magia Pura: 6pm, +4, 30PS. Magia Branca de Cura. Elemental Luz. TC14.\newline
--->Especial Primeiro Círculo (Luz da Vida): O alvo pode usar a magia para purificar corpos, livrando-os de se tornarem mortos vivos, ou causar dano em mortos vivos com o uso da magia pura.\newline
--->Especial Segundo Círculo (Holy Light): A magia de primeiro círculo pode ser usada para causar dano do elemento Luz. Além disso a magia pode receber um bônus ou redutor de acordo com a carisma do alvo. O bônus não pode ser maior que o bônus de sabedoria do usuário.

	\item Raio Ofuscante(12 xpm): Custo de heris + 4. Magia instantânea.\newline
A magia tem efeito durante 4 turnos em um raio de sab/metros em volta do usuário. Todos os inimigos dentro da área que tentem acertar o usuário da magia tem um redutor igual a metade (arredondado para cima) do bônus da magia pura. 

	\item Regeneração(20ps,10xpm): Custo normal de heris + 2. TC10.\newline
Durante concentração turnos, o alvo recupera em PV metade do bônus da magia pura (arredondado para cima) +2 PV por turno.

\item Auspan(20ps,12xpm): Custo normal de heris + 6. TC normal de heris + 4.\newline
Durante concentração turnos o alvo fica protegido com uma barreira mágica ou física cujo valor de defesa é igual ao bônus da magia heris. A defesa física é do tipo automática.

	\item Dom da Amizade(12xpm): Quando usando a magia héris, o usuário pode gastar metade do custo de heris (arredondado para baixo) reflexivamente para cada alvo que não seja o alvo da magia pura. O usuário pode curar desse alvo uma quantidade de PV igual ao bônus de heris. O número de alvos extras não pode ultrapassar o bônus de concentração do usuário.
	
	\item Encorajar da Luz(8xpm): Custo normal de héris - 1 PM.\newline
O alvo recebe um bônus para teste de medo igual ao bônus da magia pura heris. Habilidade Reflexiva.

	\item Aris(20ps,8xpm): Custo normal do heris. TC 10.\newline
O usuário pode usar essa magia apenas em uma pessoa por vez. Durante 24 horas, se o alvo ficar com PV negativos e não morrer ele se levanta com quantidade de PV igual ao carisma do mago. Os PM usados para essa magia são acometidos, ou seja, até que a magia se desfaça o usuário não pode recuperar a quantidade de PM usada para ativar a magia.

	\item Fari (8ps,8xpm): 1 pm. Magia reflexiva.\newline
O usuário com apenas o olhar pode parar o sangramento de 1 alvo, prologando o tempo de vida do alvo em 30 minutos. Pode ser usado em multiplos alvos, apenas gastando a quantidade extra de PM. O número de alvos extras é igual a concentração do usuário.

	\item Esuna (30ps,6xpm): Custo de 6 PM. TC10.\newline
O usuário tem o poder de anular uma magia ou habilidade que conceda penalidades temporárias ou condições negativas. Vale observar que se o alvo está sob efeito de várias magias que concedam penalidades, o usuário deve escolher somente um dos efeitos para cancelar. O custo dessa magia pode ser reduzido em 1 PM para cada 10Ps gastos até o custo minimo de 4 PM. O usuário pode fazer um teste reflexivo de inteligência para perceber se um alvo está sendo penalizado por uma magia/habilidade ou não. Requer nível mental maior que 4 para aprender essa magia.

\end{enumerate}

\subsection{Windan}

\begin{enumerate}
	\item Magia Pura: 6pm, +4, 30PS. Magia Branca. TC14. Elemento Vento.\newline
--->Especial Primeiro Círculo (Força do Vento): O usuário pode levantar pequenos objetos de 5kg por ponto de sabedoria(até 50 kg). O mestre pode cobrar 1 ou 2 PM de acordo com o peso excessivo do objeto. Quando usado em batalha o personagem pode empurrar o oponente alguns metros sempre que o mesmo receber dano por windan.
--->Especial Segundo Círculo (Explosão do Vento): A magia windan agora atinje todos os alvos em uma área com raio igual ao bônus de sabedoria em metros. Esse raio pode aumentar caso o usuário use alguma habilidade para causar dano em área.

\item Barreira de Vento(20ps,10xpm): Custo normal de Windan + 4 pm. TC 10.\newline
O usuário cria uma barreira mágica que o envolve, reduzindo todo o dano mágico que entrar na barreira em Sabedoria + 2 durante concentração turnos. O usuário também pode gastar 1 pm para aumentar a redução da barreira em 1 até o valor de sua inteligência. Nos últimos 2 turnos a proteção da barreira cai pela metade. Para qualquer magia de vento e de água a barreira tem o dobro da defesa. Porém para qualquer magia de raio ela tem apenas metade. Caso o usuário tenha a habilidade materializar magia, pelo mesmo custo ela pode criar uma barreira física, com dados e comportamentos semelhantes se ela fosse invocada como barreira mágica, mas como a natureza da magia é diferente são necessárias algumas observações. A barreira criada tem um valor de defesa automática igual a Sabedoria + 2 (a defesa da barreira pode ser aumentada da mesma forma se ela fosse mágica), e agora ela tem uma certa quantidade de PV igual ao dobro da consciencia do alvo + o bônus da magia pura. O oponente tem duas opções quando ataca um usuário envolto pela barreira manifestada fisicamente. A primeira é atacar diretamente a barreira, afim de destrui-la. O atacante sempre acerta a barreira, porém não recebe bônus de acerto ao faze-lo. A barreira defende com sua defesa normal e todo o dano é reduzido dos PV totais da barreira. O atacante também pode optar por tentar ultrapassar a barreira. Dessa forma ele tem um bônus de destreza durante o ataque (como se estivesse realizando um ataque inesperado), porém recebe dano por tal feito. O dano infligido pela barreira é de natureza físico, e é igual ao dobro do dano causado pela magia pura (o dano de sabedoria é considerado normal e o bônus da magia é considerado automático). Se o dano causado for maior que o bônus de resistência do alvo, o mesmo é arremessado para fora da barreira. Sua fraqueza e resistência a elementais também continuam as mesmas. 

	\item Ventania(10xpm): Custo normal de Windan + 2. TC 14\newline
Todos os alvos dentro de um cone de tamanho igual ao focus do usuário em metros devem realizar um teste de força contra o bônus de windan. windan. Caso obtenham falha, eles são deslocados sabedoria + sucesso metros. Esse ataque é considerado de natureza física podendo ser desviado com o atributo esquiva. O usuário pode usar inteligência ou destreza para jogar seu acerto. Pré-requisito: Materializar Magica ou Materializar Windan.

\item Voz do vento(10xpm): O alvo pode simular vozes diferentes da sua por 2 PM durante concentração minutos e também eliminar a emissão de sons provenientes de uma área com raio igual a sabedoria do usuário em metros por 4 PM durante o mesmo tempo. 
	
	\item Pés Leves(10xpm): Custo normal de Windan .\newline
Durante concentrãção turnos o alvo recebe um bônus em esquiva igual ao bônus de sabedoria do usuário da magia. Além disso o usuário consegue correr sobre superfícies não possíveis antes (como água ou gelo fino). 
	
\item Tornado(10PS,14xpm): Custo normal de Windam + 4 pm. TC normal de Windam + 4 pm.\newline
O usuário cria um tornado em sua volta durante concentração turnos. Todos os projéteis pequenos ou médios são desviados autómaticamente em angulos definidos pelo mestre de acordo com seu peso. Os demais ataques físicos têm um redutor no acerto igual a metade do bônus de windam. Ataques elementais de vento ou raio ignoram essa proteção.

\item Levitar(10ps,10xpm): O usuário consegue voar até uma altura igual sua sabedoria em metros. A distância horizontal é ilimitada. A velocidade de voo é igual a sua velocidade de corrida em solo. Com essa magia ativada ele sempre ganha seu bônus de esquiva quando lutando contra oponentes que não estão voando, bônus esse que pode ser acumulado com outras habilidades. Pode-se gastar pms extras para aumentar o potencial da habilidade, na ordem de 1 PM por metro aumentado até o maximo de focus metros aumentados dessa forma. Custa 5 PM para ativar essa magia de TC 14. Tem duração em minutos igual ao concentração do usuário.

\item Explosão Sônica(14xpm): Custo normal de windan + 8 pm. TC normal de windam + 6.\newline O usuário da magia cria uma área em volta de si mesmo com raio igual a sua sabedoria metros. Todas os alvos dentro dessa área recebem dano físico do tipo normal que ignora armadura. Esse dano é igual ao dano causado normalmente por windan. Os usuários da habilidade auridade podem tentar se esquivar dessa magia, assim como usuários de algum tipo de controle elemental relacionado ao vento. Algumas barreiras podem ser usadas para se defender contra o dano dessa magia. Por exemplo, caso criada próxima a uma barreira elemental física (barreira de vento por exemplo), a barreira recebe dano no lugar do usuário. 


\end{enumerate}

\subsection{Ciarú}

\begin{itemize}
	\item Magia Pura(30PS): 6PM de custo, bônus +4. Magia Verde. TC14.\newline
--->Especial Primeiro Círculo (Voz da Natureza): O alvo pode conversar com plantas ao gasto de PM. Quanto mais complexa for a conversa, mais PM o mestre deve exigir. O usuário também pode criar frutas gastando PM pelo custo normal da  magia.
--->Especial Segundo Círculo (Ajuda da Mae Gaia): O usuário pode criar um familiar temporário a partir das plantas do ambiente. O familiar auxilia o usuário em combate, concedendo +4 no acerto ou esquiva/aparar durante bônus de inteligência. O familiar tem 10 de defesa, espirito 0 e 20 PV. O familiar pode se sacrificar para reduzir 15 de dano em qualquer ataque físico ou mágico. Para invocar o familiar o usuário deve um turno ou ao usar a magia ciaru 2, com um custo extra de 4 PM. O custo normal de invocar o familiar varia entre 2 e 6 PM de acordo com a quantidade de plantas do local. Apenas um familiar pode ser invocado por vez.
 	
\item Proteção das plantas(10xpm): Custo normal de ciarú + 2 PM. TC normal de ciarú + 4.\newline
Durante concentração turnos, a pele ou armadura do alvo é envolto por uma crosta de madeira onde a sua defesa é igual ao bônus da magia pura. Essa proteção é do tipo normal. A armadura criada tem uma quantidade de PR igual ao dobro do focus do usuário mais o bônus da magia pura. Caso essa habilidade seja usada enquanto o usuário esteja usando uma armadura média ou pesada, o mesmo recebe um redutor de 2 em sua destreza e 4 em sua esquiva. Pré-requisito: Materializar Magia.

	\item Garra das Árvores(12xpm): 4PM. TC16.\newline
O usuário testa seu focus contra a agilidade ou força do alvo. Se o alvo não passar ele fica preso em uma rede de vinhas e galhos durante 4 turnos ou até a rede de plantas que o prende ser destruida por meio físico. Durante esse tempo, o alvo não pode se mexer e tem um redutor de 4 em sua destreza.

	\item Barreira de Planta(20ps,10xpm): Custo de ciarú normal + 4 pm. TC14.\newline
O usuário cria uma barreira mágica que o envolve, reduzindo todo o dano mágico que entrar na barreira em Sabedoria + 2 durante concentração turnos. O usuário também pode gastar 1 pm para aumentar a redução da barreira em 1 até o valor de sua inteligência. Nos últimos 2 turnos a proteção da barreira cai pela metade. Para qualquer magia verde e de gelo a barreira tem o dobro da defesa. Porém para qualquer magia de fogo ela tem apenas metade. Caso o usuário tenha a habilidade materializar magia, pelo mesmo custo ela pode criar uma barreira física, com dados e comportamentos semelhantes se ela fosse invocada como barreira mágica, mas como a natureza da magia é diferente são necessárias algumas observações. A barreira criada tem um valor de defesa automática igual a Sabedoria + 2 (a defesa da barreira pode ser aumentada da mesma forma se ela fosse mágica), e agora ela tem uma certa quantidade de PV igual ao dobro da consciencia do alvo + o bônus da magia ciarú. O oponente tem duas opções quando ataca um usuário envolto pela barreira manifestada fisicamente. A primeira é atacar diretamente a barreira, afim de destrui-la. O atacante sempre acerta a barreira, porém não recebe bônus de acerto ao faze-lo. A barreira defende com sua defesa normal e todo o dano é reduzido dos PV totais da barreira. O atacante também pode optar por tentar ultrapassar a barreira. Dessa forma ele tem um bônus de destreza durante o ataque (como se estivesse realizando um ataque inesperado), porém recebe dano por tal feito. O dano infligido pela barreira é de natureza físico, e é igual ao dobro do dano causado pela magia pura (o dano de sabedoria é considerado normal e o bônus da magia é considerado automático). Se o dano causado for maior que o bônus de resistência do alvo, o mesmo é arremessado para fora da barreira. Sua fraqueza e resistência a elementais também continuam as mesmas. 

	\item Animar Árvores(20PS,10xpm): 8 PM. TC16.\newline
O usuário pode animar uma árvore que estiver próximo a ele, podendo criar membros extras na mesma durante a invocação. A força e defesa (pele de madeira) da árvore têm valores iguais a sabedoria de quem a invoca. Os valores de destreza e esquiva são iguais a inteligência. O valor de PV igual ao dobro de focus, espírito igual 6 e valores extras nos atributos físicos são determinados pelo mestre de acordo com a situação (geralmente igual a metade do nível mental do usuário). O usuário também pode gastar PM extras para aumentar os atributos da criatura invocada, na ordem de 1PM para 1 ponto aumentando, até o maximo da concentração do usuário.

  	\item Portal da Floresta Sagrada(20ps,12xpm): Quando em florestas o usuário pode se teleportar a vontade até concentração kilometros. Quanto maior a distância maior o consumo de PM e PF. Para cada kilometro teleportado o custo é de 2PF e 8PM. Ele gasta 1 turno concentrando para se teleportar. Por 6 pontos totalizando PM e/ou PF vezes 2, ele pode levar 1 pessoa junto.  Nível mental e fisico mínimo necessário  igual a 4.

	\item Arma da Floresta(10xpm,10ps): 4 PM. TC 14.\newline
O usuário pode criar uma arma com dano do tipo normal igual a sua sabedoria. Ele pode adicionar 1 de dano normal por PM extra gasto até o bônus de focus. A arma tem uma quantidade de PR igual ao espírito do usuário vezes 5 e tem duração de 2 horas. A arma não tem redução de atributos devido a perda de PR, apenas quando o mesmo é reduzido a zero. A arma é considerada uma arma mágica de slot nulo (pode ser usada com qualquer arma mágica) que causa dano físico. Outras pessoas com conhecimento de magia ou forjar podem perceber que a arma é magica.

	\item Dardo Vil (10ps,10xpm): 4 PM. Magia reflexiva ativada no momento do uso de ciarú.\newline
Quando a magia ciarú tirar dano do oponente, o mesmo fica com a condição negativa veneno vital 4(perde 4 PV por turno) se não passar em um teste de resistência contra o focus + metade do bonus de ciarú 1 do usuário. Qualquer habilidade/magia que conceda bônus em resistência ou contra veneno pode ser usada no teste de resistência do defensor. Caso a magia seja de toque, o efeito é automático. 

	\item Casco da Nogueira Sagrada(16ps,12xpm): Custo de 10 PM. TC16. \newline	
Durante concentração turnos o usuário cria um casco que diminui sua destreza e esquiva em 3 pontos. O sempre que o usuario receber dano (seja físico ou mágico), quem recebe dano é o casco cuja quantidade inicial de PV é 40. A quantida de PV do casco não pode ser curada.
	
\end{itemize}


\subsection{Veru}

Veru é uma magia peculiar no caminho arcano. Assim como a magia toronto (mostrada a seguir) ela se divide em subcaminhos. A magia pura de veru é manifestada por meio do deterioramento do tecido vivo ou no enfraquecimento da matéria. De uma forma mais simples ela é manifestada no plano físico como ácido corrosivo. Após aprender a magia pura o usuário pode gastar 10 PS para poder começar a aprender as habilidades de um dos dois subcaminhos da magia, veneno e morte. Caso ela deseje liberar o outro caminho para aprendizado, ele pode faze-lo gastando mais 10 PS. Os necromantes têm livre acesso a qualquer magia do caminho veneno e morte, desde que possuam a magia pura. A magia pura de veru é considerada uma habilidade de necromante para determinar o custo de sua compra.


Magia Pura: 6pm, +6, 30PS. Magia Roxa.  TC14. \newline

--->Especial Primeiro Círculo (Corrosive Power): Caso o alvo atigindo pela magia veru esteja usando uma armadura, esta sofre uma perda de PR igual ao bônus da magia. \newline
--->Especial Segundo Círculo (Dark Venon): Gastando 3 PM extras, o usuário da magia pode adicionar seu valor de status na perda de PR sofrida ou o bônus de status no dano da magia veru.

\textbf{Habilides de Veru(Veneno)}
\begin{itemize}


	\item Barreira de Veneno(20ps,10xpm): Custo de veru normal + 4 pm. TC14.\newline
O usuário cria uma barreira mágica que o envolve, reduzindo todo o dano mágico que entrar na barreira em Sabedoria + 2 durante concentração turnos. O usuário também pode gastar 1 pm para aumentar a redução da barreira em 1 até o valor de sua inteligência. Nos últimos 2 turnos a proteção da barreira cai pela metade. Para qualquer magia veru e de terra a barreira tem o dobro da defesa. Porém para qualquer magia de vento ela tem apenas metade. Caso o usuário tenha a habilidade materializar magia, pelo mesmo custo ela pode criar uma barreira física, com dados e comportamentos semelhantes se ela fosse invocada como barreira mágica, mas como a natureza da magia é diferente são necessárias algumas observações. A barreira criada tem um valor de defesa automática igual a Sabedoria + 2 (a defesa da barreira pode ser aumentada da mesma forma se ela fosse mágica), e agora ela tem uma certa quantidade de PV igual ao dobro da consciencia do alvo + o bônus da magia veru. O oponente tem duas opções quando ataca um usuário envolto pela barreira manifestada fisicamente. A primeira é atacar diretamente a barreira, afim de destrui-la. O atacante sempre acerta a barreira, porém não recebe bônus de acerto ao faze-lo. A barreira defende com sua defesa normal e todo o dano é reduzido dos PV totais da barreira. O atacante também pode optar por tentar ultrapassar a barreira. Dessa forma ele tem um bônus de destreza durante o ataque (como se estivesse realizando um ataque inesperado), porém recebe dano por tal feito. O dano infligido pela barreira é de natureza físico, e é igual ao dobro do dano causado pela magia pura (o dano de sabedoria é considerado normal e o bônus da magia é considerado automático). Se o dano causado for maior que a resistência do alvo, o mesmo é arremessado para fora da barreira. A barreira tem a mesma duração se invocada como barreira mágica. Sua fraqueza e resistência a elementais também continuam as mesmas. 


	\item Veneno Vital(10xpm): 6 pm. tc 14.\newline
O alvo recebe a condição negativa veneno vital (perde uma certa quantia de pv por turno) se não passar em um teste de espírito contra sabedoria do usuário da mágia. Qualquer habilidade/magia que conceda bônus em resistência ou contra veneno pode ser usada no teste de espírito. A cada turno extra gasto concentrando a magia até o limite da concentração do usuário, o mesmo recebe um bônus de sabedoria no teste. O usuário pode gastar 5 PS para diminuir o custo em 1 PM até o máximo de 3 pontos diminuidos dessa forma. A intensidade do veneno (a quantia de pv perdido por turno) é igual ao número de sucessos obtidos no teste, até o valor máximo do bônus de veru 1. O veneno fica ativo durante uma quantidade de turnos igual ao bônus de concentração do usuário. Caso o alvo atinja uma quantidade negativa de PV ele deve fazer o teste novamente até passar no teste, onde a dificuldade é reduzida em 4 todo turno. Caso passe ele desmaia com a quantidade de PV negativa no momento que ele passou no teste, ou seja, antes do veneno causar dano. Essa magia não pode ser usados em alvos sem espírito, exceto se ela for materializada.

	\item Veneno Viral(10xpm): 6 pm. tc 10.\newline
O alvo recebe a condição negativa veneno viral (perde uma certa quantia de pf por turno) se não passar em um teste de espírito contra sabedoria do usuário da mágia. Qualquer habilidade/magia que conceda bônus em resistência ou contra veneno pode ser usada no teste de espírito. A cada turno extra gasto concentrando a magia até o limite da concentração do usuário, o mesmo recebe um bônus de sabedoria no teste. O usuário pode gastar 5 PS para diminuir o custo em 1 PM até o máximo de 3 pontos diminuidos dessa forma. A intensidade do veneno (a quantia de pf perdido por turno) é igual a metade do número de sucessos obtidos no teste, até o valor máximo do bônus de veru 1. O veneno fica ativo durante uma quantidade de turnos igual ao bônus de concentração do usuário. A magia perde o efeito assim que o alvo fica com 0 de PF. Essa magia não pode ser usados em alvos sem espírito.

	\item Veneno Mágico(10xpm): 6 pm. tc 10.\newline
O alvo recebe a condição negativa veneno mágico (perde uma certa quantia de pm por turno) se não passar em um teste de espírito contra sabedoria do usuário da mágia. Qualquer habilidade/magia que conceda bônus em resistência ou contra veneno pode ser usada no teste de espírito. A cada turno extra gasto concentrando a magia até o limite da concentração do usuário, o mesmo recebe um bônus de sabedoria no teste. O usuário pode gastar 5 PS para diminuir o custo em 1 PM até o máximo de 3 pontos diminuidos dessa forma. A intensidade do veneno (a quantia de pm perdido por turno) é igual ao número de sucessos obtidos no teste, até o valor máximo do bônus de veru 1. O veneno fica ativo durante uma quantidade de turnos igual ao bônus de concentração do usuário. A magia perde o efeito assim que o alvo fica com 0 de PM. Essa magia não pode ser usados em alvos sem espírito.


%se basear nessa para criar os drenos, porem a versao de veru pode ser bombada.
%	\item Sangue Frio(12xpm): Custo normal de zorin + 2 PM. TC10.\newline 
%O usuário joga focus contra a resistência do alvo. Caso falhe, o alvo perde uma quantidade de PF igual a metade do bônus (arredondado para cima) da magia pura zorin.

	\item Drenar PV(12xpm): 4 pm.\newline
Essa magia é ativada reflexivamente quando o usuário usa a magia veru. Metade do dano causado no alvo, caso esse receba dano, é absorvido em forma de PV. A quantidade máxima absorvida é igual a consciência do usuário. Essa magia não pode ser usados em alvos sem espírito, exceto se ela for materializada.
	
	\item Drenar PF(10ps,10xpm): Custo normal de veru + 6 pm. TC normal de veru + 2.\newline
O alvo realiza um teste de espírito contra sabedoria do usuário. Caso falhe, o alvo recebe dano em forma de PF igual a metade do bônus da magia veru. Além disso, metade do dano causado é absorvido pelo usuário. Essa magia não pode ser usados em alvos sem espírito.

	\item Drenar PM(10ps,10xpm): Custo normal de veru + 2 pm. TC normal de veru + 2.\newline
O alvo realiza um teste de espírito contra sabedoria do usuário. Caso falhe, o alvo recebe dano em forma de PM igual ao bônus da magia veru. Além disso, metade do dano causado é absorvido pelo usuário. Essa magia não pode ser usados em alvos sem espírito. 

\item Corrosão Mágica(12xpm): Custo normal de veru  + 2 PM. TC16.\newline
O usuário causa dano ao equipamento dos oponentes. Sempre que usa essa magia o equipamento do alvo perde uma quantidade de PR igual a consciencia do usuário + bônus cedido pela magia veru.

	\item Maldição do Corpo Fraco(20ps,12xpm): Custo normal de veru + 6 pm. TC normal de veru + 6.\newline 
O alvo realiza um teste de espírito contra sabedoria do usuário. Caso falhe, o alvo perde uma quantidade em força e defesa igual ao bônus de veru. Qualquer Bônus de resistência ou veneno podem ser usados pelo teste de espírito.


\end{itemize}


\textbf{Habilides de Veru(Morte)}

\begin{itemize}
	\item Vingança Negra(12xpm): 8 PM. TC14. \newline
Durante concentração turnos sempre que alguém receber dano em uma area igual a focus em metros, o usuário recebe um bônus no + 2 dano mágico, até o limite de igual ao nível mental do usuário.

	\item Olhar da Morte(10xpm): 3 PM. Magia instantânea.\newline
O usuário testa status ou sabedoria contra coragem ou espirito (maior dos dois) do alvo. Se o alvo não passar, ele fica sob efeito de medo em relação ao usuario e não pode ataca-lo durante uma quantidade de turnos igual ao número de falhas no teste, até o maximo igual ao bônus de concentração do usuário. O alvo pode gastar PM extras para ajudar no teste, na ordem de 1 PM para 2 sucessos até o valor de seu bônus de focus ganho dessa forma.
  
	\item Chamado das Tumbas(20PS): 3 PM por morto vivo. TC14 para cada 3 mortos vivos invocados.\newline
O alvo pode invocar mortos vivos onde seus atributos físicos são variaves de acordo o tipo de cadáver. O normal usado é destreza e esquiva variando entre 4 e 7, e força e defesa entre algo entre 7 e 10. Os mortos vivos invocados podem ser compostos de qualquer material em decomposição. Caso essa magia seja usada em locais com corpos reais, os mortos vivos invocados podem receber bônus em seus atributos (armas extras, mais PV, ataques venenoso, etc). Sua quantidade de pontos de vida é proporcional ao estado do cadáver invocado, este variando entre 15 e 30. Outro detalhe dos zumbis e que eles sempre usam sua parada de defesa máxima e não podem receber dano mágico, exceto aquele provindo da magia héris. Apesar do zumbi não ser uma criatura inteligente, ele é capaz de ser controlado mentalmente pelo invocador. Quanto maior é a concentração do invocador para os zumbis criados, mais inteligente eles são. Quando não há atenção por parte do invocador, os zumbis sabem apenas quem são os aliados do invocador, sempre atacante inconsequentemente os inimigos mais próximos. A duração do chamado é de cerca de acuidade em horas. Os PM usados para essa magia (arredondados para cima) são acometidos, ou seja, até que a magia se desfaça o usuário não pode recuperar a quantidade de PM usada para ativar a magia.

%% dizer dps q tem rituais que podem abaixar o custo da criação
%% dizer que com medicina o invocador pode invocar zumbis mais fortes.
%%

	\item Bondade dos Mortos(14ps): O usuário pode usar uma invocação normal de veru para curar um morto vivo. A quantidade curada é igual ao dano causado normalmente por veru. Além dissoo status negativo veneno causa regeneração no morto vivo igual a intensidade do veneno.

	\item Dominar Morto Vivo(10xpm): 3PM.\newline
O usuário pode obrigar 1 morto vivo a realizar uma ação simples ou com médio grau de complexidade. A ação não pode fazer com que o alvo atinga a si mesmo ou aliados de longa data. É necessário um teste combatido de Focus do usuário contra consciência do alvo. Uma vez executada a ordem, o alvo deve executa-la dentro de 48 horas, se nao o efeito da magia acaba. A vítima da magia pode fazer um teste de inteligência ou sabedoria para saber que a ação realizada foi induzida por magia.
	
	\item Chamado da Morte(20ps,14xpm): O usuário profere uma maldição contra um alvo gastando o custo normal de veru no processo. O alvo recebe um dano igual ao dobro de PM extras gastos durante a concentração da magia. Esse dano é direto na quantidade de PV, não podendo ser absorvido. A quantidade máxima de PM extras gastos não pode ultrapassar o valor de sabedoria do usuário. O personagem necessita nível 4 mental para poder aprender essa habilidade. TC 5 + PM extras gastos.


	%\item Miasma da Morte(30ps, 12xpm): Custo de 10 PM. TC20.\newline
%Cria uma area de miasma em um raio de focus metros em sua volta. Para cada PM extra gasto ao concentrar a magia, o raio é aumentado em 1 metro. Pode-se gastar até o valor de concentração em PM dessa forma. Todos aqueles dentro da área recebem dano físico do tipo automático igual ao bônus da magia pura verú. Esse dano é elemental do tipo veneno e ignora qualquer armadura no momento da jogada de absorção. Os alvos também perdem 2 PM e 1 PF por turno dentro da área. A magia tem duração em turnos igual a concentração do usuário.

  	\item Miasma da Morte(40PS,10xpm): TC22.\newline
O usuário cria uma área circular com raio em metros igual ao dobro do seu focus com duração em turnos igual a sua concentração. Dentro da área da magia qualquer habilidade, item ou magia de cura não tem efeito, ou tem efeito reduzido. Por 10 PM qualquer cura tem seu efeito reduzido pela metade. Por 20 PM os efeitos de cura são anulados. Requer a magia veru além de nível mental mínimo de 5 para aprender essa habilidade. O efeito dessa magia não é considerado um status negativo, e sim um efeito de área do ambiente. 
	
\end{itemize}


\subsection{Toronto}

A Magia toronto se divide em 3 subcaminhos; Forma, mente e espirito. A maioria desses poderes é usado indiretamente em combate. O subcaminho forma faz com que a magia altere características da matéria. O subcaminho da mente permite que o mago conecte sua vibração mágica com a vibração elétrica emitita pelo cérebro das criaturas vivas, podendo assim compreende-los ou controla-los. E finalmente o subcaminho espírito permite que o usuário possa manipular a energia espirital em sua volta, podendo interagir com o plano espiritual mais facilmente. Uma pessoa normal nescessita gastar 20 PS e 8xpm para poder começar a aprender as habilidades de um subcaminho, enquanto que aqueles que são desfavorecidos com a classe mago vermelho, devem pagar 24PS e 14xpm. Os magos vermelhos têm livre acesso a todos os subcaminhos do magia toronto. Uma vez que um personagem tem acesso a um subcaminho da magia, ele pode comprar qualquer magia daquele subcaminho pelo preço normal.

\textbf{Habilides de Toronto(Forma)}

\begin{itemize}
	
	\item Manipular Metal(10ps,12xpm): 6 PM. TC14.\newline
O usuário causa dano ao equipamento dos oponentes. Sempre que usa essa magia o equipamento do alvo perde uma quantidade de PR igual a consciência do usuário + o dobro de pm extras usados. A quantidade máxima de PM extras gastos não pode ultrapassar a concentração do usuário. O usuário dessa magia também pode recuperar PR perdidos por meios mágicos. Os dados para recuperar os PR são os mesmos usados para retirar.

	\item Fisup(20ps,12xpm): TC 0 para 3 PMs gastos. TC14 para 6 PMs gastos.\newline
O usuário concentra a magia em um alvo ou nele mesmo para aumentar uma característica física. O usuário gasta 3 PM para um valor igual a metade da sua sabedoria, arredondado para cima. Ele também pode gastar 6 PM para aumentar um valor igual a sua sabedoria em 1 atributo físico, ou metade desse valor em 2 atributos. O bônus (max 10) dura concentração turnos.
	
	\item Fisba(20ps,12xpm): TC 0 para 3 PMs gastos. TC14 para 6 PMs gastos.\newline 
O usuário concentra a magia em um alvo ou nele mesmo para reduzir uma característica física. O usuário gasta 3 PM um valor igual a metade da sua sabedoria, arredondado para cima. Ele também pode gastar 6 PM para diminuir um valor igual a sua sabedoria em 1 atributo físico, ou metade desse valor em 2 atributos. O redutor (max 10) dura concentração turnos. Se o alvo não estiver sendo tocado, ele só recebe a penalidade imposta pela magia caso perca em um teste de espírito contra sabedoria do usuário da magia. Se for tocado a redução é automatica.

	\item Atravessar paredes(20ps,6xpm): 10 PM, TC40.\newline
Durante o tempo de concentração da magia o usuário se foca em uma parede ou obstáculo. Ele pode ignorar o obstáculo e atravessa-lo normalmente até o dobro de sua sabedoria em metros.

	\item Teleporte(30ps,10xpm): O usuário pode se teleportar a vontade até concentração kilometros. Quanto maior a distância maior o consumo de PM. Para cada kilometro teleportado o custo é de 8PM. Ele gasta 1 turno concentrando para se teleportar. Para cada 6 PM gastos, ele pode levar 1 pessoa junto no teleporte.  Nível mental e fisico mínimo necessário igual a 4.
  	
	\item Respirar no vácuo (10ps,6xpm): Gastando-se 6 pm por hora o mago pode respirar usando seu poder mágico. 

\item Levitar(10ps,10xpm): O usuário consegue voar até uma altura igual sua sabedoria em metros. A distância horizontal é ilimitada. A velocidade de voo é igual a sua velocidade de corrida em solo. Com essa magia ativada ele sempre ganha seu bônus de esquiva quando lutando contra oponentes que não estão voando, bônus esse que pode ser acumulado com outras habilidades. Pode-se gastar pms extras para aumentar o potencial da habilidade, na ordem de 1 PM por metro aumentado até o maximo de focus metros aumentados dessa forma. Custa 5 PM para ativar essa magia de TC 14. Tem duração em minutos igual ao concentração do usuário.

	\item Sentidos Aguçados(10PS,8xpm): Para cada 4 PM gastos reflexivamente o usuário pode ativar uma das seguintes habilidades: Visão noturna, Visão extendida, Tato supersensível (com um teste de inteligência pode identificar substâncias em alimentos e líquidos por exemplo), Olfato preciso ou Audição profunda.

	\item Paralisia(20PS,12xpm): Custo de 8 PM. TC10.\newline
Teste entre resistência do alvo e focus do usuário da magia. Se o usuário ganhar, o alvo não pode realizar ações físicas durante uma quantidade de turnos igual ao sucesso obtido no teste pelo usuário (max bônus de concentração do usuário). O usuário pode gastar 5 PS para reduzir o custo da magia em 1 PM até o mínimo de 4 PM. Vale lembrar que qualquer bônus cedido ao atributo geral resistência pode ser usado no teste contra essa magia.

	
	\item Armas da Natureza(10PS,10xpm): 6 PM. TC20.\newline
O alvo pode criar armas ou armaduras feitas de matériais facilmente encontrados na natureza. O dano (ou defesa) normal e automático do equipamento são igual a metade da inteligência do usuário. O mesmo pode gastar 2 PM para aumentar 1 de dano/defesa automático ou 1 PM para aumentar 1 de dano/defesa normal. O usuário pode aumentar até o valor de sua sabedoria como extra de dano/defesa do equipamento criado. A arma tem duração igual a concentração em horas. A arma tem uma quantidade de PR igual a acuidade do usuário vezes 3. Outras pessoas com conhecimento de magia ou forjar podem perceber que a arma é criada pelo poder da magia toronto.
	
%	\item Sinergia Vingativa(20ps,12xpm): 8 PM. TC10. \newline
Durante concentração turnos sempre que o alvo receber dano, metade do dano levado pode ser usado para aumentar força ou defesa. O bônus para aumento de força ou defesa não deve ultrapassar o focus do usuário da magia. O efeito do bônus dura 15 turnos.

\item Metamorfose (20PS,12xpm,12xpf): O usuário pode usar a habilidade metamorfose. O bônus inicial dessa habilidade é 3. O mestre deve restringir esse aprendizado para situações MUITO especificas dentro de jogo.

\item Forma da Verdade (12PS,6xpm): 6 PM. TC14.\newline
O usuário cancela automaticamente o poder concedido pela habilidade geral especial metamorfose de 1 alvo. O usuário pode cancelar um bônus de metamorfose igual ou menor do que seu bônus de consciência.

\item Sinergia Interplanar(20ps,10xpm): 1 PM. TC14. \newline
Durante concentração turnos sempre que o alvo receber dano físico, o mesmo recupera uma quantia de PM igual a metade do dano do tipo normal levado ou do tipo automatico caso o dano seja explosivo. A quantia recuperada não pode ultrapassar a sabedoria do usuário.

\end{itemize}

\textbf{Habilides de Toronto(Mente)}

\begin{itemize}
	\item Ler Mentes(20PS,10xpm): 4 PM. Magia reflexiva.\newline
O usuario pode ler a mente de um alvo. É necessário um teste combatido de acuidade do usuário contra consciência do alvo. Quanto maior for o sucesso maior a extenção da memória que pode ser lida. O usuário dessa magia pode gastar 10 PS para reduzir 1 PM no custo da magia até a metade do custo inicial. Quando usado em combate, se bem sucedido no teste, o usuário recebe na esquiva um bônus igual a diferença entre sua inteligência e a do atacante.

	\item Libra(10xpm): O usuário pode ver os atributos de um alvo apenas ao olhar. Além disso, gastando 4 PM reflexivamente ele pode usar visão de raio x durante acuidade minutos.

	\item Manipular Memória(16xpm): 8 PM. TC20.\newline
Quando usando a magia ler mentes, o usuário pode apagar as memórias lidas. Para tal ele deve fazer um teste de sabedoria ou inteligência contra a inteligência do alvo. O mestre pode atribuir bônus para o alvo de acordo com a importância daquela memória. Uma vez apagada a memória somente pode ser recuperada por outro usuário da magia manipular memória após um intervalo de tempo em semanas igual ao focus do usuário original da magia. Para recuperar uma memória o novo usuário da magia deve gastar 8PM e testar sua sabedoria contra a sabedoria do usuário original (use o valor que o usuário original tinha no momento da magia). O mestre pode atribuir penalidades de acordo com o tempo que a memória está apagada. Pré-requisito: Ler Mentes.

	\item Dominação(20PS,10xpm): 6PM. TC 14.\newline
O usuário pode obrigar o alvo a realizar uma ação simples ou com médio grau de complexidade. A ação não pode fazer com que o alvo atinga a si mesmo ou aliados de longa data. É necessário um teste combatido de consciência do usuário contra consciência ou focus do alvo. Uma vez executada a ordem, o alvo deve executa-la dentro de 48 horas, se nao o efeito da magia acaba. A vítima da magia pode fazer um teste de inteligência ou sabedoria para saber que a ação realizada foi induzida por magia.

\item Menup(20ps,10xpm): TC 0 para 3 PMs gastos. TC14 para 6 PMs gastos.\newline
O usuário concentra a magia em um alvo ou nele mesmo para aumentar uma atributo mental, exceto espírito. O usuário gasta 3 PM para um valor igual a metade da sua sabedoria, arredondado para cima. Ele também pode gastar 6 PM para aumentar um valor igual a sua sabedoria em 1 atributo físico, ou metade desse valor em 2 atributos. O bônus (max 10) dura concentração turnos.
	
	\item Confusão(20ps,10xpm): TC 0 para 3 PMs gastos. TC10 para 6 PMs gastos.\newline 
O usuário concentra a magia em um alvo para reduzir um atributo mental, exceto espírito. O usuário gasta 3 PM para um valor igual a metade da sua sabedoria, arredondado para cima. Ele também pode gastar 6 PM para diminuir um valor igual a sua sabedoria em 1 atributo físico, ou metade desse valor em 2 atributos. O redutor (max 10) dura concentração turnos. Se o alvo não estiver sendo tocado, ele só recebe a penalidade imposta pela magia caso perca em um teste de espírito contra sabedoria do usuário da magia. Se for tocado a redução é automatica.

	\item Telepatia(10PS,10xpm): Magia instantânea.\newline 
O usuáro pode entrar em contato telepaticamente com qualquer pessoa dentro dentro do seu campo de visão ou qualquer pessoa que ele tenha encontrado nas últimas 24 hrs. Para cada turno conversando com o alvo, é necessario 2 PM. Caso a pessoa a ser comunicada esteja muito distante, o usuário deve fazer um teste de consciencia com dificuldade 18 para falar normalmente com o alvo (a dificuldade do teste aumenta de acordo com a distância e estado mental atual do alvo). Essa magia também pode ser usada para falar com múltiplos alvos, onde o teste de consciência (DF 18) deve aumentar em 2 para cada alvo extra dentro do campo de visão ou em 5 caso usado a distância. A distância máxima da telepatia é igual ao nível mental em kilometros

	\item Ilusão(14PS,8xpm): 6PM, TC14.\newline 
	O usuário gasta 6 PM para criar uma ilusão que dura concentração minutos. A ilusão criada tem complexidade igual ao valor obtido em um teste de inteligência. Ou seja, qualquer valor obtido acima de 19 pode fazer com que o usuário crie uma ilusão com alto grau de complexidade (uma cópia de si mesmo que fala por exemplo). Qualquer um que veja a ilusão pode tentar percebe-la fazendo um teste de consciência contra a consciência ou focus de quem criou a ilusão. O mestre pode atribuir penalidades para perceber a ilusão de acordo com a situação (visibilidade prejudicada, várias pessoas no mesmo local, etc). 

	\item Sleep (20PS,12xpm): Custo de 8 PM. TC14.\newline
Teste combatido entre consciência ou focus do alvo contra a consciência do usuário da magia. Se o usuário ganhar, o alvo fica na condição negativa sono durante uma quantidade de turnos iguais a quantidade de sucessos obtidas no teste (max bônus de concentração). Quando nessa condição o alvo não pode realizar nenhuma ação, incluindo o uso de habilidades reflexivas. Caso seja atacado, o seu valor de esquiva é sempre considerado 0 sem a oportunidade de rolar os dados e seu valor de defesa (não incluindo habilidades ativas ou armaduras) é reduzido pela metade. Por 10 PS ele pode reduzir o custo dessa magia em 1PM até o mínimo de 4PM. O alvo acorda caso leve uma quantidade de dano maior que seu atributo defesa. 


	\item Berserk(20PS,10xpm): Custo de 8 PM. TC14.\newline
Teste combatido entre consciência do alvo e do usuário da magia. Se o usuário ganhar, o alvo fica na condição negativa berserk durante uma quantidade de turnos iguais a quantidade de sucessos obtidas no teste. Quando nessa condição o alvo ganha 5 em força e 5 em defesa, realiza apenas ataques inconsequentes e ataca aleatoriamente oponentes e aliados próximos a ele. Por 10 PS ele pode reduzir o custo dessa magia em 1PM até o mínimo de 4PM.

	\item Transcedência da Mente(10xpm): 4 PM. Magia instantânea.\newline
O usuário pode curar efeitos de sleep e berserk. Ele também pode usar essa habilidade perceber ilusões durante focus minutos.

\item Mind Break(10ps,10xpm): 6 PM. TC14.\newline
O alvo recebe uma quantidade de dano, sem direito a absorção, igual a diferença entre a consciência do usuário com a sua. Por 10 PS ele pode reduzir o custo dessa magia em 1PM até o mínimo de 4PM.


\end{itemize}


\textbf{Habilides de Toronto(Espírito)}

\begin{itemize}
	\item Mute(20PS,10xpm): Custo de 8 PM. TC 14. \newline
Teste combatido entre espírito do alvo e sabedoria do usuário. Se o usuario ganhar, o alvo não pode realizar ataques mágicos durante sucessos turnos (max igual ao bônus de concentração). O usuário pode gastar 10 PS para reduzir 1 PM até o minimo de 4 PM.

	\item Sentido Espiritual(10xpm): 1 PM. Habilidade reflexiva.\newline
Durante 1 hora o alvo pode ver e conversar com espiritos normalmente dentro de um ambiente que ele tenha algum tipo de sobrevivência. Ele também pode ver nuances do mundo espíritual além de concentrações de mana (energia mágica). 

	\item Suspiro Espiritual(10ps,10xpm): Para cada 1 PM gasto ao concentrar a magia, o usuário pode curar 1 PF do alvo até o máximo da resistência do beneficiário. O usuário pode gastar uma quantidade de PM igual a seu focus. TC14.

	\item Suspiro Mágico(10ps,10xpm): Para cada 1 PF gasto ao concentrar a magia, o usuário pode curar 2 PM do alvo até o máximo do focus do beneficiário. O usuário pode gastar uma quantidade de PF igual ao seu nível mental. TC14.

\item Dominar Espírito(20PS,10xpm): 6PM.\newline
O usuário pode obrigar um espírito a realizar uma ação simples ou com médio grau de complexidade. A ação não pode fazer com que o alvo atinga a si mesmo ou aliados de longa data. É necessário um teste combatido de Focus do usuário contra consciência do alvo. Uma vez executada a ordem, o alvo deve executa-la dentro de 48 horas, caso contrário o efeito da magia acaba. A vítima da magia pode fazer um teste de inteligência ou sabedoria para saber que a ação realizada foi induzida por magia.

% 	\item Invocar Espírito(30ps,16xpm): O usuário gasta 1 turno para invocar espírito aliado e consome 12 PM. Por concentração turnos o elemental concede um bônus físico em força e defesa de 5. Além disso ele tem 10 pontos extras para distribuir em qualquer atributo físico da forma como bem desejar. Requer Nivel mental 3.

	\item Benção Espiritual(10xpm): Uma vez por dia o usuário pode curar uma quantidade de PM igual ao seu valor de espírito. É necessário 1 turno concentrando-se unicamente nessa habilidade. Requer nível mental mínimo de 4.

	\item Shell(10PS,8xpm): 4 PM. TC14.\newline %Amagza
O alvo da magia recebe um bônus de + 8 em espírito. O bônus tem duração em turnos igual a concentração do usuário.
	
	\item Amagza(20PS,8xpm): 4 PM. TC10.\newline
 O alvo da magia recebe uma penalidade de -8 em espírito. Para que a magia tenha efeito, o usuário deve passar em um teste de sabedoria contra o espírito do alvo. A penadalidade tem duração em turnos igual a concentração do usuário.
  
	\item Banir espiritos(10PS,10xpm): TC10. 4 PM.\newline
O usuário testa sabedoria + status contra coragem + espírito do espírito alvo. Se conseguir sucesso, o espirito alvo é banido para uma área aleatória em um raio de focus * acuidade kilometros. Se a distância for muito grande o mestre pode apagar a memória recente do espírito além de deixa-lo atordoado por algumas horas.

\item Refletir (30ps,10xpm): Qualquer magia que cause dano (físico ou mágico) igual ou menor que o focus (máximo de 30) do usuário é refletida de volta para o oponente. Custa 8 PM para se levantar o espelho que dura 4 + PMs extras turnos. TC 14. 
 	
	\item Defir(30PS,8xpm): 8 PM. TC14.\newline
O usuário pode cancelar o efeito de um artefato mágico durante inteligência horas. O mestre deve exibir um teste de focus de acordo com o poder do artefato. O mestre pode interpretar um sucesso muito próximo como cancelamento temporário de alguns poderes do artefato. Requer nível mental 4 ou mais para aprender essa magia. 

	\item Fio de Prata(30PS,6xpm): 1 PM por Alvo. TC18.\newline
O usuário se conecta aos alvos que estiverem dentro do seu campo de visão durante a concentração da magia. Durante uma quantidade de turnos igual ao espírito do usuário sempre que algum alvo consumir PM, essa quantidade de PM é retirada do usuário da magia fio de prata. Caso o alvo consuma PF, o dobro dessa quantia é retirada do usuário de fio de prata. Caso algum alvo consuma mais do que o usuário pode suportar, o fio de prata é automáticamente quebrado deixando o usuário atordoado por 1 turno. Nesse estado em específico, ele joga metade dos seus valores mentais e os oponentes que o atacarem recebem um bônus de destreza no ataque. 

\item Transferir (8xpm): O usuário pode transferir a vontade seu PM para outra pessoa. Habilidade instantânea. Ele pode transferir uma quantidade máxima de PM igual ao seu focus.

	\item Sinergia Mágica(20ps,10xpm): 1 PM. TC14. \newline
Durante concentração turnos sempre que o alvo receber dano mágico, o mesmo recupera uma quantia de PM igual a metade do dano. A quantia recuperada não pode ultrapassar o espírito do usuário.

	\item  Chave Mágica(12xpm,10PS): O usuário pode tentar desativar rituais com seu poder mágico. O mesmo deve fazer um teste de focus ou consciencia com uma dificuldade estabelecida pela mestre de acordo com a força do ritual e do nível mental de seu criador. Tem custo de 4 PM para rituais de nível baixo, 8 para rituais de nível médio e 12 para rituais de nível alto.



\end{itemize}

