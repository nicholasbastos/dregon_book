\chapter{Explorando o Mundo}
\label{cap:explorando}

O personagem pode explorar determinadas áreas em 2 amplitudes diferentes. Explorações em locais grandes e em locais pequenos. Explorando um local grande é por exemplo, explorar um deserto, uma floresta, etc. Explorar um local pequeno seria explorar um castelo, um bairro abandonado, entre outros. Quando explorando um local grande, os personagens percorrem distância maiores e gastam mais tempo durante a exploração. O mestre deve levar em consideração o local a ser explorando para descrever o que está acontecendo, e também manipular a perda/ganho de PF.

Em um dia de exploração ou locomoção, o personagem passa por 3 estágios. Descansos curtos, longos e caminhadas. Um descanso longo dura cerca de 8 horas, enquanto que um curto cerca de 30 minutos. Dentro do tempo gasto para descanso um personagem pode dormir, fazer sua higiene pessoal e se alimentar. Ele consome cerca de 10 horas de caminhada em um dia de exploração. Tomando em consideração que a velocidade média de caminhada de um ser humano é cerca de 8 km por hora, então ele caminha cerca de 80 km por dia. Porém, esse valor é usado para caminhos retos, como estradas bem construídas ou planícies. O mestre deve reduzir esses 80 km diários de acordo com a situação do terreno. Existem montanhas que esse valor é reduzido para 10 km diários por exemplo. 

Durante a exploração, o mestre pode fazer com que os personagens encontrem criaturas, armadilhas, tesouros, ou outros elementos interessantes. O mestre deve usar sua criatividade para tornar a exploração interessante e desafiadora.

\section{Exploração de Locais Grandes}

Ao explorar locais grandes, os personagens podem encontrar:

\begin{itemize}
    \item Criaturas selvagens
    \item Ruínas antigas
    \item Recursos naturais
    \item Perigos ambientais
    \item Povos nativos
\end{itemize}

\section{Exploração de Locais Pequenos}

Ao explorar locais pequenos, os personagens podem encontrar:

\begin{itemize}
    \item Armadilhas
    \item Tesouros escondidos
    \item Documentos importantes
    \item Criaturas hostis
    \item Segredos antigos
\end{itemize}

\section{Regras de Exploração}

Durante a exploração, os personagens devem:

\begin{itemize}
    \item Fazer testes de percepção para detectar perigos
    \item Fazer testes de sobrevivência para navegar
    \item Fazer testes de conhecimento para identificar elementos
    \item Gerenciar recursos como comida e água
    \item Tomar decisões estratégicas sobre rotas
\end{itemize}

\section{Consequências da Exploração}

A exploração pode resultar em:

\begin{itemize}
    \item Ganho de XP por descobertas
    \item Encontros com criaturas perigosas
    \item Descoberta de tesouros valiosos
    \item Perda de PF por fadiga
    \item Descoberta de informações importantes
\end{itemize}