%%
%% Capítulo : Equipamento
%%

\chapter{Equipamento}
\label{Cap:equipamento}

Os equipamentos são as armas, armaduras e acessórios que seu personagem utiliza em combate e aventuras. Em seguida explicaremos detalhes sobre obtenção e uso dos equipamentos no sistema Dregon. Antes de entrarmos em tais detalhes, apenas algumas observações:

\begin{itemize}
	\item Normalmente, a curta distância, pode-se aparar ataques com as mãos, porém isso não quer dizer que um artista marcial pode usar o braço para defender como se fosse uma espada ou um escudo. Ele de alguma forma impede o golpe de ser desferido para ele (segurando a base de uma espada antes que o golpe seja desferido por exemplo). O mesmo vale para armas brancas leves quando forem defender uma arma grande a curta distância, porém o teste de força continua sendo necessário em certos casos.

	\item Em alguns casos o mestre pode optar por restringir certos equipamentos de acordo com a história do personagem ou contexto da campanha. 

	\item Algumas armas exóticas podem não estar incluídas nas listagens devido a sua exclusividade de cenário. Cabe ao mestre, de acordo com a campanha, criar equipamentos novos. 
	
	\item Certas armas exigem valores mínimos de atributos para serem utilizadas. Por exemplo, um machado grande pode exigir força 6 para ser usado.

	\item Todos os personagens podem arremessar armas ou dardos para atacar a distância, porém sem bônus especiais. 

	\item Todos os personagens podem usar armaduras leves sem restrições.

\end{itemize}



\section{Tipos de Armas}

Existem três tipos principais de armas no sistema Dregon, cada uma causando um tipo específico de dano:

\begin{itemize}
	\item \textbf{Contusão:} Armas que causam dano através de impacto e força bruta, como martelos, clavas e socos. Exemplos incluem maças, martelos de guerra e armas de combate desarmado.
	
	\item \textbf{Corte:} Armas que causam dano através de lâminas afiadas, como espadas, machados e facas. Exemplos incluem espadas, machados, foices e adagas.
	
	\item \textbf{Energia:} Armas que causam dano através de energia mágica ou tecnológica, como armas de fogo, varinhas mágicas e armas de energia. Exemplos incluem pistolas, rifles, varinhas e armas de plasma.
\end{itemize}

O dano causado por todos os tipos de armas é calculado da mesma forma. A diferença está na interação com armaduras e proteções específicas. Algumas armaduras podem ter bônus ou reduções específicas contra certos tipos de dano, oferecendo proteção especializada contra um tipo específico de arma.

\section{Pontos de Resistência}

Um equipamento está sujeito a desgaste, seja esse devido ao uso ou ao passar do tempo. O que diz se um objeto está desgastado ou não são os seus PR (pontos de resistência). Os PR dizem o quão a arma ou armadura encontra-se desgastada, e o mestre deve atribuir penalidades ao equipamento quando a mesma encontra-se com poucos PR. As penalidades começam quando a quantidade de Pr é menor do que 20. Quanto menor o Pr do equipamento, maior sua penalidade. Para valores inferiores a 10, sempre que usada o equipamento tem uma chance de 50\% ser quebrado completamente.


Os PR podem ser recuperados ou terem sua quantidade total aumentada segundo alguns métodos alternativos, como magias, trabalhos de ferreiros, banhos em soluções químicas, entre outros. O PR total e atual de um equipamento deve ser anotado ao lado do mesmo na ficha, na parte de ''Info''. Cada vez que o equipamento é utilizado, o mestre deve analisar de que forma aquele uso contribuiu para o seu desgaste e assim retirar seus PR. Por exemplo, se um inimigo tem uma defesa do tipo automático alto, o mestre deve retirar mais PR da arma do que o normal (esse valor pode ser igual ao bônus de força do atacante por exemplo). Alguns inimigos explicitamente indicam que caso recebam dano ou acertem o alvo, uma quantidade de PR adicional é retirada do equipamento. A média de perda de uma arma ou armadura por batalha é de cerca de 5 PR. Armas grandes podem ser usadas para retirarem PR de outros equipamentos menores de forma semelhante a habilidade de guerreiro "Quebra de Equipamento". Por exemplo, uma armadura leve caso receba um golpe de um machado grande pode perder uma quantidade de PR igual a força do oponente, ou no máximo ate o dobro da mesma (a quantidade fica à criterio do mestre). Caso o personagem use a habilidade citada, a perda de PR é ainda maior, ou seja, 3 vezes o valor da força do usuário, normal da habilidade,  mais um bônus devido ao tamanho da arma. Esse bônus pode variar de acordo com o julgamento do mestre analisando a situação.


Algumas armas (incluindo armas extras) podem ser usadas em conjunto com a habilidade punho da stiga. Quando assim feitas, eles perdem uma quantidade de PR igual a defesa do alvo, com um limite de perda de PR maxima igual ao força mais bônus do punho da stiga do usuário. Por exemplo, um lutador tem força 6 e punho da stiga +4, e está usando um soco inglês +4. Ao atinjir um oponente com defesa 8, esse mesmo soco inglês perde 8 de PR. Porém se o inimigo tiver 20 de defesa, a arma perde apenas 10 PR (força mais bônus da habilidade punho da stiga).

Armas têm seu Pr variando entre 30 e 80, enquanto armaduras e escudos entre 50 e 100.


\section{Armaduras}

Armaduras protegem várias partes do corpo (pontos específicos como partes vitais, articulações etc), e devido a seu grau de proteção, podem privar o personagem de alguns movimentos. O uso de armaduras pesadas pode impor penalidades de movimento, reduzindo destreza e esquiva conforme o peso e tipo da armadura.

Algumas armaduras podem oferecer proteção especializada contra certos tipos de dano. Por exemplo, uma armadura de couro reforçado pode ter defesa normal 5+5 (5 de defesa normal e 5 de defesa automática) mas conceder uma redução adicional de +5 contra danos de corte, tornando-a especialmente eficaz contra espadas e lâminas. Da mesma forma, uma armadura de metal pode ser mais resistente a danos de contusão, enquanto uma armadura tecnológica pode oferecer proteção extra contra danos de energia.

Essas proteções específicas são indicadas na descrição da armadura e podem variar conforme o material, construção e tecnologia utilizada.
 

\section{Escudos}

Qualquer personagem pode usar escudos. O uso de escudos oferece bônus em aparar ataques.

O escudo tem duas características especiais. Tamanho e resistência. O tamanho do escudo diz a área de proteção do mesmo, ou seja, o bônus de aparar. Em outras palavras, sempre que um personagem tenta se defender usando um escudo, ele joga destreza + bônus do tamanho do escudo. A resistência diz o quanto um escudo pode suportar impactos, livrando o defensor de sofrê-los. O valor de resistência de um escudo é manifestado como um bônus de força quando usando a manobra aparar (veja o capítulo batalha para maiores informações). Resumindo, além do escudo conceder bônus em aparar, ele pode ajudar o defensor a suportar a força do atacante pois mesmo conseguindo interceptar o golpe, o defensor pode não suportar a força imposta pelo mesmo. Para projéteis, o teste de força na maioria das vezes é ignorado. 

De acordo com certas situações de combate (defensor sendo flanqueado por 2 personagens agéis por exemplo), o personagem que deseje aparar com o escudo pode ser impedido de realizar tal manobra ou faze-la com redutores.


\section{Arco e Flecha}

\subsection{Dano e Força}

O dano de um arco é igual à força do usuário, limitado pela capacidade máxima do arco. Cada arco tem um limite de força que pode suportar. Por exemplo, um arco que suporte até força 6 terá dano 6 mesmo se o usuário tiver força 10, mas terá dano 4 se o usuário tiver força 4.

\subsection{Alcance e Distâncias}

O alcance de um arco depende de sua construção e força máxima. Existem duas distâncias importantes:

\begin{itemize}
	\item \textbf{Distância Média:} A energia da flecha está concentrada, causando dano normal completo.
	\item \textbf{Distância Longa:} A energia foi dissipada para alcançar a distância, causando dano reduzido.
\end{itemize}

\textbf{Exemplo:} Um arco com força máxima 4 tem distância média de 0-35 metros e distância longa de 35-135 metros.

\subsection{Alcance em Combate}

Em situações de combate real, o alcance efetivo de um arqueiro é limitado por sua percepção. O alcance médio em combate é igual ao dobro da percepção do arqueiro (em metros), limitado pela capacidade do arco.

\textbf{Exemplo:} Um arqueiro com percepção 10 tem alcance médio de 20 metros em combate, mesmo que seu arco permita 50 metros.

O arqueiro pode aumentar esse alcance concentrando-se: cada turno mirando aumenta o alcance em concentração metros (até o máximo do arco).

\subsection{Características dos Arcos}

\begin{itemize}
	\item \textbf{Arcos Pequenos:} Ideais para combate próximo e médio alcance, fáceis de usar.
	\item \textbf{Arcos Longos:} Melhores para posições fixas e defesa territorial.
	\item \textbf{Arcos Pesados:} Podem ter redução no acerto devido ao dano elevado.
\end{itemize}

\subsection{Flechas}

O dano total é a soma do dano do arco + dano da flecha. Flechas podem quebrar quando atingem alvos com defesa automática maior que seu dano. Existem diferentes tipos:

\begin{itemize}
	\item \textbf{Flechas Anti-armadura:} Dano normal maior para penetrar armaduras.
	\item \textbf{Flechas de Combate Próximo:} Ponta mais pesada para maior impacto.
\end{itemize}

\subsection{Características Gerais}

\begin{itemize}
	\item Arcos devem ser usados com ambas as mãos
	\item Consomem pouco PF
	\item Perdem poucos PR por uso (principalmente por fatores externos como água e tempo)
	\item Distâncias muito longas são difíceis de mirar com precisão
\end{itemize}

\section{Armas Perfurantes}

Armas que tem o atributo perfurante ignoram um certo valor de armadura do alvo. Por exemplo, uma flecha 2+3 perfurante 2, ignora 2 pontos de armadura do alvo. 
Geralmente armas com atributo perfurante são armas cuja superficie de contato no mometo do dano é pequena em relação ao tamanho da arma em si, por exemplo, flechas, balas, etc.

\section{Armas de Fogo}

O uso de armas de fogo é bastante comum em alguns cenários. O uso dessas armas muitas vezes mortal tem alguns detalhes, que iremos expor a seguir.

\subsection{Cadência de tiro}
 
Armas mais antigas disparam apenas uma bala por ataque (tem apenas um ataque por rodada de ataque), porém armas de fogo automáticas ou semi-automáticas podem disparar diversos tiros em um único turno de ataque. A quantidade de tiros que uma arma pode realizar por turno é chamada de cadência de tiro. Existem duas formas de usar a cadência de tiro de uma arma. 

A primeira é a forma manual. O atirador divide sua destreza normalmente, da mesma forma quando está usando a manobra multíplos ataques. O número máximo de ataques realizados não pode ultrapassar a cadência de tiro da arma. 

A segunda forma é a automática. Nessa forma a arma realiza automáticamente varios tiros por turno, sem a necessidade de dividir a destreza. Porém, a medida que a arma realiza os disparos ela vai se tornando mais difícil de se controlada. Esse efeito é chamado de repuxo da arma. O repuxo é manifestado em um redutor cumulativo no acerto, onde cada arma tem seu próprio redutor de empuxo. A maioria das armas automáticas podem ser colocadas em modo normal de operação. O atacante não pode usar a manobra ataques multíplos nessa situação. A pessoa que tenta se desviar dos ataques poderá jogar apenas uma vez sua jogada de esquiva normal, porem vários tiros poderão acerta-la. 

Vamos a um exemplo geral: Biggs tem uma arma com cadência de tiro 4 e um repuxo de -1. Ele irá atacar com os 3 tiros graças a cadência de tiro da arma. O primeiro ataque será com sua destreza normal, o segundo com -1 e o terceiro com -2. Wedge tenta pular para se esquivar, porém apenas consegue um resultado maior do que o terceiro tiro, ou seja, dois tiros o acertam com dano normal. Vale lembrar que ele poderá jogar sua defesa para os dois danos dos tiros recebidos; uma jogada de defesa para cada tiro.

\subsection{Pontos de resitência de armas de fogo}

Observações sobre pontos de resistência em relação as armas de fogo.
\begin{itemize}
	\item Armas de fogo se desgastam a medida que são usadas, principalmente quando usadas em modo automático. Porém a perda de PF de uma arma de fogo é bem menor em comparação com outras armas e armaduras. 

	\item Ao contrário das armas normais, uma arma de fogo com baixo PR não tem redução em seus atributos, e sim uma chance de falha de funcionamento. Essa chance é de 10\% para valores de PR abaixo de 30\% do total, e de 50\% para valores de PR abaixo de 10\% do total. 

	\item Para recuperar os PR de uma arma de fogo, é necessário a habilidade manutenção de armas de fogo, e não forjar como na maioria das armas e armaduras normais.

	\item Geralmente armas de fogo básicas têm valor inicial de 30 PR. 
\end{itemize}

\subsection{Balas}

O dano total de uma arma de fogo é dividido em duas partes. O dano do impacto da arma e o dano da bala. Ou seja, se uma arma de fogo tem dano 6+4 e uma bala tem dano +6, o dano total será 6+10 para cada tiro. Uma arma de fogo geralmente é feita para ser compatível com apenas um tipo de munição, porém existem casos de armas de fogos que são feitas para dispararem diversas balas diferentes de acordo com seu modo de operação. Vale notar que, ao contrário de uma flecha que talvez possa ser reaproveitada após seu uso, sempre que uma bala é usada ela é descartada.


\subsection{Efeito mangueira}

Algumas armas têm a capacidade disparar todo o pente a elas equipado, dando um poder de ataque em área maior. Essa capacidade é chamada de efeito mangueira. A arma atinge todos os alvos dentro do campo do efeito mangueira, que é por volta de 45 a 60 graus partindo do atacante. Esse ataque é igual para todos os alvos, e tem um bônus de acordo com o bônus de efeito mangueira da arma. Os alvos atingidos por um ataque usando o efeito mangueira recebem uma quantidade de tiros igual a cadência da arma - 1. Além de perder todas as balas de um pente, a arma perde 10 PR sempre que o efeito mangueira for utilizado.

Para realizar o efeito mangueira existem algumas limitações, sitadas a seguir:

\begin{itemize}
	\item O alvo deve gastar todo o turno de ataque para realizar o efeito  mangueira;

	\item Algumas armas exigem uma força mínima para realizarem o efeito mangueira. Essa força mínima pode ser reduzida com a ajuda de suportes;

	\item Uma arma só pode realizar efeito mangueira se tiver munição suficiente dentro do seu pente;

	\item Mesmo que vários pentes possam ser acoplados a uma arma, ela deve esperar aproximadamente 2 turnos para resfriar e poder ser usada novamente.

\end{itemize}


\section{Limitação de Itens Mágicos}

Alguns equipamentos normais são melhorados graças a poderes sobrenaturais. Esses itens únicos e poderosos são conhecidos como itens mágicos ou artefatos. Uma espada artefato tem poderes especiais quando comparado a uma espada normal. Mas o uso desse tipo de item tem suas limitações. A maioria dos artefatos permite que um certo limite de outros artefatos sejam utilizados simultaneamente. Em outras palavras, se o limite de um item mágico é 2, então o usuário desse item mágico pode equipar no máximo outros 2 itens mágicos. Se um item for mágico, ele deve ser identificado como tal, e também ter sua limitação exposta.

Com conhecimento de magia um personagem pode "desligar/ligar" um artefato mágico gastando cerca de 5 minutos. Dessa forma ele pode carregar vários artefatos, porém usar somente os que desejar. 


Aqui mostraremos o padrão usado para criação e uso dos equipamentos. Tudo aqui mostrado é uma referência, podendo ser alterado pelo mestre.

\section{Dinheiro}
Segue a base do sistema monetário usado no mundo de ederu.
\begin{itemize}
	\item 1 ponto de ouro equivale a 10 ponto de jade.
	\item 1 ponto de jade equivale a 10 pontos de prata.
	\item 1 ponto de jade equivale a aproximadamente 10 dolares.
	\item 1 ponto de jade é poder monetário suficiente para dormir 1 noite em uma estalagem, com comida.
	\item Com 1 ponto de jade pode-se comprar 1 provisão que alimente por 1 dia durante uma exploração longa.	
	\item Ouro e jade são usados em qualquer parte do mundo.
	%\item Ouro e jade são normais em montanhas, ou seja, tem custo padrão.
	%\item Jade tem um valor maior em áreas costeiras.
	%\item Ouro tem um valor maior em florestas e cidades grandes.
	%\item Jade não tem muito valor em desertos, mas ouro tem muito valor.
	%\item Ouro nao tem muito valor em areas geladas, mas jade sim.
	\item Todo o dinheiro aceito dentro de jogo tem uma marcação realizada únicamente por algum governo local, dificultando ou impedindo a falsificação do dinheiro.
		\item Apenas humanos e gorions tem moedas proprias.
	
\end{itemize}

\section{Criando um equipamento}


A estrutura básica de criação de um equipamento é semelhante a criação de um personagem. Da mesma forma que para criar um personagem, o jogador escolhe uma raça e classe, para criar um equipamento, o mestre escolhe o seu tipo (escudo, espada média, armadura leve, etc). O tipo da arma vai determinar algumas limitações. Armas tem seu PR inicial igual a 60, armaduras Pr inicial igual a 80 e escudos e escudos têm seu PR inicial igual a 30. O valor máximo de PR que uma arma pode alcançar varia de acordo com a natureza da mesma.

Após escolher o tipo da arma, o mestre deve distribuir os pontos da mesma, semelhante a como um jogador distribui os pontos nos atributos de seu personagem, seguindo as regras mostradas a seguir. Para as informações abaixo leve em consideração que a pontuação base de um equipamento mundano é igual a 10. Vale notar que nem todo equipamento deve começar com 10 pontos, essa pontuação é a uma base. Uma faca simples de um ladrão pode ser construida com 5 pontos, enquanto que uma espada longa feita de um material raro pode ser construida com 14 pontos, por exemplo. Uma arma que use a pontuação máxima permitida, já considerando todos as restrições que concedem bônus, é chamada de obra prima.

\begin{enumerate}
	\item Cada ponto extra concedido em aparar ou acerto deve ser descontado da pontuação total da arma.
	\item Da mesma forma, cada ponto extra retirado em destreza ou esquiva deve ser acrescentado na pontuação total da arma.
	\item Cada ponto de dano/defesa do tipo automático do equipamento, deve ser descontado diretamente da pontuação total da arma.
	\item Cada 3 pontos de dano/defesa do tipo normal do equipamento, devem retirar 2 pontos da pontuação total da arma.
	\item Equipamentos com consumo médio de PF devem acrescentar 2 pontos na pontuação total da arma.
	\item Equipamentos com consumo grande de PF devem acrescentar 3 pontos na pontuação total da arma.
	\item Cada 10 pontos de PR retirados/acrescentados do limite do equipamento, acrescentam/retiram 1 ponto de sua pontuação total.
	\item Para armaduras, cada 2 pontos de cobertura devem reduzir 1 ponto da pontuação total da arma.
	\item Algumas armas têm restrições de atributo para serem usadas. Armas com restrição de atributo igual a 8 ou 9, recebem 1 ponto em sua pontuação total. Restrições de atributo a 10, acrescentam 2 pontos na pontuação total da arma.
	\item Outras restrições (usar a arma com duas mãos, locais aberto, etc) podem concedem de 1 até 4 pontos. Essa quantidade será definida pelo mestre segundo a severidade da restrição. 
	\item Restrições de material (que material a arma é feita) e também de manufatura podem concender bônus definidos pelo mestre. Por exemplo, a manufatura de armas de fogo é bastante restrita, o que concede a armas de fogo um bônus considerável em sua pontuação inicial
	\item Poderes e vantagens extras da arma podem reduzir pontos da pontuação total da arma. Por exemplo, uma arma de fácil criação ou com baixo consumo de PF pode ter sua pontuação máxima reduzida severamente.
\end{enumerate}

%O que "pontuação total máxima" quer dizer? Isso representa a pontuação usada para criar uma determinada obra prima (equipamento no limite mundano). Ou seja, se de acordo com os pontos acima, uma determinada arma fica com 14 pontos para serem usados em sua criação, quer dizer que se todos os 14 pontos forem usados nos atributos da arma (dano, acerto, Pr, etc), então aquela arma criada é considerada uma obra prima. Em poucos momentos da campanha os personagens vão se deparar com armas desse tipo. Essa medida acima é usada para orientar o mestre no momento de criar qualquer equipamento, seja ele obra prima ou não.

%Armas de ataque a distância (arco e flecha, bestas, armas de fogo, etc) seguem as mesmas regras de criação de armas, porém de sua pontuação total é descontado/acrescentado alguns fatores relacionados a natureza da arma. Pelo fato delas podem ser usadas a distância do oponente, isso concede uma grande vantagem na maioria das situações. Por isso são descontados de 2 a 4 pontos da pontuação da arma de ataque a distância no momento de sua criação. Essa redução varia de acordo com o alcance efetivo da arma e velocidade de uso, ou seja, a maioria das armas de fogo são criadas com 6 pontos como base. Porém o fator restritivo "construção" pode ser usado para conceder mais pontos a uma arma de ataque a distância. Por exemplo, uma besta ou uma arma de fogo são de dificil acesso para a maioria das pessoas, pois sua construção é dificil, exigindo uma tecnologia maior. Devido a esse fator, o mestre pode acrescentar de 2 a 6 pontos na arma de acordo com a tecnologia usada para sua criação. Outro fator restritivo importante é que uma arma de ataque a distância requer munição para ser usada. Devido a essa restrição o mestre pode acrescentar de até 4 pontos de acordo com a raridade da munição da arma. Essa exceção é mais usada para armas de fogo. Como esse tipo de arma tem um funcionamento diferente da maioria das armas, mais a frente é dedicado uma sub-sessão voltado unicamente para sua explicação. 

Armas de ataque a distância (arco e flecha, bestas, armas de fogo, etc) seguem as mesmas regras de criação de armas, porém de sua pontuação total é descontado/acrescentado alguns fatores relacionados a natureza da arma. Existem pontos positivos (distância de ataque), e pontos negativos (manufatura complexa) que colocam as armas de fogo em uma situação diferente das demais armas. Arcos são criados usando valores máximos de pontuação variando entre 8 e 12, enquanto que armas de fogo usando valores máximos entre 12 e 16. A criação de projéteis deve ser realizada usando menos pontos (6 pontos). Outro detalhe é se o projétil é descartável ou não, ou seja, um projétil não tem PR e sim uma certa percentagem dele se quebrar ou não após seu uso. A maioria das flechas poder ser utilizada caso o defensor não tenha tido um teste muito grande de defesa, o que não acontece em balas, onde elas são descartadas depois de seu uso. Geralmente projéteis reaproveitáveis concedem menos pontos do que os descartáveis.

Armas de arremesso indireto, ou seja, que o dano não é baseado na força a qual os projeta (granadas por exemplo), devem ter seus valores baseados na tecnologia que a constroi.

Alguns equipamentos podem conceder bônus auxiliares, sem contar o dano ou defesa do equipamento. Por exemplo, certas armas concedem bônus no acerto ou para o teste de aparar. Em armaduras esse bônus auxiliar é manifestado na forma de um incremento no Pr normal. Geralmente armas com um valor grande de dano automático tem um valor baixo de PR ou não concedem bônus em aparar ao usuário. Um machado com dano /8 pode conceder +2no teste de aparar e ter 80 PR.

Na hora de criar um equipamento o mestre deve levar em consideração bônus, restrições e a natureza da arma (não seria lógico um porrete ter dano automático grande). Mais a frente mostraremos uma lista de alguns exemplos de equipamentos criados com esses parametros para ajudar o mestre a criar seus proprios equipamentos.


Os equipamentos de ataque/defesa mundanos normalmente têm seu dano/absorção variando de 2 até 10. Quanto maior for o bônus automático do equipamento, mais avançada ele é. Ou seja, uma espada +8 é considerada uma arma de boa qualidade, porém uma espada +12 pode ser considerada uma obra prima de um artesão. Uma espada média com dano 10 + 10 não é considerada mundana e sim uma arma mágica, pois ela ultrapassa os limites de uma arma mundana. É considerado equipamento mágico qualquer equipamento que se encaixe em um dos fatores descritos a seguir:

\begin{itemize}
	\item Qualquer arma ou armadura que conceda bônus relacionados a fatores não diretamente relacionados com o equipamento; 
	\item Qualquer arma ou armadura que ultrapasse excessivamente os valores analisados para equipamentos mundanos;
	\item Qualquer equipamento auxiliar que conceda bônus (um anel que conceda bônus em força por exemplo).
\end{itemize}

Um anel mágico que conceda bônus em sabedoria ou dano mágico, uma armadura que conceda bônus em PV ou PF, ou uma bota que conceda bônus em esquiva e destreza são exemplos de iténs mágicos. Qualquer equipamento que esteja no seu limite de melhoria é chamado de obra prima.
	
Para determinar se uma arma ou armadura ultrapassou os valores mundanos o mestre deve analisar todos os atributos do equipamento, não somente o dano ou sua defesa. Por exemplo, uma espada média com dano 6+6, +2 acerto, +2 aparar e 40 PR, não é considerada arma mágica. Somando os bônus de dado e acerto temos 14 pontos. Porém considerando a retirada de 20 PR, concede a arma + 2 pontos em sua pontuação. Finalmente, devido a ela ser uma arma de consumo mediano de PF, ela recebe + 2, totalizando 14 pontos a ser distribuido. Como o bônus automático não é tão grande, o mestre nessa situação considera essa espada média como no limite de um equipamento mundano. Qualquer alteração nessa espada deve ser feita por um ferreiro especial, para transforma-la em arma mágica. O mestre também pode considerar alguns fatores negativos para equilibrar a arma fazendo com que ela seja considerada mundana, como exigir um valor mínimo de atributo para usa-lá, ou restringir para que ela só possa ser usada com 2 mãos. Por exemplo, o bastão longo e pesado conhecido como dai bo. O dai bo tem dano /12, PR 80, +2 acerto. Totalizando seus bônus temos ao todo 12 pontos distribuidos na arma. Levando em consideração que todo o seu dano é do tipo normal (mais fácil de se reduzir com defesa) e que é uma arma que deve ser usada com duas mãos, consideramos essa arma como mundana. Na verdade podemos ainda melhora-la em alguns pontos para que ela fique em seu limite de melhoramento, tornando essa arma uma obra prima. 

O mestre deve levar em conta todos esses fatores (tipo do dano, bônus extras, penalidades, natureza da arma, restrições de uso, etc) para criar uma arma ou limitar seu melhoramento. Lembre-se que da mesma forma que pontos negativos no uso de uma arma concedem bônus para a melhoria da mesma, o contrario pode ocorrer, como por exemplo, uma arma com baixo consumo de PF que pode ser facilmente escondida não terá tanto dano quanto um machado de guerra que só pode ser usado com duas mãos.	


Na hora de criar um equipamento o mestre deve ter conhecimento sobre a arma criada, e distribuir seus pontos e determinar seus limites de acordo com a natureza da arma. Por exemplo, não tem muita lógica criar uma espada curta com alto PR ou alto dano, já que elas são armas de porte pequeno e com pouco consumo de PF. A mesma linha de pensamento deve ser usada para a melhoria do equipamento, aonde o mestre pode atribuir redução no custo básico para aumentar certos atributos, por exemplo, fazer com que uma faca possa aumentar seu dano automático de forma mais rápida do que uma espada grande, esta que teria seu dano normal aumentado mais rápido. O mestre tem liberdade para alterar os fatores gerais de melhoria e criação de uma arma de acordo com a natureza da mesma. Os dados mostrados aqui são apenas para orientar o mestre. O erro de calculo de poucos pontos não torna o jogo desbalanceado.


\section{Melhorando e Criando Equipamentos}
Pode-se melhorar o dano ou defesa de um equipamento obedecendo a seguinte regra.

\begin{itemize}
	\item O personagem deve gastar 6 de jade para aumentar o dano/defesa em 1 ponto do tipo normal.
	\item O personagem deve gastar 8 de jade para converter 1 ponto do tipo normal em 1 ponto do tipo automático.
	\item O personagem deve gastar 4 de jade para aumentar o PR máximo da arma em 5 pontos.
	\item O personagem deve gastar 1 PO para inserir um bônus auxilar na arma (acerto por exemplo).
	\item O personagem deve gastar 1 ponto de jade para recuperar 3 Pr perdidos. %1 pr 4 pj magico
	\item O personagem deve gastar 4 ponto de jade para recuperar 1 Pr perdido de uma arma mágica. %1 pr 4 pj magico
	\item Uma arma mágica pode ser melhorada gastando os mesmos pontos de uma arma normal, porém deve-se usar material mágico como materia prima conseguindo dentro de jogo.
	\item Para aumentar a cobertura de uma armadura o personagem deve gastar uma quantidade de pontos de jade igual a 4 vezes o  nível desejado.
	\item Cada 2 pontos de jade gastos no teste de forjar/gunsmith reduz em 1 ponto a dificuldade do mesmo. Essa redução não pode ser maior do que o valor original do habilidade geral forjar/gunsmith.	
\end{itemize}

Caso o personagem tenha a habilidade geral forjar e equipamentos de forja, ele pode criar/melhorar uma arma fazendo um teste de destreza ou concentração + forjar. A dificuldade desse teste é igual a 12 + dano/defesa do tipo normal desejado + dobro do valor de dano/defesa do tipo automático desejado + eventuais bônus auxiliares desejados (acerto por exemplo). Por exemplo, se um personagem deseja criar uma arma 2+2, ele deve fazer um teste de forjar com dificuldade 18. Para melhorar essa mesma arma para 2+4, ele deve realizar outro teste com dificuldade 22. Cada teste dura mais ou menos 3 horas. Alguns materiais especiais podem ser usados para diminuir a dificuldade desse teste. Por exemplo, um pouco de mythril pode diminuir em ate 4 a dificuldade de um teste.

Para criar/melhorar um equipamento o personagem deve ter ferramentas. Além disso ele deve ter uma quantidade inicial de material de forja para construir cada equipamento. Em termos de material usado, o personagem gasta 1 PJ para cada ponto desejado de dano/defesa do tipo normal, e o dobro para cada ponto do tipo automático. Esse valores são usados para equipamentos com dano/defesa abaixo de 6 pontos. Acima disso, o mestre deve usar os mesmos valores usados na melhora de equipamento. Por exemplo, o personagem deseja fazer uma espada  média 3 + 3. Ele deve gastar 7 de jade para fazer uma espada 3 + 2 (totalizando até o momento 5 pontos limite inicial). Após isso ele deve gastar  mais 6 PJ para adicionar /1 de dano, e em seguida converter esse dano normal em automático, gastando 8 PJ nesse último passo. Totalizando 21 PJ no processo todo, que é aproximadamente 2 PO. Analisando todo o processo de criação vemos que os testes de criação e melhora devem ser feitos separadamente. Em relação a melhora ou recuperação de PR nenhum teste é exigido. Além diso, não é necessário nenhum teste por parte do ferreiro para inserir cobertura da armadura.


Para balas/flechas o processo é o mesmo, porém, o bônus do sucesso obtido determina a quantidade de balas/flechas criadas, sendo o valor máximo de 4 por teste. O personagem gasta 1 PJ por teste para balas e a mesma quantia para 5 flechas. Para testes relacionados com balas o personagem deve usar gunsmith, enquanto que para flechas o personagem pode usar tanto forjar ou oficios. Além disso projéteis dificilmente podem ser melhorados após a sua criação. O que ocorre é do ferreiro desmontar o projétil e refazer ele do zero, de uma forma melhor. Quando isso ocorre o personagem pode receber um certo valor em material para ser usado na redução da dificuldade do teste.

%Essa regra citada acima é usada para criar armas normais de forma rápida, que dura no máximo algumas horas de trabalho pesado. Em alguns casos especiais, o processo de criação de uma arma pode levar meses, fazendo com que o processo de criação seja mais complexo, envolvendo várias fases (preparação do material, melhora da base do equipamento, etc). Para tais casos o mestre, para termos de sistema, pode considerar todo o processo de criação como um processo de criação simples seguido de vários processos de melhoria do equipamento.


O melhorando de um equipamento não pode ultrapassar o valor básico de um equipamento mundando de acordo com sua natureza, ou seja, se um equipamento já distribuiu todos os pontos disponíveis a ele, o máximo que o artesão pode fazer é criar penalidades para aumentar a pontuação da arma ou de alguma forma redistribuir os pontos do equipamento.

A maioria dos ferreiros encontrados dentro de jogo têm forjar/gunsmith e atributos usados nos testes (destreza e/ou concentração) variando entre 6 a 8, ou seja, os valores usados para testes varia entre de 12 até 16. Além disso, boa parte dos ferreiros usa equipamentos que concedem bônus nos testes. De modo geral, testes com dificuldade variando entre 20 até 25 podem ser realizados sem problemas por ferreiros comuns. Qualquer valor acima disso, deve ser cobrado um preço adicional, que é de mão de obra mais pontos de jade adicionais usados na redução do teste.

O mestre deve levar em consideração que esses preço são apenas valores aproximados. Alguns ferreiros podem cobrar mais ou menos de acordo com a situação. Por exemplo, um ferreiro pode alugar gratuitamente seus equipamentos e sua forja para um grupo, se esse grupo prender um grupo de ladrões que o atormenta há algumas semanas. O mestre tem total liberdade para manipular tais fatores. 


\section{Material de Forja/Alquimia/Herbalismo/Gunsmith}


Material de Forja/Alquimia/Herbalismo/Gunsmith é uma espécie de moeda de cambio, porém que só pode ser usada para Forja/Alquimia/Herbalismo/Gunsmith. A relação é que 1 ponto de jade equivale a 1 ponto de material. Por exemplo, no lugar do personagem ir para um ferreiro e gastar 3 pontos de jade para recuperar 1 pr de sua arma, ele pode recuperar a mesma quantia usando 3 pontos de Material para Forja. Esse material pode ser obtido ao longo da aventura ou até mesmo comprado. Em certos locais abundantes desse material (minas, montanhas, etc), ele pode ser comprado a um valor muito baixo (1 ponto de jade valendo até 10 pontos de material para forja por exemplo).

Ao longo da aventura o mestre também pode (e DEVE) criar materiais especiais para serem usados pelos personagens com o mesmo objetivo do material para forja. Por exemplo, ele pode criar material para gunsmith, material para herbalismo, madeira especial (abaixa o custo de fazer flechas por exemplo), entre outros. Para facilitar o uso quantitativo, todo e qualquer material do mestre deve ser visto como uma moeda usada especificamente para aquele fim. Alguns materias podem valer mais do que outros, por exemplo, cada ponto de uma madeira especial (material criado pelo mestre) pode valer 1 po quando se tratando de compra de materiais feitos com madeira. Para determinados equipamentos (uma espada próxima de se tornar uma obra prima), o mestre pode incluir na lista de materiais necessários alguns materiais raros (jade vermelho do deserto de lorén por exemplo). Isso é criado com o objetivo de valorizar as habilidades dos personagens e também de incentivar a exploração de materiais, seja eles raros ou não.

\section{Poções}
Um personagem pode usar poções mágicas para PV, PF, PM, ou para atacar inimigos. As poções de cura de efeito imediato emitem uma espécie de radiação mágica e são consideradas poções mágicas. Cada personagem tem um limite de quantas poções mágicas pode carregar, caso contrário essa radiação mágica pode causar efeitos não desejáveis (envenenamento do personagem ou até explosão das poções). Levando isso em consideração, a quantidade máxima de poções mágicas que um personagem pode carregar é de 4. A baixo segue uma referência para poções de cura de efeitos imediatas ou não, além de algumas poções de ataque.

As poções podem ser criadas com as mesmas parecidas da criação de equipamento, onde o usuário deve realizar um teste com dificuldade igual a 12 + cura/dano da poção (dobre esse último valor caso a poção de cura tenha efeito automático). Cada ponto de jade/material gasto ao criar a poção vale 2 pontos. Por exemplo, para criar uma poção de cura de 8 PV de efeito imediato, o personagem deve realizar um teste com dificuldade 26, gastando 4 pontos de jade/material. Ele também pode usar 2 pontos de jade/material para diminuir a dificuldade do test em 1 até um valor igual ao nível mental do personagem.

\begin{itemize}

	\item Potion 10 pv, Efeito após 5 minutos, Custo de 4 de jade.

	\item Potion 20 pv, Efeito após 5 minutos, Custo de 1 PO.

	\item Potion 10 pv, Efeito imediato, Custo de 1 PO.

	\item Potion 20 pv, Efeito imediato, Custo de 3 PO.

	\item Ginseng 10 pf, Efeito após 5 minutos, Custo de 1 PO.

	\item Ginseng 10 pf, Efeito imediato, Custo de 3 PO.

	\item Ether 10 pm, Efeito imediato, Custo de 1 PO.

	\item Ether 18 pm, Efeito imediato, Custo de 2 PO.

	\item Poção de fogo, +16 de dano físico de fogo em uma área de 3 metros, Custo de 6 a 12 de jade.

	\item Granada Incendiária, /8+18 de dano físico de fogo em uma área de 5 metros, Custo de 2 a 4 PO.

	\item Granada de Projéteis /20+10 de dano físico em uma área de 5 metros, Custo de 2 a 4 PO.

	\item Bomba Venenosa, Os alvos dentro de uma área de 4 metros devem fazer teste de resistência df de 20 a 25 para não ficar envenenados (veneno 2). Custo de 1 a 2 PO.

		\item Bomba Paralisadora, 1 alvo , resistencia 25 , se passar -2dex-2esquiva(resi20 ou mais nao pega). 1 ou 2 PO.

\end{itemize}

\section{Equipamento Mundano}
Aqui mostraremos os dados e informações básicas de alguns equipamentos que podem ser usadas no sistema. A listagem aqui feita é baseada na lista de péricias bélicas. Vale lembrar que as armas aqui mostradas não são armas mágicas.

\subsection{Espada Curta}
Armas leves de fácil uso. Consomem pouco PF quando utilizadas, por isso é a melhor arma para lutadores que usem vários ataques. Além disso podem ser usadas como armas de arremesso. Apesas do dano relativamente pequeno, algumas concedem bônus no acerto devido a sua leveza. Não podem ser usadas para aparar armas grandes. 

\begin{itemize}
	\item Dagger = Dano +2, PR 30, Custo 2 PJ (ponto de jade).
	\item Peixeira Ginzu = Dano 2+2, PR 40, Custo 1 PO.
	\item Silver Dagger = Dano +4, +2 acerto, PR 40, Custo 5 PO.
	\item Katara = 3+3, +3 acerto, PR 40. 6 PO.
	\item Faca Élfica = 4+4, PR 50. 7 PO.
	\item Kris Malai = Dano 2+3, +1 acerto, PR 45, Custo 3 PO.
	\item Adaga de Assassino = Dano +3, +2 acerto, PR 35, Custo 4 PO.
\end{itemize}

\subsection{Espada Média}
Armas extramamente versáteis, podendo ser usadas para defesa ou ataque. Consomem uma quantidade normal de PF quando usada em batalha (1 pf é consumido a cada 3 ataques aproximadamente).

\begin{itemize}
	\item Espada Média = Dano 3+3, PR 40, Custo 2 PO.
	\item Espada Romana = Dano 5+2, PR 60, Custo 3 PO.
	\item Espada de Prata = Dano 4+4, PR 50, Custo 5 PO.
	\item Katana = Dano 1+5, +2 Acerto, PR 40, Custo 4 PO.
	\item Sabre de Cavalaria = Dano 4+3, +1 acerto, PR 55, Custo 4 PO.
	\item Espada Bastarda = Dano 3+4, +1 aparar, PR 45, Custo 3 PO.
\end{itemize}

\subsection{Espada Longa}

Armas pesadas que podem ser usadas apenas com duas mãos. O seu peso dificulta que outros oponentes usem a manobra aparar quando atacados por espadas longas. Consomem mais PF que as espadas médias e podem ser usadas para atacar oponentes em área. Recebem uma penalidade ou não podem ser usadas para aparar armas de pequeno porte quando o inimigo encontra-se muito próximo.

\begin{itemize}
	\item Espada Grande = Dano 4+4, PR 80, Custo 4 PO, Min força 6 com 2 mãos.
	\item Katana Grande = Dano 6+4, PR 60, Custo 5 PO, Min força 6 com 2 mãos.
	\item Buster Sword = Dano 8+4, -1 Destreza e Esquiva, PR 90, Custo 8 PO. Requer força 8 ou mais para usar essa arma com 2 mãos.
	\item Claymore Escocesa = Dano 5+5, +1 aparar, PR 85, Custo 6 PO, Min força 7 com 2 mãos.
	\item Zweihänder = Dano 7+3, +2 aparar, PR 95, Custo 7 PO, Min força 8 com 2 mãos.
\end{itemize}

\subsection{Lança}
A péricia bélica Lança concede ao personagem usar qualquer tipo de lança, seja ela média ou longa.

Lanças médias são armas boas pois podem receber bônus no acerto devido ao seu tamanho. Geralmente não PR elevados e são usadas em conjunto com outra arma ou escudo. Consomem a mesma quantidade de Pf que outras armas de porte médio. Devido a sua natureza de perfuração tem um bom dano automático.

\begin{itemize}

	\item Javelin De Mogno = Dano +3, PR 30, +1 acerto, Custo 6 de Jade.
	\item Javelin = Dano 1+4, PR 30, +2 acerto, Custo 3 PO.
	\item Javelin de Prata = Dano 2+7, +4 acerto e 50 PR, Custo 12 PO.
	\item Tridente = Dano 2+6, +2 aparar, PR 40, Custo 7 PO.
	\item Lança de Caça = Dano 1+5, +1 acerto, PR 35, Custo 4 PO.
	\item Pilum Romano = Dano 2+3, +1 aparar, PR 45, Custo 5 PO.
	
\end{itemize}

Lanças longas são armas usadas com as duas mãos, podendo ser usadas para o ataque ou para defesa, devido ao seu tamanho prolongado que pode ser usado para impedir a aproximação de oponentes. Têm um bom dano, porém baixo PR. Consomem mais PF que o normal. Recebem uma penalidade ou não podem ser usadas para aparar armas de pequeno porte quando o inimigo encontra-se muito próximo.

\begin{itemize}	
	\item Bambu Yari = Dano +5, PR 30, +2 acerto, -2 esquiva, Custo 2 PO, Min força 5 com 2 mãos.
	\item Alabarda = Dano 4+4, PR 50, +4 aparar, +2 acerto, -3 esquiva, Custo 8 PO, Min força 7 com 2 mãos.
	\item Naginata = Dano 1+7, PR 40, +2 aparar, +4 acerto, -2 esquiva, Custo 8 PO , Min força 6 com 2 mãos.
	\item Lança de Cavalaria = Dano 3+6, +3 aparar, PR 45, Custo 6 PO, Min força 6 com 2 mãos.
	\item Guisarme = Dano 2+8, +3 aparar, +1 acerto, PR 55, Custo 7 PO, Min força 7 com 2 mãos.
\end{itemize}	


\subsection{Bastão}
O bastão médio assemelhasse muito a lança média. Tem um dano maior, porém do tipo normal. Consome a mesma quantidade de PF que uma arma de porte médio.

\begin{itemize}	
	\item Jo = Dano /6, +4 aparar (2 mãos), +2 aparar(1 mão), +1 acerto,  PR 70, Custo 3 PO.
	\item Jo de Bambu = Dano /4, +2 acerto, +2 aparar(2 mãos)  PR 60, Custo 2 de jade.
	\item Bastão de Carvalho = Dano /5, +3 aparar, PR 65, Custo 2 PO.
	\item Cajado de Peregrino = Dano /3, +1 acerto, +1 aparar, PR 55, Custo 1 PO.
\end{itemize}	

O bastão longo consome mais PF que o normal, assim como as espadas longas, e tem um dano do tipo normal maior. Também tem grande quantidade de PF, mas devido ao seu peso e tamanho podem causar redutores na agilidade do usuário. Devem ser usadas com ambas as mãos.

\begin{itemize}	
	\item Bo = Dano /8, PR 70, +1 acerto, Custo 2 PO, Min força 6 com 2 mãos.	
	\item Daibo = Dano /12, PR 80, +2 acerto, -2 esquiva, Custo 5 PO, Min força 7 com 2 mãos.	
	\item Bastão de Ferro = Dano /10, +2 aparar, PR 85, Custo 4 PO, Min força 7 com 2 mãos.
	\item Quarterstaff = Dano /9, +1 aparar, +1 acerto, PR 75, Custo 3 PO, Min força 6 com 2 mãos.
\end{itemize}	



\subsection{Esmagador}
Um esmagador simples usa como principio a capacidade da arma em esmagar objetos. Têm alto dano normal, são facilmente encontradas, e tem uma boa quantidade de PR. O consumo de PF varia de acordo com o tamanho do esmagador, pois uma maior força será usada para esmagadores simples maiores. Provavelmente é o tipo de arma mais usado desde a antiguidade, levando em consideração que qualquer pedaço de pau pode ser usasdo como um esmagador. Alguns tipos de esmagadores, conhecidos como esmagadores compostos, são versões melhoradas de um esmagador simples (um machado que é uma versao melhorada de um martelo por exemplo). Um mangual é considerado um esmagador composto.


\begin{itemize}	
	\item Tonfa = Dano /3, PR 80, +3 aparar. Custo 3 de jade.
	\item Porrete = Dano /5, PR 80, +2 aparar, +2 agarrar. Custo 6 de jade.		
	\item Takapi = Dano /4+1, PR 60, Custo 6 de jade.
	\item Maça = Dano /6, PR 90, Cada ataque bem	sucedido retira 5 PR mais bônus de força da armadura/escudo. Custo 3 PO.
	\item Kanabo = Dano /6+1, PR 80, Custo 2 PO.	
	\item Clava de Guerra = Dano /4, PR 75, +1 aparar, Custo 4 de jade.
	\item Mangual = Dano /5+1, PR 70, +1 agarrar, Custo 5 de jade.		
\end{itemize}	


\subsection{Esmagador Grande}

\begin{itemize}
	\item Martelo de Guerra = Dano /8, +2 Aparar, PR100, Min força 6 com 2 mãos, -1 esquiva.	Custo 4 PO.
	\item Kanabo Grande = Dano /10+2, PR 90, Min força 7 com 2 mãos, -2 esquiva, Custo 4 PO.	
	\item Goremaul de Jade Reforçado (obra-prima) = Dano /14+2, +2 aparar, -2 Agilidade, Min força 8 com 2 mãos, Cada ataque bem sucedido retira 6 PR mais bônus de força da armadura/escudo, PR 100, Custo 18 PO.
	\item Maça de Duas Mãos = Dano /9, +1 aparar, PR 95, Min força 6 com 2 mãos, Custo 5 PO.
	\item Martelo de Thor = Dano /12, +3 aparar, PR 110, Min força 8 com 2 mãos, -2 esquiva, Custo 8 PO.
\end{itemize}	

\subsection{Arma de Fogo Leve}
Armas de fogo leve estão entre as armas mais usadas no mundo. Relativamente baratas em comparação com as outras armas, podem causar um grande dano se usado por um expert. Geralmente são armas não-autômatica e de cadência de tiro baixa, mas devido a sua leveza podem ser usadas em ambas as mãos. Não podem ser usadas para aparar armas. Seu alcance atinge até 50 metros.

\begin{itemize}
	\item Bereta = Dano 8, Cadência de Tiro 1, PR 20, Custo 8 PO.
	\item M1911 = Dano 8, Acerto +2, Cadência de Tiro 2 manual, PR 20, Custo 10 PO.
	\item Deseart Eagle Dano = 11, Acerto -2, PR 25, Custo 11 PO.
	\item Glock 17 = Dano 7, +1 acerto, Cadência de Tiro 2 manual, PR 20, Custo 9 PO.
	\item Revolver .357 = Dano 9, +1 acerto, Cadência de Tiro 1, PR 25, Custo 10 PO.
\end{itemize}

A maioria das armas leves aceita balas de calibre 2+2 (custo de 8 PJ por caixa), +4(custo de 2 PO por caixa), 6+2 (custo de 3 PO por caixa). O calibre da bala especifica o dano causado pela mesma. Cada caixa vêm com geralmente 6 balas. 

\subsection{Arma de Fogo Média}
Mais caras do que as armas de fogo, têm um poder de destruição maior, seja pela poder do calibre ou pela cadência de tiro elevada. A maioria das armas de fogo média devem ser usadas com ambas as mãos. Algumas requerem um maior nível de gunsmith para que seus PR sejam reparados. O seu alcance pode variar até 500 metros, dependendo do tipo da arma. Vale apenas observar que o dano de muitas armas automáticas não é o dano de 1 so tiro, mas sim de uma sequencia de tiros. Mesmo assim, cada ataque deve ser considerado, para termo de sistema, como 1 tiro separado, assim como o gasto daquela "bala".

\begin{enumerate}
	\item Sniper Rifle = Dano +8, Acerto +4, Alcance entre 20m e 500m, Calibre + 7, Cadência de Tiro 1, Custo de 15 PO, 1 PO por bala, PR30.
	\item Shotgun = Dano 12, Acerto -2, Alcance até 10m e alvos próximos, Calibre +4, Custo de 10 PO, 4 PJ por bala, PR30. 
	\item AK-47 = Dano 8+2, Acerto -2, Cadência de Tiro 3 Auto, Calibre 2+4, Custo 12 PO, Custo de 2 PO por caixa da bala, PR30. 
	\item M16 = Dano 7+3, +1 acerto, Cadência de Tiro 3 Auto, Calibre 2+3, Custo 13 PO, Custo de 2 PO por caixa da bala, PR30.
	\item Winchester = Dano 9, +2 acerto, Cadência de Tiro 1, Alcance até 200m, Custo 11 PO, 3 PJ por bala, PR30.
\end{enumerate}

Vale lembrar que o mostrado aqui para as balas é apenas um padrao. Então não seria surpresa um jogador poder encontrar uma bala com dano 4+6 para a arma AK-47.

\subsection{Arma de Fogo Pesada}
Armas raramente encontradas, tem munição limitada e preço elevado. São extremamente perigosas e de dificil  mauseio.

\begin{enumerate}
	\item RPG-7V = Dano /10, Alcance entre 50m e 100m, Calibre /10+15, Cadência de Tiro 1, Custo de 10 PO, 3 PO por bala, PR30.
	\item Rail Gun = Dano +22, dano elétrico. Acerto +2, Alcance até 100m, Calibre feito de balas especiais, 1 turno para concentrar. Custo de 20 PO. PR50. A rail gun é uma arma que descarrega uma corrente elétrica no alvo. Cada tiro consome um pouco de sua bateria, de forma que a bateria deve ser trocada por uma nova sempre que for descarregada. Em termo de sistema, cada tiro custa 4 PJ.	
	\item Bazuca = Dano /12, Alcance entre 30m e 80m, Calibre /12+18, Cadência de Tiro 1, Custo de 12 PO, 4 PO por bala, PR30.
	\item Metralhadora Pesada = Dano 10+5, Cadência de Tiro 5 Auto, Alcance até 300m, Calibre 3+5, Custo de 18 PO, 3 PO por caixa da bala, PR40.
\end{enumerate}


\subsection{Manopla}
Manopla é um equipamento que pode ser usado para reduzir o dano ou para aparar ataques, servindo como um escudo pequeno. A absorção da manopla não acumulada com outras armaduras (média e pesada) que o personagem esteja usando, podendo em alguns casos aumentar um pouco a absorção da armadura como um todo. A grande vantagen da manopla em relação ao escudo e armadura, é que o seu consumo de PF é muito baixo. Os PR máximos e mínimos de uma manopla são a metade dos considerados normais para uma armadura.

\begin{itemize}
	\item Bracelete Samurai = +2 Aparar, /2 Absorção, PR 60, Custo 4 de jade. Se usado com armadura concede cobertura +1.
	\item Manopla de Espinhos  =  +3 Aparar, PR50.	Custo 5 de jade.
	\item Manopla de Couro = +1 Aparar, /1 Absorção, PR 40, Custo 2 de jade.
	\item Manopla de Ferro = +2 Aparar, /3 Absorção, PR 70, Custo 6 de jade.
\end{itemize}	


\subsection{Armadura Leve}
Qualquer personagem pode usar uma armadura leve sem nenhuma restrição. Geralmente sua área de proteção é restrita a algumas partes do corpo, fazendo assim com que o atributo cobertura desse tipo de armadura não ultrapasse 3. A faixa de PR de uma armadura leve é igual a usada em armas, ou seja, de 30 a 80.

\begin{itemize}
	\item Colete Simples = Absorção /3, PR 40, Custo 2 de jade.
	\item Cota de Couro  =  Absorção /2+2, PR 50, Custo 1 PO.
	\item Kimono Reforçado  =  Absorção /4, PR 70, Custo 1 PO.	
	\item Armadura Romana  =  Absorção /2+4, PR 50, cobertura 2, Custo 7 PO.	
	\item Cota élfica de mythril =  Absorção +8, PR 60, Custo 13 PO.	
	\item Armadura de Couro = Absorção /3+1, PR 45, Custo 3 de jade.
	\item Armadura de Tecido = Absorção /2, PR 35, Custo 1 de jade.		
\end{itemize}	


\subsection{Armadura Média}
Facilmente encontadas, protegem boa parte do corpo. Aqueles que não estão acostumados com esse tipo de equipamento tem dificuldade na locomoção e movimento de membros. A cobertura máxima desse tipo de armadura é 6. A redução máxima em um atributo é igual a 3. Geralmente não apresentam requisito de força para seu uso.

\begin{enumerate}
	\item Armadura de Ferro = Absorção /4+2, PR 80, Custo 3 PO.
	\item Armadura de Mitril = Absorção /2+8, PR 50, cobertura 1. Custo 11 PO.
	\item Armadura Samurai = Absorção /5+5, PR 60, cobertura 1. Custo 6 PO.
	\item Armadura Jade Escuro = Absorção /8+2, PR 100, cobertura 2. Custo 7 PO.
	\item Colete Tatsunamei = Absorção +8, PR 30, Custo 6 PO.		
	\item Armadura Média Obra-Prima Anã = Absorção +12, PR 100, cobertura 4, -2 esquiva. Custo 20 PO.	
	\item Armadura de Aço = Absorção /5+3, PR 85, Custo 4 PO.
	\item Armadura de Bronze = Absorção /3+4, PR 75, Custo 3 PO.		
\end{enumerate}


\subsection{Armadura Pesada}
Armaduras pesadas sempre concedem penalidades no movimento do usuário, porém combrem uma área de proteção geralmente maior que 90\% do corpo do usuário, fazendo assim com que a cobertura natural dessa armadura (sem contar os pontos de distribuição de equipamento) seja igual ao maior redutor em um atributo. Mover-se com esse tipo de equipamento pesado consome uma quantidade considerável de PF. 

\begin{enumerate}
	\item Armadura Completa = Absorção /8+4, PR 100, -3 esquiva, -3 Dex, cobertura extra 3 (6 no total), Min força 6. Custo 8 PO.
	\item Armadura Pesada de Veridiun Vermelho = Absorção /8+8, -4 esquiva, Min força 8, -4 Dex, PR 100. Custo 13 PO.
	\item Colete Tatsunamei Anti Explosão = Absorção +13, -4 esquiva, -2 Dex, PR 40. Custo 13 PO.
	\item Armadura De Couro de Dragão (obra-prima) = Absorção /15+8, Min força 10, -5 esquiva, -2 Dex, PR 120, cobertura extra 2 (7 no total). Custo 18 PO.
	\item Armadura de Placas = Absorção /10+6, PR 110, -4 esquiva, -3 Dex, cobertura extra 2 (5 no total), Min força 7. Custo 10 PO.
	\item Armadura Gótica = Absorção /9+5, PR 105, -3 esquiva, -2 Dex, cobertura extra 1 (4 no total), Min força 6. Custo 9 PO.
\end{enumerate}


\subsection{Foice Pequena}
Essas armas são adaptações de instrumentos usados na agricultura. De fácil criação, podem ser fatais se usadas contra oponentes desprotegidos. Devido ao seu formato curvo, podem ser usadas para aparar algumas armas.

\begin{enumerate}
	\item Foice Simples = Dano 5+1, PR 50, +1 aparar. Custo 8 de jade.
	\item Foice Gorion = Dano 5+3, PR 60, +2 aparar. Custo 5 PO .
	\item Makraka = Dano 8+1, PR 70, +1 aparar. Custo 4 PO.
	\item Foice de Caça = Dano 4+2, PR 45, +1 aparar, Custo 6 de jade.
	\item Foice de Batalha = Dano 6+2, PR 55, +2 aparar, Custo 1 PO.
\end{enumerate}


\subsection{Foice Longa}
As armas da categoria foice longa também são conhecidas como gadanhas. São fatais a distâncias médias, recebendo um bônus no acerto e em aparar devido ao seu formato.

\begin{itemize}	
	\item Gadanha Simples = Dano +5, PR 30, +2 acerto, -2 esquiva, Custo 1 PO, Min força 5 com 2 mãos.
	\item Foice Longa de Cobre Negro = Dano 4+4, PR 50, +4 aparar, +2 acerto, -3 esquiva, Custo 7 PO, Min força 7 com 2 mãos.
	\item Gadanha de Guerra = Dano 3+6, PR 40, +3 aparar, +1 acerto, -2 esquiva, Custo 3 PO, Min força 6 com 2 mãos.
	\item Foice de Duas Mãos = Dano 5+3, PR 45, +3 aparar, +2 acerto, -2 esquiva, Custo 4 PO, Min força 6 com 2 mãos.
\end{itemize}	

\subsection{Escudo}
Escudo são excelentes equipamentos de proteção, uma vez que eles podem aparar ataques com mais facilidade que uma arma comum. Podem ser usados para aparar projéteis. Geralmente concedem bônus no teste de aparar (bônus esse variante de acordo com o tamanho do escudo) e no teste para resistir a força do oponente (bônus esse variante a resistência do escudo).

\begin{enumerate}
	\item Escudo de Madeira = Tamanho (+1 aparar). PR30.  Custo 1 de jade.
	\item Escudo do Guerreiro = Tamanho (+3 aparar), Resitência (+3). PR70.  Custo 4 PO.
	\item Escudo Nobre = Tamanho (+4 aparar), Resitência (+2). PR80.  Custo 5 PO.	
	\item Escudo Pesado Anão = Tamanho (+5 aparar), Resitência (+4), - 2 Esquiva, -1 Destreza. PR100.  Custo 8 PO.
	\item Escudo de Couro = Tamanho (+2 aparar), Resistência (+1). PR40. Custo 2 de jade.
	\item Escudo de Ferro = Tamanho (+3 aparar), Resistência (+2). PR60. Custo 3 PO.
\end{enumerate}



\subsection{Zarabatana}
São armas de simples, leves e facilmente camufladas. Devido a esses fatores seu dano não é tão elevado. A periculosidade dessa arma concentra-se na natureza do projétil usado, esse sendo muitas vezes venenoso. Geralmente atiram apenas um projétil por vez. O que diferencia uma zarabatana da outra é sua quantidade de PR e alguns bônus que podem ser concedidos na destreza para acerto. Como é uma arma de sopro, geralmente a força usada serve apenas para mover o projétil até seu alvo. Abaixo seguem algumas sugestões de projéteis que podem ser usados com a zarabatana. Alguns desses projeteis somente tem efeito caso acertem a pele do alvo, ou seja, o mestre pode incluir uma penalidade no acerto devido a cobertura da armadura que o alvo esteja usando.


\begin{enumerate}

	\item Dardo Venenoso = Causa Veneno (2) se o alvo não passar em um teste de resistência DF 15. 3 por 2 pontos de jade.	 
	\item Dardo Tapu = Causa Veneno (3) se o alvo não passar em um teste de resistência DF 20. 4 Pontos de Jade. 
	\item Dardo Enfraquecedor = O alvo recebe um redutor na força de 4 se não passar em um teste de resistência DF 15. 3 Pontos de jade.
	\item Dardo Ácido = Dano +5. Ignora armadura. 2 pontos de jade por dardo.
	\item Dardo Sonífero = O alvo deve fazer teste de resistência DF 18 ou ficar inconsciente por 1d4 turnos. 5 pontos de jade por dardo.
	\item Dardo Paralisante = O alvo recebe -3 destreza e -3 esquiva se não passar em teste de resistência DF 16. 4 pontos de jade por dardo.
	
\end{enumerate}


\subsection{KusariGama}
Arma exótica criadas pelos omayushas. Consiste de uma foice pequena anexada à uma corrente, com um peso extra no final desta. Podem ser usada de dois modos, de perto (usando a foice) ou a distância (usando a corrente).

\begin{enumerate}
	\item KusariGama (Essa arma pode ser usada de dois modos, usando a pequena foice para atacar ou a bola de ferro ). Modo Foice = Dano +4, Aparar +2. Modo Bola de Ferro = Dano /6, Acerto +4, Agarrar +4, Alcance de 2 até 6 metros. Nessa distância a arma é considerada um projétil e recebe os bônus de acerto e agarrar. PR 50. 5 PO. 
	\item HatariGama (Essa arma pode ser usada de dois modos, usando a pequena foice para atacar ou a estrela de ferro ). Modo Foice = Dano /6, Aparar +3. Modo Estrela de Ferro = Dano +6, Acerto +2, Agarrar +2, Alcance de 2 até 6 metros. Nessa distância a arma é considerada um projétil e recebe os bônus de acerto e agarrar. PR 50. 5 PO. 
	\item KusariGama Simples (Modo Foice = Dano +3, Aparar +1. Modo Bola = Dano /4, Acerto +2, Agarrar +2, Alcance 2-4m). PR 40. 3 PO.
	\item KusariGama Avançado (Modo Foice = Dano +5, Aparar +3. Modo Bola = Dano /8, Acerto +5, Agarrar +5, Alcance 2-8m). PR 60. 7 PO.
\end{enumerate}


\subsection{Bastão Curto Composto}
As armas enquadradas na categoria de bastão curto composto são todas as armas de porte pequeno e médio ligadas por uma corrente, como por exemplo Nunchaku, martelo meteoro, etc. Apesar de não terem um dano elevado, são facilmente transportadas e consomem pouco PF durante a luta. Além disso, podem ser usada para defender armas de porte médio de modo mais seguro. Algumas podem ser usadas como bastões pequenos.

\begin{enumerate}
	\item Nunchaku = Dano /4, PR 60, +2 Destreza. Se usado com 1 mão a arma perde o bônus em destreza. Custo 8 de jade.	
	\item Nunchaku de Ferro = Dano /5, PR 70, +2 Destreza, +1 aparar. Custo 1 PO.
	\item Martelo Meteoro = Dano /6, PR 65, +1 Destreza, +1 agarrar. Custo 6 de jade.
\end{enumerate}


\subsection{Lutar Desarmado}
Poucas armas são usadas para atribuir bônus a péricia bélica lutar desarmado. Armas que se enquadram nessa categoria são soqueiras e luvas.


\begin{itemize}	
	\item Soco Inglês = Dano /2, PR 30. Custo 2 Pontos Jade.
	\item Bracelete de Espinhos = Dano +1, PR 50, +2 aparar. Custo 3 PO.	
	\item Luva de Couro = Dano /1, PR 25, +1 aparar. Custo 1 ponto de jade.
	\item Soqueira de Ferro = Dano /3, PR 40, +1 aparar. Custo 4 de jade.
\end{itemize}	


\subsection{Chicote}
O chite é uma arma que não é considerava uma arma de ataque a distância, mas pode realizar ataques a distâncias maiores que uma arma branca normal. Também tem bônus para agarrar. Algumas podem causar dano extra, como dano por eletricidade ou veneno por exemplo.

\begin{enumerate}

\item Chicote = Dano /4, PR 40, +2 Destreza. Custo 2 de jade.
\item Chicotede Prata = Dano /4+2, PR 50, +1 Destreza. Custo 4 PO.
\item Chicote Tatsunamei = Dano /8, PR 30, +2 Destreza. O dano dessa arma ignora armadura, mas o usuário não usa a força para calculo de dano. So o alvo aparar, também leva o dano. Custo 10 PO.
\item Chicote de Couro = Dano /3, PR 35, +1 Destreza, +1 agarrar. Custo 1 de jade.
\item Chicote de Guerra = Dano /6, PR 45, +2 Destreza, +2 agarrar. Custo 3 PO.

  
\end{enumerate}


\subsection{Bestas}
Bestas são armas feitas com a capacidade de arremessar projéteis sem a necessidade de força do usuário. Geralmente são mais caras e de manutenção mais complexa do que arcos. Bestas pequenas podem ser usadas com uma mão, enquanto bestas médias só podem ser usadas com uma mão caso a força do usuário seja alta. A dano da besta não é somado com a força do usuário. O dano final de uma besta é seu dano base mais o dano do projétil acoplado à besta. O alcance da besta geralmente é menor do que um arco, porém o mestre pode usar a mesma logica de alcance médio e longo usada em arco para bestas. Os dados de munição para bestas podem ser os mesmos das flechas usadas para arcos, porém uma flecha de besta não pode ser usada em um arco. 

\begin{itemize}	
	\item Besta Simples = Dano /5, Alcance Máximo(Médio = 15m, Longo = 35m), PR 30, Custo 2 PO.	
	\item Besta do Caçador = Dano /3+2, +1 Acerto, Alcance Máximo(Médio  = 10m, Longo = 25m), PR 30, Custo 5 PO.
	\item Besta Mata Urso = Dano /10 , Alcance Máximo(Médio  = 15m, Longo = 25m), -3 acerto, Força acima de 8 pode usar com 1 mão, PR 30, Custo 7 PO.
	\item Besta Leve = Dano /4, Alcance Máximo(Médio = 12m, Longo = 30m), +1 acerto, PR 25, Custo 1 PO.
	\item Besta Pesada = Dano /8, Alcance Máximo(Médio = 18m, Longo = 40m), -1 acerto, Min força 6, PR 35, Custo 4 PO.
\end{itemize}	



\section{Artefatos Mágicos}


\subsection{Natureza}
Alguns materiais são capazes de emanar energia mágica, alterando assim a propriedade da materia ao seu redor. Essa alteração pode ser algo útil ou perigoso para todos aqueles em volta do material. O nível de seguranca de um material mágico é determinado pela sua natureza e também pela forma como ele é tratado. Por exemplo, exposição continua a materiais mágicos pode causar doenças físicas e até mentais. Alguns animais expostos a determinados materiais podem se tornar bestas violentas e perigosas. Podemos fazer uma analógia de um material mágico com algum material radiativo. Da mesma forma, equipamentos construidos com esses materiais mágicos tem sua propriedade alterada. Por exemplo, fisicamente falando é impossível uma espada de madeira cortar uma parede de ferro. Porém, se essa mesma espada de madeira tiver sido construida com algum material mágico (uma madeira mágica por exemplo), essa espada de madeira mágica pode cortar a parede. Equipamentos construidos com material mágico são chamados de artefatos mágicos.

\subsection{Slots}

O slot é uma propriedade de uma arma mágica que diz a sua capacidade de equilibrio com outros itens mágicos. Quanto maior o slot de uma arma, maior é o número de artefatos mágicos que podem "conviver" com aquela arma. 

A energia mágica emanada por um artefato não é tao intensa quanto a materia prima encontada na natureza. Isso ocorre devido ao fato de, no momento da sua criação, o material é lapidado de forma a emenar a energia mágica de forma mais harmoniosa. Porém, mesmo mais seguro que um material magico bruto, um artefato mágico ainda pode apresentar problemas em relação a sua emanação de energia mágica. Se a energia mágica emanada por vários artefatos entrar em ressonância, uma reação mágica perigosa acontece, como por exemplo, a explosão dos artefatos.

Devido a esse fator, um personagem deve carregar um número máximo de itens mágicos. O mesmo acontece com as poções mágicas, como explicado anteriormente. O límite de armas que um personagem pode carregar é igual ao menor slot de uma arma que ele esteja carregando. Por exemplo, se um personagem tem 3 artefatos com slots 8, 6 e 2, respectivamente, ele deve escolher somente 2 artefatos para poder usar.

O usuário pode gastar 2 PM para "desligar" um artefato mágico. Dessa forma ele pode carregar consigo esse artefato porém sem usufruir de seus poderes. Esse dura alguns minutos, e não pode ser feito instantâneamente. Ele pode ligar o mesmo artefato sem custo de PM extra, gastando alguns minutos concentrado-se.

Artefatos mágicos com slots "nulos" são raros ou muito fracos, pois eles conseguem conviver com qualquer outro tipo de equipamento mágico.

\subsection{Pr dos Artefatos}
O consumo de PR de artefatos mágico ocorre de mesma forma que equipamentos mundanos. A diferença é que boa parte dos artefatos mágicos não é exposta a perda de PR. Por exemplo, se uma armadura recebe um bônus devido a uma pedra mágica acoplada à ela, quando a armadura recebe dano quem perde PR é a armadura e não a pedra mágica. Mas vale notar que um item mágico não é indestrutível. O mestre deve levar em consideração esses fatores durante a campanha. 


\subsection{Acumulo de Bônus}
Não pode existir acumulo de bônus em artefatos mágicos, ou seja, se determinado personagem usa 2 artefatos que concedam bônus para 1 atributo em comum, apenas o maior é considerado.

Além disso, bônus temporários de habilidades que duram múltiplos turnos não acumulam com bônus de equipamentos mágicos no mesmo atributo. Quando um personagem possui tanto um bônus de habilidade quanto um bônus de equipamento mágico no mesmo atributo, apenas o maior é considerado. Por exemplo, se uma habilidade concede +10 em força durante 5 turnos e um equipamento mágico concede +5 em força, o personagem recebe apenas o bônus de +10 da habilidade.

Bônus que duram apenas um instante (como bônus de teste ou bônus de ataque único) não são afetados por esta regra e podem acumular normalmente com outros bônus.

\subsection{Exemplos}

\begin{itemize}	

\item Acessório Mágico = Um acessório mágico consiste de qualquer acessório (anel, brinco, luva, etc) capaz de conceder bônus em um atributo. Esse bônus é igual a 3 pontos em um atributo. Ou seja, um anel mágico da força, faz com que o usuário aumente sua força em 3 enquanto estiver usando o artefato. Esse bônus aumenta também atributo geral, porém o bônus do atributo continua o mesmo. Esse bônus também pode ser convertido em PF (4 pontos), PM ou PV (6 pontos). O slot de um acessório mágico desse nível é igual a 3. O seu custo varia de 7 até 10 PO em uma loja de itens mágicos.

\item Sigil do Carvalho = O sigil do carvalho é um símbolo mágico escrito com tinta mágica. Qualquer equipamento que tenha o sigil do carvalho tem sua quantidade de PR aumentada em 100. Slot 4. Custo de 5 a 7 PO.

\item Espada Mágica = Espadas desse tipo não possuem a lâmina, sendo esta feita de energia mágica. O usuário gasta 4 PM para "ligar" a lâmina de energia, que tem duração em minutos igual a sabedoria do usuário. Quanto maior for a concentração do usuário, mais bela e harmiosa é a lâmina. Qualquer ataque realizado por essa arma não pode ser aparado ou impedido por qualquer meio físico. O dano final é igual ao bônus de inteligência ou sabedoria (a escolha do atributo é feita pelo forjador da arma) do usuário + 6, ou seja,  não tem influência da força do usuário ou outros bônus relacionado, como especialização ou habilidades. Além disso o oponente deve jogar espírito para reduzir o dano, uma vez que a arma causa dano de toque. Arma de slot 3. Custo de 10 a 13 PO.

\item Capa de Proteção Mágica = Essa capa concede 6 de bônus contra ataques de dano mágico. Slot 3. Custo de 7 a 10 PO.

\item Pedra de Proteção = Essa pedra concede +6 de bônus contra ataques de dano físico. Slot 3. Custo de 10 a 14 PO.

\item Firane gun, Terrato shot gun = sabedoria + 6 firane, dano magico toque, o cara joga dex para acertar contra esquiva. Projetil carregado com PM. 4 projeteis, 8 PM. 8 a 12 PO. Tem que ter a pericia respectiva para poder usar a arma.

\item Twilight Armour = Absorção /5+7. PR 50. Além disso concede 4 de defesa mágica para o usuário. O usuário pode gastar reflexivamente 1 PM para receber +1 de defesa mágica por e ataque, até o máximo de 4. Slot 3. Custo de 13 a 16 PO.

\item Earth Protection = Absorção /8, +4PF, Slot 4, PR100 preco de 8 ate 10 PO.

% sugestoes de itens
% colar da sorte, o cara gasta tantos pm para poder jogar um dado a mais no teste
% magic essence gun, o cara pode atirar magias com um cajado;arma etc
% colar da virtude, o cara ganha bonus em todos os atributo sociais
%	Dark fog cloath +5 em testes de furtividade, slot 4
%	Armadura da dor. +15 de defesa fisica e +7 defesa magica. Custa 6PV acometido para ativar durante 1 hora. 40 pr slot2
% sussuros da noite.Por 4 pm o alvo pode usar seu focus como bonus automatico em defesa fisica/magica (max20). +6 fisica/ Magica Slot2 Requer consciencia maior que 15. Sempre que a habilidade for usda, a armadura perde 5 pr.

% para melhorar;criar item magico flexivel.

% pode usar as mesmas regras de forjar normalmente usadas, porem, poucas diferencas.
% cada ponto de atributo vale 2 PO.
% cada ponto aumenta 3 a DF basica.
% Df basica igual a 20.
% -1df basica igual a 3 pj.

% slot nullo, 
% slot 3, entre 10 e 20 XP (podendo ir ate 30 com uso de loot nivel 10 relacionado ao artefato ou acrescentar metade do xp maximo de acordo com uma restricao de igual custo em xp).
% slot 2, entre 30 e 45 XP (podendo ir ate 60 com uso de loot nivel 10 relacionado ao artefato ou acrescentar metade do xp maximo de acordo com uma restricao de igual custo em xp).
% slot 1, entre 80 e 100 XP (podendo ir ate 120 com uso de loot nivel 10 relacionado ao artefato ou acrescentar metade do xp maximo de acordo com uma restricao de igual custo em xp).



 
% os item embutidos podem dar poderes extras free de acordo com sua natureza.
% maior desvantagen de um item magico embutido eh q se a armadura quebrar o item tb quebra.
% alem disso, voce pode querer apenas uma caracteristica de um item embutido, e nao ele todo 
% por exemplo, um martelo que de sabedoria. um mago q nao use um martelo nunca vai querer.
% melhorar um item magico eh mais caro q o normal pq tem q usar item magico, entao um embutido pode ser um problema.


\end{itemize}	
