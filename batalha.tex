%%
%% Cap咜ulo 4: Batalha
%%

\chapter{Batalha}
\label{Cap:batalha}

O objetivo principal do sistema de batalha de dragon é fazer com que a luta se torne mais din穃ica poss咩el. ノ muito importante em um sistema de RPG que o seu sistema de combate tenha tr黌 principais caracter﨎ticas. Liberdade, velocidade e coer麩cia. 
	 
V疵ios detalhes como habilidades, pericias, magias e etc, ser縊 explicados em cap咜ulos posteriores. Vale notar que o jogador n縊 é obrigado a conhecer as regras de combate, quem deve conhecê-las é o mestre. O desconhecimento do sistema de batalha n縊 impede que um bom jogador realize lutas ruins. O conhecimento do mesmo torna mais r疳ido o fluxo da batalha.

O sistema de batalha de dragon será descrito atrav駸 de diversas situa鋏es de combate. Cabe ao mestre encaixar as a鋏es desejadas pelos jogadores, nessas a鋏es descritas neste capitulo. Esse conjunto de situa鋏es b疽icas podem dar origem a outras situa鋏es mais espec凬icas. Por exemplo, atacar é uma a鈬o b疽ica. Desarmar o oponente, é uma varia鈬o da a鈬o b疽ica atacar, por駑 com outro objetivo. Se um personagem desjar desarmar o oponente, ele tem duas op鋏es. Seguir a orienta鈬o de regra dado nesse cap咜ulo, ou alterar ao bel prazer a regra de ataque. Isso cabe ao mestre e jogador fazer durante o jogo, deixando o jogo mais fluido e divertido. Ou seja, nesse cap咜ulo s縊 dadas a鋏es de combate b疽icas, e sugests de como o mestre deve proceder caso o jogador queira outras a鋏es mais complexas.

\section{Momento de A鈬o}

Um momento de a鈬o corresponde a um intervalo de tempo que é igual チ soma de dois turnos, um turno de ataque e um turno de esquiva. Durante o turno de ataque o personagem pode realizar a鋏es mais complexas, como atacar, usar uma habilidade ou uma magia por exemplo. Durante o turno de esquiva, a鋏es mais reflexivas devem ser feitas tais como pular rapidamente para poder se desviar de um ataque, dar um grito sinalizando uma ordem, etc. Portanto podemos dizer que durante o turno de ataque o tempo para se realizar a鋏es é maior do que o tempo para se realizar a鋏es no tempo de esquiva. Se algu駑 tenta realizar uma a鈬o mais longa no turno de esquiva, terá penalidades para realizar tal a鈬o, por outro lado quem desejar realizar a鋏es simples no turno de ataque terá bus.
	
	No momento que acontece o seu turno de ataque, acontece o turno de esquiva do inimigo, e assim por diante. Mas isso n縊 quer dizer que você n縊 possa tentar realizar um ataque em seu turno de esquiva (manobra contra-ataque). Mais detalhes explicados posteriormente.
	
	Em suma, para termos de sistema um momento de a鈬o é dividido em dois turnos. Um turno de ataque e um turno de esquiva. Um momento de a鈬o tem mais ou menos dura鈬o de 3 a 4 segundos.  O mestre deve usar esse valor apenas como refer麩cia para se basear na dura鈬o de determinadas batalhas. Mas o contexto da situa鈬o de combate descreve melhor a dura鈬o de todo o combate. 

\section{Iniciativa}

Para calcular a ordem das a鋏es, todos jogam agilidades para a鋏es f﨎icas e percep鈬o para a鋏es mentais ou ativa鈬o de habilidades. A ordem dos sucessos determina a ordem das a鋏es. 
O personagem pode tamb駑, caso ganhe a iniciativa, optar por n縊 atacar. Nesse caso ele terá um bus em qualquer a鈬o que ele quiser realizar no turno de esquiva (apenas em a鋏es que n縊 sejam de ataque, ou seja, a鋏es de esquiva) e esse bus depende de sua agilidade/percep鈬o. 

\section{Ataque}

\subsection{Um ataque normal}

O atacante realiza um teste de destreza para atacar. Guarde esse valor para usar posteriormente se o defensor for esquivar. Em outras palavras, ele faz um teste combatido contra o defensor. Se o atacante conseguir ganhar no teste ele consegue acertar o golpe. O numero de sucessos extras no ataque concede um aumento no dano final (detalhes explicados posteriormente).
Ataques usando magias n縊 obedecem essa regra, tendo regra prria explicando em capitulo posterior.

\subsection{Ataque Inconsequente}

Nesta manobra, o personagem ataca de forma louca e sem precau鈬o, visando apenas a destrui鈬o do oponente. Ele vai perder o turno de esquiva (podendo neste turno realizar apenas uma manobra de defesa, a de absor鈬o total, explicada mais a frente) e n縊 pode realizar as manobras de combate ataque em alvo espec凬ico ou m伃tiplos ataques. O mestre deve considerar tamb駑 o bus de for軋 para determinar a diferen軋 da for軋 atacante-defensor para fins da manobra aparar.

O personagem que realizar a manobra ataque inconsequente ganha bus de atributo em destreza e em forca para 1 ataque nesse turno. Essa manobra só é valida para ataques f﨎icos feitos corpo a corpo, ou seja, n縊 é usada para armas projet疱eis, como arcos, armas de fogo, etc. 

No seu primo turno de esquiva, como dito anteriormente, o personagem n縊 pode esquivar. Ou seja, o valor do teste de esquiva do usu疵io é igual a sua esquiva, sem a jogada de dados (ele n縊 pode substituir esse valor pelo usado na manobra aparar). Use esse valor como base para calcular bus no dano proveniente dos sucessos obtidos no ataque.

\section{Esquiva}

O personagem pode usar 3 manobras de esquiva.

\subsection{Esquiva Normal}

Ele pode esquivar-se normalmente jogando sua esquiva e testando contra a destreza de quem o atacou. Em outras palavras, o que define se o defensor se esquivou ou n縊 é o resultado do teste combatido da destreza (do atacante) contra a esquiva (do defensor). O numero de sucessos extras na esquiva serve apenas para descrever o grau de 黜ito da mesma.

\subsection{Aparar}

Ele pode aparar o golpe usando um objeto. Uma espada, lan軋, escudo, ou algum equipamento que ele saiba utilizar pode ser usado para desviar o golpe do atacante. Nesse caso ele joga destreza no lugar de esquiva. ノ um movimento bastante 偀il para quem tem baixa esquiva, por駑 esse movimento cansa mais do que uma simples esquiva (consome mais pf no momento da batalha). Cada 2 ou 3 usos da manobra aparar (de acordo com a tamanho da arma, diferen軋 entre for軋s defensor/atacante, etc), o usu疵io perde 1 PF. Al駑 disso o mestre pode retirar uma quantidade extra de PR da arma por ser usada como objeto parar aparar o golpe do advers疵io.

No caso do defensor ter uma for軋 menor em rela鈬o ao atacante em 4 ou mais pontos, ou da arma do atacante ser pelo menos 1 porte maior que a arma do defensor, o atacante recebe um bus em sua jogada de acerto igual ao seu bus de for軋. Al駑 disso, o mestre tem total liberdade para alterar os valores nesse teste, por exemplo, no caso do atacante usar armas leves a uma curta dist穗cia de um defensor que tente aparar com uma arma grande, o defensor recebe uma penalidade. O mestre pode usar qualquer interpreta鈬o da situa鈬o de combate para torna-lo mais din穃ico.

%Por exemplo, quando a diferen軋 entre for軋s é consider疱el, o atacante pode jogar destreza com um bus de for軋 contra a destreza do defensor. O mestre tamb駑 pode dar bus ao atacante de acordo com a dist穗cia de ataque, natureza favoravel contra a arma do atacante/defensor, etc. Al駑 disso o mestre pode reduzir o dano de acordo com a situa鈬o, por exemplo, em caso de empate no teste de aparar. 
	
Se o defensor pretende usar essa manobra para se defender de um proj騁il (um ataque a dist穗cia), dobre o valor do teste final da destreza do atacante, já que a 疵ea de um proj騁il é bem menor que de uma espada, por exemplo. Al駑 disso, o mestre pode decidir aumentar ainda mais a penalidade para se defender projeteis (alguns n縊 permitem a defesa de projeteis usando armas). A maioria dos projeteis pode ser defendida independente da for軋 do defensor. Alguns proj騁eis n縊 podem ser aparados usando essa manobra, o caso de esferas de energia ou proj騁eis pequenos.
	
\subsection{Absor鈬o Total}

Nesse caso o defensor n縊 tem direito a jogada de esquiva ou destreza(no caso de aparar), usando apenas seu valor bruto sem a soma de qualquer dado. Por駑, n縊 precisa jogar defesa, minimizando o dano com a defesa total. ノ o 佖ico movimento de esquiva que pode ser feito no turno de esquiva quando se usa um ataque inconsequente. Teoricamente nessa manobra o movimento de esquiva foi o m匤imo poss咩el, pois o corpo fica mais posicionado para receber o impacto do que para se esquivar dele. Caso o atacante acerte o defensor, os sucessos extras do atacante resultam em certo bus no acerto. O atacante somente erra o alvo caso tenha um numero de sucessos menor do que o atributo esquiva do defensor que esta usando a manobra absor鈬o total. Essa manobra consome 1 Pf do usu疵io. Com esse consumo o usu疵io pode usar sua defesa m痊ima para um n伹ero de ataques igual a seu bus de defesa. Ele pode gastar novamente 1 PF para "re-ativar" a manobra.
 
\section{Tipos de Dano}

\subsection{Dano Normal}

Representa o dano que pode ser reduzido sem auxilio de armadura especial. Dano por contus縊 (socos, chutes) e afins s縊 exemplos de danos normais. O dano normal é representado por uma " / " 
	Todo dano usando apenas o atributo for軋 (sem auxilio de habilidade especial) é considerado como dano normal. Você pode somar todos os danos normais para fins de calculo de dano normal total. Por exemplo, se personagem com for軋 6 usa um porrete /3, pode-se considerar a como dano normal total 9 (/6 + /3 = /9).


\subsection{Dano Autom疸ico}

O dano letal é um tipo de dano mais dif兤il de absorver sem armadura apropriada. Normalmente é causado por ataques cortantes, quentes, 當idos e afins. ノ representado por um \" + \". Ou seja, quando a for軋 de um personagem mostrar 5 + 1 n縊 quer dizer que a for軋 dele é 6, e sim que a for軋 dele é 5 por駑 esta associada com +1 de dano autom疸ico. Para se minimizar +1 de dano é necess疵io /3 de defesa. Apesar de ser mais dif兤il de defender, 1 de dano autom疸ico retira a mesma quantidade de dano que 1 de dano normal. 
Vamos supor que uma pessoa seja atacada por uma espada +2 e tenha conseguido no seu resultado de defesa um valor igual a /5. Ela conseguira reduzir +1 de dano, reduzindo o dano total para +1. Ent縊 ela vai perder apenas 1 de dano e n縊 3. Se por exemplo a espada fosse um porrete, seria /2 (dano normal), e a pessoa com /5 de defesa poderia reduzir o dano total a 0, n縊 perdendo nenhum PV com o ataque.

\subsection{Lembrete Sobre Dano}

Tanto o dano normal quanto o dano autom疸ico s縊 semelhantes em termos de dano (quantidade de pv tirada), o que os diferencia é a capacidade de um ser absorvido facilmente enquanto o outro n縊 . +1 de dano autom疸ico pode ser absorvido com defesa /3, por駑 /1 de dano normal pode ser absorvido com /1 de defesa.

	Da mesma forma existe a defesa autom疸ica e normal. A defesa autom疸ica serve para defender igualmente +1 dano, por駑 na mesma intensidade de defender /1 de dano. Em outras palavras, uma armadura +4 defende da mesma forma +4 ou /4.
	
\section{Dano}

Para se calcular o dano de um ataque (dano final), deve se calcular o dano inicial e a redu鈬o de dano. O dano que o personagem vai receber é o dano inicial menos a redu鈬o de dano. 

Para se calcular o dano inicial, execute o seguinte procedimento. Primeiramente some no dano do atacante o bus do sucesso no teste de acerto como sendo dano normal. Esse bus n縊 pode exceder o valor do bus de for軋 permanente do mesmo para ataques corpo a corpo, e no caso de ataques a dist穗cia o bus proveniente do sucessos extras n縊 pode ultrapassar o dano base da arma (n縊 do proj騁il). Se por exemplo o atacante consegue 17 no teste de acerto e o defensor consegue 10 no teste de esquiva, a diferen軋 é de 7. Ent縊 o atacante terá um bus de /3 no dano (3 é o valor do bus de 7). Se o atacante tiver for軋 5, esse bus será /2, pois o bus de sua for軋 é 2. Isso pode ser explicado analisando que quanto maior o sucesso do atacante em cima do defensor, mais eficiente será o golpe. Existem algumas armas e habilidades que aumentam esse limite. Ap calcular o dano proveniente de sucessos extras, o atacante soma separadamente os tipos de dano (autom疸ico e normal). Esse valor será o dano inicial.
	
Para saber quanto de redu鈬o de dano o defensor tem, joga-se um dado e o numero respectivo na tabela de defesa será a quantidade de dano normal que ele vai reduzir. Consulte o cap咜ulo sobre a descri鈬o da ficha para saber como se preencher a tabela de defesa. A defesa que é jogada e consultada na tabela é do tipo normal(/).  N縊 é necess疵io jogar o dado para se calcular a defesa autom疸ica.  Nesse momento, o defensor saberá sua absor鈬o para o dano normal e autom疸ico.
Subtraia dano autom疸ico com a absor鈬o para autom疸ico e o dano normal com absor鈬o normal. O excedente da absor鈬o autom疸ica pode reduzir o excedente de dano normal assim como o excedente da absor鈬o normal pode reduzir o excedente de dano autom疸ico, obedecendo a regra de que /3 de absor鈬o normal reduzir +1 de dano autom疸ico.  O dano ap os c疝culos é chamado de dano final.
A jogada de absor鈬o é uma rea鈬o do corpo, uma rea鈬o ao golpe, ou seja, mesmo que ele seja atacado varias vezes no mesmo turno sempre que ele for atingido o defensor deve executar esse procedimento. A regra explicada acima vale para danos f﨎icos, detalhes sobre o calculo de dano m疊ico explicado no capitulo sobre magias.
	O mestre pode interpretar como o alvo sofreu o dano de acordo com a quantidade m痊ima de pv. O mestre pode interpretar um golpe que tire 15 de PV de um personagem que tem 20 de PV total como quase um bra輟 decepado ou um olho perfurado por exemplo. A mesma quantidade de dano em um personagem com 100 de PV pode ser vista como um arranh縊.

\textbf{Exemplo de c疝culo de dano.}
	
O atacante tem for軋 5 e uma faca +4. Ele teve 16 no teste de destreza e o defensor 10 no de esquiva. Portanto, o atacante teve 6 sucessos extras, dando um bus em seu dano de /2. O seu dano inicial vai ser /7+4. O defensor joga o dado e consultando a tabela, conseguindo uma absor鈬o normal igual a /10.  Ele tem um colete +2, totalizando /10+2. Esse será o valor de sua redu鈬o de dano. Primeiro diminu匇os os tipos de danos semelhantes. /7 de ataque é reduzido pelos /10 defesa, sobrando /3 para a defesa. Os +4 de ataque s縊 reduzidos em +2 pela defesa +2 do defensor. Sobra /3 de defesa para reduzir +2 de dano. Como /3 de defesa reduz +1 no dano, no final o defensor perde apenas 1 pv.


\section{Gastando PF}

Durante uma batalha é normal o desgaste f﨎ico dos combatentes. Isso é refletido na perda de PF. Normalmente em uma batalha um lutador, atacando e se esquivando 1 vez a cada turno usando armas leves, perde 1 PF a cada 4 turnos aproximadamente, 1 PF a cada 3 turnos quando usando equipamento de porte m馘io e 1 PF a cada 2 turnos quando usando equipamento de porte pesado. Mas existem alguns fatores que aceleram a perda de PF:

\begin{itemize}
	\item Manobras de combate em geral. Atacar v疵ias vezes, aparar, entre outros. Qualquer manobra de combate utilizada acelera a perda de PF. O mestre deve retirar 1 PF extra por cada tr黌 manobras realizadas, exceto ataque inconsequente e absor鈬o m痊ima, que cada uso dessas manobras consomen 1 PF autom疸icamente. O uso excessivo da manobra ataques mult厓los (mais de 2 ou 3 ataques em um turno por exemplo) pode retirar 1 PF por turno. 

	\item Uso de abusivo de equipamento pesado. Quanto maior for o nä¼¹ero de equipamentos pesados usados durante o combate, mais PF é perdido. O uso de 3 equipamentos pesados ao  mesmo tempo acarreta a perda de 1 PF por turno. 
	\item Uso de habilidades espec凬icas. A maioria das habilidades consomem PF.

	\item Locomo鈬o. Seja para fugir ou para alcan軋r alvos, o mestre deve exigir o gasto de PF quanto maior for a locomo鈬o do alvo na luta.

	\item Condi鋏es clim疸icas desfavor疱eis tamb駑 podem ser usados como fatores agravantes na perda de PF em rela鈬o ao desgaste f﨎ico natural sofrido nas batalhas.
\end{itemize}

No lugar de monitorar o que cada personagen faz em certos intervalos e ir tirando os PF de acordo com as regras acima, no final da batalha o mestre retira de 1 a 4 PF dos personagens por crit駻io pessoal, levando em considera鈬o condi鋏es que retirem mais PF que o normal (usar muitas manobras de combate, peso do equipamento, dura鈬o do combate etc).


\section{Usando Itens}

ノ comun durante as lutas os personagens usares determinados medicamentos de a鈬o instant穗ea para recuperarem seus PV,PM ou PF, ou drogas para aumento tempor疵io de atributos. Um item pode ser usado em você mesmo ou em um aliado dentro do seu campo de a鈬o durante o seu turno de ataque. Para realizar tal a鈬o durante a batalha o mestre deve ficar atendo as situa鋏es de luta para atribuir penalidades ou até mesmo privar o jogador de realizar ataques nesse turno. 

Sempre que um personagem for usar um item em um aliado, ele perde todo o turno de ataque realizando essa a鈬o. O mestre pode privar o uso para os aliados チ ele primos, ou dentro do seu campo de a鈬o. Caso um personagem deseje usar um item nele mesmo, e esse item encontra-se de f當il acesso, o personagem pode usa-lo sem penalidade nenhuma em seu ataque. Por駑 se o personagem usa alguma armadura ou encontra-se com as m縊s ocupadas (segurando um arco e flecha por exemplo) o mestre pode fazer com que o personagem perca seu turno de ataque localizando o item.

 

\section{Regras Opcionais para Combate}

Abaixo segue a explica鈬o de v疵ias manobras de combate utilizadas no sistema dragon. Essas manobras servem para tornar o combate mais din穃ico poss咩el. Sempre que o mestre tiver d俿ida como proceder perante uma manobra, ele pode usar as informa鋏es a seguir para ajudar em sua deci鈬o final de como prosseguir. Todas as manobras de combate consumem mais PF que as a鋏es padrs normais. 

As regras abaixo tem o 佖ico objetivo de mostrar como as regras b疽icas de combate s縊 alteradas para que diferentes situa鋏es mais complexas de combate possam ser usados. Em outras palavras, o mestre pode altera-l疽 ou até mesmo n縊 utilizar as regras abaixo que nada vai atrapalhar o funcionamento b疽ico do sistema. Elas foram criadas para dar um maior grau de detalhes aos combates.
	
\subsection{Ataques M伃tiplos}

Um personagem pode optar por no lugar de dar apenas um golpe bem visado, dar v疵ios golpes por駑 com sua destreza menor. Isso pode ser feito se o personagem dividir a destreza para realizar um golpe, ou seja, um personagem que tem destreza 4 dar dois golpes com destreza 2, realizando 1 ataque extra. Cada mestre pode estabelecer um numero limite de golpes extras que o personagem pode dar. Aqui iremos dar uma m馘ia que pode ser usada. De cada 4 em 4 pontos em destreza o personagem pode dar um golpe extra, ou seja, um personagem com destreza 7 poderia dar 2 golpes (1 normal e um extra) com destreza dividia (4 para um e 3 para outro ou 5 para um e 2 para outro) mas n縊 poderia dar sete golpes com destreza 1. Vale observar que que qualquer bus recebido em destreza entra no c疝culo antes da divis縊 de ataques. Caso alguma habilidade conceda ao personagem bus em um ataque, o personagem deve escolher em qual dos ataques (caso executando a manobra ataques m伃tiplos) esse bus deve contar.

A distribui鈬o de destreza entre os golpes deve ser escolhida pelo jogador, obedecendo a seguinte regra: O valor da destreza dividida para cada ataque n縊 deve ser maior que o bus de destreza do usu疵io. Por exemplo, um personagem com destreza 8 (bus igual a 3), pode dividir 2 ataques em um ataque com a destreza 5 e o outro ataque com a destreza 3, mas n縊 pode alocar 2 de destreza para 1 ataque e 6 para outro ataque, pois a diferen軋 entre o primeiro ataque que tem destreza 2 e o segundo ataque que tem destreza 6 é 4, valor esse maior que o bus de destreza que é 3.

Uma regra opcional para a forma como a destreza é dividida é que caso o personagem aloque 2 vezes consecutivas 1 ponto quando dividindo a destreza, o defensor n縊 tem penalidade na terceira esquiva (penalidade essa descrita mais a frente). O uso de duas armas pode aumentar o numero de golpes extras e reduzir um pouco o consumo de PF ao usar essa manobra. 

Quando um personagem utiliza a manobra ataques m伃tiplos no primo turno de esquiva qualquer a鈬o envolvendo destreza ou esquiva tem um redutor igual ao numero de ataques extras.
	
No caso de você está dando v疵ios golpes e o oponente n縊 poder esquivar, o mestre pode atribuir um valor de dificuldade m匤imo para você acertar o golpe. Isso fica a crit駻io do narrador.
	
Tanto a quantidade de golpes, quanto as penalidades que cada golpe tiver (se tiver) podem ser estabelecidas pelo mestre. Existem mestres que n縊 limitam a quantidade de golpes outros se guiam pelo bus de destreza (cada bus de destreza é um golpe a mais). Isso cabe ao mestre decidir. Aqui n apenas mostramos uma m馘ia que pode ser usada, uma vez que o narrador pode utilizar do modo como bem desejar. Existem algumas habilidades que retiram esses limites, cedendo ataques extras ou bus aos atacantes. 

\subsection{Esquivas M伃tiplas}

Existem duas situa鋏es diferentes para se usar esquivas m伃tiplas. A primeira é se esquivando de um só oponente, este realizando v疵ios ataques. Nesse caso, para cada ataque extra do atacante o defensor recebe um redutor cumulativo de -2. Por exemplo, o atacante realiza 3 ataques. O defensor n縊 terá nenhum redutor no primeiro golpe, por駑 recebe um redutor de -2 no segundo (primeiro ataque extra) e -4 no terceiro (segundo ataque extra).

A outra situa鈬o é v疵ios oponentes te atacando.  Para cada oponente extra, o defensor recebe um redutor cumulativo de -1 para aquele atacante, n縊 importando quantos ataques outros oponentes realizaram. Por exemplo, 3 atacantes atacam 1 alvo. Para se esquivar dos ataques do primeiro, o defensor n縊 tem redutor (tendo redutor apenas se o atacante realizar v疵ios golpes), já durante os ataques do segundo oponente, o defensor recebe uma penalidade de -1 para todos os ataques realizados por aquele atacante. O m痊imo de atacantes usando armas de combate corpo a corpo em um só oponente é 5. Quando usando armas de ataque a dist穗cia, esse nä¼¹ero aumenta para 20 ou de acordo com o mestre.

Existe tamb駑 outra ocasi縊 de batalha, onde se envolve 1 佖ica esquiva para v疵ios ataques. Se você está lutando contra v疵ios alvos eles podem atacar na mesma hora. Nessa situa鈬o o mestre pode permitir o uso de apenas uma esquiva para todos os golpes. O mestre deve decidir quando você pode fazer isso ou n縊. Você cercado n縊 poderá se esquivar desse modo, por exemplo, al駑 de que usando esse tipo de esquiva seus PF ser縊 mais consumidos mais rapidamente, geralmente 1 PF pelo turno de esquiva todo. O mesmo vale para um só oponente usando v疵ios ataques. O mestre tamb駑 pode exigir um teste de percep鈬o em alguns casos. Essa manobra n縊 pode ser realizada caso o defensor tente aparar os golpes, exceto situa鋏es muitos raras, como por exemplo o defensor usando 2 espadas para defender 2 golpes separados ao mesmo tempo.

Vale lembrar que mesmo sendo atacando por v疵ios golpes num 佖ico instante, você deve jogar uma defesa separada para cada golpe. Se o mestre assim permitir, você pode jogar apenas uma vez a defesa, e usar o mesmo valor para absorver de todos os golpes recebidos em um intervalo curt﨎simo de tempo (ataques quase simult穗eos). Se o defensor utilizar a manobra absor鈬o m痊ima, n縊 poderá se desviar de nenhum golpe, por駑 todos ser縊 minimizados com toda sua defesa.
 

\subsection{Ataque coordenado}

Em certas situa鋏es algu駑 do grupo pode usar seu turno de ataque para ditar a鋏es fazendo com que outros personagens  ataquem coordenadamente, deste modo aumentando a efici麩cia do grupo. Essa manobra pode acarretar um dos dois efeitos descritos a seguir. O personagem alvo, ou seja, aquele que vai receber o ataque coordenado, tem uma penalidade maior para cada alvo que o esta atacando. Normalmente a penalidade é de -1 cumulativo para cada inimigo extra, por駑 essa penalidade é aumentada para -2 nessa situa鈬o. O outro efeito é de que o nä¼¹ero m痊imo de inimigos atacando o alvo pode aumentar em até 5 inimigos extras, de acordo com a intelig麩cia do alvo e a habilidade geral guerra. O membro que cordena o ataque deve escolher o efeito antes da manobra ser realizada. As 佖icas restri鋏es da manobra é de que os membros do grupo (personagens que est縊 sendo coordenado) devem poder se comunicar com o lider e atacar ap o turno de ataque do mesmo (membro que cordena o ataque).

O alvo do ataque coordenado pode tentar perceber a estrat馮ia usada fazendo um teste de percep鈬o contra a acuidade do l冝er. Alguns bus devem ser atributos, tanto ao alvo quanto ao l冝er de acordo com certas situa鋏es. Por exemplo, se o l冝er tiver a habilidade geral guerra, adicione o bus da habilidade nesse teste ou se o alvo n縊 compreender o idioma do l冝er, ele terá dificuldade para compreender a estrat馮ia usada (nesse caso a penalidade atribuida é determinada pelo mestre).

%\subsection{Detalhes sobre Aparar}

%A regra a seguir pode ser usada por mestres mais exigentes em rela鈬o a detalhes da batalha.

%Caso o defensor passe no teste (consiga aparar o golpe) ele deve realizar um teste de for軋 ou destreza (ele pode escolher) contra a for軋 do atacante em duas situa鋏es distintas. A primeira é quando a arma do atacante é de grande porte, o defensor está usando uma arma m馘ia para aparar e sua for軋 é menor do que a for軋 do atacante. A segunda é quando as duas armas s縊 de mesmo porte, por駑 a for軋 do atacante é maior que a for軋 do defensor em 4 pontos ou mais. No caso do defensor escolher destreza para jogar contra a for軋 do alvo, ele usa o mesmo resultado do teste inicial de aparar. 

%Esses dois testes s縊 interpretados da seguinte forma. O primeiro serve para o posicionar o objeto que vai ser usado para aparar o golpe. Esse teste é o teste de aparar em si, aonde bus de habilidades e equipamentos que concedam bus em aparar podem ser usados. O segundo teste serve para anular ou direcionar a for軋 proveniente do atacante. Para realizar tal feito, o defensor pode fazê-lo usando sua for軋 diretamente contra a do oponente ou utilizando sua destreza para desviar essa for軋 em outra dire鈬o. Ou seja, o atributo aparar do equipamento deve ser usado como um bus em qualquer teste de destreza envolvendo essa manobra. De forma semelhante, o atributo resit麩cia do equipamento deve ser usado como um bus em qualquer teste de for軋.


%Se o defensor n縊 conseguir passar nesse teste de for軋 combatido, isso pode ser interpretado que o defensor conseguiu posicionar algum objeto na trajetia do ataque, mas a for軋 do ataque deslocou o objeto. Ent縊 o defensor recebe o golpe, por駑 reduzido. O valor da redu鈬o do golpe deve ser definida pelo mestre. Por exemplo, o mestre pode reduzir do dano recebido pelo defensor um valor igual a for軋 do defensor. O defensor tamb駑 pode ter feito um teste de for軋 cujo resultado foi t縊 baixo, que o golpe n縊 perdeu nada de sua for軋 original. Nesse contexto, no lugar do defensor levar o golpe reduzido, o mestre tamb駑 pode retirar uma quantidade maior de PR da arma ou até mesmo fazer com que a arma caia no ch縊, primo ao defensor. O mestre pode interpretar a manobra da forma como desejar.

%Em outras palavras, para realizar a manobra aparar o defensor deve realizar no m痊imo dois testes. O primeiro é de destreza contra a destreza do oponente. O segundo deve ser realizado caso se encaixe na situa鈬o descrita acima. Esse segundo teste é de for軋 ou destreza (o defensor escolhe qual atributo usar) contra a for軋 do oponente. Caso escolha a destreza, ele usa o mesmo valor obtido no primeiro teste. 

%Em algumas situa鋏es o mestre pode (e deve) proibir o uso dessa manobra. Por exemplo, algu駑 tentando aparar um martelo grande com uma espada curta. Apesar da per兤ia b駘ica "lutar desarmado" poder ser usada para aparar ataques, o mestre deve considera-lá como arma curta ou m馘ia de acordo com a situa鈬o. Por exemplo, ela pode ser considerada arma m馘ia caso o defensor esteja primo do oponente, tornando assim o uso da p駻icia mais f當il. No caso inverso, ou seja, o defensor usa alguma arma para se defender de um atacante que usa lutar desarmado, o defensor n縊 pode usar a manobra aparar (apesar com alguma p駻icia defensiva como escudo ou manopla) caso o atacante esteja muito primo. Isso é determinado pelo mestre de acordo com a batalha, onde o mesmo pode exigir teste por parte do atacante para se aproximar a tal ponto do defensor (um teste de saltar por exemplo).	


\subsection{Atacando Alvos Espec凬icos}

Para tornar mais 疊il e simples o combate, em dragon adotamos a seguinte intepreta鈬o para ataques em alvos espec凬icos. Ao inv駸 de acertar um local especifico de um alvo para retirar uma quantidade maior de PV, sempre que um atacante retirar um sucesso muito grande (dano e acerto) esse ataque bem sucedido é interpretado como sendo nesse local. Por exemplo, uma grande quantidade de dano retirada subtamente pode ser interpretada como um membro decepado por exemplo.

Caso o personagem deseje atacar algum alvo espec凬ico do oponente (furar uma bolsa com itens, por exemplo), a jogada de ataque é feita normalmente, por駑, o defensor tem um bus em sua jogada de esquiva (esquivando ou aparando) de acordo com o tamanho do objeto alvo. Quanto menor o alvo, maior o bus, variando de 2 até 8.

O mestre tamb駑 pode usar a seguinte regra opcionais para redu鈬o de atributo de acordo com dano causando em certas partes do corpo. Caso o ataque acerte com 黜ito membros do corpo (usando os redutores citados acima), o mestre pode impor redutores em atributos (for軋, destreza ou esquiva) de acordo com o dano sofrido. A quantidade de redu鈬o de atributo será avaliada perante a quantidade de dano sofrido. O redutor é igual ao bus do dano (um dano de 4 irá causar um redutor de -1 e assim por diante).
  

%\begin{itemize}
%	\item Alvos m馘ios (bolsas, bra輟s, etc) = bus de destreza ou esquiva +2.
%	\item Alvos pequenos(pesco輟, cabe軋, m縊s, etc) = bus de agilidade  + 4.
%	\item Alvos muito pequenos(olhos, brincos, etc) = bus de agilidade  + 6.
%\end{itemize}

O personagem tamb駑 pode desejar atacar um alvo espec凬ico do oponente com o objetivo de reduzir a absor鈬o da armadura do alvo. Para fazer isso é simples. Para cada ponto reduzido no teste de atacar, o dano (caso o ataque acerte) ignora 1 ponto da armadura do alvo (come軋ndo pela defesa normal). Outro detalhe é que para cada ponto de cobertura da armadura, a mesma consegue ignorar 1 ponto essa redu鈬o. Por exemplo, caso o atacante deseje retirar 2 pontos de absor鈬o de uma armadura com cobertura zero ele deve jogar o teste de atacar com um redutor igual a 2. Se ele deseja reduzir a mesma quantidade de uma armadura com cobertura 3, ent縊 ele deve ter um redutor de 5 em seu ataque. A penalidade recebida n縊 pode ser maior que o atributo destreza - 1. Vale observar que armaduras pesadas podem ter sua absor鈬o reduzida até a metade com o uso dessa manobra.


%Caso o ataque atinja partes vitais e cause dano, o alvo sofre um dano extra igual a uma percentagem do PV do alvo (o mestre decide se 25\% ou 50\% do PV total).

Vale destacar que determinados inimigos n縊 tem penalidades de receber danos em alvos espec凬icos, seja devido a sua natureza ou cobertura excepcional de alguma armadura. Al駑 disso, como dito anteriormente, o mestre pode intepretar os efeitos da perda de PV, ou seja, n縊 será somente com o uso dessa manobra que um alvo pode ter seu bra輟 ferido ou decepado. Cabe ao mestre escolher de acordo com a situa鈬o de combate quando essa manobra pode ser usada e como.

\subsection{Ataque em チrea}

Determinadas ocasis permitem você acertar v疵ios oponentes com apenas um golpe. Seja devido a arma ser grande (um machado ou espada montante), ou por que v疵ios alvos est縊 muito primos a você. No caso da arma poder fazer isso, vem especificado nela quantos metros ela pode alcan軋r com um golpe. No caso de personagens primo demais, quem decide é o mestre. Nesse caso se faz um teste de destreza(no caso quem ataca). O alvo mais primo recebe o golpe com destreza normal e os alvos consecutivos v縊 esquivar com um bonus cumulativo de +2 e uma penalidade cumulativa no dano para o atacante, em outras palavras, o ataque vai perdendo a for軋 ao longo do trajeto. De acordo com a descri鈬o da cena, o dano pode diminuir em 3 ate 10 para cada alvo atingido. Por exemplo, se algu駑 com uma espada longa ataca v疵ios alvos e o segundo absorve tudo, o mestre pode decidir por reduzir o dano em 13 para o terceiro alvo, sendo 3 do primeiro e 10 do segundo. O mestre pode tamb駑 interromper o ataque, dizendo que o ataque em 疵ea n縊 efetuou sua trajetia completa. 

\subsection{Contra Ataque - Atacando no Turno de Esquiva}

Você pode no seu turno de esquiva optar por tentar atacar o oponente no lugar de esquivar-se de seu golpe (algo como um contra ataque). Primeiramente você tem que fazer um teste para ver se você é r疳ido o suficiente para poder atacar no turno de esquiva. Você vai fazer um teste de agilidade contra o oponente que está te atacando nesse turno de ataque. Por駑, como você está atacando no turno de esquiva e o oponente n縊 (pois o intervalo de tempo que ele disp para realizar uma a鈬o e mais bem ajustado) você vai ter penalidades nesse teste de agilidade. A penalidade será igual ao bus de agilidade do oponente (se for um oponente lento o turno de ataque dele poderá ser facilmente mais lento proporcionando oportunidade do seu ataque). Se você passar no teste poderá atacar normalmente o oponente. Algumas observa鋏es devem ser levantadas a respeito de tal manobra.
\begin{itemize}
	\item Se v疵ios alvos estiverem te atacando no turno de esquiva, você poderá atacar eles. Fa軋 um teste agilidade contra cada um (com redutores baseados nas agilidades de cada um separadamente). Os que você passar poderá atacar normalmente (dividindo a destreza com regras de m伃tiplos ataques), por駑 se algu駑 for mais r疳ido (ganhar no teste de agilidade), você n縊 poderá mais contra-atacar nesse turno.
	\item O alvo que recebe essa manobra n縊 pode tentar se mais r疳ido que você(um contra-contra-ataque).
	\item Nem você nem o alvo ter縊 direito de esquiva. O maximo que pode se feito é o que está no turno de ataque fazer um teste de percep鈬o para poder usar a manobra defesa m痊ima. Esse teste é um teste contra a agilidade de quem tenta dar o contra golpe.
	\item Essa manobra n縊 pode ser feita no primeiro momento de a鈬o da batalha.
	\item O mestre tem total direito para alterar tais regras segundo sua interpreta鈬o. Em casos como já foi citado, se os nä¼¹eros do teste de agilidade ou de destreza forem primos ou iguais, o mestre pode interpretar de modo prrio. Os golpes podem se chocar, os dois atacam ao mesmo tempo e assim por diante.
	\item Como outras manobras de combate n縊 usuais, ela consome mais PF do que o normal.
\end{itemize}


\subsection{Esquivando no Turno de Ataque }

Ao inv駸 de atacar o oponente, você pode desviar de um prov疱el golpe que pode vir a receber no primo turno. Você ganha um bus de agilidade em destreza ou esquiva para qualquer a鈬o que você fizer no turno seguinte a este em que você ficou parado (no caso seu turno de esquiva). Você amplia seu turno de ataque para se fundir com o turno de esquiva de modo a ganhar bus para se esquivar ou defender usando a destreza (esse bus e determinado pela agilidade). O bus n縊 pode ser maior que o atributo usado. Por exemplo, certo personagem tem destreza 8 e esquiva 2, totalizando agilidade 10. O bus da sua agilidade é 4, por駑 ele so pode usar 2 pontos desse bus caso use a esquiva para se esquivar do ataque.


\subsection{Movimento Durante a Batalha}

ノ normal durante uma batalha algu駑 se mover, seja para fugir, seja para ativar algum mecanismo ou apenas se distanciar de um alvo para poder atacar de longe. O personagem que deseja se mover durante a luta, pode fazê-lo durante o turno de esquiva ou de ataque, por駑, quando o fizer, n縊 poderá mais usar esse turno para qualquer outra a鈬o. Ou seja, se você gastar o turno de ataque para mover-se uma dist穗cia grande só poderá atacar usando a manobra contra ataque. Existem habilidades onde você pode se mover muito e continuar atacando, mas a priori qualquer grande deslocamento que você realizar em batalha você deve gastar turnos. Obviamente durante uma luta os alvos est縊 se locomovendo pelo cen疵io. Essa manobra de locomo鈬o somente é usada quando uma distancia consider疱el vai ser percorrida por algum personagem.

	Um alvo pode mover-se o seu valor de esquiva em metros durante um turno completo (incluindo esquiva e ataque), podendo atacar e se esquivar normalmente enquanto se locomove sem penalidade. Some o bonus de agilidade no caso de tirar todo o turno para locomover-se (situa鈬o de fuga). Em determinadas situa鋏es o mestre pode atribuir a habilidade Corrida para aumentar a distancia na situa鈬o de fuga. Da mesma forma, cargas levadas pelo personagem pode reduzir esse valor de acordo com o mestre
	
	
\subsection{Desarmar Oponente}

A manobra desarmar é usada para tirar a arma de um oponente de suas m縊s. O uso dessa manobra é bastante simples, por parte do atacante um teste de destreza mais um bus, definido de acordo com a situa鈬o do desarme. Esse bus é igual ao bus do atributo for軋 caso o atacante use esse atributo para desarmar o oponente ou de acordo com a arma utilizada. De forma semelhante, o bus é de destreza para desarmes usando armas que possibilitem essa op鈬o (armas leves ou pequenas geralmente se encaixam nessa situa鈬o).

Por parte do defensor, um teste de destreza ou esquiva mais um bus. Da mesma forma que o atacante escolhe atribuir o bus de for軋 ou destreza, o defensor tamb駑 pode faze-lo. O mestre pode adicionar outros bus extras de acordo com a situa鈬o, por exemplo, concender um bus maior de for軋 se a arma utilizada for de duas m縊s, se ela tiver algum acessio que ligue a arma ao defensor ou até mesmo de acordo com a descri鈬o da manobra (dist穗cia dos alvos, usar uma arma m馘ia atr疽 de um escudo, etc). ノ muito importante que o mestre preste aten鈬o nesses detalhes, por exemplo, na maioria dos casos é muito dificil desarmar um oponente usando determinadas armas (espadas, machados, etc). Boa parte das armas s縊 feitas para causar dano, e seu formato n縊 auxilia na manobra desarmar. Para facilitar, o mestre pode usar o seguinte referencial. Se uma arma n縊 conceder bus em aparar, o defesor recebe seu bus (de destreza ou esquiva) duas vezes para se defender. Se a descri鈬o da manobra por parte do atacante for coerente, o defensor joga o atributo mais bus de atributo mais bus de dist穗cia (que pode ser ignorado de acordo com a situa鈬o).
 
Essa manobra n縊 retira dano do defensor caso seja bem sucedida. O n伹ero de sucessos determina qual longe a arma foi arremessada. Em alguns de sucesso muito baixo favorecendo o atacante, o defensor pode realizar um teste de concentra鈬o para tentar pegar a arma "no ar", ou atacar usando a arma no primo turno com um redutor (como se uma parte do turno de ataque dele tivesse sido usada para recuperar a arma caida).

O mestre deve decidir quais atributos ser縊 usados e os bus que ser縊 estabelecidos de acordo com a situa鈬o. Por exemplo, um samurai com uma espada m馘ia tentando desarmar um ranger com uma lan軋. O samurai pode escolher destreza (recebendo assim tamb駑 o bus de destreza) para tentar desarmar o ranger. O ranger escolhe pular para o lado enquanto puxa sua lan軋. Nesse caso ele joga esquiva mais bus de destreza mais um bus definido pelo mestre de acordo com a dist穗cia que o samurai encontra-se do ranger.

Em termos de sistema n縊 é uma manobra complexa, por駑 seu uso da brecha para muitas possibilidades diferentes, possibilidades essas que podem conceder bus e us tantos para o defensor quanto para o atacante. O mestre deve ficar atendo para a descri鈬o da manobra. Na dä¿¿ida, favorecer a defesa.

\subsection{Agarrar Oponente}

Para realizar a manobra agarrar é simples. Basta realizar sua jogada de ataque normal, ou seja, destreza contra esquiva ou destreza do oponente. Caso acerte o alvo, o atacante deve realizar outro teste, por駑 dessa vez usando for軋 contra a for軋 ou contra a destreza do oponente. Se obtiver sucesso mais uma vez, o defensor é considerado agarrado pelo atacante. O mestre deve incluir bus ou us de acordo com a situa鈬o, por exemplo, um atacante pequeno tentando prender um alvo grande apenas com as m縊s, ou o atacante usando cordas para amarrar o alvo. Quando agarrado, dependendo da situa鈬o do agarr縊, o mestre deve definir as limita鋏es de a鋏es durante o agarr縊. Por exemplo, uma pessoa com os bra輟s imobilizados, n縊 pode atacar usando os mesmos. O mestre tamb駑 deve definir o dano que o alvo fica recebendo enquanto fica agarrando. Esse dano tamb駑 pode ser representado por perda de PF (em situa鈬o de estrangulamento), ou perda de atributo (no caso de um membro quebrado).

Para se soltar do agarr縊, o defensor deve realizar um teste novamente de destreza ou for軋 com a for軋 do oponente.

\subsection{Regra Opcional sobre PF}

Para mestres mais exigentes exige um m騁odo mais detalhista para monitoramento de perda de PF para cada personagem. Para cada personagem presente na batalha o mestre deve criar uma barra, com 10 campos, variando de 10\% até 100\%, como mostrado a seguir. Vamos chamar essa barra de "fego".

\begin{table}[htbp]
\begin{center}
\begin{tabular}{|c|c|c|c|c|c|c|c|c|c|} \hline 
10\%&	 20\%&	30\%&	 40\%&	 50\%&	 60\%&	 70\%&	 80\%&	 90\%&	 100\%\\\cline{1-10} 
 &  &   & &  &  &  &  &   &  
\\ \hline
\end{tabular}
\end{center}
\caption{Barra de acompanhamento de perda de PF}
\label{}
\end{table}

Sempre que o personagem realizar uma a鈬o, o mestre deve marcar com um "X" os campos equivalentes a percentagem que aquela a鈬o consome. Por exemplo, se um personagem realizar um ataque que consuma 30\% dessa barra, a tabela deve ficar como a mostrada a seguir (partindo do princ厓io que o fego inicialmente estava vazio). 
  

\begin{table}[htbp]
\begin{center}
\begin{tabular}{|c|c|c|c|c|c|c|c|c|c|} \hline 
10\%&	 20\%&	30\%&	 40\%&	 50\%&	 60\%&	 70\%&	 80\%&	 90\%&	 100\%\\\cline{1-10} 
X &X  &X   & &  &  &  &  &   &  
\\ \hline
\end{tabular}
\end{center}
\caption{Exemplo usando ataque que consuma 30\% de seu fego}
\label{}
\end{table}

A percentagem de quanto cada a鈬o consome do fego é mostrada na tabela a seguir.

\begin{table}[htbp]
\begin{center}
\begin{tabular}{|c|c|c|c|} \hline 
Equip/A鈬o &	 Esquiva &	Ataque&	 Aparar\\\cline{1-4} 
Leve & 5\%  & 20\%   &  10\% \\\cline{1-4} 
M馘io & 10\%  & 30\%   &  20\% \\\cline{1-4} 
Pesado & 30\%  & 50\%   &  30\%  
\\ \hline
\end{tabular}
\end{center}
\caption{Gasto de percentagem de PF}
\label{}
\end{table}


Sempre que essa barra atingir 100\% o personagem perde 1 PF. Para cada a鈬o extra realizada no mesmo turno, o personagem perde metade do que normalmente perde quando realizado aquela a鈬o pela primeira vez. Em outras palavras, para cada 2 a鋏es extras o personagem perde a percentagem associada aquela a鈬o na tabela acima. Por exemplo, se um personagem realizar 3 ataques com uma arma m馘ia no mesmo turno ele perde 60\% de seu fego. O consumo do fego da esquiva está associado na tabela acima esta relacionado com o peso do equipamento usado. Em outras palavras, em um turno um personagem com uma arma grande e uma armadura m馘ia, perde 60\% de seu fego quando realizado um ataque e uma esquiva. 

Vale lembrar que em certas batalhas um personagem n縊 é atingido ou realiza nenhum ataque. Nessas situa鋏es um personagem n縊 perde PF na batalha. Isso tamb駑 cria situa鋏es de que um grupo salvar os PF de um personagem para serem gastos em situa鋏es mais urgentes. 

Por駑 vale observar que em algumas batalhas o consumo de PF por ser crucial para definir o seu resultado. Por exemplo, um personagem usando uma arma de porte grande (um machado por exemplo), pode usar a manobra aparar v疵ias vezes (uma vez que usando uma arma de grande porte, ele pode aparar armas de porte m馘io e pesado). Caso a destreza desse personagem seja muito grande, pode acontecer de que nenhum de seus oponentes consigam atingi-lo. Nesse caso, os oponentes podem bolar uma estrat馮ia de consumo de PF, fazendo com que o alvo use a arma de porte grande v疵ias vezes para aparar, consumindo seus PF lentamente com isso. Nesse caso o mestre deve retirar PF cautelosamente dos personagens e n縊 somente uma vez ao termino da luta, pois o consumo de PF durante ela é decisivo nesse caso.

Para agilizar todo o processo de perda de PF durante a luta, o mestre pode ir anotando as a鋏es realizadas por cada personagem (1 ataque, 3 aparar, 2 ataques, n縊 foi atacado, etc), e de 4 em 4 turnos ir retirando de cada personagem o PF perdido até aquele momento. Vale lembrar que o mestre tem total liberdade para interpretar a perda de PF e usa-lá dentro de jogo como bem desejar. Alguns mestres retiram PF apenas quando habilidades ou manobras s縊 usadas.


\subsection{Recuperando PF durante a Batalha}

Sempre que um personagem n縊 ataca, ele recupera seu fego. Al駑 disso, caso um personagem passe 1 turno de batalha sem realizar nenhuma a鈬o, seja de ataque ou de defesa, ele pode recuperar 1 PF. Essa recupera鈬o n縊 pode fazer com que o personagem tenha mais PF do que tinha no inicio da batalha nem que recupere uma quantidade maior que seu bus de defesa. Essa regra é opcional e so é valida para condi鋏es clim疸icas normais.

