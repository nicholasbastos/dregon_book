%%
%% Capítulo 4: Batalha
%%

\chapter{Batalha}
\label{Cap:batalha}

O objetivo principal do sistema de batalha de Dregon é fazer com que a luta se torne mais dinâmica possível. É muito importante em um sistema de RPG que o seu sistema de combate tenha três principais características: liberdade, velocidade e coerência. 
	 
Vários detalhes como habilidades, perícias, magias e etc, serão explicados em capítulos posteriores. Vale notar que o jogador não é obrigado a conhecer as regras de combate, quem deve conhecê-las é o mestre. O desconhecimento do sistema de batalha não impede que um bom jogador realize lutas ruins. O conhecimento do mesmo torna mais rápido o fluxo da batalha.

O sistema de batalha de Dregon será descrito através de diversas situações de combate. Cabe ao mestre encaixar as ações desejadas pelos jogadores, nessas ações descritas neste capítulo. Esse conjunto de situações básicas podem dar origem a outras situações mais específicas. Por exemplo, atacar é uma ação básica. Desarmar o oponente, é uma variação da ação básica atacar, porém com outro objetivo. Se um personagem desejar desarmar o oponente, ele tem duas opções. Seguir a orientação de regra dado nesse capítulo, ou alterar ao bel prazer a regra de ataque. Isso cabe ao mestre e jogador fazer durante o jogo, deixando o jogo mais fluido e divertido. Ou seja, nesse capítulo são dadas ações de combate básicas, e sugestões de como o mestre deve proceder caso o jogador queira outras ações mais complexas.

\section{Momento de Ação}

Um momento de ação corresponde a um intervalo de tempo que é igual à soma de dois turnos, um turno de ataque e um turno de esquiva. Durante o turno de ataque o personagem pode realizar ações mais complexas, como atacar, usar uma habilidade ou uma magia por exemplo. Durante o turno de esquiva, ações mais reflexivas devem ser feitas tais como pular rapidamente para poder se desviar de um ataque, dar um grito sinalizando uma ordem, etc. Portanto podemos dizer que durante o turno de ataque o tempo para se realizar ações é maior do que o tempo para se realizar ações no tempo de esquiva. Se alguém tenta realizar uma ação mais longa no turno de esquiva, terá penalidades para realizar tal ação, por outro lado quem desejar realizar ações simples no turno de ataque terá bônus.

	No momento que acontece o seu turno de ataque, acontece o turno de esquiva do inimigo, e assim por diante. Mas isso não quer dizer que você não possa tentar realizar um ataque em seu turno de esquiva (manobra contra-ataque). Mais detalhes explicados posteriormente.
	
	Em suma, para termos de sistema um momento de ação é dividido em dois turnos. Um turno de ataque e um turno de esquiva. Um momento de ação tem mais ou menos duração de 3 a 4 segundos.  O mestre deve usar esse valor apenas como referência para basear-se na duração de determinadas batalhas. Mas o contexto da situação de combate descreve melhor a duração de todo o combate. 

\section{Iniciativa}

Para calcular a ordem das ações, todos jogam ágilidades para ações físicas e percepção para ações mentais ou ativação de habilidades. A ordem dos sucessos determina a ordem das ações. 
O personagem pode também, caso ganhe a iniciativa, optar por não atacar. Nesse caso ele terá um bônus em qualquer ação que ele quiser realizar no turno de esquiva (apenas em ações que não sejam de ataque, ou seja, ações de esquiva) e esse bônus depende de sua ágilidade/percepção. 

\section{Ataque}

\subsection{Um ataque normal}

O atacante realiza um teste de destreza para atacar. Guarde esse valor para usar posteriormente se o defensor for esquivar. Em outras palavras, ele faz um teste combatido contra o defensor. Se o atacante conseguir ganhar no teste ele consegue acertar o golpe. O número de sucessos extras no ataque concede um aumento no dano final (detalhes explicados posteriormente).
Ataques usando magias não obedecem essa regra, tendo regra própria explicando em capítulo posterior.

\subsection{Ataque Inconsequente}

Nesta manobra, o personagem ataca de forma louca e sem precaução, visando apenas a destruição do oponente. Ele vai perder o turno de esquiva e não pode realizar as manobras de combate ataque em alvo específico ou múltiplos ataques. O mestre deve considerar também o bônus de força para determinar a diferença da força atacante-defensor para fins da manobra aparar.

O personagem que realizar a manobra ataque inconsequente ganha bônus de atributo em destreza e em força para 1 ataque nesse turno. Essa manobra só é válida para ataques físicos feitos corpo a corpo, ou seja, não é usada para armas projetáveis, como arcos, armas de fogo, etc. \textbf{Um personagem só pode realizar um ataque inconsequente por batalha.} 

No seu próximo turno de esquiva, como dito anteriormente, o personagem não pode esquivar. Ou seja, o valor do teste de esquiva do usuário é igual a sua esquiva, sem a jogada de dados (ele não pode substituir esse valor pelo usado na manobra aparar). Use esse valor como base para calcular bônus no dano proveniente dos sucessos obtidos no ataque.

\section{Esquiva}

O personagem pode usar 3 manobras de esquiva.

\subsection{Esquiva Normal}

Ele pode esquivar-se normalmente jogando sua esquiva e testando contra a destreza de quem o atacou. Em outras palavras, o que define se o defensor se esquivou ou não é o resultado do teste combatido da destreza (do atacante) contra a esquiva (do defensor). O número de sucessos extras na esquiva serve apenas para descrever o grau de êxito da mesma.

\subsection{Aparar}

Ele pode aparar o golpe usando um objeto. Uma espada, lança, escudo, ou algum equipamento que ele saiba utilizar pode ser usado para desviar o golpe do atacante. Nesse caso ele joga destreza no lugar de esquiva. É um movimento bastante útil para quem tem baixa esquiva. %porém esse movimento cansa mais do que uma simples esquiva (consome mais pf no momento da batalha). Cada 2 ou 3 usos da manobra aparar (de acordo com a tamanho da arma, diferença entre forças defensor/atacante, etc), o usuário perde 1 PF. Além disso o mestre pode retirar uma quantidade extra de PR da arma por ser usada como objeto parar aparar o golpe do adversário.

No caso do defensor ter uma força menor em relação ao atacante em 4 ou mais pontos, ou da arma do atacante ser pelo menos 1 porte maior que a arma do defensor, o atacante recebe um bônus em sua jogada de acerto igual ao seu bônus de força. Além disso, o mestre tem total liberdade para alterar os valores nesse teste, por exemplo, no caso do atacante usar armas leves a uma curta distância de um defensor que tente aparar com uma arma grande, o defensor recebe uma penalidade. O mestre pode usar qualquer interpretação da situação de combate para torna-lo mais dinâmico.
	
%Por exemplo, quando a diferença entre forças é considerável, o atacante pode jogar destreza com um bônus de força contra a destreza do defensor. O mestre também pode dar bônus ao atacante de acordo com a distância de ataque, natureza favorável contra a arma do atacante/defensor, etc. Além disso o mestre pode reduzir o dano de acordo com a situação, por exemplo, em caso de empate no teste de aparar. 
	
Se o defensor pretende usar essa manobra para se defender de um projétil (um ataque a distância), dobre o valor do teste final da destreza do atacante, já que a área de um projétil é bem menor que de uma espada, por exemplo. Além disso, o mestre pode decidir aumentar ainda mais a penalidade para se defender projéteis (alguns não permitem a defesa de projéteis usando armas). A maioria dos projéteis pode ser defendida independente da força do defensor. Alguns projéteis não podem ser aparados usando essa manobra, o caso de esferas de energia ou projéteis pequenos.

 
\section{Tipos de Dano}

\subsection{Dano Normal}

Representa o dano que pode ser reduzido sem auxilio de armadura especial. Dano por contusão (socos, chutes) e afins são exemplos de danos normais. O dano normal é representado por uma " / " 
	Todo dano usando apenas o atributo força (sem auxilio de habilidade especial) é considerado como dano normal. Você pode somar todos os danos normais para fins de cálculo de dano normal total. Por exemplo, se personagem com força 6 usa um porrete /3, pode-se considerar a como dano normal total 9 (/6 + /3 = /9).


\subsection{Dano Automático}

O dano letal é um tipo de dano mais difícil de absorver sem armadura apropriada. Normalmente é causado por ataques cortantes, quentes, ácidos e afins. é representado por um \" + \". Ou seja, quando a força de um personagem mostrar 5 + 1 não quer dizer que a força dele é 6, e sim que a força dele é 5 porém está associada com +1 de dano automático. Para minimizar-se +1 de dano é necessário +1 de defesa automática. Apesar de ser mais difícil de defender, 1 de dano automático retira a mesma quantidade de dano que 1 de dano normal. 
Vamos supor que uma pessoa seja atacada por uma espada +2 e tenha conseguido no seu resultado de defesa um valor igual a /5. Ela conseguirá reduzir +1 de dano, reduzindo o dano total para +1. Então ela vai perder apenas 1 de dano e não 3. Se por exemplo a espada fosse um porrete, seria /2 (dano normal), e a pessoa com /5 de defesa poderia reduzir o dano total a 0, não perdendo nenhum PV com o ataque.

\subsection{Dano Crítico}

Quando um ataque resulta em um acerto crítico (explicado no capítulo de jogadas), o dano causado por esse ataque é considerado dano crítico. O dano crítico possui duas características importantes:

\begin{itemize}
	\item \textbf{Não pode ser curado durante o combate}: Magias de cura, poções e outras formas de recuperação de PV não afetam o dano crítico.
	\item \textbf{Reduz temporariamente o PV máximo}: O dano crítico reduz o PV máximo do personagem temporariamente até que ele realize um descanso adequado (geralmente uma noite de sono).
\end{itemize}

Por exemplo, se um personagem com 20 PV máximo recebe 5 de dano crítico, ele fica com 15 PV máximo até descansar. Durante o combate, ele não pode recuperar esses 5 PV perdidos por dano crítico.

\subsection{Dano de Toque}

O dano de toque é um tipo especial de dano que representa ataques que ignoram parcialmente a proteção física. Este tipo de dano possui duas características importantes:

\begin{itemize}
	\item \textbf{Reduz a defesa pela metade}: A defesa do alvo é reduzida pela metade (arredondada para baixo) para fins de redução de dano.
	\item \textbf{Ignora armadura}: Equipamentos de proteção (armaduras, coletes, etc.) não fornecem proteção contra dano de toque.
\end{itemize}

Por exemplo, se um personagem com defesa natural /6 e armadura /2+1 recebe um ataque de dano de toque /4, apenas sua defesa natural (/6) é considerada, mas reduzida pela metade (/3), resultando em /1 de dano efetivo.

\subsection{Lembrete Sobre Dano}

Tanto o dano normal quanto o dano automático são semelhantes em termos de dano (quantidade de pv tirada), o que os diferencia é a capacidade de um ser absorvido facilmente enquanto o outro não . +1 de dano automático pode ser absorvido com +1 de defesa automática, porém /1 de dano normal pode ser absorvido com /1 de defesa normal.

	Da mesma forma existe a defesa automática e normal. A defesa automática serve para defender igualmente +1 dano, porém na mesma intensidade de defender /1 de dano. Em outras palavras, uma armadura +4 defende da mesma forma +4 ou /4.
	
\section{Dano}

Para calcular-se o dano de um ataque (dano final), deve-se calcular o dano inicial e a redução de dano. O dano que o personagem vai receber é o dano inicial menos a redução de dano. 

Para calcular-se o dano inicial, execute o seguinte procedimento. Primeiramente some no dano do atacante o bônus do sucesso no teste de acerto como sendo dano normal. Esse bônus não pode exceder o valor do bônus de força permanente do mesmo para ataques corpo a corpo, e no caso de ataques a distância o bônus proveniente do sucessos extras não pode ultrapassar o dano base da arma (não do projétil). Se por exemplo o atacante consegue 17 no teste de acerto e o defensor consegue 10 no teste de esquiva, a diferença é de 7. Então o atacante terá um bônus de /3 no dano (3 é o valor do bônus de 7). Se o atacante tiver força 5, esse bônus será /2, pois o bônus de sua força é 2. Isso pode ser explicado analisando que quanto maior o sucesso do atacante em cima do defensor, mais eficiente será o golpe. Existem algumas armas e habilidades que aumentam esse limite. Após calcular o dano proveniente de sucessos extras, o atacante soma separadamente os tipos de dano (automático e normal). Esse valor será o dano inicial.

Para calcular-se a redução de dano, compare diretamente os valores de dano com os valores de defesa do defensor. A defesa é calculada diretamente sem jogadas de dados, usando apenas os valores base do personagem e equipamentos. Subtraia dano automático com defesa automática e dano normal com defesa normal. O excedente da defesa automática pode reduzir o excedente de dano normal, mas o excedente da defesa normal não pode reduzir dano automático. O dano após os cálculos é chamado de dano final.

A regra explicada acima vale para danos físicos, detalhes sobre o cálculo de dano mágico explicado no capítulo sobre magias.
	O mestre pode interpretar como o alvo sofreu o dano de acordo com a quantidade máxima de pv. O mestre pode interpretar um golpe que tire 15 de PV de um personagem que tem 20 de PV total como quase um braço decepado ou um olho perfurado por exemplo. A mesma quantidade de dano em um personagem com 100 de PV pode ser vista como um arranhão.
	
\textbf{Exemplo de cálculo de dano.}

O atacante tem força 5 e uma faca +4. Ele teve 16 no teste de destreza e o defensor 10 no de esquiva. Portanto, o atacante teve 6 sucessos extras, dando um bônus em seu dano de /2. O seu dano inicial vai ser /7+4. O defensor tem defesa /4+2 (valores base + equipamento). Primeiro diminuímos os tipos de danos semelhantes: /7 de ataque é reduzido pelos /4 de defesa, sobrando /3 de dano normal. Os +4 de ataque são reduzidos em +2 pela defesa +2 do defensor, sobrando +2 de dano automático. O excedente da defesa automática (+2) pode reduzir o excedente de dano normal (/3), reduzindo-o para /1. No final o defensor perde 3 PV (/1 + +2).


\section{Gastando PF}

Durante uma batalha é normal o desgaste físico dos combatentes. Isso é refletido na perda de PF. Normalmente em uma batalha um lutador, atacando e se esquivando 1 vez a cada turno usando armas leves, perde 1 PF a cada 4 turnos aproximadamente, 1 PF a cada 3 turnos quando usando equipamento de porte médio e 1 PF a cada 2 turnos quando usando equipamento de porte pesado. Mas existem alguns fatores que aceleram a perda de PF:

\begin{itemize}

	\item Uso de abusivo de equipamento pesado. Quanto maior for o número de equipamentos pesados usados durante o combate, mais PF é perdido. O uso de 3 equipamentos pesados ao  mesmo tempo acarreta a perda de 1 PF por turno. 
	\item Uso de habilidades específicas. A maioria das habilidades consomem PF.

	\item Locomoção. Seja para fugir ou para alcançar alvos, o mestre deve exigir o gasto de PF quanto maior for a locomoção do alvo na luta.

	\item Condições climáticas desfavoráveis também podem ser usados como fatores agravantes na perda de PF em relação ao desgaste físico natural sofrido nas batalhas.
\end{itemize}

No lugar de monitorar o que cada personagem faz em certos intervalos e ir tirando os PF de acordo com as regras acima, no final da batalha o mestre retira de 1 a 4 PF dos personagems por critério pessoal, levando em consideração condições que retirem mais PF que o normal (usar muitas manobras de combate, peso do equipamento, duração do combate etc).


\section{Usando Itens}

é comum durante as lutas os personagems usares determinados medicamentos de ação instantânea para recuperarem seus PV,PM ou PF, ou drogas para aumento temporário de atributos. Um item pode ser usado em você mesmo ou em um aliado dentro do seu campo de ação durante o seu turno de ataque. Para realizar tal ação durante a batalha o mestre deve ficar atento as situações de luta para atribuir penalidades ou até mesmo privar o jogador de realizar ataques nesse turno. 

Sempre que um personagem for usar um item em um aliado, ele perde todo o turno de ataque realizando essa ação. O mestre pode privar o uso para os aliados ele próximo, ou dentro do seu campo de ação. Caso um personagem deseje usar um item nele mesmo, e esse item encontra-se de fácil acesso, o personagem pode usa-lo sem penalidade nenhuma em seu ataque. Porém se o personagem usa alguma armadura ou encontra-se com as mãos ocupadas (segurando um arco e flecha por exemplo) o mestre pode fazer com que o personagem perca seu turno de ataque localizando o item.

 

\section{Regras Opcionais para Combate}

Abaixo segue a explicação de várias manobras de combate utilizadas no sistema Dregon. Essas manobras servem para tornar o combate mais dinâmico possível. Sempre que o mestre tiver dúvida como proceder perante uma manobra, ele pode usar as informações a seguir para ajudar em sua decisão final de como prosseguir. Todas as manobras de combate consumem mais PF que as ações padrões normais. 

As regras abaixo tem o único objetivo de mostrar como as regras básicas de combate são alteradas para que diferentes situações mais complexas de combate possam ser usados. Em outras palavras, o mestre pode alterá-las ou até mesmo não utilizar as regras abaixo que nada vai atrapalhar o funcionamento básico do sistema. Elas foram criadas para dar um maior grau de detalhes aos combates.
	
\subsection{Ataques Múltiplos}

Um personagem pode optar por no lugar de dar apenas um golpe bem visado, dar vários golpes porém com sua destreza menor. Isso pode ser feito se o personagem dividir a destreza para realizar um golpe, ou seja, um personagem que tem destreza 4 dar dois golpes com destreza 2, realizando 1 ataque extra. Cada mestre pode estabelecer um número limite de golpes extras que o personagem pode dar. Aqui iremos dar uma média que pode ser usada. De cada 4 em 4 pontos em destreza o personagem pode dar um golpe extra, ou seja, um personagem com destreza 7 poderia dar 2 golpes (1 normal e um extra) com destreza dividida (4 para um e 3 para outro ou 5 para um e 2 para outro) mas não poderia dar sete golpes com destreza 1. Vale observar que que qualquer bônus recebido em destreza entra no cálculo antes da divisão de ataques. Caso alguma habilidade conceda ao personagem bônus em um ataque, o personagem deve escolher em qual dos ataques (caso executando a manobra ataques múltiplos) esse bônus deve contar.

A distribuição de destreza entre os golpes deve ser escolhida pelo jogador, obedecendo a seguinte regra: O valor da destreza dividida para cada ataque não deve ser maior que o bônus de destreza do usuário. Por exemplo, um personagem com destreza 8 (bônus igual a 3), pode dividir 2 ataques em um ataque com a destreza 5 e o outro ataque com a destreza 3, mas não pode alocar 2 de destreza para 1 ataque e 6 para outro ataque, pois a diferença entre o primeiro ataque que tem destreza 2 e o segundo ataque que tem destreza 6 é 4, valor esse maior que o bônus de destreza que é 3.

Uma regra opcional para a forma como a destreza é dividida é que caso o personagem aloque 2 vezes consecutivas 1 ponto quando dividindo a destreza, o defensor não tem penalidade na terceira esquiva (penalidade essa descrita mais a frente). O uso de duas armas pode aumentar o número de golpes extras e reduzir um pouco o consumo de PF ao usar essa manobra. 

Quando um personagem utiliza a manobra ataques múltiplos no próximo turno de esquiva qualquer ação envolvendo destreza ou esquiva tem um redutor igual ao número de ataques extras.
	
No caso de você está dando vários golpes e o oponente não poder esquivar, o mestre pode atribuir um valor de dificuldade mínimo para você acertar o golpe. Isso fica a critério do narrador.
	
Tanto a quantidade de golpes, quanto as penalidades que cada golpe tiver (se tiver) podem ser estabelecidas pelo mestre. Existem mestres que não limitam a quantidade de golpes outros se guiam pelo bônus de destreza (cada bônus de destreza é um golpe a mais). Isso cabe ao mestre decidir. Aqui nós apenas mostramos uma média que pode ser usada, uma vez que o narrador pode utilizar do modo como bem desejar. Existem algumas habilidades que retiram esses limites, cedendo ataques extras ou bônus aos atacantes. 

\subsection{Esquivas Múltiplas}

Quando um personagem recebe múltiplos ataques, seja de um único oponente realizando vários ataques ou de múltiplos oponentes, ele recebe penalidades cumulativas em suas esquivas.

\textbf{Regra Simplificada}: Para cada ataque extra recebido, o defensor recebe um redutor de -2 na esquiva.

\textbf{Exemplos}:
\begin{itemize}
	\item Um oponente realiza 3 ataques: primeiro ataque sem penalidade, segundo ataque -2, terceiro ataque -4.
	\item Três oponentes atacam simultaneamente: primeiro ataque sem penalidade, segundo ataque -2, terceiro ataque -4.
\end{itemize}

O mestre pode permitir que o defensor use uma única esquiva para ataques quase simultâneos, mas isso consome mais PF e pode exigir um teste de percepção dependendo da situação.
 


%\subsection{Detalhes sobre Aparar}

%A regra a seguir pode ser usada por mestres mais exigentes em relação a detalhes da batalha.

%Caso o defensor passe no teste (consiga aparar o golpe) ele deve realizar um teste de força ou destreza (ele pode escolher) contra a força do atacante em duas situações distintas. A primeira é quando a arma do atacante é de grande porte, o defensor esté usando uma arma média para aparar e sua força é menor do que a força do atacante. A segunda é quando as duas armas são de mesmo porte, porém a força do atacante é maior que a força do defensor em 4 pontos ou mais. No caso do defensor escolher destreza para jogar contra a força do alvo, ele usa o mesmo resultado do teste inicial de aparar. 

%Esses dois testes são interpretados da seguinte forma. O primeiro serve para o posicionar o objeto que vai ser usado para aparar o golpe. Esse teste é o teste de aparar em si, aonde bônus de habilidades e equipamentos que concedam bônus em aparar podem ser usados. O segundo teste serve para anular ou direcionar a força proveniente do atacante. Para realizar tal feito, o defensor pode fazé-lo usando sua força diretamente contra a do oponente ou utilizando sua destreza para desviar essa força em outra direééo. Ou seja, o atributo aparar do equipamento deve ser usado como um bônus em qualquer teste de destreza envolvendo essa manobra. De forma semelhante, o atributo resiténcia do equipamento deve ser usado como um bônus em qualquer teste de força.


%Se o defensor não conseguir passar nesse teste de força combatido, isso pode ser interpretado que o defensor conseguiu posicionar algum objeto na trajetória do ataque, mas a força do ataque deslocou o objeto. Então o defensor recebe o golpe, porém reduzido. O valor da redução do golpe deve ser definida pelo mestre. Por exemplo, o mestre pode reduzir do dano recebido pelo defensor um valor igual a força do defensor. O defensor também pode ter feito um teste de força cujo resultado foi téo baixo, que o golpe não perdeu nada de sua força original. Nesse contexto, no lugar do defensor levar o golpe reduzido, o mestre também pode retirar uma quantidade maior de PR da arma ou até mesmo fazer com que a arma caia no chéo, próximo ao defensor. O mestre pode interpretar a manobra da forma como desejar.

%Em outras palavras, para realizar a manobra aparar o defensor deve realizar no méximo dois testes. O primeiro é de destreza contra a destreza do oponente. O segundo deve ser realizado caso se encaixe na situação descrita acima. Esse segundo teste é de força ou destreza (o defensor escolhe qual atributo usar) contra a força do oponente. Caso escolha a destreza, ele usa o mesmo valor obtido no primeiro teste. 

%Em algumas situações o mestre pode (e deve) proibir o uso dessa manobra. Por exemplo, alguém tentando aparar um martelo grande com uma espada curta. Apesar da perécia bélica "lutar desarmado" poder ser usada para aparar ataques, o mestre deve considera-lé como arma curta ou média de acordo com a situação. Por exemplo, ela pode ser considerada arma média caso o defensor esteja próximo do oponente, tornando assim o uso da péricia mais fácil. No caso inverso, ou seja, o defensor usa alguma arma para se defender de um atacante que usa lutar desarmado, o defensor não pode usar a manobra aparar (apesar com alguma péricia defensiva como escudo ou manopla) caso o atacante esteja muito próximo. Isso é determinado pelo mestre de acordo com a batalha, onde o mesmo pode exigir teste por parte do atacante para se aproximar a tal ponto do defensor (um teste de saltar por exemplo).	


\subsection{Atacando Alvos Específicos}

Para tornar mais ágil e simples o combate, em Dregon adotamos a seguinte interpretação para ataques em alvos específicos. Ao invés de acertar um local específico de um alvo para retirar uma quantidade maior de PV, sempre que um atacante retirar um sucesso muito grande (dano e acerto) esse ataque bem sucedido é interpretado como sendo nesse local. Por exemplo, uma grande quantidade de dano retirada subitamente pode ser interpretada como um membro decepado por exemplo.

Caso o personagem deseje atacar algum alvo específico do oponente (furar uma bolsa com itens, por exemplo), a jogada de ataque é feita normalmente, porém, o defensor tem um bônus em sua jogada de esquiva (esquivando ou aparando) de acordo com o tamanho do objeto alvo. Quanto menor o alvo, maior o bônus, variando de 2 até 8.

O mestre também pode usar a seguinte regra opcionais para redução de atributo de acordo com dano causando em certas partes do corpo. Caso o ataque acerte com êxito membros do corpo (usando os redutores citados acima), o mestre pode impor redutores em atributos (força, destreza ou esquiva) de acordo com o dano sofrido. A quantidade de redução de atributo será avaliada perante a quantidade de dano sofrido. O redutor é igual ao bônus do dano (um dano de 4 irá causar um redutor de -1 e assim por diante).
  

%\begin{itemize}
%	\item Alvos médios (bolsas, braços, etc) = bônus de destreza ou esquiva +2.
%	\item Alvos pequenos(pescoéo, cabeéa, mãos, etc) = bônus de ágilidade  + 4.
%	\item Alvos muito pequenos(olhos, brincos, etc) = bônus de ágilidade  + 6.
%\end{itemize}

O personagem também pode desejar atacar um alvo específico do oponente com o objetivo de reduzir a absorção da armadura do alvo. Para fazer isso é simples. Para cada ponto reduzido no teste de atacar, o dano (caso o ataque acerte) ignora 1 ponto da armadura do alvo (começando pela defesa normal). Outro detalhe é que para cada ponto de cobertura da armadura, a mesma consegue ignorar 1 ponto essa redução. Por exemplo, caso o atacante deseje retirar 2 pontos de absorção de uma armadura com cobertura zero ele deve jogar o teste de atacar com um redutor igual a 2. Se ele deseja reduzir a mesma quantidade de uma armadura com cobertura 3, então ele deve ter um redutor de 5 em seu ataque. A penalidade recebida não pode ser maior que o atributo destreza - 1. Vale observar que armaduras pesadas podem ter sua absorção reduzida até a metade com o uso dessa manobra.


%Caso o ataque atinja partes vitais e cause dano, o alvo sofre um dano extra igual a uma percentagem do PV do alvo (o mestre decide se 25\% ou 50\% do PV total).

Vale destacar que determinados inimigos não tem penalidades de receber danos em alvos específicos, seja devido a sua natureza ou cobertura excepcional de alguma armadura. Além disso, como dito anteriormente, o mestre pode interpretar os efeitos da perda de PV, ou seja, não será somente com o uso dessa manobra que um alvo pode ter seu braço ferido ou decepado. Cabe ao mestre escolher de acordo com a situação de combate quando essa manobra pode ser usada e como.

\subsection{Ataque em área}

Determinadas ocasiões permitem você acertar vários oponentes com apenas um golpe. Seja devido a arma ser grande (um machado ou espada montante), ou por que vários alvos estão muito próximos a você. No caso da arma poder fazer isso, vem especificado nela quantos metros ela pode alcançar com um golpe. No caso de personagems próximo demais, quem decide é o mestre. Nesse caso se faz um teste de destreza(no caso quem ataca). O alvo mais próximo recebe o golpe com destreza normal e os alvos consecutivos vão esquivar com um bonus cumulativo de +2 e uma penalidade cumulativa no dano para o atacante, em outras palavras, o ataque vai perdendo a força ao longo do trajeto. De acordo com a descrição da cena, o dano pode diminuir em 3 ate 10 para cada alvo atingido. Por exemplo, se alguém com uma espada longa ataca vários alvos e o segundo absorve tudo, o mestre pode decidir por reduzir o dano em 13 para o terceiro alvo, sendo 3 do primeiro e 10 do segundo. O mestre pode também interromper o ataque, dizendo que o ataque em área não efetuou sua trajetória completa. 

\subsection{Contra Ataque - Atacando no Turno de Esquiva}

Você pode no seu turno de esquiva optar por tentar atacar o oponente no lugar de esquivar-se de seu golpe (algo como um contra ataque). Primeiramente você tem que fazer um teste para ver se você é rápido o suficiente para poder atacar no turno de esquiva. Você vai fazer um teste de ágilidade contra o oponente que está te atacando nesse turno de ataque. Porém, como você esté atacando no turno de esquiva e o oponente não (pois o intervalo de tempo que ele dispõe para realizar uma ação e mais bem ajustado) você vai ter penalidades nesse teste de ágilidade. A penalidade será igual ao bônus de ágilidade do oponente (se for um oponente lento o turno de ataque dele poderá ser facilmente mais lento proporcionando oportunidade do seu ataque). Se você passar no teste poderá atacar normalmente o oponente. Algumas observações devem ser levantadas a respeito de tal manobra.
\begin{itemize}
	\item Se vários alvos estiverem te atacando no turno de esquiva, você poderá atacar eles. Faça um teste ágilidade contra cada um (com redutores baseados nas ágilidades de cada um separadamente). Os que você passar poderá atacar normalmente (dividindo a destreza com regras de múltiplos ataques), porém se alguém for mais rápido (ganhar no teste de ágilidade), você não poderá mais contra-atacar nesse turno.
	\item O alvo que recebe essa manobra não pode tentar se mais rápido que você(um contra-contra-ataque).
	\item Nem você nem o alvo terão direito de esquiva. O maximo que pode se feito é o que está no turno de ataque fazer um teste de percepção para poder usar a manobra defesa máxima. Esse teste é um teste contra a ágilidade de quem tenta dar o contra golpe.
	\item Essa manobra não pode ser feita no primeiro momento de ação da batalha.
	\item O mestre tem total direito para alterar tais regras segundo sua interpretação. Em casos como já foi citado, se os números do teste de ágilidade ou de destreza forem próximos ou iguais, o mestre pode interpretar de modo próprio. Os golpes podem se chocar, os dois atacam ao mesmo tempo e assim por diante.
	\item Como outras manobras de combate não usuais, ela consome mais PF do que o normal.
\end{itemize}


\subsection{Esquivando no Turno de Ataque }

Ao invés de atacar o oponente, você pode desviar de um provável golpe que pode vir a receber no próximo turno. Você ganha um bônus de ágilidade em destreza ou esquiva para qualquer ação que você fizer no turno seguinte a este em que você ficou parado (no caso seu turno de esquiva). Você amplia seu turno de ataque para se fundir com o turno de esquiva de modo a ganhar bônus para se esquivar ou defender usando a destreza (esse bônus e determinado pela ágilidade). O bônus não pode ser maior que o atributo usado. Por exemplo, certo personagem tem destreza 8 e esquiva 2, totalizando ágilidade 10. O bônus da sua ágilidade é 4, porém ele so pode usar 2 pontos desse bônus caso use a esquiva para se esquivar do ataque.


\subsection{Movimento Durante a Batalha}

Qualquer personagem pode se mover até um valor igual ao seu bônus de esquiva em metros durante um turno normal (podendo atacar e se esquivar normalmente). Se o personagem tirar o turno só para se mover, esse valor sobe para o seu valor completo de esquiva em metros.

O mestre pode reduzir esse valor de acordo com cargas levadas pelo personagem ou outras condições específicas da situação.
	
	
\subsection{Desarmar Oponente}

A manobra desarmar é usada para tirar a arma de um oponente de suas mãos. O uso dessa manobra é bastante simples, por parte do atacante um teste de destreza mais um bônus, definido de acordo com a situação do desarme. Esse bônus é igual ao bônus do atributo força caso o atacante use esse atributo para desarmar o oponente ou de acordo com a arma utilizada. De forma semelhante, o bônus é de destreza para desarmes usando armas que possibilitem essa opção (armas leves ou pequenas geralmente se encaixam nessa situação).

Por parte do defensor, um teste de destreza ou esquiva mais um bônus. Da mesma forma que o atacante escolhe atribuir o bônus de força ou destreza, o defensor também pode faze-lo. O mestre pode adicionar outros bônus extras de acordo com a situação, por exemplo, conceder um bônus maior de força se a arma utilizada for de duas mãos, se ela tiver algum acessério que ligue a arma ao defensor ou até mesmo de acordo com a descrição da manobra (distância dos alvos, usar uma arma média atrás de um escudo, etc). é muito importante que o mestre preste atenção nesses detalhes, por exemplo, na maioria dos casos é muito difícil desarmar um oponente usando determinadas armas (espadas, machados, etc). Boa parte das armas são feitas para causar dano, e seu formato não auxilia na manobra desarmar. Para facilitar, o mestre pode usar o seguinte referencial. Se uma arma não conceder bônus em aparar, o defensor recebe seu bônus (de destreza ou esquiva) duas vezes para se defender. Se a descrição da manobra por parte do atacante for coerente, o defensor joga o atributo mais bônus de atributo mais bônus de distância (que pode ser ignorado de acordo com a situação).

Essa manobra não retira dano do defensor caso seja bem sucedida. O número de sucessos determina qual longe a arma foi arremessada. Em alguns de sucesso muito baixo favorecendo o atacante, o defensor pode realizar um teste de concentração para tentar pegar a arma "no ar", ou atacar usando a arma no próximo turno com um redutor (como se uma parte do turno de ataque dele tivesse sido usada para recuperar a arma caída).
 
O mestre deve decidir quais atributos serão usados e os bônus que serão estábelecidos de acordo com a situação. Por exemplo, um samurai com uma espada média tentando desarmar um ranger com uma lança. O samurai pode escolher destreza (recebendo assim também o bônus de destreza) para tentar desarmar o ranger. O ranger escolhe pular para o lado enquanto puxa sua lança. Nesse caso ele joga esquiva mais bônus de destreza mais um bônus definido pelo mestre de acordo com a distância que o samurai encontra-se do ranger.

Em termos de sistema não é uma manobra complexa, porém seu uso da brecha para muitas possibilidades diferentes, possibilidades essas que podem conceder bônus e ônus tantos para o defensor quanto para o atacante. O mestre deve ficar atento para a descrição da manobra. Na dúvida, favorecer a defesa.

\subsection{Agarrar Oponente}

Para realizar a manobra agarrar é simples. Basta realizar sua jogada de ataque normal, ou seja, destreza contra esquiva ou destreza do oponente. Caso acerte o alvo, o atacante deve realizar outro teste, porém dessa vez usando força contra a força ou contra a destreza do oponente. Se obtiver sucesso mais uma vez, o defensor é considerado agarrado pelo atacante. O mestre deve incluir bônus ou ônus de acordo com a situação, por exemplo, um atacante pequeno tentando prender um alvo grande apenas com as mãos, ou o atacante usando cordas para amarrar o alvo. Quando agarrado, dependendo da situação do agarrão, o mestre deve definir as limitações de ações durante o agarrão. Por exemplo, uma pessoa com os braços imobilizados, não pode atacar usando os mesmos. O mestre também deve definir o dano que o alvo fica recebendo enquanto fica agarrando. Esse dano também pode ser representado por perda de PF (em situação de estrangulamento), ou perda de atributo (no caso de um membro quebrado).

Para se soltar do agarrão, o defensor deve realizar um teste novamente de destreza ou força com a força do oponente.

%\subsection{Regra Opcional sobre PF}

%Para mestres mais exigentes exige um método mais detalhista para monitoramento de perda de PF para cada personagem. Para cada personagem presente na batalha o mestre deve criar uma barra, com 10 campos, variando de 10\% até 100\%, como mostrado a seguir. Vamos chamar essa barra de "félego".

%\begin{table}[htbp]
%\begin{center}
%\begin{tabular}{|c|c|c|c|c|c|c|c|c|c|} \hline 
%10\%&	 20\%&	30\%&	 40\%&	 50\%&	 60\%&	 70\%&	 80\%&	 90\%&	 100\%\\\cline{1-10} 
% &  &   & &  &  &  &  &   &  
%\\ \hline
%\end{tabular}
%\end{center}
%\caption{Barra de acompanhamento de perda de PF}
%\label{}
%\end{table}

%Sempre que o personagem realizar uma ação, o mestre deve marcar com um "X" os campos equivalentes a percentagem que aquela ação consome. Por exemplo, se um personagem realizar um ataque que consuma 30\% dessa barra, a tabela deve ficar como a mostrada a seguir (partindo do princépio que o félego inicialmente estáva vazio). 
  

%\begin{table}[htbp]
%\begin{center}
%\begin{tabular}{|c|c|c|c|c|c|c|c|c|c|} \hline 
%10\%&	 20\%&	30\%&	 40\%&	 50\%&	 60\%&	 70\%&	 80\%&	 90\%&	 100\%\\\cline{1-10} 
%X &X  &X   & &  &  &  &  &   &  
%\\ \hline
%\end{tabular}
%\end{center}
%\caption{Exemplo usando ataque que consuma 30\% de seu félego}
%\label{}
%\end{table}

%A percentagem de quanto cada ação consome do félego é mostrada na tabela a seguir.

%\begin{table}[htbp]
%\begin{center}
%\begin{tabular}{|c|c|c|c|} \hline 
%Equip/Aééo &	 Esquiva &	Ataque&	 Aparar\\\cline{1-4} 
%Leve & 5\%  & 20\%   &  10\% \\\cline{1-4} 
%Médio & 10\%  & 30\%   &  20\% \\\cline{1-4} 
%Pesado & 30\%  & 50\%   &  30\%  
%\\ \hline
%\end{tabular}
%\end{center}
%\caption{Gasto de percentagem de PF}
%\label{}
%\end{table}


%Sempre que essa barra atingir 100\% o personagem perde 1 PF. Para cada ação extra realizada no mesmo turno, o personagem perde metade do que normalmente perde quando realizado aquela ação pela primeira vez. Em outras palavras, para cada 2 ações extras o personagem perde a percentagem associada aquela ação na tabela acima. Por exemplo, se um personagem realizar 3 ataques com uma arma média no mesmo turno ele perde 60\% de seu félego. O consumo do félego da esquiva esté associado na tabela acima está relacionado com o peso do equipamento usado. Em outras palavras, em um turno um personagem com uma arma grande e uma armadura média, perde 60\% de seu félego quando realizado um ataque e uma esquiva. 

%Vale lembrar que em certas batalhas um personagem não é atingido ou realiza nenhum ataque. Nessas situações um personagem não perde PF na batalha. Isso também cria situações de que um grupo salvar os PF de um personagem para serem gastos em situações mais urgentes. 

%Porém vale observar que em algumas batalhas o consumo de PF por ser crucial para definir o seu resultado. Por exemplo, um personagem usando uma arma de porte grande (um machado por exemplo), pode usar a manobra aparar várias vezes (uma vez que usando uma arma de grande porte, ele pode aparar armas de porte médio e pesado). Caso a destreza desse personagem seja muito grande, pode acontecer de que nenhum de seus oponentes consigam atingi-lo. Nesse caso, os oponentes podem bolar uma estratégia de consumo de PF, fazendo com que o alvo use a arma de porte grande várias vezes para aparar, consumindo seus PF lentamente com isso. Nesse caso o mestre deve retirar PF cautelosamente dos personagems e não somente uma vez ao termino da luta, pois o consumo de PF durante ela é decisivo nesse caso.

%Para ágilizar todo o processo de perda de PF durante a luta, o mestre pode ir anotando as ações realizadas por cada personagem (1 ataque, 3 aparar, 2 ataques, não foi atacado, etc), e de 4 em 4 turnos ir retirando de cada personagem o PF perdido até aquele momento. Vale lembrar que o mestre tem total liberdade para interpretar a perda de PF e usa-lé dentro de jogo como bem desejar. Alguns mestres retiram PF apenas quando habilidades ou manobras são usadas.



