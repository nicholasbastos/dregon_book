


		 
	
	
  
  




 
 Oraculo / Eremita 
 	





Habilidades Gerais Especiais

Algumas habilidades gerais merecem ser uma explica鈬o mais detalhada devido a sua natureza 佖ica. Al駑 disso, as habilidades gerais especiais só podem ser aprendidas em situa鋏es muito espec凬icas, e n縊 a qualquer momento.

Afinidade Natural
 A habilidade geral afinidade natural faz com que um personagem se comunique e sinta as for軋s da natureza de um local por ele escolhido. Esse contato faz com que o personagem dificilmente se perca nesse habitat. Al駑 disso, ele tamb駑 mantem uma sincronia com os animais da regi縊, podendo pedir socorro ou escutar pedidos de ajuda. Para aprender essa habilidade o personagem deve passar anos no ambiente e ter um envolvimento pac凬ico com o mesmo.
 
 
 Metamorfose
 
 Metamorfose é a capacidade que algumas criaturas t麥 de alterar sua forma f﨎ica e em alguns casos até sua composi鈬o. Ao adquirir metamorfose o personagem ganha a capacidade de alterar partes do seu corpo atrav駸 de sua vontade, representada pelo bus concedido pela habilidade. A metamorfose é um poder extremamente vers疸il, onde cada uso diferente da habilidade tem um custo em bus que deve ser descontada do bus total da habilidade. Em outras palavras, um personagem com um bus de +8 em metamorfose pode invocar 4 poderes de 2 pontos. 

O bus inicial é +3, assim como as habilidades gerais normais. Por駑 sua evolu鈬o é um pouco diferente. O personagem deve gastar 4 pontos de XP (mental ou f﨎ica definidos no momento da compra da habilidade) e 4 PS para aumentar +1 ponto no bus da habilidade. A partir do d馗imo ponto, essa habilidade recebe um redutor para aumentar seu bus definido da mesma forma que o redutor para aumentar atributos acima de dez é definido.

Para ativar a metamorphose o usu疵io deve gastar em PM um valor igual ao bus utilizado. Ou seja, para ativar metamorfose +6 o usu疵io gasta 6 PM. Para fins de ativa鈬o a metamorphose é considerada uma habilidade instant穗ea. Por駑 o usu疵io pode ativa-la reflexivamente no turno de ataque gastando 2 PF extras. A metamorfose fica ativa durante concentra鈬o em minutos ou até que o poder diga o contr疵io. O usu疵io pode "renovar" a ativa鈬o da habilidade gastando reflexivamente metade do custo em PM usado para ativar a habilidade. 

Apesar de ser uma habilidade extremamente 偀il ela pode ser perigosa quando usada desfreadamente. O usu疵io pode receber o bus de metamorfose sem penalidades a seu corpo até o valor de seu bus de resist麩cia. Para qualquer valor de bus concedido acima desse limite, o usu疵io perde o dobro desse valor distribuido entre PF e/ou PV. Por exemplo, se um personagem tem resist麩cia 12 (bus de resist麩cia igual a 5) e usar metamorfose +8 ele deve perder 6 pontos entre PF e/ou PF, por exemplo, 4 PV e 2 PF.
	
	Sempre que o usu疵io alterar o seu corpo usando a metamorfose, toda a vestimenta e equipamento atrelados a ele continuam normais, n縊 tendo nenhuma altera鈬o no seu formato.

	Segue algumas sugests de poderes que podem ser usados a partir da metamorfose, assim com o quanto elas consomem do bus da habilidade:


	 Aumento de atributo: Cada ponto no bus em metamorfose concede aumentar 1 ponto de atributo. 

	 		Armas Extras: Permite ao personagem melhorar suas armas naturais criando dentes afiados ou transformando m縊s e p駸 em garras, ferrs, l穃inas, etc. Cada  pontos do bus concede adicionar +1 de dano autom疸ico a arma criada.

		 	Armadura Natural: Outra utiliza鈬o muito simples da metamorfose, o personagem pode transformar sua pele em couro, escamas, carapa軋, etc. Cada  pontos do bus concede adicionar +1 de dano autom疸ico a armadura criada.

	 		Asas: O metamorfo cria asas e ganha capacidade de planar ou voar. A sua velocidade de locomo鈬o é semelhante a usada em terra. Custa 6 pontos do bus cedido pela metamorfose.

	 		Anatomia Mutante:  O usu疵io tem 50% de chance de anular o bus no dano proveniente do bonus de acerto. Custa 4 pontos do bus cedido pela metamorfose.
	
	 		Org縊 de Marmore:  O usu疵io cancela qualquer bus no dano proveniente do bus de acerto. Custa 8 pontos do bus cedido pela metamorfose.

	 		Alcance Estendido: O metamorfo pode alterar seus membros, tornando-os extens咩eis, aumentando o alcance de seus ataques f﨎icos. Esse efeito permite ao personagem aumentar a dist穗cia de ataques corpo a corpo. Custa 4 pontos do bus cedido pela metamorfose.

	 		Ataque Tico: O metamorfo pode alterar seus fluidos corporais e torná-los ticos ou c疼sticos. O dano causado é igual a quantidade de pontos retirados do bus total de metamorfose + 3. O ataque pode ser realizado a dist穗cia até o valor de for軋 metros do usu疵io. A jogada de acerto é feita usando-se destreza. 

	 		Membros Extras: O metamorfo pode criar ap麩dices à partir de seu tax ou coluna formando membros novos, que podem variar de simples complementos, como caudas ou antenas, até membros h畸eis, como bra輟s, pernas, etc. Uma cauda pode garantir bus em testes espec凬icos (saltos, equil兊rio, nata鈬o, etc.) e membros h畸eis em testes de esfor輟 prolongado (agarrar, empurrar, erguer...), facilitar o uso de armas ou ferramentas grandes, ou simplesmente usar e carregar mais coisas. O bus recebido no membro extra é igual ao bus retirado do bus total da metamorfose.

	 		Mil Faces: O metamorfo pode mudar sua apar麩cia, recebendo o bus da metamorfose para disfarces, podendo se passar por outra pessoa ou ra軋 (podendo até alterar o g麩ero com uma habilidade alta o bastante). Quanto mais pontos do bus total forem retirados, maior será a transforma鈬o do rosto do usu疵io.

	 		Reorganizar Metabolismo: O metamorfo pode alterar a forma como seu corpo funciona tornando-o resistente à efeitos de enfraquecimento, recebendo o bus de metamorfose em testes para resistir ou superar venenos, doen軋s e outras toxinas. O bus recebido para veneno é igual ao valor retirado do bus total da habilidade.

	 		Respira鈬o Anf兊ia: O metamorfo pode alterar seu sistema respiratio, tornando-se capaz de respirar dentro e fora da 疊ua (ou outro ambiente semelhante). Custa 3 pontos do bus cedido pela metamorfose.

	 		Sentidos Agu軋dos: O metamorfo pode alterar suas estruturas sensoriais, agu軋ndo seus sentidos ou desenvolvendo capacidades sensoriais completamente novas. No momento da ativa鈬o ele deve especificar qual sentido esta sendo agu軋do de acordo com a habilidade sentidos agu軋dos. Custa 3 pontos do bus cedido pela metamorfose.
	
	 		Forma da Besta: O usu疵io pode adquirir a forma de um animal que ele conhe軋. Apesar de seus atributos continuarem os mesmos, o mestre pode fazer algumas altera鋏es de acordo com o tipo do animal transformado. Custa 12 pontos do bus cedido pela metamorfose. A dura鈬o dessa habilidade é igual a concentra鈬o em horas ou até o usu疵io desejar.
	
	 		Forma de N騅oa: Enquanto na forma de n騅oa o usu疵io n縊 pode receber ataques f﨎icos normais, ele so pode ser atacado ou atacar usando dano elemental, mas pode usar seus atributos normalmente para defender, atacar e se esquivar. Da mesma forma ele so pode atacar com ataques elementais. O usu疵io n縊 pode falar nessa forma. Custa 14 pontos do bus cedido pela metamorfose.



 
Armas Extras
 
Armas extras é a capacidade de um personagem usar partes do seu corpo como armas. Garras, presas, caldas, s縊 apenas exemplos de armas extras. Geralmente está acessivel a determinadas ra軋s ou classes, sendo de dificil acesso para outras. Quando usando armas extras, o personagem n縊 necessita saber nenhuma pericia belica para usar as armas extras. Elas podem ser melhoradas com XP f﨎ico, por駑 elas como qualquer arma tem PR. Esses PR s縊 recuperados de uma forma diferente das armas normais. Sempre que um personagem desgastar suas armas extras, ele so pode recuperar os PR perdidos com o uso da habilidade medicina ou com medicamentos apropriados.

Todas as armas extras com formato semelhante a uma arma, s縊 considerados como tal para fins de uso em conjunto com a pericia belica espec凬ica e para outros detalhes na batalha. Por exemplo, se uma garra se assemelha a uma espada curta, o usu疵io pode usar o bus de especializa鈬o cedido pela p駻icia b駘ica espada curta mas n縊 pode ser usada em certas ocasis em conjunto com a manobra aparar.
 
Mediunidade
 
 A habilidade mediunidade diz o quanto você pode interagir com o mundo dos esp叝itos e seus habitantes. O n咩el dessa intera鈬o é dito pelo seu atributo esp叝ito. Com espirito 6 você consegue ver esp叝itos sem nenhuma penalidade ou gasto de pontos. Com espirito 8 você pode conversar com eles normalmente. A linguagem dos esp叝itos é universal, ou seja, o medium e o esp叝ito podem conversar em qualquer idioma e mesmo assim pode haver total entendimento da conversa. Com espirito 10 você pode incorporar algum esp叝ito. Nesse caso seus atributos f﨎icos n縊 mudam, por駑 seus atributos mentais assim como suas habilidades podem receber um bus consideravel de acordo com os atributos do esp叝ito a ser incorporado.
 
Apesar de ser uma habilidade 偀il em muitas situa鋏es, pode ser perigosa. Um usu疵io da mediunidade com esp叝ito baixo esta sujeito a pertuba鋏es e até mesmo incorpora鋏es involunt疵ias.

Essa habilidade está relacionada a classe ou ra軋, por駑 o mestre pode liberar sua compra de acordo com algum fator decorrente da campanha. 

Familiar
O familiar, ou companheiro animal, é um parceiro animal que o ajuda sempre que necess疵io. Apesar de n縊 ser uma habilidade propriamente dita do personagem, em termos de sistema é tratada como tal. Para um personagem come軋r com um familiar ele deve ter a aprova鈬o do mestre e sacrificar 1 conhecimento inicial. Geralmente classes como druida e ranger podem trocar 1 conhecimento por um familiar sem maiores exig麩cias do mestre, mas de acordo com a campanha todas as classes podem ter o mesmo privilegio. Isso quem deve decidir é o mestre. Por exemplo, o mestre n縊 deve permitir um familiar muito forte que desequilibre o grupo. Um familiar nunca é melhor do que o seu dono. Ele foi criado para complementar um dos campos de atua鈬o que o personagem n縊 é t縊 bom nele. Apesar disso, um familiar sempre é mais destac疱el que um animal normal de seu tipo. Por exemplo, um familiar coruja é mais r疳ido e inteligente que uma coruja normal.

Para criar um familiar o mestre deve usar a seguinte regra:

	 O familiar tem 22 pontos de atributo f﨎icos iniciais para distribuir.
	 O valor de um atributo n縊 pode ser zero nem maior do que 12.
	 O familiar tem 14 pontos de vida iniciais.
	 O familiar tem 8 PF iniciais.
	 Cada ponto colocado em defesa, aumenta em 1 a quantidade de PV do familiar.
	 Cada 2 pontos colocados em for軋, aumenta em 1 a quantidade de PF do familiar.
	 Cada 2 pontos em agilidade, concede 1 ponto em uma habilidade geral que o familiar possa aprender.


Cabe ao mestre decidir se o jogador tem liberdade para criar seu familiar ou n縊. O mestre pode criar o familiar para o jogador.

As regras para cria鈬o de familiar servem apenas durante sua cria鈬o. A partir de ent縊 ele deve evoluir da mesma forma que um personagem normal. Em termos de evolu鈬o, o familiar evolui bem mais devagar que o jogador. O familar recebe 1 a 2 de xp f﨎ico por sess縊. No final de uma aventura (quest) o familiar recebe tamb駑 ponto de bus. Esse valor pode se igualar ao valor recebido pelo jogador, de acordo com a vontade do mestre analisando o qu縊 importante o familiar foi nessa aventura. 

Se o jogador tiver a habilidade adestrar animais ele pode converter xpf ou xpm prrios em abp do familiar. Sempre que o jogador desejar fazer isso com seu familiar, ele deve treina-lo. Esse per卲do de treinamento varia de alguns dias para até semanas de acordo com o valor do atributo que o familiar deseje aumentar. Al駑 disso, para cada per卲do de treinamento o jogador tem um limite de xpf e xpm que ele pode converter. Esse valor é igual aos sucessos obtidos em um teste (dificuldade 18) de acuidade ou consci麩cia mais bus em adestrar animais. O mestre pode conceder bus ao jogador nesse teste caso o mesmo possa se comunicar com o familiar em quest縊 com o uso da habilidade conversar com animais. Um familiar n縊 pode aprender habilidades de classe, somente pode aprender a magia pura de alguns elementos em raros casos.

O uso de familiar tamb駑 pode ser interessante para que pessoas que n縊 sejam fixas no grupo, participem de certas sesss jogando com o familiar.
 
 
 TechnoAlquimia
 
 A technoalquimia é um tipo especial de habilidade que une os poderes de 3 habilidades: Alquimia, herbalismo e medicina. As drogas produzidas pelos usu疵ios dessa habilidade tem os efeitos de cura, melhorando de atributos al駑 de poder serem usadas ofensivamente. Apesar de usar b疽icamente as mesmas materias primas das outras habilidades similares, as ferramentas de produ鈬o s縊 um pouco diferentes. Al駑 do mais, geralmente as drogas produzidas pelas technoalquimia s縊 mais potentes do que as produzidas normalmente. 
 
Os usu疵ios da technoalquimia s縊 exclusivos da tatsunamei. O mestre deve restringir seu aprendizado dentro de jogo. Outro detalhe é que o uso continuo de drogas techalquimicas pode levar o usu疵io a depend麩cia quimica.
 
 
Conhecimento Ritual
 
Por defini鈬o um conhecimento n縊 tem uma pontu鈬o como a maioria das habilidades gerais. Por駑 o conhecimento ritual é um pouco diferente. Ele tem uma pontua鈬o total que deve ser usada para comprar rituais. O que isso quer dizer? Que um personagem com Conhecimento Ritual 6, pode comprar 2 rituais de 3 pontos, ou um ritual de 6 pontos, e assim por diante.

Como um conhecimento normal, um personagem que deseje aprender outros rituais deve faze-lo dentro da campanha, e n縊 com gasto de experi麩cia. O mestre pode exigir uma quantia m匤ima de intelig麩cia ou algum outro conhecimento espec凬ico por exemplo.

Um ritual usa a energia m疊ica do mundo para realizar um efeito m疊ico, usando alguns objetos como catalisadores. Esses objetos s縊 conhecidos como fetiches, e todo ritual teu os seus. Por usar a energia da natureza, um personagem para realizar um ritual n縊 gasta PM, PV, ou PF. Quando os fetiches est縊 desgastados ou quebrados, o ritual simplesmente n縊 pode ser realizado. Digamos que o usu疵io de um ritual pede para a natureza a energia m疊ica necess疵ia para realizar um efeito m疊ico. O jeito dele pedir isso é determinado como o usu疵io realiza o ritual usando os catalisadores (fetiches).
 
Abaixo segue uma lista de alguns rituais, assim como a quantidade de pontos relacionados a cada um. Como um ritual é um conhecimento alcan軋do e n縊 aprendido com o gasto de pontos, a quantidade de pontos aqui mostrada é utilizada para comprar rituais na cria鈬o de personagem ou em certos periodos durante a campanha. 
 
 

	 Ritual Tent(3 pontos): Fetiches: Um pouco de comida, madeira e panos. 30 Minutos para ser realizado. Cria uma tenda que pode abrigar 3 pessoas durante 10 horas, com agua, alimentos e vestimentas. Um ritualista so pode usa-la uma vez por semana.
 
 	 Ritual ADE(3 pontos): Fetiches: Liquido transparente, giz e areia. 5 minutos para ser realizado. O personagem ganha 6 pontos em for軋 e 6 pontos em defesa durante 30 minutos. Pode ser usado apenas uma vez por dia.
 
 	 Ritual ADES(6 pontos): Fetiches: Liquido transparente, giz e areia. 5 minutos para ser realizado. O personagem ganha 7 pontos em for軋 e 7 pontos em defesa durante 30 minutos. N縊 tem limite de uso di疵io.
 
 	 Ritual bussola(3 pontos): Fetiches: Peda輟 de Ferro pequeno e carv縊, giz ou barro. 5 minutos para ser realizado. Aponta a maior concentra鈬o de pessoas num raio de 5 km.
 
 	 Ritual de prote鈬o da Madeira Fraca(6 pontos): Madeira velha, Agua limpa, panos limpos. 1 hora para ser executado. O ritual é realizado
 sobre uma arma ou equipamento. Durante 24 horas, o equipamento perde metade dos PR normalmente perdidos com dano ou desgaste natural.
 
 	 Ritual da voz amiga(3 pontos): Papel limpo e carv縊. O personagem escreve uma mensagem, rasga o papel e joga no ar. Uma mensagem é enviada para qualquer pessoa que ele tenha encontrado num periodo de 3 anos. 
 
 	 Ritual da voz fraternal(3 pontos): Papel limpo, carv縊. O personagem escreve uma mensagem, rasga o papel e joga no ar. Uma mensagem é enviada para qualquer pessoa que ele tenha encontrado num periodo de 10 anos. Requer o ritual da voz amiga.
 
 	 Ritual da voz infinita(3 pontos): Papel limpo, carv縊. O personagem escreve uma mensagem, rasga o papel e joga no ar. Uma mensagem é enviada para qualquer pessoa que ele tenha encontrado durante qualquer periodo de sua vida. Requer o ritual da voz fraternal.
 
 	 Ritual do Corpo fechado(9 pontos): Bronze, barro, giz ou carvao e leite. 2 horas. O personagem constroi um boneco com os rituais, escreve seu nome com giz e ao final derrama leite no boneco enquanto recita palavras m疊icas. Enquanto o boneco n縊 for destruido, todo o dano que o personagem levar, é passado para o boneco. O boneco tem 30 PV. O ritualista so pode fazer um boneco por vez.
 
 	 Ritual da Boa Sorte (9 pontos): Trevo de quatro folhas, sal e p縊. 30 minutos. Durante o resto do dia e para os primos 7 sucessos o ritualista joga 3 dados ao inv騷 de 2 para escolher o maior resultado.

 