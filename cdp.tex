%%
%% Capítulo 3: Criação de Personagem
%%


\chapter{Criação de Personagem}
\label{Cap:cdp}


A criação de personagem no sistema Dregon foi feita para que o processo de criação fosse divertido, rápido e interativo, fazendo assim com que os jogadores e o mestre possam criar personagens de forma rápida porém robusta. Aqui iremos mostrar sequencialmente os passos que devem ser tomados para se criar um personagem no sistema Dregon.


\section{História}
É também chamada de background ou de prelúdio. A primeira coisa que um personagem deve fazer antes de criar sua ficha é criar sua história. Os caminhos que o levaram a tomar atuais rumos de vida, amizades e inimizades que ao longo de sua vida você plantou e colheu. Provavelmente esses fatos construíram o caráter de seu personagem. Fale como ele vê a vida, como é seu temperamento, seu modo de agir perante algumas ocasiões, características que fazem de seu personagem único.

Essa parte da construção do personagem é extremamente importante, pois aqui que você define importantes conceitos de seu personagem (como personalidade, trama que ele está envolvido e etc) além de justificar para o mestre algumas de suas habilidades e conhecimentos (já que muitas capacidades você apenas tem acesso com uma boa explicação em sua história, não apenas com gasto de pontos). O sistema aqui proposte é orientado a conceitos, ou seja, o personagem pode criar seu conceito com liberdade e o sistema vai propor ferramentas para que você crie um personagem com aquele conceito. Você não deve mudar o conceito do seu personagem para que ele se adapte ao sistema.

O mestre pode e deve auxiliar na construção de detalhes na história do personagem. Quanto mais detalhada for essa parte, mais vivo e característico será o PC (personagem). O mestre pode atribuir pontos em atributos sociais baseado no background do personagem. Lembre também de dar uma certa importância ao seu personagem quando for criar a sua história, fazendo com que ele de alguma forma se destaque no meio em que se desenvolveu. 


\section{Raça}

Dependendo do mundo (cenário) em que você joga, existem diversas raças ou até mesmo apenas uma (como na terra por exemplo onde somente existem humanos em determinados tipos de campanhas).

A raça define algumas características importantes de seu personagem. Sua aparência, seu tempo de vida, seu tipo de comportamento, entre outros aspectos. Porém, não é preciso que você siga o estilo de vida de sua raça a risca. Existem muitas pessoas de raças que são conhecidas por sua calma e inteligência e outras que são pessoas irritadas e raivosas. Esses dados são apenas um \"guia\" de como você deve agir perante tais situações.

Em termos de sistema a raça te diz alguns pontos importantes. Ela diz a quantidade de pontos iniciais para os atributos principais, assim como a quantidade de PV e PM do seu personagem. Também te diz algumas habilidades, habilidades essas conhecidas como \"habilidades básicas da raça\". O mestre pode limitar o uso de determinadas raças pela sua raridade no mundo a ser jogado.
Maiores detalhes do sistema em relação às raças, no capitulo \"Raças\".

\section{Classe}

A classe diz como o seu personagem usa as habilidades dele. Ela determina uma grande gama de habilidades, conhecimentos e perícias que o personagem terá dificuldade e facilidade de aprender.

Por exemplo, um pianista. A classe pianista concede bônus de aprendizado para todas as habilidades que envolvam música. Então, se sua classe for pianista além de iniciar com algumas habilidade de músico você terá facilidade em aprender outras no futuro. Em compensação, você terá dificuldade em aprender técnicas de sobrevivência em floresta por exemplo. 

Em termos de sistema, diz certos bônus que seu personagem pode ter, seja de atributos ou habilidades, além de definir a quantidade de PF inicial do seu personagem. 

\section{Toques Finais e Observações}

Quando tiver escolhido a classe, raça e sua história, distribua os pontos da sua raça, juntamente com os bônus da classe. O mestre deve ditar quais equipamentos que você vai começar, assim como a quantidade de dinheiro que você pode ter. Depois disso é só jogar!
Agora diremos algumas observações importantes sobre a criação de personagem.


\begin{itemize}
	\item Em Dregon não existem as famosas vantagens/qualidades e desvantagens/defeitos dos personagens. Seu personagem não vai ganhar pontos por ser cego ou ter um inimigo a mais. Do mesmo modo que ele não precisa ter pontos para ter uma espada celestial ou ser uma pessoa com muito dinheiro. Cabe ao mestre analisar se tais possíveis fatores são aceitáveis dentro da campanha, de acordo com a história do personagem.
O que pode ocorrer é de você ser cego e por causa disso ter os outros sentidos mais aguçados, mas não justifica somente por que você tem problema de visão terá mais contatos no submundo. Do mesmo modo, se você tiver uma espada sagrada provavelmente não saberá como usá-la, e será alvo fácil para os que sabem (mais fácil ainda já que é um personagem iniciante). Cabe ao mestre juntamente com o jogador analisar tais itens. Apenas habilidades técnicas devem ser compradas com pontos, do mesmo modo alguns poderes necessitam de fetiches para serem realizados.
	\item Cada 5 pontos em consciência dão ao personagem um conhecimento extra. Com esses pontos de conhecimentos extras você pode comprar novos conhecimentos. Os conhecimentos não têm nível próprio - quando você usa um conhecimento, você joga com o valor de inteligência ou sabedoria (conforme o tipo de conhecimento). Maiores detalhes na sessão sobre conhecimentos/habilidades.
 	\item O personagem recebe um bônus em PV igual ao bônus do atributo resistência. Esse bônus também é utilizado após a criação de personagem quando se aumenta os atributos gerais de forma permanente. Por exemplo: um personagem com resistência 8 (bônus de 3) recebe +3 PV. Quando ele aumenta sua resistência para 9 (bônus de 4), recebe mais 1 PV adicional.
	\item Para cada 3 pontos no atributo focus, o personagem recebe 1 PM. Esse bônus também é utilizado após a criação de personagem quando se aumenta os atributos gerais de forma permanente. Por exemplo: um personagem com focus 11 recebe 3 PM extras (11÷3 = 3, arredondado para baixo). Se ele aumentar seu focus para 12, seu PM extra aumenta em 1 (12÷3 = 4).
	\item Na criação você pode trocar experiência por PS. A relação é 2 PS para 1 experiência, mas isso é somente utilizado durante a criação de personagem.
	\item Cada personagem tem 10 pontos de experiência de início, não importa qual raça ou classe ele seja. A experiência pode ser usada para aumentar qualquer atributo. Os pontos não usados permanecem como experiência.
	\item Todo personagem começa com nível 1. O mestre pode, em determinadas aventuras, mudar isso dando um, dois, ou mais níveis para o personagem.
	\item O mestre pode limitar durante a criação de personagem um valor máximo para os atributos dos personagens, que por base deve ser 10.	
	\item O personagem pode trocar habilidades de classe por experiência. Se o jogador não desejar certa habilidade de classe ele pode trocá-la por experiência. Cada habilidade cede 6 pontos de experiência ao personagem. O mesmo pode ser feito para algumas habilidades de raça.
	\item O personagem recebe 10 PO para gastar em itens durante sua criação, caso o mestre permita.
	\item Todo personagem tem direito a uma habilidade inútil. O uso de habilidade inútil foi criado no sistema Dregon para dar uma caracterização única e divertida para seu personagem. Uma habilidade inútil deve ser algo que incremente a personalidade do seu personagem, mas não o ajudará em testes e situações de vida e morte. Exemplos de habilidades inúteis são: poses de vitória, lamber o cotovelo, arrotar cantando, saber hinos de times de esporte, criar frases estranhas, cantadas horríveis, cantar música brega, entre outras.
	\item O personagem deve gastar toda a sua experiência inicial, ou convertê-la em PS segundo o mestre.
	\item Um personagem não pode ter dois pontos favorecidos em apenas um atributo. Cada atributo pode ter no máximo um ponto favorecido.
\end{itemize}

Ao final deste livro serão mostrados maiores exemplos de criação de personagem utilizando o sistema.


