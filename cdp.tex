%%
%% Cap咜ulo 4: Batalha
%%


\chapter{Criação de Personagem}
\label{Cap:cdp}


A criação de personagem no sistema dragon foi feita para que o processo de criação fosse divertido, rápido e interativo, fazendo assim com que o jogadores e o mestre possam criar personagens de forma rápida porém robusta. Aqui iremos mostrar sequencialmente os passos que devem ser tomados para se criar um personagem no sistema dragon.


\section{Historia}
 Também chamado de background ou de preludio. A primeira coisa que um personagem deve fazer antes de criar sua ficha é criar sua histia. Os caminhos que o levaram a tomar atuais rumos de vida, amizades e inimizades que ao longo de sua vida você plantou e colheu. Provavelmente esses fatos constru叝am o car疸er de seu personagem. Fale como ele vê a vida, como é seu temperamento, seu modo de agir perante algumas ocasis, caracter﨎ticas que fazem de seu personagem 佖ico.

Essa parte da construcao do personagem é extremamente importante, pois aqui que você define importantes conceitos de seu personagem (como personalidade, trama que ele está envolvido e etc) al駑 de justificar para o mestre algumas de suas habilidades e conhecimentos (já que muitas capacidades você apenas tem acesso com uma boa explicacao em sua historia, n縊 apenas com gasto de pontos). O sistema aqui proposte é orientado a conceitos, ou seja, o personagem pode criar seu conceito com liberdade e o sistema vai propor ferramentas para que você crie um personagem com aquele conceito. Você n縊 deve mudar o conceito do seu personagem para que ele se adapte ao sistema.

O mestre pode e deve auxiliar na construcao de detalhes na histia do personagem. Quanto mais detalhada for essa parte, mais vivo e caracter﨎tico será o PC (personagem). O mestre pode atribuir pontos em atributos sociais baseado no background do personagem. Lembre tamb駑 de dar uma certa import穗cia ao seu personagem quando for criar a sua historia, fazendo com que ele de alguma forma se destaque no meio em que se desenvolveu. 


\section{Ra軋}

Dependendo do mundo (cen疵io) em que você joga, existem diversas ra軋s ou até mesmo apenas uma (como na terra por exemplo onde somente existem humanos em determinados tipos de campanhas).

A ra軋 define algumas caracter﨎ticas importantes de seu personagem. Sua apar麩cia, seu tempo de vida, seu tipo de comportamento, entre outros aspectos. Por駑, n縊 é preciso que você siga o estilo de vida de sua ra軋 a risca. Existem muitas pessoas de ra軋s que s縊 conhecidas por sua calma e intelig麩cia e outras que s縊 pessoas irritadas e raivosas. Esses dados s縊 apenas um \"guia\" de como você deve agir perante tais situa鋏es.

Em termos de sistema a ra軋 te diz alguns pontos importantes. Ela diz a quantidade de pontos iniciais para os atributos principais, assim como a quantidade de PV e PM do seu personagem. Tamb駑 te diz algumas habilidades, habilidades essas conhecidas como \"habilidades b疽icas da ra軋\". O mestre pode limitar o uso de determinadas ra軋s pela sua raridade no mundo a ser jogado.
Maiores detalhes do sistema em relacao 灣 ra軋s, no capitulo \"Ra軋s\".

\section{Classe}

A classe diz como o seu personagem usa as habilidades dele. Ela determina uma grande gama de habilidades, conhecimentos e per兤ias que o personagem terá dificuldade e facilidade de aprender.

Por exemplo, um pianista. A classe pianista concede bus de aprendizado para todas as habilidades que envolvam m俍ica. Ent縊, se sua classe for pianista al駑 de iniciar com algumas habilidade de m俍ico você terá facilidade em aprender outras no futuro. Em compensacao, você terá dificuldade em aprender t馗nicas de sobreviv麩cia em floresta por exemplo. 

Em termos de sistema, diz certos bus que seu personagem pode ter, seja de atributos ou habilidades, al駑 de definir a quantidade de PF inicial do seu personagem. 

\section{Toques Finais e Observa鋏es}

Quando tiver escolhido a classe, ra軋 e sua histia, distribua os pontos da sua ra軋, juntamente com os bus da classe. O mestre deve ditar quais equipamentos que você vai come軋r, assim como a quantidade de dinheiro que você pode ter. Depois disso é só jogar!
Agora diremos algumas observa鋏es importantes sobre a criacao de personagem.


\begin{itemize}
	\item Em dragon n縊 existe as famosas vantagens/qualidades e desvantagens/defeitos dos personagens. Seu personagem n縊 vai ganhar pontos por ser cego ou ter um inimigo a mais. Do mesmo modo que ele n縊 precisa ter pontos para ter uma espada celestial ou ser uma pessoa com muito dinheiro. Cabe ao mestre analisar se tais poss咩eis fatores s縊 aceit疱eis dentro da campanha, de acordo com a historia do personagem.
O que pode ocorrer e de você ser cego e por causa disso ter os outros sentidos mais agu軋dos, mas n縊 justifica somente por que você tem problema de vis縊 terá mais contatos no submundo. Do mesmo modo, se você tiver uma espada sagrada provavelmente n縊 saberá como usá-la, e será alvo f當il para os que sabem (mais f當il ainda já que é um personagem iniciante). Cabe ao mestre juntamente com o jogador analisar tais itens. Apenas habilidades t馗nicas devem ser compradas com pontos, do mesmo modo alguns poderes necessitam de fetiches para serem realizados.
	\item Cada 5 pontos em consci麩cia d縊 ao personagem um conhecimento extra. Com esses pontos de conhecimentos extras você pode comprar mais conhecimentos ou melhorar os que você tem(ganhando +4 nesse conhecimento). Se você tem um conhecimentos com bus + 3 e deseja gastar 1 ponto extra de conhecimento nesse que você já tem, ele vai aumentar em +3, ficando ao todo +7, e assim por diante (se você tivesse dois conhecimentos extras e desejassa concentrar todos em um so conhecimento, ele ficaria +9 no final). Maiores detalhes na sess縊 sobre conhecimentos/habilidades. Essa regra so serve para os conhecimentos que tem bus.
 	\item O personagem recebe um bus em PV igual ao bus do atributo resist麩cia. Esse bus tamb駑 é utilizado ap a criacao de personagem quando se aumenta os atributos gerais de forma permanente. Ou seja, um personagem com resist麩cia 8 (bonus de 3) recebe 1 PV quando aumenta sua resist麩cia para 9 (bus de 4).
	\item  Para cada 3 pontos no atributo focus, o personagem recebe 1PM. Esse bus tamb駑 é utilizado ap a criacao de personagem quando se aumenta os atributos gerais de forma permanente. Ou seja, um personagem com focus 11 recebe mais 3 PM extras. Se ele aumentar seu focus para 12, seu PM extra aumenta em 1.
	\item Na criacao você pode trocar ponto de bus por ps. A relacao é 2 ps para 1 ponto de bus, mas isso é somente utilizado durante a criacao de personagem.
	\item Cada personagem tem 10 pontos de bus de inicio, n縊 importa qual ra軋 ou classe ele seja. Ponto de bus serve como XP f﨎ico e mental, podendo ser usado para aumentar qualquer um dos dois tipos de atributos. Os pontos n縊 usados podem ser convertidos em XP f﨎ico ou mental na relacao 1 para 1.
	\item Todo personagem come軋 com n咩el mental e f﨎ico 1. O mestre pode, em determinadas aventuras, mudar isso dando um, dois, ou mais n咩eis para o personagem.
	\item O mestre pode limitar durante a criacao de personagem um valor maximo para os atributos dos personagem, que por base deve ser 10.	
	\item O personagem pode trocar habilidades de classe por pontos de bus. Se o jogador n縊 desejar certa habilidade de classe ele pode trocá-la por pontos. Cada habilidade cede 6 pontos de bus ao personagem. O mesmo pode ser feito para algumas habilidades de ra軋.
	\item O personagem recebe 10 PO para gastar em itens durante sua criacao, caso o mestre permita.
	\item Todo personagem tem direito a uma habilidade in偀il. O uso de habilidade in偀il foi criado no sistema dragon para dar uma caracterizacao diferente para seu personagem. Uma habilidade in偀il deve ser algo que incremente seu personagem, mas n縊 o ajudará em testes e situa鋏es de vida e morte. Exemplos de habilidades in偀eis s縊, poses de vitoria, lamber o cotovelo, arrotar cantando, saber hinos de times de esporte, criar frases estranhas, cantadas horr咩eis, cantar musica brega, entre outras.
	\item O personagem deve gastar todo os seus pontos de bus iniciais, ou converte-los em outro tipo de experi麩cia segundo o mestre.
	\item O mestre pode usar a seguinte regra opcional para alcan軋r maior coer麩cia no personagem. Caso o personagem coloque dois pontos favorecendo apenas um atributo, o mesmo deve ser 9 ou 10. Essa regra opcional n縊 deve ser usada caso o personagem tenha escolhido alguma ra軋 que tenha poucos pontos para distribuir em f﨎ico.
\end{itemize}

Ao final deste livro ser縊 mostrados maiores exemplos de criacao de personagem utilizando o sistema.


